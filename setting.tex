\documentclass[aspectratio=169, dvipdfmx, 12pt, compress]{beamer}% dvipdfmxしたい

%%%%% Packages %%%%%
\usepackage{bxdpx-beamer}% dvipdfmxなので必要
\usepackage{pxjahyper}% 日本語で'しおり'したい
\usepackage{tikz}
\usepackage{tcolorbox}
\usetikzlibrary{shapes}
\usepackage{xcolor}
\usepackage[absolute,overlay]{textpos}
\usepackage{adjustbox}
\usepackage{caption}


%%%%% Settings %%%%%
\usetheme[sectionpage=progressbar, subsectionpage=progressbar]{metropolis}
\metroset{block=fill}
\setbeamerfont{frametitle}{size=\large}
\addtobeamertemplate{frametitle}{}{\vspace{-1em}}
\addtobeamertemplate{footline}{}{\vspace{-1em}}
\definecolor{coolblack}{rgb}{0.0, 0.18, 0.39}
\setbeamercolor{frametitle}{bg=coolblack,fg=white}
\definecolor{lightgray}{rgb}{0.83, 0.83, 0.83}
\setbeamercolor{progress bar}{bg=lightgray, fg=coolblack}
\makeatletter
\setlength{\metropolis@frametitle@padding}{1.4ex}% <- default 2.2 ex
\setbeamersize{text margin left=15pt, text margin right=15pt}
\usefonttheme{professionalfonts}
\definecolor{cobalt}{rgb}{0.0, 0.28, 0.67}
% Change standard block width
\addtobeamertemplate{block begin}{%
    \centering
    \begin{columns}\begin{column}{0.9\textwidth}
        \centering
        }{}
        \addtobeamertemplate{block end}{}{\end{column}\end{columns}}
% Change alert block width
\addtobeamertemplate{block alerted begin}{%
    \centering
    \begin{columns}\begin{column}{0.9\textwidth}
        \centering
        }{}
        \addtobeamertemplate{block alerted end}{}{\end{column}\end{columns}}
% Change example block width
\addtobeamertemplate{block example begin}{%
    \centering
    \begin{columns}\begin{column}{0.9\textwidth}
        \centering
        }{}
        \addtobeamertemplate{block example end}{}{\end{column}\end{columns}}
% Itemize color
\setbeamertemplate{itemize item}{\color{black}\scriptsize$\blacksquare$}
\setbeamertemplate{itemize subitem}{\color{black}\scriptsize$-$}
% Assumption color
\setbeamercolor{block title example}{fg=black!80,bg=cobalt!35}
\setbeamercolor{block body example}{fg=black,bg=cobalt!15}
% Definition
\BeforeBeginEnvironment{definition}{
    \setbeamercolor{block title}{use=alerted text, bg=alerted text.fg!70,fg=white}
    \setbeamercolor{block body}{use=alerted text, bg=alerted text.fg!20}
}
\AfterEndEnvironment{definition}{% return to default
    \setbeamercolor{block title}{use=structure,fg=structure.fg,bg=structure.fg!20!bg}
    \setbeamercolor{block body}{parent=normal text,use=block title,bg=block title.bg!50!bg, fg=black}
}
% Theorem
\definecolor{seagreen}{rgb}{0.18, 0.55, 0.34}
\setbeamertemplate{theorems}[numbered]
\BeforeBeginEnvironment{theorem}{
    \setbeamercolor{block title}{fg=black!80,bg=seagreen!40}
    \setbeamercolor{block body}{fg=black,bg=seagreen!15}
}
\AfterEndEnvironment{theorem}{% return to default
    \setbeamercolor{block title}{use=structure,fg=structure.fg,bg=structure.fg!20!bg}
    \setbeamercolor{block body}{parent=normal text,use=block title,bg=block title.bg!50!bg, fg=black}
}
% Lemma
\undef{\lemma}
\newtheorem{lemma}{\translate{Lemma}}
\BeforeBeginEnvironment{lemma}{
    \setbeamercolor{block title}{fg=black!80,bg=seagreen!20}
    \setbeamercolor{block body}{fg=black,bg=seagreen!10}
}
\AfterEndEnvironment{lemma}{% return to default
    \setbeamercolor{block title}{use=structure,fg=structure.fg!80,bg=structure.fg!20!bg}
    \setbeamercolor{block body}{parent=normal text,use=block title,bg=block title.bg!50!bg, fg=black}
}


%%%%% Original Command %%%%%
\newcommand{\subt}[1]{\vspace{-2mm}{\fontsize{10pt}{0cm}\selectfont \textcolor{lightgray}{#1}}\vspace{-1mm}}
\newcommand{\lastpage}[0]{\begin{frame}\begin{textblock*}{1.0\linewidth}(0pt, 50pt)\includegraphics[scale=0.512]{\beamerDir/master_figure/last.pdf}\end{textblock*}\end{frame}}
\newcommand{\todo}[1]{\al{\LARGE\textbf{TODO:} #1}}
\newcommand{\headerheight}[0]{8mm}
\newcommand{\tabml}[1]{\hspace{-2.1mm}\begin{tabular}{l} #1 \end{tabular}}
\newcommand{\al}[1]{\alert{#1}}
\newcommand{\assume}[1]{\begin{exampleblock}{Assumption}#1\end{exampleblock}}% Assumption

\tcbset{
    framebox/.style={
            enhanced,
            boxsep=0pt,       % 箱の上下左右の余白を指定
            colback=white,
            boxrule=1pt,
            colframe=#1
        },
    framebox/.default=red
}
\newcommand{\upbln}[3]{
    \tcboxmath[
        framebox=#2,
        top=0.5ex,bottom=0.5ex,    % 箱の上下の余白を指定
        left=0.5ex,right=0.5ex,    % 箱の左右の余白を指定
        overlay={
                \node[
                    above,
                    rectangle callout,                         % nodeを吹き出しの形に
                    callout absolute pointer={(frame.north)},  % 吹き出しの先端を絶対的に指定
                    fill=#2!20
                ] at ([yshift=2ex]frame.north) {\footnotesize#3};
            }
    ]{#1}
}
\newcommand{\lwbln}[3]{
    \tcboxmath[
        framebox=#2,
        top=0.5ex,bottom=0.5ex,    % 箱の上下の余白を指定
        left=0.5ex,right=0.5ex,    % 箱の左右の余白を指定
        overlay={
                \node[
                    below,
                    rectangle callout,                         % nodeを吹き出しの形に
                    callout absolute pointer={(frame.south)},  % 吹き出しの先端を絶対的に指定
                    fill=#2!20
                ] at ([yshift=-2ex]frame.south) {\footnotesize#3};
            }
    ]{#1}
}
\tcbuselibrary{theorems,skins}
