\documentclass[aspectratio=169, dvipdfmx, 14pt, compress]{beamer}% dvipdfmxしたい

%%%%% Packages %%%%%
\usepackage{bxdpx-beamer}% dvipdfmxなので必要
\usepackage{pxjahyper}% 日本語で'しおり'したい
\usepackage{tikz}
\usepackage{tcolorbox}
\usetikzlibrary{shapes}
\usepackage{xcolor}
\usepackage[absolute,overlay]{textpos}
\usepackage{adjustbox}
\usepackage{caption}


%%%%% Settings %%%%%
\usetheme[sectionpage=none, subsectionpage=progressbar]{metropolis}
\metroset{block=fill}
\setbeamerfont{frametitle}{size=\normalsize}
\addtobeamertemplate{frametitle}{}{\vspace{-2em}}
\definecolor{coolblack}{rgb}{0.0, 0.18, 0.39}
\setbeamercolor{frametitle}{bg=coolblack,fg=white}
\definecolor{lightgray}{rgb}{0.83, 0.83, 0.83}
\setbeamercolor{progress bar}{bg=lightgray, fg=coolblack}
\makeatletter
\setlength{\metropolis@frametitle@padding}{1.4ex}% <- default 2.2 ex
\setbeamersize{text margin left=15pt, text margin right=15pt}


%%%%% Original Command %%%%%
\newcommand{\subt}[1]{\vspace{-2mm}{\fontsize{10pt}{0cm}\selectfont \textcolor{lightgray}{#1}}\vspace{-1mm}}
\newcommand{\lastpage}[0]{\begin{frame}\begin{textblock*}{1.0\linewidth}(0pt, 50pt)\includegraphics[scale=0.512]{\beamerDir/master_figure/last.pdf}\end{textblock*}\end{frame}}

\tcbset{
    framebox/.style={
            enhanced,
            boxsep=0pt,       % 箱の上下左右の余白を指定
            colback=white,
            boxrule=1pt,
            colframe=#1
        },
    framebox/.default=red
}
\newcommand{\upbln}[3]{
    \tcboxmath[
        framebox=#2,
        top=0.5ex,bottom=0.5ex,    % 箱の上下の余白を指定
        left=0.5ex,right=0.5ex,    % 箱の左右の余白を指定
        overlay={
                \node[
                    above,
                    rectangle callout,                         % nodeを吹き出しの形に
                    callout absolute pointer={(frame.north)},  % 吹き出しの先端を絶対的に指定
                    fill=#2!20
                ] at ([yshift=2ex]frame.north) {\footnotesize#3};
            }
    ]{#1}
}
\newcommand{\lwbln}[3]{
    \tcboxmath[
        framebox=#2,
        top=0.5ex,bottom=0.5ex,    % 箱の上下の余白を指定
        left=0.5ex,right=0.5ex,    % 箱の左右の余白を指定
        overlay={
                \node[
                    below,
                    rectangle callout,                         % nodeを吹き出しの形に
                    callout absolute pointer={(frame.south)},  % 吹き出しの先端を絶対的に指定
                    fill=#2!20
                ] at ([yshift=-2ex]frame.south) {\footnotesize#3};
            }
    ]{#1}
}
\tcbuselibrary{theorems,skins}
