\documentclass[aspectratio=169, dvipdfmx, 14pt]{beamer}% dvipdfmxしたい

%%%%% Packages %%%%%
\usepackage{bxdpx-beamer}% dvipdfmxなので必要
\usepackage{pxjahyper}% 日本語で'しおり'したい
\usepackage{tikz}
\usepackage{tcolorbox}
\usetikzlibrary{shapes}
\usepackage{xcolor}


%%%%% Settings %%%%%
\usetheme{metropolis}
\metroset{sectionpage=progressbar}
\metroset{block=fill}

%%%%% Original Command %%%%%
\newcommand{\hlcap}[3][yellow]{\tikz[baseline=(x.base)]{\node[rectangle,rounded corners,fill=#1!10](x){$#2$} node[below of=x, color=#1]{#3};}}
\newcommand{\subt}[1]{\vspace{-2mm}{\fontsize{10pt}{0cm}\selectfont \textcolor{lightgray}{#1}}\vspace{-1mm}}

\tcbset{
    framebox/.style={
            enhanced,
            boxsep=0pt,       % 箱の上下左右の余白を指定
            colback=white,
            boxrule=1pt,
            colframe=#1
        },
    framebox/.default=red
}
\newcommand{\upbln}[3]{
    \tcboxmath[
        framebox=#2,
        top=0.5ex,bottom=0.5ex,    % 箱の上下の余白を指定
        left=0.5ex,right=0.5ex,    % 箱の左右の余白を指定
        overlay={
                \node[
                    above,
                    rectangle callout,                         % nodeを吹き出しの形に
                    callout absolute pointer={(frame.north)},  % 吹き出しの先端を絶対的に指定
                    fill=#2!20
                ] at ([yshift=2ex]frame.north) {\footnotesize#3};
            }
    ]{#1}
}
\newcommand{\lwbln}[3]{
    \tcboxmath[
        framebox=#2,
        top=0.5ex,bottom=0.5ex,    % 箱の上下の余白を指定
        left=0.5ex,right=0.5ex,    % 箱の左右の余白を指定
        overlay={
                \node[
                    below,
                    rectangle callout,                         % nodeを吹き出しの形に
                    callout absolute pointer={(frame.south)},  % 吹き出しの先端を絶対的に指定
                    fill=#2!20
                ] at ([yshift=-2ex]frame.south) {\footnotesize#3};
            }
    ]{#1}
}
\tcbuselibrary{theorems,skins}
