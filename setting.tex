\documentclass[aspectratio=169, dvipdfmx, 12pt, compress]{beamer}


%%%%% Packages %%%%%
\usepackage{bxdpx-beamer}
\usepackage{pxjahyper}
\usepackage{tikz}
\usepackage{tcolorbox}
\usetikzlibrary{shapes}
\usepackage{xcolor}
\usepackage[absolute,overlay]{textpos}
\usepackage{adjustbox}
\usepackage{caption}
\usepackage{ifthen}
\usepackage{graphics}
\usepackage{hyperref}


%%%%% Settings %%%%%
\usetheme[sectionpage=progressbar, subsectionpage=progressbar]{metropolis}
% -- space between line --
\renewcommand{\baselinestretch}{1.3}
% -- space between item --
\newlength{\wideitemsep}
\setlength{\wideitemsep}{0.9\itemsep}
% \addtolength{\wideitemsep}{1.0pt} <- more space
\let\olditem\item
\renewcommand{\item}{\setlength{\itemsep}{\wideitemsep}\olditem}
% -- Itemize style --
\setbeamertemplate{itemize item}{\color{black}\scriptsize$\blacksquare$}
\setbeamertemplate{itemize subitem}{\color{black}\scriptsize$-$}
% -- image border color --
\renewcommand\fbox{\fcolorbox{black!20}{white}}
% -- frame title --
\definecolor{coolblack}{rgb}{0.0, 0.18, 0.39}
\makeatletter
\setbeamertemplate{frametitle}
{
    \ifbeamercolorempty[bg]{frametitle}{}{\nointerlineskip}%
    \@tempdima=\textwidth%
    \advance\@tempdima by\beamer@leftmargin%
    \advance\@tempdima by\beamer@rightmargin%
    \begin{beamercolorbox}[sep=0.3cm,left,wd=\the\@tempdima]{frametitle}
        \vbox{}\vskip-1.8ex%
        \if@tempswa\else\csname beamer@fteleft\endcsname\fi%
        {\strut\color{coolblack!90}\bfseries\insertframetitle\strut\par%
            {%
                \ifx\insertframesubtitle\@empty%
                    \else%
                    {\usebeamerfont{framesubtitle}\usebeamercolor[fg]{framesubtitle}\insertframesubtitle\strut\par}%
                \fi
            }%
            \vskip-2.0ex%
            \rule{\dimexpr\paperwidth-0.6cm\relax}{0.4pt}}
        \if@tempswa\else\vskip-.3cm\fi
    \end{beamercolorbox}%
    %
}
\makeatother
\setbeamercolor{frametitle}{bg=black!2,fg=black}
\setbeamerfont{frametitle}{size=\large}
\addtobeamertemplate{frametitle}{}{\vspace{-1em}}
\makeatletter
\setlength{\metropolis@frametitle@padding}{1.4ex}% <- default 2.2 ex
% -- foot line --
\addtobeamertemplate{footline}{}{\vspace{-1em}}
% -- normal text color --
\setbeamercolor{normal text}{fg=black!80}
% -- progress bar --
\definecolor{lightgray}{rgb}{0.83, 0.83, 0.83}
\setbeamercolor{progress bar}{bg=lightgray, fg=coolblack}
\setbeamersize{text margin left=15pt, text margin right=15pt}
% -- equation font --
\usefonttheme{professionalfonts}
% -- block style --
\metroset{block=fill}
% -- block border --
\useinnertheme[shadow,rounded]{tcolorbox}
\makeatletter
\tcbset{
    boxrule=0pt,
    borderline={1.5pt}{0pt}{beamer@tcb@titlebg}
}
\makeatother
% -- Change standard block width --
\addtobeamertemplate{block begin}{%
    \vspace{-3mm}
    \centering\begin{columns}\begin{column}{0.9\textwidth}\centering
            }{}
            \addtobeamertemplate{block end}{}{\end{column}\end{columns}}
% -- Change alert block width --
\addtobeamertemplate{block alerted begin}{%
    \vspace{-3mm}
    \centering
    \begin{columns}\begin{column}{0.9\textwidth}
            \centering
            }{}
            \addtobeamertemplate{block alerted end}{}{\end{column}\end{columns}}
% -- Change example block width --
\addtobeamertemplate{block example begin}{%
    \vspace{-3mm}
    \centering
    \begin{columns}\begin{column}{0.9\textwidth}
            \centering
            }{}
            \addtobeamertemplate{block example end}{}{\end{column}\end{columns}}
% -- block color --
\setbeamercolor{block title}{fg=black!80,bg=black!15}
\setbeamercolor{block body}{fg=black!80,bg=white}
% -- Simplification color --
\definecolor{cobalt}{rgb}{0.0, 0.28, 0.67}
\setbeamercolor{block title example}{fg=black,bg=cobalt!20}
\setbeamercolor{block body example}{fg=black!80,bg=white}
% -- Definition --
\BeforeBeginEnvironment{definition}{
    \setbeamercolor{block title}{use=alerted text, bg=alerted text.fg!50,fg=black}
    \setbeamercolor{block body}{use=alerted text, fg=black!80,bg=white}
}
\AfterEndEnvironment{definition}{% return to default
    \setbeamercolor{block title}{use=structure,fg=structure.fg,bg=structure.fg!20!bg}
    \setbeamercolor{block body}{parent=normal text,use=block title,bg=block title.bg!50!bg, fg=black}
}
% -- Theorem --
\definecolor{seagreen}{rgb}{0.18, 0.55, 0.34}
\setbeamertemplate{theorems}[numbered]
\BeforeBeginEnvironment{theorem}{
    \setbeamercolor{block title}{fg=black,bg=seagreen!40}
    \setbeamercolor{block body}{fg=black!80,bg=white}
}
\AfterEndEnvironment{theorem}{% return to default
    \setbeamercolor{block title}{use=structure,fg=structure.fg,bg=structure.fg!20!bg}
    \setbeamercolor{block body}{parent=normal text,use=block title,bg=block title.bg!50!bg, fg=black}
}
% -- Lemma --
\undef{\lemma}
\newtheorem{lemma}{\translate{Lemma}}
\BeforeBeginEnvironment{lemma}{
    \setbeamercolor{block title}{fg=black,bg=seagreen!20}
    \setbeamercolor{block body}{fg=black!80,bg=white}
}
\AfterEndEnvironment{lemma}{% return to default
    \setbeamercolor{block title}{use=structure,fg=structure.fg!80,bg=structure.fg!20!bg}
    \setbeamercolor{block body}{parent=normal text,use=block title,bg=block title.bg!50!bg, fg=black}
}


%%%%% Original Command %%%%%
\newcommand{\lastpage}[0]{\begin{frame}\begin{textblock*}{1.0\linewidth}(0pt, 50pt)\includegraphics[scale=0.512]{\beamerDir/master_figure/last.pdf}\end{textblock*}\end{frame}}
\newcommand{\todo}[1]{\al{\LARGE\textbf{TODO:} #1}}
\newcommand{\headerheight}[0]{8mm}
\newcommand{\footerheight}[0]{3mm}
\newcommand{\slideheight}[0]{\textheight-\headerheight-\footerheight}
\newcommand{\tabml}[1]{\hspace{-2.1mm}\begin{tabular}{l} #1 \end{tabular}}
\newcommand{\al}[1]{\alert{#1}}
\newcommand{\full}[1]{
    \centering
    \begin{columns}
        \begin{column}{\textwidth}
            \begin{minipage}[t][\textheight][c]{\textwidth}
                #1
            \end{minipage}
        \end{column}
    \end{columns}
}
\newcommand{\fullimage}[1]{
    \full{
        \centering
        \begin{columns}
            \begin{column}{\textwidth}
                \centering
                \adjustbox{max width=\textwidth, max height=\slideheight}{
                    \fbox{\includegraphics{#1}}
                }
            \end{column}
        \end{columns}
    }
}
% fit image
\newlength\fitimageht
\newlength\fitotherht
\newsavebox\fitimagebox
\newcommand{\fitimage}[2]{%
    \sbox\fitimagebox{%
        \parbox{\textwidth}{%
            #1\par
        }%
    }%
    \settototalheight{\fitotherht}{%
        \usebox\fitimagebox
    }%
    \setlength\fitimageht{\textheight}%
    \addtolength\fitimageht{-\fitotherht-\headerheight-\footerheight-1\baselineskip}%
    #1\par
    \centering
    \fbox{\includegraphics[width=\textwidth,height=\fitimageht,keepaspectratio]{#2}}
}
% Simplification
\newcommand{\assume}[1]{
    \begin{exampleblock}{Simplification}
        \setlength{\linewidth}{0.98\columnwidth}
        #1
    \end{exampleblock}
}
% Re-post
\setbeamercolor{RepostBox}{fg=black!50, bg=coolblack!5}
\newcommand{\repost}[1]{
    \vspace{2mm}
    \centering
    \begin{columns}
        \begin{column}{0.86\textwidth}
            \begin{beamercolorbox}[wd=\textwidth, sep=1pt, rounded=true, shadow=true]{RepostBox}
                \begin{tabular}{|p{0.95\textwidth}}
                    {\small #1}
                \end{tabular}
            \end{beamercolorbox}
        \end{column}
    \end{columns}
}

% equation ballon
\tcbset{
    framebox/.style={
            enhanced,
            boxsep=0pt,       % 箱の上下左右の余白を指定
            colback=white,
            boxrule=1pt,
            colframe=#1
        },
    framebox/.default=red
}
\newcommand{\upbln}[3]{
    \tcboxmath[
        framebox=#2,
        top=0.5ex,bottom=0.5ex,    % 箱の上下の余白を指定
        left=0.5ex,right=0.5ex,    % 箱の左右の余白を指定
        overlay={
                \node[
                    above,
                    rectangle callout,                         % nodeを吹き出しの形に
                    callout absolute pointer={(frame.north)},  % 吹き出しの先端を絶対的に指定
                    fill=#2!20
                ] at ([yshift=2ex]frame.north) {\footnotesize#3};
            }
    ]{#1}
}
\newcommand{\lwbln}[3]{
    \tcboxmath[
        framebox=#2,
        top=0.5ex,bottom=0.5ex,    % 箱の上下の余白を指定
        left=0.5ex,right=0.5ex,    % 箱の左右の余白を指定
        overlay={
                \node[
                    below,
                    rectangle callout,                         % nodeを吹き出しの形に
                    callout absolute pointer={(frame.south)},  % 吹き出しの先端を絶対的に指定
                    fill=#2!20
                ] at ([yshift=-2ex]frame.south) {\footnotesize#3};
            }
    ]{#1}
}
\tcbuselibrary{theorems,skins}
