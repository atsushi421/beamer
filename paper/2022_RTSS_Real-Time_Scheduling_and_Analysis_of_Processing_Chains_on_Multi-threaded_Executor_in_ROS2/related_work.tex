% !TeX root = main.tex

\section{RELATED WORK}
\label{sec: related work}


\begin{frame}{}
    \begin{itemize}
        \item ROS [18], [19], [22] でのリアルタイムパフォーマンスの向上を目的とした多くの作業が行われている
\item たとえば, [23] は, 透過的な CPU/GPU 調整メカニズムのリアルタイム拡張を提案した
\item [14] は, 分散システムの ROS 2 のエンドツーエンドのレイテンシを研究した
\item [9] ROS 2 の標準の rclcpp Executor を Callback-group-level Executor と呼ばれるリアルタイムの Executor に置き換え, その結果のパフォーマンスを調査した
\item ただし, これらの研究はリアルタイム機能を向上させるだけで, 分析のリアルタイム性を保証するものではない
    \end{itemize}
\end{frame}

\begin{frame}{}
    \begin{itemize}
        \item Casiniは,ROS 2エグゼキュータの形式的モデリングと解析に関する最初の仕事[8]を行い,シングルスレッドエグゼキュータのスケジューリング動作をモデル化し,その後の解析の基礎を築いた.[10]は,応答時間分析技術を改善し,応答時間を最適化するための優先度割り当て戦略を提案した.[12]は,実際の実行時間のばらつきを考慮して[8]を改良し,ROS 2のスケジューリング機能をさらに探求しました.[13]は,ROS 2フレームワークのための新しい優先順位駆動型チェーンアウェアスケジューラを提案し,提案スケジューラのエンドツーエンドレイテンシ分析を実施しました.[24] は ROS Live latency manager (ROSLlama) を導入し,エンドツーエンド遅延を制御するために既存のリアルタイムメカニズムを使用し,ランタイムに必要なすべてのスケジューリングパラメータを自動的に推定しました.特に,[24]は応答時間を束縛するために[8]の結果に依存しています.要約すると,すべての既存研究は,シングルスレッド実行ファイルの分析に重点を置いています.
    \end{itemize}
\end{frame}

\begin{frame}{}
    \begin{itemize}
        \item ROS 2 での作業にもかかわらず, 処理チェーンの分析も研究されている
\item 主に 2 つのカテゴリがある
\item チェーン内のタスクが個別にトリガされるデータチェーン [25], [26] と, タスクが先行タスクによってトリガされるトリガチェーンである
\item 広い意味で, このホワイト 論文で検討している ROS 2 エグゼキュータの処理チェーンは, トリガチェーンのカテゴリに属す
\item トリガチェーンの応答時間を分析するための最も一般的なフレームワークの 2 つは, 構成パフォーマンス分析 [27] とリアルタイム微積分 [28] であり, これらに基づいて [29] は, 同期通信と非同期通信を使用する一連のスレッドの応答時間を計算した
\item [30] 複数のプロセッサを横断するトリガチェーンのエンドツーエンドのレイテンシを最適化した
\item マルチプロセッサ上のシーケンシャルリアルタイムタスクの従来のノンプリエンプティブスケジューリングについて, [31] は, EDF でスケジュールされた周期的タスクに対して, 十分ではあるが必須ではない多項式時間のスケジューリング可能性テスト条件を提案した
\item [32] は, EDF の下で線形時間の複雑さを備えた新しい十分なテストを提案し, さらに, EDF と FP スケジューリングの疑似多項式の時間複雑さの改善されたテスト条件をそれぞれ提示した
\item それにもかかわらず, これらのすべての作業は, ROS 2 エグゼキュータの特定のスケジューリング機能を捉えることができないため, 直接適用することはできない
    \end{itemize}
\end{frame}
