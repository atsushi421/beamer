% !TeX root = main.tex

\section{SYSTEM MODEL}
\label{sec: system_model}

\begin{frame}{セクションサマリ}
    \begin{itembox}[l]{\textbf{目的}}
        実装から抽出したマルチスレッドエグゼキュータのリアルタイム関連の動作をカバーするスケジューリングモデルを定義
    \end{itembox}
\end{frame}

\begin{frame}{}
    \begin{itemize}
        \item 始めに, 本論文で登場する表記法・用語の表を示す
        \item 基本的な表記法・用語は資料中で説明無しで使用する
        \item 暗記するのは困難なため, 別ファイルで開く・印刷するなどして, 常に参照できる状態にしておくことを推奨する
    \end{itemize}
\end{frame}

% !TeX root = main.tex

\begin{frame}{表記法・用語 1}
    \full{
        \begin{table}[tb]
            \adjustbox{max width=\textwidth, max height=\slideheight}{
                \centering\begin{tabular}{|c|l|} \hline
                    $m$                                                                                             & スレッド数                                                                                        \\\hline
                    $\Gamma=\left\{\mathcal{C}_{1}, \mathcal{C}_{2}, \cdots, \mathcal{C}_{|\Gamma|}\right\}$        & チェインのセット                                                                                  \\\hline
                    $|\Gamma|$                                                                                      & $\Gamma$内のチェインの数                                                                          \\\hline
                    $\mathcal{C}_{i}=\left\{c_{i, 1}, c_{i, 2}, \cdots, c_{i,\left|\mathcal{C}_{i}\right|}\right\}$ & チェイン                                                                                          \\\hline
                    $c_{i,j}$                                                                                       & $\mathcal{C}_{i}$の$j$番目のコールバック                                                          \\\hline
                    $\left|\mathcal{C}_{i}\right|$                                                                  & $\mathcal{C}_{i}$内のコールバックの数                                                             \\\hline
                    先行要素                                                                                        & $J_{i}^{k}$ 内の連続する要素 $c_{i, j}^{k}$ と $c_{i, j+1}^{k}$ の各ペアにおける $c_{i, j}^{k}$   \\\hline
                    後続要素                                                                                        & $J_{i}^{k}$ 内の連続する要素 $c_{i, j}^{k}$ と $c_{i, j+1}^{k}$ の各ペアにおける $c_{i, j+1}^{k}$ \\\hline
                    ソースコールバック                                                                              & $\mathcal{C}_{i}$ の最初のコールバック                                                            \\\hline
                    シンクコールバック                                                                              & $\mathcal{C}_{i}$ の最後のコールバック                                                            \\\hline
                \end{tabular}
            }
        \end{table}
    }
\end{frame}

\begin{frame}{表記法・用語 2}
    \full{
        \begin{table}[tb]
            \adjustbox{max width=\textwidth, max height=\slideheight}{
                \centering\begin{tabular}{|c|l|} \hline
                    $T_i$                 & \tabml{$\mathcal{C}_{i}$の周期                \\\underline{周期}: 2 つの連続するチェーンインスタンスのリリース時刻の間の最小間隔}                       \\\hline
                    $D_i$                 & \tabml{$\mathcal{C}_{i}$の相対デッドライン    \\\underline{相対デッドライン}: 時間 $r$ でリリースされた $\mathcal{C}_{i}$ の各チェーンインスタンスは, \\その絶対デッドライン $r+D_{i}$ までに終了する必要がある}           \\\hline
                    $e_{i,j}$             & $c_{i,j}$の最悪実行時間 (WCET)                \\\hline
                    $E_i$                 & $\mathcal{C}_{i}$内のコールバックのWCETの合計 \\\hline
                    $U_{i}=E_{i} / T_{i}$ & $\mathcal{C}_{i}$の利用率                     \\\hline
                \end{tabular}
            }
        \end{table}
    }
\end{frame}

\begin{frame}{表記法・用語 3}
    \full{
        \begin{table}[tb]
            \adjustbox{max width=\textwidth}{
                \centering\begin{tabular}{|c|l|} \hline
                    $J_{i}^{k}$                & $\mathcal{C}_{i}$ の $k$番目のチェーンインスタンス             \\\hline
                    $c_{i, j}^{k}$             & $J_{i}^{k}$ に含まれる$c_{i, j}$ のコールバックインスタンス    \\\hline
                    $R\left(J_{i}^{k}\right)$  & $J_{i}^{k}$の応答時間                                          \\\hline
                    $\mathcal{R}_i^{wc}$       & $\mathcal{C}_{i}$の最悪応答時間                                \\\hline
                    $\Omega$                   & ready セット                                                   \\\hline
                    $h p\left(c_{i, j}\right)$ & コールバック $c_{i, j}$ よりも優先度の高いコールバックのセット \\\hline
                \end{tabular}
            }
        \end{table}
    }
\end{frame}

\begin{frame}{表記法・用語 4}
    \full{
        \begin{table}[tb]
            \adjustbox{max width=\textwidth, max height=\slideheight}{
                \centering\begin{tabular}{|c|l|} \hline
                    updated  & $\Omega$ に新しい要素が追加されること                                                      \\\hline
                    バッチ   & \tabml{複数のコールバックインスタンスが同じポーリングポイントで $\Omega$ に追加された場合, \\これらのインスタンスは同じバッチである} \\\hline
                    ビジー   & スレッドがコールバックインスタンスが実行している状態                                       \\\hline
                    アイドル & スレッドがコールバックインスタンスを実行していない状態                                     \\\hline
                \end{tabular}
            }
        \end{table}
    }
\end{frame}

\begin{frame}{表記法・用語 5}
    \full{
        \begin{table}[tb]
            \adjustbox{max width=\textwidth, max height=\slideheight}{
                \centering\begin{tabular}{|c|l|} \hline
                    $J$   & 分析対象のチェーン                     \\\hline
                    $r$   & $J$のリリース時刻                      \\\hline
                    $f$   & $J$の終了時刻                          \\\hline
                    $c_i$ & $J$の$i$番目のコールバックインスタンス \\\hline
                    $r_i$ & $c_i$がリリースされる時刻              \\\hline
                    $s_i$ & $c_i$が実行開始する時刻                \\\hline
                    $|J|$ & $J$内のコールバックの数                \\\hline
                \end{tabular}
            }
        \end{table}
    }
\end{frame}


\begin{frame}{表記法・用語 6}
    \full{
        \begin{table}[tb]
            \adjustbox{max width=\textwidth, max height=\slideheight}{
                \centering\begin{tabular}{|c|l|} \hline
                    $\mathcal{S}_{k, i}=\left\langle e_{k, a}^{\prime}, e_{k, b}^{\prime}, \cdots\right\rangle$                                                         & $c_i$に対する$\mathcal{C}_{k}$のサブ干渉シーケンス                                            \\\hline
                    $e_{k, a}^{\prime}$                                                                                                                                 & コールバックインスタンス $c_{k, a}$ が $\left[r_{i}, s_{i}\right)$ の間に実行された時間の長さ \\\hline
                    $\mathcal{S}_{k}=\left\{\mathcal{S}_{k, 1}, \mathcal{S}_{k, 2}, \cdots, \mathcal{S}_{k,|\mathcal{C}|}\right\}$                                      & $J$に対する$\mathcal{C}_{k}$の干渉シーケンス                                                  \\\hline
                    $\mathcal{I}_{k,i}$                                                                                                                                 & $c_i$に対する$\mathcal{C}_{k}$の干渉作業                                                      \\\hline
                    $\mathcal{I}_{k}$                                                                                                                                   & $J$に対する$\mathcal{C}_{k}$の干渉作業                                                        \\\hline
                    $\mathcal{I}_{k,i}^\mathcal{P} $                                                                                                                    & \tabml{$c_i$がブロックされている間に, $c_i$と同じコールバックグループに属す                   \\$\mathcal{C}_k$のコールバックインスタンスが実行した時間の総和} \\\hline
                    $\mathcal{I}_{k,i}^\mathcal{E} $                                                                                                                    & \tabml{$c_i$がブロックされている間に, $c_i$と異なるコールバックグループに属す                 \\$\mathcal{C}_k$のコールバックインスタンスが実行した時間の総和} \\\hline
                    $\mathcal{I}_{k,i}^\mathcal{B}  $                                                                                                                   & \tabml{$[r_i, s_i)$の間に, $c_i$と同じmutually exclusiveコールバックグループに属す            \\$\mathcal{C}_k$のコールバックインスタンスが実行した時間の総和} \\\hline
                    $\mathcal{Q}_{k}=\sum_{i=1}^{|\mathcal{C}|}\left(\mathcal{I}_{k, i}+(m-1) \mathcal{I}_{k, i}^{\mathcal{B}}-\mathcal{I}_{k, i}^{\mathcal{E}}\right)$ & $\mathcal{C}_{k}$の実行が$J$の終了時間に与える影響を特徴付けるために開発した項                \\\hline
                    $\Phi_{k,i}$                                                                                                                                        & $\mathcal{Q}_k$に対する $\mathcal{S}_{k,i}$の寄与                                             \\\hline
                \end{tabular}
            }
        \end{table}
    }
\end{frame}

\begin{frame}{表記法・用語 7}
    \full{
        \begin{table}[tb]
            \adjustbox{max width=\textwidth, max height=\slideheight}{
                \centering\begin{tabular}{|c|l|} \hline
                    $L$                                    & 問題ウィンドウの長さ                                                                                         \\\hline
                    $n_{k}(L)$                             & $J$ の問題ウィンドウ中に実行できる $\mathcal{C}_{k}$ のチェーンインスタンスの最大数                          \\\hline
                    $\overrightarrow{\mathcal{M}}_{k}$     & $J$に対する$\mathcal{C}_{k}$の超干渉シーケンス                                                               \\\hline
                    $ \hat{\Phi} $                         & $\Phi_{k,i}$の上界                                                                                           \\\hline
                    $\mathcal{G}(c_{i,j}) $                & \tabml{$c_{i,j}$が属すmutually exclusiveコールバックグループのインデックス}                                  \\\hline
                    $\theta_i$                             & \tabml{$\mathcal{C}_{i}$ の各コールバックが属すmutually exclusiveコールバックグループの集合                  \\ $\theta_{i}=\cup_{\forall c_{i, j} \in \mathcal{C}_{i}}\left\{\mathcal{G}\left(c_{i, j}\right)\right\}$} \\\hline
                    $\mathcal{I}_{k, i}^{\mathcal{E}^{*}}$ & $\mathcal{S}_{k, i}$ 内の $c_{i}$ とは異なるコールバックグループに属すコールバックインスタンスの合計実行時間 \\\hline
                    $\mathcal{Q}_{k}(\mathcal{Y})$         & 各サブ干渉シーケンス $\mathcal{S}_{k, i}$ に関する $\hat{\Phi}_{k, i}$ の合計                                \\\hline
                \end{tabular}
            }
        \end{table}
    }
\end{frame}

\begin{frame}{表記法・用語 8}
    \full{
        \begin{table}[tb]
            \adjustbox{max width=\textwidth, max height=\slideheight}{
                \centering\begin{tabular}{|c|l|} \hline
                    $\Phi_{k, i}^{p, q}$                            & \tabml{$\overrightarrow{\mathcal{M}}_{k}$ の$ p $番目のコールバックインスタンスの開始時刻から \\$ q $番目のコールバックインスタンスの開始時刻までの \\範囲内にある任意のウィンドウによって発生しうる最大の $\hat{\Phi}_{k, i}$} \\\hline
                    $\left|\overrightarrow{\mathcal{M}}_{k}\right|$ & $\overrightarrow{\mathcal{M}}_{k}$ のコールバックインスタンスの数                             \\\hline
                    $\Theta_{i, p}$ ($i \in[1,|\mathcal{C}|])$      & \tabml{$ p $番目のコールバックインスタンスの開始時刻から                                      \\$\overrightarrow{\mathcal{M}}_{k}$ の最後のコールバックインスタンスの終了時刻までの \\範囲に収まる任意のウィンドウによって発生しうる最大の $\sum_{j=i}^{|\mathcal{C}|} \hat{\Phi}_{k, j}$} \\\hline
                \end{tabular}
            }
        \end{table}
    }
\end{frame}


\subsection{Workload Model}
\label{ssec: workload_model}

\begin{frame}{スレッドに関する簡単化}
    \assume{各スレッドは専用プロセッサに静的に展開されており, スレッドはいつでもプロセッサを使用できる}
\end{frame}

\begin{frame}{チェーンモデル}
    \begin{itemize}
        \item 本論文では, $m$ スレッド上のマルチスレッドエグゼキュータによってスケジュールされる一連の独立したチェーン $\Gamma=\left\{\mathcal{C}_{1}, \mathcal{C}_{2}, \cdots, \mathcal{C}_{|\Gamma|}\right\}$ を検討する
        \item 各チェーン $\mathcal{C}_{i} \in \Gamma$ は, コールバックのシーケンス $\mathcal{C}_{i}=\left\{c_{i, 1}, c_{i, 2}, \cdots, c_{i,\left|\mathcal{C}_{i}\right|}\right\}$ で構成される
        \item 非シンクコールバックインスタンス $c_{i, j}^{k}$ が実行を終了すると, 後続のインスタンスを呼び出すメッセージを生成する
        \item チェーン $\mathcal{C}_{i}$ は, チェーンインスタンスの無限シーケンスをリリースする
        \item 各チェーンは制約付きデッドラインを持つ, すなわち $D_{i} \leq T_{i}$
    \end{itemize}
\end{frame}

\begin{frame}{応答時間の定義}
    \begin{definition}[チェーンインスタンスの応答時間 $R\left(J_{i}^{k}\right)$]
        チェーンインスタンスの応答時間 $R\left(J_{i}^{k}\right)$ は,  $J_{i}^{k}$ のリリース時刻と終了時刻の間の時間間隔
    \end{definition}
    \begin{definition}[チェーンの最悪応答時間$\mathcal{R}_{i}^{w c}$]
        チェーン $\mathcal{C}_{i}$ の最悪応答時間 $\mathcal{R}_{i}^{w c}$ は, その全てのインスタンスの中で最大の応答時間
    \end{definition}
\end{frame}

\begin{frame}{スケジュール可能の定義}
    \begin{definition}[チェーン$\mathcal{C}_{i}$がスケジュール可能]
        チェーンの全てのインスタンスがデッドラインを満たしている場合 (すなわち, $\mathcal{R}_{i}^{w c} \leq D_{i}$), チェーン $\mathcal{C}_{i}$はスケジュール可能である
    \end{definition}
    \begin{definition}[チェーンセット$\Gamma$がスケジュール可能]
        $\Gamma$ の全てのチェーンがスケジュール可能である場合 (すなわち, $\forall i: \mathcal{R}_{i}^{w c} \leq D_{i}$), $\Gamma$はスケジュール可能である
    \end{definition}
\end{frame}

\begin{frame}{本論文の目的}
    本論文の目的は, 以下2つ
    \begin{itemize}
        \item システム $\Gamma$ がマルチスレッドエグゼキュータでスケジュール可能かを判断
        \item $\Gamma$ がスケジュール可能である場合, 各チェーン $\mathcal{C}_{i} \in \Gamma$ の $\mathcal{R}_{i}^{w c}$ の安全な上界を計算
    \end{itemize}
\end{frame}


\subsection{Scheduling Model}
\label{ssec: scheduling_model}


\begin{frame}{コールバックインスタンスのリリースタイミング}
    \begin{itemize}
        \item \tb{コールバックインスタンスのリリースタイミングは以下2パターン}
              \begin{block}{ソースコールバックインスタンス}
                  $J_{i}^{k}$ 内のソースコールバックインスタンス $c_{i, 1}^{k}$ は, $J_{i}^{k}$ のリリースと同時にリリースされる
              \end{block}
              \begin{block}{非ソースコールバックインスタンス}
                  $J_{i}^{k}$ 内の非ソースコールバックインスタンスは, その先行の終了時, すなわち, 先行からの入力メッセージが利用可能になった時にリリースされる
              \end{block}
              \vspace{5mm}
        \item 同じコールバックの複数のインスタンスがリリースされた場合, リリース時刻を基準にFIFOにエンキューされる
    \end{itemize}
\end{frame}

\begin{frame}{pending}
    \begin{definition}[pending]
        コールバックインスタンス$c_{i, j}$がリリースされ, 自身の $j-1$ 以前のインスタンスが全て実行中か終了しているとき, そのインスタンスは pending である
    \end{definition}
\end{frame}

\begin{frame}{ready, Pブロック}
    \begin{definition}[Pブロック]
        $c_{i, j}$ が mutually exclusive コールバックグループに属しており, $c_{i, j}$ が pending かつ, $c_{i, j}$ と同じグループ内のコールバックのインスタンス ($c_{i, j}$ 自体を含む) が実行されている場合, $c_{i, j}$ はPブロックされている
    \end{definition}
    \begin{definition}[ready]
        コールバックインスタンスが pending であり, Pブロックされていない場合, そのコールバックインスタンスは ready である
    \end{definition}
\end{frame}

\begin{frame}{pendingコールバックインスタンスの状態}
    \tb{pendingコールバックインスタンス$c_{i, j}$は, ready または Pブロックされている}
    \begin{block}{$c_{i, j}$ が mutually exclusive コールバックグループに属している場合}
        \setlength{\linewidth}{0.98\columnwidth}
        \begin{itemize}
            \item $c_{i, j}$ と同じコールバックグループ内のコールバックのインスタンスが実行されていない場合, $c_{i, j}$ はready
            \item それ以外の場合, $c_{i, j}$ はPブロックされている
        \end{itemize}
    \end{block}
    \begin{block}{$c_{i, j}$ がreentrantコールバックグループに属している場合}
        $c_{i, j}$ は常に ready
    \end{block}
\end{frame}

\begin{frame}{$\Omega$}
    \begin{itemize}
        \item エグゼキュータは readyコールバックインスタンスを保持する共通のreadyセット $\Omega$ を持つ
        \item \tb{コールバックインスタンスが実行のために選択されると, $\Omega$ から削除される}
    \end{itemize}
\end{frame}

\begin{frame}{Rブロック, eligible}
    \tb{$\Omega$ 内の ready のコールバックインスタンス $c_{i, j}$ が実行のために選択される条件は eligible であること}
    \begin{definition}[Rブロック]
        $c_{i, j}$ が mutually exclusive コールバックグループに属すかつ, $\Omega$ 内に存在し, $c_{i, j}$ と同じグループ内のコールバックのインスタンスが実行されている場合, $c_{i, j}$ は Rブロックされている
    \end{definition}
    \begin{definition}[eligible]
        $c_{i, j}$ が $\Omega$ 内に存在し, Rブロックされていない場合, eligible である
    \end{definition}
\end{frame}

\begin{frame}{$\Omega$ 内のコールバックインスタンスの状態}
    \tb{$\Omega$ 内のコールバックインスタンスは eligible であるか, Rブロックされている}
    \begin{block}{$c_{i, j}$ が mutually exclusive コールバックグループに属す場合}
        \setlength{\linewidth}{0.98\columnwidth}
        \begin{itemize}
            \item $c_{i, j}$ と同じコールバックグループ内のコールバックのインスタンスが実行されていない場合, $c_{i, j}$ は eligible
            \item それ以外の場合, $c_{i, j}$ はRブロックされている
        \end{itemize}
    \end{block}
    \begin{block}{$c_{i, j}$ が reentrant コールバックグループに属す場合}
        $c_{i, j}$ は 常に eligible
    \end{block}
\end{frame}

\begin{frame}{mutually exclusiveコールバックグループの特徴}
    上記の設計により, \tb{同じmutually exclusiveコールバックグループに属すコールバックのインスタンスは並列に実行されない}
\end{frame}


\begin{frame}{ポーリングポイント}
    コールバックインスタンスは ready でもすぐには $\Omega$ に追加されず, ポーリングポイントのみで $\Omega$ に追加される
    \begin{definition}[ポーリングポイント]
        $\Omega$ に eligible なコールバックが無く, スレッドがアイドル状態の場合に, ready のコールバックインスタンスを $\Omega$ に追加する時点
    \end{definition}
\end{frame}

\begin{frame}{スレッド}
    \begin{itemize}
        \item \tb{$\Omega$ 内のeligibleなコールバックインスタンスは, $m$ 個のスレッドによって1つずつ選択され, ノンプリエンプティブに実行される}
        \item スレッドとは, 処理リソースの観点から, 対応するプロセッサを表すもの
    \end{itemize}
\end{frame}

\begin{frame}{コールバックインスタンスの優先度}
    \begin{itemize}
        \item \tb{eligibleなコールバックインスタンスを選択する順序はそれらの優先度によって決定される}
        \item 各コールバックには固定の固有の優先度があり, 全てのインスタンスがこの優先度を継承する
    \end{itemize}
\end{frame}

\begin{frame}{ブロック}
    \begin{definition}[コールバックインスタンス $c_{i, j}$ がブロックされている]
        $c_{i, j}$ が Pブロックまたは Rブロックされている場合, ブロックされている
    \end{definition}
    \begin{definition}[チェーンインスタンス $J_{i}^{k}$ がブロックされている]
        $J_{i}^{k}$ 内のコールバックインスタンスがブロックされている場合, $J_{i}^{k}$ はブロックされている
    \end{definition}
\end{frame}

\begin{frame}{実行中}
    \begin{definition}[チェーンインスタンス $J_{i}^{k}$ が実行中]
        $J_{i}^{k}$内のコールバックインスタンスが実行中の場合, $J_{i}^{k}$ は実行中である
    \end{definition}
\end{frame}

\begin{frame}{例の前提1}
    \fitimage{
        \begin{itemize}
            \item $\mathcal{C}_{1}=\left\{c_{1,1}, c_{1,2}\right\}, \mathcal{C}_{2}=\left\{c_{2,1}, c_{2,2}\right\}$ と $\mathcal{C}_{3}=\left\{c_{3,1}, c_{3,2}, c_{3,3}, c_{3,4}\right\}$ の 2 つのスレッドでスケジュールされた 3 つのチェーンがある
            \item $c_{2,2}$ と $c_{3,2}$ は同じmutually exclusiveコールバックグループに属し, 他のコールバックはreentrant コールバックグループに属す
        \end{itemize}
    }{system_model_example_a.png}
\end{frame}

\begin{frame}{例の前提2}
    \fitimage{
        \begin{itemize}
            \item 表は全てのコールバックの優先度と WCET を示す
            \item 数字が小さいほど優先度が高くなる
        \end{itemize}
    }{system_model_example_b.png}
\end{frame}

\begin{frame}{例の前提3}
    \fitimage{
        \begin{itemize}
            \item 上矢印はチェーンインスタンスのリリース時刻を表す
            \item 右の表は各ポーリングポイントでの $\Omega$ 内のコールバックインスタンスを示す
        \end{itemize}
    }{system_model_example_c_d.png}
\end{frame}

\begin{frame}{Rブロック例}
    \fullimage{system_model_example_sup1.jpg}
\end{frame}

\begin{frame}{Pブロック例}
    \fullimage{system_model_example_p_blocked.jpg}
\end{frame}
