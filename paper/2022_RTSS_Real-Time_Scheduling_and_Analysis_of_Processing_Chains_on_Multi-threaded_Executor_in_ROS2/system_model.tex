\section{SYSTEM MODEL}
\label{sec: system_model}


\begin{frame}{}
    以下では, 上記の説明から抽出したマルチスレッド エグゼキューターのリアルタイム関連の動作をカバーするスケジューリング モデルを紹介する.簡単にするために, wait\_set にアクセスまたは更新するスレッドによって発生するオーバーヘッドはゼロであると仮定する
\end{frame}

\begin{frame}{Notations1}
    \begin{table}[tb]
        \adjustbox{max width=\textwidth}{
            \centering\begin{tabular}{|c|l|} \hline
                \textbf{Notations}      & \textbf{Descriptions}                                                                          \\\hline
                $C_i$                   & チェイン                                                                                       \\\hline
                $m$                     & スレッド数                                                                                     \\\hline
                $\Gamma $               & チェインのセット                                                                               \\\hline
                $c_{i,j}$               & $C_i$の$j$番目のコールバック                                                                   \\\hline
                $e_{i,j}$               & $c_{i,j}$のWCET                                                                                \\\hline
                $E_i$                   & $C_i$内のコールバックのWCETの合計                                                              \\\hline
                $D_i$                   & $C_i$のデッドライン                                                                            \\\hline
                $T_i$                   & $C_i$の周期                                                                                    \\\hline
                $U_i$                   & $C_i$の利用率                                                                                  \\\hline
                $\mathcal{G}(c_{i,j}) $ & $c_{i,j}$が属すmutually exclusiveコールバックグループのインデックス                            \\\hline
                $\theta_i$              & \tabml{$\mathcal{C}_{i}$ の各コールバックが属すmutually exclusiveコールバック グループの集合. \\ $\theta_{i}=\cup_{\forall c_{i, j} \in \mathcal{C}_{i}}\left\{\mathcal{G}\left(c_{i, j}\right)\right\}$} \\\hline
            \end{tabular}
        }
    \end{table}
\end{frame}

\begin{frame}{Notation2}
    \begin{table}[tb]
        \adjustbox{max width=\textwidth}{
            \centering\begin{tabular}{|c|l|} \hline
                \textbf{Notations} & \textbf{Descriptions}                                        \\\hline
                $J_{i}^{k}$        & $\mathcal{C}_{i}$ の $k$番目のインスタンス                   \\\hline
                $c_{i, j}^{k}$     & $J_{i}^{k}$ に含まれる$c_{i, j}$ のコールバック インスタンス \\\hline
            \end{tabular}
        }
    \end{table}
\end{frame}

\subsection{Workload Model}
\label{ssec: workload_model}

\begin{frame}{チェーンの定義}
    \begin{itemize}
        \item $m$ スレッド上のマルチスレッド エグゼキュータによってスケジュールされた一連の独立した処理チェーン (略してチェーンと呼ぶ) $\Gamma=\left\{\mathcal{C}_{1}, \mathcal{C}_{2}, \cdots, \mathcal{C}_{|\Gamma|}\right\}$ を検討する
        \item 各スレッドは専用プロセッサに静的に展開されており,スレッドはいつでもプロセッサを使用できる
        \item 各チェーン $\mathcal{C}_{i} \in \Gamma$ は, コールバック $\mathcal{C}_{i}=\left\{c_{i, 1}, c_{i, 2}, \cdots, c_{i,\left|\mathcal{C}_{i}\right|}\right\}$ の順序付けられたシーケンスで構成される.
        \item $\mathcal{C}_{i}$ の最初のコールバック $c_{i, 1}$ (それぞれ, 最後のコールバック $c_{i,\left|\mathcal{C}_{i}\right|}$ ) は, $\mathcal{C}_{i}$ のソース コールバック (それぞれ, シンク コールバック) と呼ばれる
    \end{itemize}
\end{frame}

\begin{frame}{}
    \begin{itemize}
        \item 各コールバック $c_{i, j}$ は最悪実行時間 (WCET) $e_{i, j}$ によって特徴付けられる.
        \item コールバックは, 固有の相互に排他的なコールバック グループまたは再入可能なコールバック グループのいずれかに属している可能性がある.
        \item コールバック $c_{i, j}$ が相互に排他的なコールバック グループに属していない場合, $c_{i, j}$ が再入可能なコールバック グループに属していることを意味する.
    \end{itemize}
\end{frame}

\begin{frame}{}
    \begin{itemize}
        \item チェーン $\mathcal{C}_{i}$ は, チェーン インスタンスの無限シーケンスをリリースする.
        \item $T_{i}$ の期間は, $\mathcal{C}_{i} \cdot \mathcal{C}_{i}$ の 2 つの連続するチェーン インスタンスのリリース時刻の間の最小間隔
        \item 相対デッドライン $D_{i}$ があり, 時間 $r$ でリリースされた $\mathcal{C}_{i}$ の各チェーン インスタンスは, その絶対デッドライン $r+D_{i}$ までに終了する必要がある.
        \item チェーンにはデッドラインが制限されている, つまり $D_{i} \leq T_{i}$ があると想定している.
        \item $\mathcal{C}_{i}$ の使用率は $U_{i}=E_{i} / T_{i}$ である.
    \end{itemize}
\end{frame}

\begin{frame}{}
    \begin{itemize}
        \item $J_{i}^{k}$ 内の連続する要素 $c_{i, j}^{k}$ と $c_{i, j+1}^{k}$ の各ペアについて, $c_{i, j}^{k}$ を $c_{i, j+1}^{k}$ の先行要素, $c_{i, j+1}^{k}$ を $c_{i, j}^{k}$ の後続要素と呼びます.
        \item 非シンク コールバック インスタンス $c_{i, j}^{k}$ が実行を終了すると, 後続のインスタンスを呼び出すメッセージが生成される.
    \end{itemize}
\end{frame}

\begin{frame}{応答時間の定義}
    \begin{itemize}
        \item 応答時間 $R\left(J_{i}^{k}\right)$ は,  $J_{i}^{k}$ のリリース時刻と終了時間の間の時間間隔である.
        \item チェーン $\mathcal{C}_{i}$ の最悪の場合の応答時間 $\mathcal{R}_{i}^{w c}$ は, そのすべてのインスタンスの中で最大の応答時間である.
        \item チェーンは, そのすべてのインスタンスがデッドラインを満たしている場合 (つまり, $\mathcal{R}_{i}^{w c} \leq D_{i}$), スケジュール可能であると言われ, $\Gamma$ は, すべてのチェーンがスケジュール可能である場合 (つまり, $\forall i: \mathcal{R}_{i}^{w c} \leq D_{i}$), スケジュール可能であると言われます
    \end{itemize}
\end{frame}

\begin{frame}{本論文の目的}
    このホワイト ペーパーの目的は, システム $\Gamma$ がマルチスレッド エグゼキュータでスケジュール可能かどうかを判断し, $\Gamma$ がスケジュール可能である場合, 各チェーン $\mathcal{C}_{i} \in \Gamma$ の $\mathcal{R}_{i}^{w c}$ の安全な上限を計算することである.
    \todo{どういうこと?}
\end{frame}
