% !TeX root = main.tex

\section{INTRODUCTION}
\label{sec: introduction}

\begin{frame}{ROS (Robot Operating System)}
    \begin{itemize}
        \item ROS (Robot Operating System) は, ロボット工学アプリケーションで最も人気のあるオープンソースソフトウェアフレームワークの 1 つ
        \item ROS の主な哲学は, ソフトウェアのモジュール性と構成可能性を促進することにより, ロボットソフトウェア開発の生産性を向上させること
        \item ただし, ROS にはリアルタイム機能がなく, 安全性が重要な環境での実装が制限される

    \end{itemize}
\end{frame}

\begin{frame}{ROS 2}
    \begin{itemize}
        \item ROS 2 は 2017 年にリリースされた第 2 世代の ROS であり, ROS から肯定的な側面を継承し, リアルタイムロボットソフトウェアをサポートする強力なリアルタイム機能を提供する
        \item 例えば, ROS 2 は, リアルタイムデータ交換のためのプロセス間通信ミドルウェアとして DDS (Data Distribution Service) を採用し, リアルタイムオペレーティングシステム上に展開できる
    \end{itemize}
\end{frame}

\begin{frame}{ハードリアルタイム制約}
    \begin{itemize}
        \item セーフクリティカルドメインのロボットソフトウェアはハードリアルタイム制約がある
        \item 設計者はタイミング動作を正確にモデル化して分析を行い, リアルタイム制約を実行時に常に保証する必要がある
    \end{itemize}
\end{frame}

\begin{frame}{エグゼキュータ}
    \begin{itemize}

        \item ROS 2 のリアルタイムパフォーマンスに関連するコアコンポーネントはエグゼキュータ
        \item エグゼキュータはプラットフォームとなる OS のスレッドでワークロードの実行を調整する
        \item ROS 2 executor のスケジューリング動作は, 過去のリアルタイムスケジューリングの研究で研究されたものとはまったく異なり, 新しい分析手法が必要

    \end{itemize}
\end{frame}

\begin{frame}{シングル/マルチスレッドエグゼキュータ}
    \begin{itemize}
        \item ROS 2 には, ワークロードをシーケンシャルに実行するシングルスレッドエグゼキュータと, ワークロードの実行を複数のスレッドに分散するマルチスレッドエグゼキュータの 2 つの組み込みエグゼキュータがある
        \item 最近の研究で, シングルスレッドエグゼキュータのリアルタイムパフォーマンスのモデリングと分析が行われたが, マルチスレッドエグゼキュータは未だ研究されていない
        \item マルチコアプロセッサのコンピューティング能力を十分に活用するには, マルチスレッドエグゼキュータが必要
    \end{itemize}
\end{frame}

\begin{frame}{本論文の貢献}
    \begin{itemize}
        \item ROS 2 マルチスレッドエグゼキュータのリアルタイムパフォーマンス分析に関する最初の作業を行う
        \item マルチスレッドエグゼキュータ上に展開される処理チェーンとして実装されたシステムのスケジューリングモデルを提示し, その応答時間分析手法を開発する
        \item エグゼキュータのスケジューリング動作による応答時間への影響を調査し, 複数のシングルスレッドエグゼキュータから 1 つのマルチスレッドエグゼキュータにワークロードをマージするときに応答時間が長くなるリスクを特定する
    \end{itemize}
\end{frame}
