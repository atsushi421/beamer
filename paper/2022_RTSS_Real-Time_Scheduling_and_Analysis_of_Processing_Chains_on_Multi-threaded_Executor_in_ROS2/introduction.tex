% !TeX root = main.tex

\section{INTRODUCTION}
\label{sec: introduction}

\begin{frame}{}
    \begin{itemize}
        \item ROS (Robot Operating System) は, ロボット工学アプリケーションで最も人気のあるオープンソースソフトウェアフレームワークの 1 つであり, 急速に開発が進んでいる
        \item ROS の主な哲学は, ソフトウェアのモジュール性と構成可能性を促進することにより, ロボットソフトウェア開発の生産性を促進することである
        \item ただし, ROS にはリアルタイム機能がないという大きな欠点もあり, 安全性が重要な環境での実装が制限される
        \item これに関連して, ROS 2 は 2017 年にリリースされた第 2 世代の ROS であり, ROS から最も肯定的な側面を継承し, リアルタイムロボットソフトウェアをサポートする強力なリアルタイム機能を提供することを目的としている
        \item 例えば, ROS 2 は, リアルタイムデータ交換のためのプロセス間通信ミドルウェアとして DDS (Data Distribution Service) を採用し, リアルタイムオペレーティングシステム上に展開できる
        \item ROS 2 の新しいアーキテクチャは優れたプラットフォームを提供するが, それだけではハードリアルタイムロボットソフトウェアをサポートするには不十分である
    \end{itemize}
\end{frame}

\begin{frame}{}
    \begin{itemize}
        \item セーフクリティカルドメインのロボットソフトウェアは通常, ハードリアルタイム制約の影響を受けるため, 設計者はそのタイミング動作を正式にモデル化および分析して, リアルタイム制約が実行時に常に尊重されることを保証する必要がある
        \item ROS 2 のリアルタイムパフォーマンスに関連するコアコンポーネントはエグゼキュータである
        \item エグゼキュータは, プラットフォームとなるオペレーティングシステム (OS) のスレッドでワークロードの実行を調整する
        \item ROS 2 には, ワークロードをシーケンシャルに実行するシングルスレッドエグゼキュータと, ワークロードの実行を複数のスレッドに分散するマルチスレッドエグゼキュータの 2 つの組み込みエグゼキュータがある
        \item 最近, いくつかの研究 [8]-[14] で, ROS 2 シングルスレッドエグゼキュータのリアルタイムパフォーマンスのモデリングと分析が研究されている
        \item ROS 2 executor のスケジューリング動作は, 過去のリアルタイムスケジューリングの研究で研究されたものとはまったく異なり, 新しい分析手法が必要であることが判明した
        \item これらの研究により, ROS 2 シングルスレッドエグゼキュータに展開されたシステムのリアルタイムパフォーマンスを保証することが可能になったが, ROS 2 マルチスレッドエグゼキュータのモデル化と分析の問題は未解決のままである
    \end{itemize}
\end{frame}

\begin{frame}{}
    \begin{itemize}
        \item 最新のロボットアプリケーションはますます複雑になっているため, マルチコアプロセッサのコンピューティング能力を十分に活用するには, マルチスレッドエグゼキュータにアプリケーションを実装することが必要になることがよくある
        \item 本論文では, ROS 2 マルチスレッドエグゼキュータのリアルタイムパフォーマンス分析に関する最初の作業を行う
        \item マルチスレッドエグゼキュータ上に展開される処理チェーンとして実装されたシステムのスケジューリングモデルを提示し, それらの応答時間分析手法を開発する
        \item さらに, さまざまなエグゼキューターでの応答時間に対するスケジューリング動作の影響を調査し, 設計で複数のシングルスレッドエグゼキューターから 1 つのマルチスレッドエグゼキューターにワークロードをマージするときに応答時間が長くなるリスクを特定する
        \item ランダムに生成されたワークロードと現実的な ROS 2 プラットフォームでのケーススタディの両方を使用して実験を行い, 結果を評価して実証する
    \end{itemize}
\end{frame}
