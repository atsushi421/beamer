% !TeX root = main.tex

\section{INTRODUCTION}
\label{sec: introduction}

\begin{frame}{ROS (Robot Operating System)}
    \begin{block}{ROS (Robot Operating System)}
        \begin{itemize}
            \item ロボット工学アプリケーションで最も人気のあるオープンソースソフトウェアフレームワークの 1 つ
            \item ソフトウェアのモジュール性と構成可能性を促進し, ロボットソフトウェア開発の生産性を向上
        \end{itemize}
    \end{block}
    \vspace{5mm}
\end{frame}

\begin{frame}{ROS 2}
    ROS にはリアルタイム機能がないため, \tb{ROS 2への移行が進んでいる}
    \begin{block}{ROS 2}
        \begin{itemize}
            \item 2017 年にリリースされた第 2 世代の ROS
            \item ROS から肯定的な側面を継承し, ロボットソフトウェアをサポートする強力なリアルタイム機能を提供
        \end{itemize}
    \end{block}
\end{frame}

\begin{frame}{ハードリアルタイム制約}
    \tb{セーフクリティカルなロボットソフトウェアはハードリアルタイム制約がある}ため, 設計者はタイミング動作を正確にモデル化して分析を行い, リアルタイム制約を実行時に常に保証する必要がある
\end{frame}

\begin{frame}{ROS 2 エグゼキュータ}
    \begin{itemize}
        \item ROS 2 のリアルタイムパフォーマンスに関連するコアコンポーネントはエグゼキュータ
        \begin{block}{エグゼキュータ}
            OS のスレッドでワークロードの実行を調整
        \end{block}
        \vspace{5mm}
        \item \tb{ROS 2 エグゼキュータのスケジューリング動作は過去の研究とは全く異なり, 新しい分析手法が必要}
    \end{itemize}
\end{frame}

\begin{frame}{シングル/マルチスレッドエグゼキュータ}
    \begin{itemize}
        \item ROS 2 には以下2つの組み込みエグゼキュータが存在
            \begin{block}{シングルスレッドエグゼキュータ}
                ワークロードをシーケンシャルに実行するエグゼキュータ
            \end{block}
            \begin{block}{マルチスレッドエグゼキュータ}
                ワークロードの実行を複数のスレッドに分散するエグゼキュータ
            \end{block}
        \vspace{5mm}
        \item 最近の研究でシングルスレッドエグゼキュータのモデリングと分析が行われたが, \tb{マルチスレッドエグゼキュータは未研究}
    \end{itemize}
\end{frame}

\begin{frame}{本論文の貢献}
    \begin{itemize}
        \item \tb{ROS 2 マルチスレッドエグゼキュータのリアルタイムパフォーマンス分析に\\関する最初の作業を実施}
        \item \tb{マルチスレッドエグゼキュータ上に展開される処理チェーンとして\\実装されたシステムのスケジューリングモデルを提示し,\\その応答時間分析手法を開発}
        \item \tb{エグゼキュータのスケジューリング動作による応答時間への影響を調査し, マルチスレッドエグゼキュータのリスクを特定}
    \end{itemize}
\end{frame}
