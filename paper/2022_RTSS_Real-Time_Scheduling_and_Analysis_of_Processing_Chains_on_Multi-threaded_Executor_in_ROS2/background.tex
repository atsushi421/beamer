% !TeX root = main.tex


\section{BACKGROUND}
\label{sec: background}


\subsection{ROS2}
\label{ssec: ros2}

\begin{frame}{}
    \begin{itemize}
        \item 本論文ではマルチスレッドエグゼキュータに焦点を当てる
    \end{itemize}
\end{frame}


\begin{frame}{エグゼキュータ}
    \begin{itemize}
        \item ROS 2 アプリケーションがプラットフォームに展開される際には, 個々のノードが OS のプロセスにマップされる
        \item ROS 2 クライアントライブラリの\al{エグゼキュータ}は, OS プロセス内のノードのコールバックの実行を調整する
    \end{itemize}
\end{frame}


\begin{frame}{処理チェーン}
    \fitimage{
        \begin{itemize}
            \item \al{処理チェーン}はコールバックの列
            \item 処理チェーンは特定のイベントによってトリガされ, 複数のエグゼキュータにまたがる可能性がある
        \end{itemize}
    }{multi_threaded_executor.png}
\end{frame}


\begin{frame}{本論文の分析の焦点}
    \fitimage{
        本論文では単一のエグゼキュータにおけるチェーンの応答時間に焦点を当てる
        \notes{CPAアプローチ\cite{casini2019response}によってチェーンが複数のエグゼキュータにもまたがる場合に拡張できる}
    }{multi_threaded_executor.png}
\end{frame}


\begin{frame}{ノード}
    \begin{itemize}
        \item コールバックは異なる\al{ノード}に属す場合もある
        \item ROS 2 では, 同じノード内のコールバックはデフォルトで同じコールバックグループとなる
        \item これを除いてノードの概念は本論文とは無関係
    \end{itemize}

    \assume{簡単にするために, モデルにノードレベルの情報を含めない}
\end{frame}


\subsection{Scheduling of multi-threaded executor}
\label{ssec: scheduling_of_multi_threaded_executor}

\begin{frame}{セクションサマリ}
    \begin{itembox}[l]{\textbf{目的}}
        ROS 2 マルチスレッドエグゼキュータのスケジューリング動作を説明
        \notes{本論文の内容は ROS 2 Foxy Fitzroy に基づいている}
    \end{itembox}
\end{frame}

\begin{frame}{コールバック}
    \begin{itemize}
        \item \al{コールバック}は ROS 2 の最小のスケジューリングエンティティ
        \item マルチスレッドエグゼキュータでは, スレッドは抽象 DDS レイヤからメッセージを受け取り, 対応するコールバックを実行する
    \end{itemize}
\end{frame}

\begin{frame}{コールバックの優先度決定方法}
    コールバックの優先度は以下 2 つの基準で決定される

    \begin{block}{コールバックタイプ}
        \setlength{\linewidth}{0.98\columnwidth}
        \begin{itemize}
            \item コールバックはタイマ・サブスクライバ・サービス・クライアントの 4 つのタイプに分類される
            \item 優先度は, タイマ $\succ$ サブスクライバ $\succ$ サービス $\succ$ クライアント \notes{\desc{$\succ$}{より高い優先度を持つ}}
        \end{itemize}
    \end{block}

    \begin{block}{登録順}
        先に登録されたコールバックが優先される
    \end{block}
\end{frame}

\begin{frame}{コールバックタイプに関する簡単化}
    \full{
        \assume{本論文ではコールバックタイプに関する情報は含めない}
    }
\end{frame}

\begin{frame}{yield\_before\_execute}
    \begin{itemize}
        \item \al{yield\_before\_execute} が false に設定されている場合, スレッドは選択されたコールバックの実行をすぐに開始する
        \item yield\_before\_execute が true に設定されている場合, スレッドは再度スケジュールされるまでプロセッサを OS に譲る
        \item 本論文ではデフォルトの設定である yield\_before\_execute $=$ false とする
    \end{itemize}
\end{frame}

\begin{frame}{メッセージキュー}
    各コールバックに到着したメッセージをバッファリングするためのキューがあり, その深さはユーザによって指定される
    \assume{
        \setlength{\linewidth}{0.98\columnwidth}
        \begin{itemize}
            \item キューが十分に大きく, メッセージが上書きされない
            \item コールバックは最も早く到着したメッセージでノンプリエンプティブに実行される
        \end{itemize}
    }
\end{frame}

\begin{frame}{wait\_set}
    エグゼキュータ内のスレッドは共通のセット \al{wait\_set} によって, 利用可能なメッセージを持つコールバックを保持する
    \assume{スレッドが wait\_set にアクセスする際に発生するオーバヘッドはゼロであるとする}
\end{frame}

\begin{frame}{reentrant/mutually exclusiveコールバックグループ}
    \tb{コールバックグループには以下2つの種類がある}
    \begin{block}{Reentrantコールバックグループ}
        wait\_set内のコールバックはいつでも選択できる
    \end{block}
    \begin{block}{Mutually exclusiveコールバックグループ}
        \setlength{\linewidth}{0.98\columnwidth}
        \begin{itemize}
            \item 各グループに排他制御フラグ \al{can\_be\_taken\_from} がある
            \item グループのコールバックが選択されるとフラグは false に設定され, コールバックの終了後に true に設定される
            \item mutually exclusive コールバックグループに属すコールバックは,  can\_be\_taken\_from が true の場合にのみ選択できる
        \end{itemize}
    \end{block}
\end{frame}

\begin{frame}{スレッドのワークフロー全体像}
    \fullimage{thread_workflow.png}
\end{frame}

\begin{frame}{[ワークフロー] low\_priority\_wait\_mutex}
    \fitimage{
        \begin{itemize}
            \item スレッドは wait\_set にアクセスする前に, 排他制御ロック \al{low\_priority\_wait\_mutex} を獲得する必要がある
            \item すなわち \tb{wait\_set を変更できるのは一度に 1 つのスレッドのみ}
            \item それ以外の場合, スレッドはブロックされる
        \end{itemize}
    }{workflow1.jpg}
\end{frame}

\begin{frame}{[ワークフロー] mutually exclusiveコールバックグループの挙動1}
    \fullimage{workflow2.jpg}
\end{frame}

\begin{frame}{[ワークフロー] コールバック選択後}
    \fitimage{
        コールバックが選択されると, \tb{スレッドは wait\_set からコールバックを削除}し, low\_priority\_wait\_mutex をリリースする
    }{workflow4.jpg}
\end{frame}

\begin{frame}{[ワークフロー] mutually exclusiveコールバックグループの挙動2}
    \fullimage{workflow3.jpg}
\end{frame}


\begin{frame}{[ワークフロー] コールバックが選択できない場合}
    \fitimage{
        全てのコールバックがチェックされ, 実行するコールバックを選択できない場合, \tb{スレッドは wait\_set をリセットし, wait\_set を更新する}
    }{workflow5.jpg}
\end{frame}

\begin{frame}{[ワークフロー] コールバックを wait\_set に追加できない場合}
    \begin{columns}
        \begin{column}{0.4\textwidth}
            コールバックを wait\_set に追加できない場合, スレッドは low\_priority\_wait\_mutex をリリースしてアイドル状態になる
        \end{column}
        \begin{column}{0.6\textwidth}
            \fullimage{workflow6.jpg}
        \end{column}
    \end{columns}
\end{frame}
