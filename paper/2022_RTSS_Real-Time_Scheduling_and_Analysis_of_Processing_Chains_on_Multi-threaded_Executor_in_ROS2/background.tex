% !TeX root = main.tex


\section{BACKGROUND}
\label{sec: background}


\subsection{ROS2}
\label{ssec: ros2}

\begin{frame}{ROS 2}
    \fitimage{
        \al{ROS 2} は OS 上の複数の抽象化レイヤの統合実装
    }{ros2_system.png}
\end{frame}

\begin{frame}{Publish/Subscribeパラダイム}
    ROS 2 アプリケーションは, \al{publish/subscribeパラダイム}を使用して相互に通信する複数のノードで構成される

    \begin{block}{Publish/Subscribeパラダイム}
        \setlength{\linewidth}{0.98\columnwidth}
        \begin{itemize}
            \item ノードはトピックにメッセージをpublishし, そのトピックをsubscribeしているノードにメッセージをブロードキャストする
            \item ノードはコールバックをアクティブにして各メッセージを処理することにより, publish メッセージに反応する
            \item コールバックはメッセージ自体をpublishすることも可能で, 複雑な動作をトピックとコールバックのネットワークとして実装できる
        \end{itemize}
    \end{block}
\end{frame}

\begin{frame}{エグゼキュータ}
    \begin{itemize}
        \item ROS 2 アプリケーションがプラットフォームに展開される際には, 個々のノードが OS のプロセスにマップされる
        \item ROS 2 クライアントライブラリの\al{エグゼキュータ}は, OS プロセス内のノードのコールバックの実行を調整する
    \end{itemize}
\end{frame}

\begin{frame}{シングル/マルチスレッドエグゼキュータ}
    \begin{itemize}
        \item ROS 2 には以下 \tu{2 つの組み込みエグゼキュータがある}
              \begin{block}{シングルスレッドエグゼキュータ}
                  1 つのスレッドでコールバックを実行するエグゼキュータ
              \end{block}
              \begin{block}{マルチスレッドエグゼキュータ}
                  コールバックの処理を複数のスレッドに分散するエグゼキュータ
              \end{block}
              \vspace{5mm}

        \item \textbf{\underline{本論文ではマルチスレッドエグゼキュータに焦点を当てる}}
    \end{itemize}
\end{frame}

\begin{frame}{DDS (Data Distribution Service)}
    ノードのpublisherとsubscriberの間のメッセージ交換は, 抽象\al{DDS} (Data Distribution Service) 層によって実現される

    \begin{block}{抽象DDS}
        DDS とクライアントライブラリ間の通信インタフェース
    \end{block}
    \begin{block}{DDS}
        publisherとsubscriber間でメッセージを交換するための業界標準の分散通信システム
    \end{block}
\end{frame}

\begin{frame}{処理チェーン}
    \fitimage{
        \begin{itemize}
            \item \al{処理チェーン}は特定のイベントによってトリガされ, 複数のエグゼキュータにまたがる
            \item チェーンの各頂点はコールバックを表す
        \end{itemize}
    }{multi_threaded_executor.png}
\end{frame}

\begin{frame}{本論文の分析の焦点}
    \fitimage{
        \tu{本論文では単一のエグゼキュータにおけるチェーンの応答時間に焦点を当てる}
        \notes{CPAアプローチ*によってチェーンが複数のエグゼキュータにまたがる場合に拡張できる}
    }{multi_threaded_executor.png}

    \source{*D.Casini et al., “Response time analysis of ROS 2 processing chains under reservation-based scheduling,” in ECRTS, 2019.}
\end{frame}

\begin{frame}{コールバックグループ}
    \fitimage{
        \begin{itemize}
            \item コールバックは必ず1つの\al{コールバックグループ}に属す
            \item エグゼキュータ内のコールバックは, 異なるコールバックグループに属す可能性がある
        \end{itemize}
    }{multi_threaded_executor.png}

\end{frame}

\begin{frame}{ノード}
    \begin{itemize}
        \item コールバックは異なる\al{ノード}に属す場合もある
        \item ROS 2 では, 同じノード内のコールバックはデフォルトで同じコールバックグループとなる
        \item これを除いて, ノードの概念は本論文とは無関係
    \end{itemize}

    \assume{簡単にするために, モデルにノードレベルの情報を含めない}
\end{frame}


\subsection{Scheduling of multi-threaded executor}
\label{ssec: scheduling_of_multi_threaded_executor}

\begin{frame}{}
    \begin{itemize}
        \item ROS 2 マルチスレッドエグゼキュータのスケジューリング動作を紹介する
        \item \tu{本論文の内容は ROS 2 Foxy Fitzroy に基づいている}
    \end{itemize}

\end{frame}

\begin{frame}{コールバック}
    \begin{itemize}
        \item \al{コールバック}は ROS 2 の最小のスケジューリングエンティティ
        \item マルチスレッドエグゼキュータでは, スレッドは抽象 DDS レイヤからメッセージを受け取り, 対応するコールバックを実行する
    \end{itemize}
\end{frame}

\begin{frame}{コールバックの優先順位決定方法}
    ROS 2 では, \tu{コールバックの優先度は 2 つのレベルで決定される}

    \begin{block}{コールバックタイプ}
        \setlength{\linewidth}{0.98\columnwidth}
        \begin{itemize}
            \item コールバックはタイマ・サブスクライバ・サービス・クライアントの 4 つのタイプに分類される
            \item 優先順位は, タイマ $\succ$ サブスクライバ $\succ$ サービス $\succ$ クライアント \notes{$\succ$: より高い優先度を持つ}
        \end{itemize}
    \end{block}

    \begin{block}{登録順}
        先に登録されたコールバックが優先される
    \end{block}
\end{frame}

\begin{frame}{コールバックタイプに関する簡単化}
    \full{
        \assume{本論文ではコールバックタイプに関する情報は含めない}
    }
\end{frame}

\begin{frame}{yield\_before\_execute}
    \begin{itemize}
        \item \al{yield\_before\_execute} が false に設定されている場合, スレッドは選択されたコールバックの実行をすぐに開始する
        \item yield\_before\_execute が true に設定されている場合, スレッドは再度スケジュールされるまでプロセッサを OS に譲る
        \item \tu{本論文ではデフォルトの設定である yield\_before\_execute $=$ false とする}
    \end{itemize}
\end{frame}

\begin{frame}{メッセージキュー}
    ROS 2 では, 各コールバックに到着したメッセージをバッファリングするためのキューがあり, その深さはユーザによって指定される
    \assume{
        \setlength{\linewidth}{0.98\columnwidth}
        \begin{itemize}
            \item キューが十分に大きく, メッセージが上書きされない
            \item コールバックは最も早く到着したメッセージでノンプリエンプティブに実行される
        \end{itemize}
    }
\end{frame}

\begin{frame}{wait\_set}
    エグゼキュータ内のスレッドは共通のセット \al{wait\_set} によって, 利用可能なメッセージを持つコールバックを保持する
\end{frame}

\begin{frame}{reentrant/mutually exclusiveコールバックグループ}
    \tu{コールバックグループには以下2つの種類がある}
    \begin{block}{Reentrantコールバックグループ}
        wait\_set内のコールバックはいつでも選択できる
    \end{block}
    \begin{block}{Mutually exclusiveコールバックグループ}
        \setlength{\linewidth}{0.98\columnwidth}
        \begin{itemize}
            \item 各グループに排他制御フラグ \al{can\_be\_taken\_from} がある
            \item グループのコールバックが選択されるとフラグは false に設定され, コールバックの終了後に true に設定される
            \item mutually exclusive コールバックグループに属すコールバックは,  can\_be\_taken\_from が true の場合にのみ選択できる
        \end{itemize}
    \end{block}
\end{frame}

\begin{frame}{スレッドのワークフロー全体像}
    \fullimage{thread_workflow.png}
\end{frame}

\begin{frame}{[ワークフロー] low\_priority\_wait\_mutex}
    \fitimage{
        \begin{itemize}
            \item スレッドは wait\_set にアクセスする前に, 排他制御ロック \al{low\_priority\_wait\_mutex} を獲得する必要がある
            \item すなわち wait\_set を変更できるのは一度に 1 つのスレッドのみ
            \item それ以外の場合, スレッドはブロックされる
        \end{itemize}
    }{workflow1.jpg}
\end{frame}

\begin{frame}{[ワークフロー] mutually exclusiveコールバックグループの挙動1}
    \fullimage{workflow2.jpg}
\end{frame}

\begin{frame}{[ワークフロー] コールバック選択後}
    \fitimage{
        コールバックが選択されると, スレッドは wait\_set からコールバックを削除し, low\_priority\_wait\_mutex をリリースする
    }{workflow4.jpg}
\end{frame}

\begin{frame}{[ワークフロー] mutually exclusiveコールバックグループの挙動2}
    \fullimage{workflow3.jpg}
\end{frame}


\begin{frame}{[ワークフロー] コールバックが選択できない場合}
    \fitimage{
        \begin{itemize}
            \item 全てのコールバックがチェックされ, 実行するコールバックを選択できない場合, スレッドは wait\_set をリセットし, wait\_set を更新する
            \item mutually exclusive コールバックグループのコールバックは, can\_be\_taken\_from が true の場合にのみ wait\_set に追加できる
        \end{itemize}
    }{workflow5.jpg}
\end{frame}

\begin{frame}{[ワークフロー] コールバックを wait\_set に追加できない場合}
    \begin{columns}
        \begin{column}{0.4\textwidth}
            コールバックを wait\_set に追加できない場合, スレッドは low\_priority\_wait\_mutex をリリースしてアイドル状態になる
        \end{column}
        \begin{column}{0.6\textwidth}
            \fullimage{workflow6.jpg}
        \end{column}
    \end{columns}
\end{frame}
