% !TeX root = main.tex


\section{BACKGROUND}
\label{sec: background}


\subsection{ROS 2}
\label{ssec: ros2}

\begin{frame}{分析対象のエグゼキュータ}
    本論文ではマルチスレッドエグゼキュータに焦点を当てる
\end{frame}

\begin{frame}{処理チェーン}
    \fitimage{
        \begin{itemize}
            \item 処理チェーンはコールバックの列
            \item 処理チェーンは特定のイベントによってトリガされ, 複数のエグゼキュータにまたがる可能性がある
        \end{itemize}
    }{multi_threaded_executor.png}
\end{frame}

\begin{frame}{本論文の分析の焦点}
    \fitimage{
        本論文では単一のエグゼキュータにおけるチェーンの応答時間に焦点を当てる
        \notes{CPAアプローチ\cite{casini2019response}によってチェーンが複数のエグゼキュータにもまたがる場合に拡張できる}
    }{multi_threaded_executor.png}
\end{frame}

\begin{frame}{ノードに関する簡単化}
    同じノード内のコールバックはデフォルトで同じコールバックグループとなることを除いて, ノードの概念は本論文とは無関係

    \assume{簡単にするために, モデルにノードレベルの情報を含めない}
\end{frame}


\subsection{Scheduling of multi-threaded executor}
\label{ssec: scheduling_of_multi_threaded_executor}

\begin{frame}{コールバックタイプに関する簡単化}
    コールバックはタイマ・サブスクライバ・サービス・クライアントの 4 つのタイプに分類される
    \assume{本論文ではコールバックタイプに関する情報は含めない}
\end{frame}

\begin{frame}{yield\_before\_execute の設定}
    \begin{block}{yield\_before\_execute}
    \setlength{\linewidth}{0.98\columnwidth}
    \begin{itemize}
        \item yield\_before\_execute が false に設定されている場合, スレッドは選択されたコールバックの実行をすぐに開始する
        \item yield\_before\_execute が true に設定されている場合, スレッドは再度スケジュールされるまでプロセッサを OS に譲る
    \end{itemize}
    \end{block}
    \vspace{5mm}
    \begin{itemize}
        \item 本論文ではデフォルトの設定である yield\_before\_execute $=$ false とする
    \end{itemize}
\end{frame}

\begin{frame}{メッセージキューに関する簡単化}
    各コールバックに到着したメッセージをバッファリングするためのキューがあり, その深さはユーザによって指定される
    \assume{
        \setlength{\linewidth}{0.98\columnwidth}
        \begin{itemize}
            \item キューが十分に大きく, メッセージが上書きされない
            \item コールバックは最も早く到着したメッセージでノンプリエンプティブに実行される
        \end{itemize}
    }
\end{frame}

\begin{frame}{wait\_setに関する簡単化}
    エグゼキュータ内のスレッドは共通のセット wait\_set によって, 利用可能なメッセージを持つコールバックを保持する
    \assume{スレッドが wait\_set にアクセスする際に発生するオーバヘッドはゼロであるとする}
\end{frame}
