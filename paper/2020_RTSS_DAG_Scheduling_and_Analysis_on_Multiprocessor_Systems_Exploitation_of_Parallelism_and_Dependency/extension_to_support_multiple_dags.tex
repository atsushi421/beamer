% !TeX root = main.tex

\section{EXTENSION TO SUPPORT MULTIPLE DAGS}
\label{sec: EXTENSION TO SUPPORT MULTIPLE DAGS}

\begin{frame}{複数DAGタスクの場合の最悪応答時間}
    \begin{itemize}
        \item $n$ 個DAGタスク $\Gamma=\left\{\tau_{1}, \ldots, \tau_{n}\right\}$ を持つ散発的タスクモデルを考える
        \item 各タスク $\tau_{x}$ には一意の優先度 $P_{x}$ が割り当てられている
              \begin{block}{$\tau_{x}$の最悪応答時間 $R_{x}^{\diamond}$}
                  \begin{equation*}
                      R_{x}^{\diamond}=R_{x}+\max _{\tau_{y} \in l p(x)}\left\{R_{y}\right\}+\sum_{\tau_{y} \in h p(x)}\left\lceil\frac{R_{x}^{\diamond}}{T_{y}}\right\rceil R_{y}
                  \end{equation*}
                  \setlength{\linewidth}{0.98\columnwidth}
                  \begin{itemize}
                      \item \desc{$lp(x)$}{$\tau_x$以下の優先度を持つDAGタスクセット}
                  \end{itemize}
              \end{block}
    \end{itemize}
\end{frame}

% \begin{frame}{}
%     \begin{itemize}
%         \item このセクションでは, 提案されたスケジューリングと分析方法を拡張して, $n$ DAGタスク $\Gamma=\left\{\tau_{1}, \ldots, \tau_{n}\right\}$ を持つ一般的な散発的タスクモデル, その中で各タスク $\tau_{x}$ には一意のデッドライン単調優先度 $P_{x}$ が割り当てられている
%         \item 複数のDAGタスクでは, スケジュールは, リリースする準備ができている全てのDAGのタスクとノードに対して, 最高優先度のタスク($\left.P_{x}\right)$ 最初に, 次にタスク内で最高優先度のノード $\left(p_{j}\right)$ 最初に, の原則に従う
%     \end{itemize}
% \end{frame}

% \begin{frame}{}
%     \begin{itemize}
%         \item 完全にノンプリエンプティブなDAGレベルスケジューリングでは, レディキュー内の最高優先度のタスクは, 現在実行中のタスクが終了した後に必ず実行されるようにスケジュールされる
%         \item すなわち, タスク優先度はレディキューで次に実行するタスクを選択するために使用されるが, ノード優先度はスケジュールされたDAG内のノードの正確な実行順序を与える
%         \item 我々は, このスケジュールが作業保存型ではなく, レディタスクが実行待ちの状態で特定のコアがアイドル状態になる可能性があることを認識している
%         \item この場合, 最悪の場合, 応答時間が長くなる
%         \item しかし, これにより, 現在実行中のタスクは, 利用可能なリソースをクリティカルパスを形成するノードに集中させることができる
%     \end{itemize}
% \end{frame}

% \begin{frame}{}
%     \begin{itemize}
%         \item DAGタスク $\tau_{x}$ は, $\tau_{x}$ のビジーウィンドウにリリースされた高優先度タスクの全ジョブと, 完了時間が最も長い低優先度タスクの1ジョブによってレイテンシさせることができる
%         \item $R_{x}^{\diamond}$ は, $\tau_{x}$ のmultiDAGの場合の最悪応答時間を示すとする
%         \item $R_{x}^{\diamond}$ は式14で与えられ, その中で $R_{x}$ は (式2により) シングルDAGの場合の $\tau_{x}$ の最悪完了時間を与え, $l p(x)$ は $P_{x}$ より低い優先度を持つ全てのタスクを返し, $h p(x)$ は $\tau_{x}$ の高い優先度を持つタスクであることを示している
%         \item レディタスクは, 現在実行中のタスクが完了した後にリリース されるため, 干渉タスク $\tau_{y}$ のジョブによる最悪レイテンシは, 式 2 により $R_{y}$ に実質的に束縛される

%               \begin{equation*}
%                   R_{x}^{\diamond}=R_{x}+\max _{\tau_{y} \in l p(x)}\left\{R_{y}\right\}+\sum_{\tau_{y} \in h p(x)}\left\lceil\frac{R_{x}^{\diamond}}{T_{y}}\right\rceil R_{y}
%               \end{equation*}
%     \end{itemize}
% \end{frame}

% \begin{frame}{}
%     \begin{itemize}
%         \item 最後に, 現在のタスクの「ファンイン」段階 (DAGの並列度が終了するまで単調に減少する段階) でレディキューにある次のタスクを開始することにより, 現在のタスクに影響を与えずに, 全てのタスクの総合所要時間を短縮できることに注目する
%         \item これは, プロセッサのインオーダーパイプライン実行と同じ原理であり, その実行を制限するための実証済みの分析が存在する[21]
%         \item 予想より早くリリースしても, ノードが最悪境界より遅く終了することはないため, これは分析を危険にさらすことはない
%         \item しかし, これはスケジューリングを複雑にし, 各DAGのファンインフェーズを識別するオンライン分析を必要とし, これは実際のアプリケーションで常に実行可能であるとは限りません
%         \item さらに, この早期リリースの分析は, クリティカルパスの開始点とタスクの非クリティカルノード間の余分なオフセットのために, $(\alpha, \beta)$-pair分析をさらに複雑にする可能性がある
%         \item 注目すべきは, [2]では, 異なる周期を持つマルチDAGを単一の周期的DAGとして記述できるため, 提案した分析がそのまま適用できることである
%         \item しかし, これは本論文の範囲外であり, 将来の研究に先送りする
%     \end{itemize}
% \end{frame}
