% !TeX root = main.tex

\section{EVALUATIONS}
\label{sec: EVALUATIONS}

\begin{frame}{}
    \begin{itemize}
        \item この評価の目的は,  (1) スケジューリングと分析 (セクションV-Aのrta-cpfとセクションV-Cのrta-cpf-eo) が最悪のメイクスパンを改善することを実証すること (古典境界を参照として) ,  (2) 提案手法がメイクスパンの改善につながる条件を確立すること,  (3) 提案する実行順序 $(E O)$ がスケジューリング性を改善し, 提案する分析によって最悪境界がタイト化することを示すこと,  (4) 複数DAGケースでのスケジュービリティテストによって改善度を評価すること, といった多角的に行うことである
\item 提案したノード順序 $E O$ は, 分析と並行してノード優先度割り当てを提案した $\mathrm{He}$ ら[11] (以下, He2019と表記) と比較される
    \end{itemize}
\end{frame}

\begin{frame}{}
    \begin{itemize}
        \item この実験は, ランダムに生成されるDAGによって評価される
\item 各 DAG タスクは以下のように生成される
\item 生成器はソースノードから開始し, ノードを 1層ずつ生成する
\item 最大深度 (層数) は 5 から 8 の間でランダムに選択される
\item 各層の生成ノード数は2から並列度パラメータ $p$ まで一様に分布している
\item オープンエンドノードはランダムに $p_{c}=0.5$ の確率で接続を追加し, 前の層の他のノードに参加する
\item そして, 全てのエンドノードはシンクノードに接続される
\item ソースノードとシンクノードはノードグラフを構成する役割を果たし, 両者とも実行時間は1単位である
\item 最後に, $W^{1}$ の総ワークロードが与えられたノードに, 実行時間がランダムに割り当てられる
    \end{itemize}
\end{frame}


\subsection{Evaluation of the worst-case makespan}
\label{ssec: Evaluation of the worst-case makespan}

\begin{frame}{}
    \begin{itemize}
        \item この実験では, コア数 $(m)$ でスケーリングした性能を評価した
\item 各設定 (タスクとシステム設定) に対して, 比較した手法で1,000回の試行が行われた
\item 各試行では, 1つのDAGタスクがランダムに生成される
\item 指標として, 規格化された最悪マークスパンを用いる
    \end{itemize}
\end{frame}

\begin{frame}{}
    \begin{itemize}
        \item 図3は, $p=8$ で生成されたDAGに対して, コア数を変化させた既存手法と提案手法の最悪時間幅を示したものである
\item $m \leq 4$ では, rta-cpfはclassic boundと同様の結果, すなわち, ほとんどの結果がclassic boundの上界となる
\item これは, コア数が少ない場合, DAGの並列度が制限され, 各非重要ノードの最悪終了時間が高くなるためである (式3参照) 
\item これにより, 各プロバイダの $\alpha_{i}$ 境界が低くなり($\beta_{i}$ 境界も高くなる) , 最悪のレイテンシ近似が長くなる
\item $m$ がさらに増加すると, rta-cpf が有効になり($m=6$ から) , 例えば, 平均で $15.7 \%$ と $16.2 \%$  (および $31.7 \%$ と $32.2 \%$ まで) , それぞれ $m=7$ と $m=8$ と, 古典的な境界を上回 ります
\item この場合, より多くのワークロードがクリティカルパスと並行して実行できるようになり, すなわち, $\alpha_{i}$ が増加し, $\beta_{i}$ が減少することになる
\item このように, rta-cpfは, このようなワークロードを明示的に考慮することによって, よりタイトな結果を導き出し, その結果, クリティカルパス上の干渉を安全に減少させることができる
    \end{itemize}
\end{frame}

\begin{frame}{}
    \begin{itemize}
        \item 同様の結果はrta-cpf-eoとHe2019の比較でも得られ, rta-cpf-eoは $m \geq 4$ で, 例えば $11.1 \%$ と $12.0 \%$ まで, それぞれ $m=7$ と $m=8$ で最悪のレイテンシ近似を短くできる
\item 両手法におけるノードの実行順序も分析的な最悪境界値に影響を与える可能性があることに注意セクションVII-Cでは, スケジューリングと分析の方法を別々に比較する
\item さらに, $m=7$ では, rta-cpf (ランダムな実行順序) は $\mathrm{He}$ 2019と同様の結果をもたらし, $m=8$ ではHe2019を上回ることを観察している
\item この観測は, 提案する分析の有効性をさらに実証している
    \end{itemize}
\end{frame}


\subsection{Sensitivity of DAG properties on the evaluated methods}
\label{ssec: Sensitivity of DAG properties on the evaluated methods}

\begin{frame}{}
    \begin{itemize}
        \item セクション VII-A の結果から, DAG の特性がどのように最悪のメークスパンに影響を与えるかを理解するのは簡単ではない
\item これに対応するために, この実験では, 評価された分析が特定のDAGの特性に対してどのように敏感であるかを示している
\item すなわち, DAGのパラメータを制御し, 規格化した値でメイクスパンを評価することで, 分析の性能がどの程度変化するかを確認できるのである
\item これは, そうでなければ, 最悪のメイクスパンやスケジューラビリティの分析では区別がつかない
\item 具体的には, 本実験では以下のようなパラメータを考えている (コア数は固定) 1) DAG並列度 (ランダム化DAG生成時に可能な最大幅)  $p$, 2) 全ワークロードに対するDAGクリティカルパス比率 $\% L$, ここで $\% L=L / W \times 100 \%$ は, $\% L=L / W \times 100 \%$
    \end{itemize}
\end{frame}

\begin{frame}{}
    \begin{itemize}
        \item 図4は, 並列化パラメータを変化させた場合の提案手法の最悪時間($m=4$ 使用)を示している
\item まず, コア数が一定であれば, 一般にrta-cpfは古典的な境界を上回ります
\item しかし, $p$ を増加させると, 両手法の性能差はあまり大きくなくなる
\item この観測の背後にある直感は, 同時実行ノード数が増加すると, 各ノードの干渉セットも増加し (式4参照) , その結果, 最悪の終了時間が増加することである
\item これは, 最悪終了時間に基づいて $\alpha_{i}$ と $\beta_{i}$ を説明する $r t a$-cpfの有効性を損ないる
    \end{itemize}
\end{frame}

\begin{frame}{}
    \begin{itemize}
        \item しかし, rta-cpf-eoは強力な性能を示し, その有効性は $p$ 上の変更の影響を受けず, 全てのシステム設定において一貫して他の手法を上回った
\item これは, 実行順序が明示されている場合, 同時実行ノードが増加してもノードの終了時間に大きな影響を与えることができず, 優先度の高いノードはレイテンシなくすぐに実行できるためである (式12を参照) 
\item したがって, 並列化DAGを用いたrta-cpf-eoは, 依然として実際の干渉ワークロードを効果的に考慮し, 最も低い最悪マークスパンを提供できる
    \end{itemize}
\end{frame}

\begin{frame}{}
    \begin{itemize}
        \item 図5は, クリティカルパスの長さが提案手法の効果に与える影響を, $m=2$ で評価したものである
\item クリティカルパスは, 生成されたDAGの総ワークロードの $60 \%$ から $90 \%$ の範囲で変化させた
\item この実験では, 提案する分析が既存の手法と比較して最も顕著な性能を示している
    \end{itemize}
\end{frame}

\begin{frame}{}
    \begin{itemize}
        \item 提案手法では, $r t a$-cpfは, $\% L$ が少ないと, 生成されるDAGの内部構造が変化するため, 最悪の場合メイクスパンが変化する (例えば, $L=0.6$ )
\item しかし, $\% L, r t a$-cpf をさらに増加させると, 全ての非クリティカルなワークロードがクリティカルパスと並列に実行できるため, 一定のメイクスパンが実現される
\item この場合, メイクスパンは, クリティカルパスの長さに直接等しくなる
\item 同様の結果は, rta-cpf-eoでも得られており, 全ての実験設定において一定のメイクスパン (すなわち, クリティカルパスの長さ) が得られている
\item なお, $\% L, \mathrm{He} 2019$ がさらに増加すると, 完全にrta-cpfに支配される (評価に基づくものであるが, ページの制限のため, 提示されていない) 
    \end{itemize}
\end{frame}

\begin{frame}{}
    \begin{itemize}
        \item 上記の実験に基づき, 提案手法は一般的なケースにおいて, 古典的手法と最先端技術を凌駕する性能を持つことが確認された
\item また, テストした全てのパラメータ $m, p, \% L$ が提案手法の性能に影響を与えることが確認された
\item $r t a-c p f$ については, $m, p$ との関係に敏感であり, $m$ が低いか, $p$ が高いと, 手法の有効性が損なわれることがわかった
\item 両要素は, 全ての非重要ノードの終了時間に直接影響を与える
\item さらに, $\% L$ はrta-cpfの性能にも大きく影響し, 一般にクリティカルパスが長いと, より正確なメイクスパン近似が可能になる
\item rta-cpfと同様に, rtacpf-eoも $\% L$ の増加に伴い, より良い性能を示す
\item しかし, rta-cpf-eoは, その明示的な実行順序により, rta-cpfよりはるかに強い性能を示し, パラメータ $p$ による影響を受けない
    \end{itemize}
\end{frame}


\subsection{Effectiveness of the proposed schedule and analysis}
\label{ssec: Effectiveness of the proposed schedule and analysis}

\begin{frame}{}
    \begin{itemize}
        \item 本実験では, 提案する優先度割り当てを, 最先端のノードレベル優先度割り当て手法であるHe2019と比較する
\item さらに, 優先度割り当てを考慮した場合の最悪マークスパンを実証する
\item 目的は, 優先度割り当てによって達成される改善された最悪シナリオを実証することである
\item 全体として, 各構成の下で1000個のランダムなタスクセットが生成される
\item (a) 提案するrta-cpf-eo分析が比較対象手法よりも優れている割合, (b) 改善されたケース内の正規化されたmakespanの削減量である
    \end{itemize}
\end{frame}

\begin{frame}{}
    \begin{itemize}
        \item 図6は, 提案する順序付け方法と $\mathrm{He} 2019$ の方法を, コア数を変化させて比較した結果を報告したものである
\item 頻度」という用語は, 提案するスケジュールがHe2019よりも短い (赤色) または長い (青色) メイクスパンを持つケースの数を示す
\item 公平性を保つため, 提案した明示的順序の最悪マークスパン分析 (セクションV-C) を両方の順序に適用しているので, 性能の違いは全て順序のポリシーに起因するものである
    \end{itemize}
\end{frame}

\begin{frame}{}
    \begin{itemize}
        \item この結果から, 提案手法は一般に周波数が高いほど, 特にコア数が少ない場合, 例えば $m=2$ と $m=3$ では周波数600付近でHe2019を上回る性能を発揮することが分かりました
\item $m$ の増加に伴い, 各手法の周波数の差は徐々に小さくなり, $m=7,8$ では見分けがつかなくなる
\item このような場合, ほとんどのノードが並列に実行できるため, 実行順序の違いは最終的なメイクスパンに影響を与えるほど大きくはならない
    \end{itemize}
\end{frame}

\begin{frame}{}
    \begin{itemize}
        \item 表IIIは, 両手法の有利なケースでの詳細な比較を, 改善率で示したものである
\item EO $\succ \mathrm{He} 2019$ (すなわち, 提案スケジュールが He2019 を上回る) の場合, 全てのケースで $5.4 \%$ ($7.89 \%$ まで) よりも高い平均改善度 (最悪マークスパン) が観測される
\item $\mathrm{EO} \prec \mathrm{He} 2019$ の場合(すなわち, He2019の方が良い), 改善度は $\mathrm{EO} \succ \mathrm{He} 2019$ の場合よりも常に低くなる
    \end{itemize}
\end{frame}

\begin{frame}{}
    \begin{itemize}
        \item 表 IV は, EO $\succ$ He2019 と EO $\prec \mathrm{He} 2019$ の両方において, 改善された利点の件数とその科学的意義を報告したものである
\item 表IVの大きさは, 科学的意義を反映するために,  (無視できる効果, 小さな効果, 中程度の効果, 大きな効果) のカテゴリー値である[22]
\item すなわち, 科学的有意性とは, 何らかの違いがランダムな偶然以上のものであるかどうか, そしてその違いの大きさを知らせるものである
\item データの列は, あるアプローチが他のアプローチより短いmakespanを持つ回数を示している
\item 全ての場合において, 我々のアプローチは, 最新技術のHe2019を上回っている
\item 例えば, $m=4$ では, EOがHe2019を上回った場合の効果量は, He2019が $\mathrm{EO}$ を上回った場合は小さく, $m=8$ では, データが同様の値であっても, 小さく, 無視できる程度になる
    \end{itemize}
\end{frame}

\begin{frame}{}
    \begin{itemize}
        \item 同様に, 両手法に同じ順序を適用して, 我々の分析とHe2019の分析の比較を行ったところ, 一貫した結果が得られた (ページの都合上, 紹介していない) 
\item したがって, 提案するスケジューリングとアナライジングは有効であり, 一般的なケースにおいて最新技術を凌駕すると結論付けている
    \end{itemize}
\end{frame}


\subsection{Schedulability test with multi-DAGs}
\label{ssec: Schedulability test with multi-DAGs}

\begin{frame}{}
    \begin{itemize}
        \item 優先度割り当てをさらに評価するために, ランダムに生成された複数DAGタスクセットのスケジューラビリティをテストした
\item 実験の設定は以下の通りである:コア数は $m=6$ に固定である
\item 全てのDAGタスクの総利用率 (コアごとの平均) は, $0.1$ から $1.0$ の範囲で, $0.05$ のステップサイズを設定する
\item タスクセット内のDAGタスクの利用率は, UUniFast-discardアルゴリズム[23]を介して生成される
\item 各DAGの利用率は $m$ より小さくなければならず, そうでなければ破棄される
    \end{itemize}
\end{frame}

\begin{frame}{}
    \begin{itemize}
        \item 1つのタスクセットには10個のDAGタスクがあり, 各DAGは先に紹介したのと同じ方法でランダムに生成される
\item DAGタスクの周期は $T_{i} \in(1000,2000)$ に対してランダムに生成され, デッドラインは周期と等しくなる
\item そして, DAGタスク内のノードの実行時間は, そのワークロード $W_{i}=U_{i} / T_{i}$ に基づいて生成される
\item マルチDAGのスケジューラビリティは, 式14を用いて検証し, その中で $R_{i}$ は, rta-cpf-eo分析とHe2019分析を用いてそれぞれ算出す
\item ランダム実行, すなわちノードレベルの命令がない場合のスケジューラビリティは, 古典的な応答時間の式で評価される
\item 優先度はデッドラインモノトニックポリシーに基づいてDAGに割り当てられる
    \end{itemize}
\end{frame}

\begin{frame}{}
    \begin{itemize}
        \item 図7に示すように, random法は, 各DAGのノードがランダムにスケジューリングされ, 参照境界を与える
\item この結果から, 提案するスケジューリング手法と分析手法は, ほとんどの場合において, 最新技術よりも優れたシステムのスケジューラビリティを提供する (すなわち, $\sum U / m=$  $0.30-0.45)$ の場合
\item この結果は, DAGタスクが非作業保存的に (すなわち, 一度に一つのタスク) , 優先度で実行されるため, 単一DAGの場合と一致する
\item 表 $\mathrm{V}$ は, 詳細なスケジューラビリティの結果を報告している
\item この表から, 提案手法は $\sum U / m=0.35$ のとき, $53.3 \%$ までは最先端技術を凌駕していることがわかる
    \end{itemize}
\end{frame}

\begin{frame}{サマリ}
    \begin{itemize}
        \item 今回の実験では, 提案する最悪メイクスパン分析と優先度割り当てにより, 最悪境界を厳しくすることで, 概ねスケジューラビリティを向上させることができることが示された
\item また, 本手法の有効性は, 既存のアプローチと比較して, マルチDAGを持つスケジューラブルなタスクの数が改善されることによって示される
    \end{itemize}
\end{frame}
