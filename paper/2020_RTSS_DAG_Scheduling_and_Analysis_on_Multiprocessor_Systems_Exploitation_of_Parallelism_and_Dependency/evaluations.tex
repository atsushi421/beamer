% !TeX root = main.tex

\section{EVALUATIONS}
\label{sec: EVALUATIONS}

% \begin{frame}{評価の目的}
%     \begin{itemize}
%         \item スケジューリングと分析が最悪のメイクスパンを改善することを実証する
%         \item 提案手法がメイクスパンの改善につながる条件を確立する
%         \item 提案する実行順序 ($EO$) がスケジューリング性を改善し, 提案する分析によって最悪境界が厳密化することを示す
%         \item 複数DAGケースでのスケジュービリティテストによって改善度を計測する
%     \end{itemize}
% \end{frame}

\begin{frame}{比較対象}
    \begin{block}{classic}
        一般的に知られた最悪応答時間
        \begin{equation*}
            R_{x}=L_{x}+\left\lceil\frac{1}{m}\left(W_{x}-L_{x}\right)\right\rceil+\sum_{\tau_{y} \in h p(x)} I_{x, y}
        \end{equation*}
    \end{block}
    \begin{block}{He2019}
        既存手法 \cite{he2019intra}
    \end{block}
\end{frame}

\begin{frame}{提案手法の表記}
    \begin{block}{rta-cpf}
        提案スケジューリングアルゴリズム及び分析
    \end{block}
    \begin{block}{rta-cpf-eo}
        明示的実行順序に基づく提案スケジューリングアルゴリズム及び分析
    \end{block}
\end{frame}

\begin{frame}{実験に使用するDAG生成方法}
    \begin{itemize}
        \item 生成器はソースノードから開始し, ノードを 1層ずつ生成する
        \item 層数 は 5 から 8 の間でランダムに選択される
        \item 各層の生成ノード数は2から並列度パラメータ $p$ まで一様に分布している
        \item オープンエンドノードはランダムに $p_{c}=0.5$ の確率で接続を追加し, 前の層の他のノードに参加する
        \item 全てのエンドノードはシンクノードに接続される
        \item ソースノードとシンクノードはノードグラフを構成する役割を果たし, 両者とも実行時間は1単位である
        \item $W$ の総ワークロードが与えられたノードに, 実行時間をランダムに割り当てる
    \end{itemize}
\end{frame}


\subsection{Evaluation of the worst-case makespan}
\label{ssec: Evaluation of the worst-case makespan}

\begin{frame}{実験方法}
    \begin{itemize}
        \item この実験では, コア数 $(m)$ でスケーリングした性能を評価する
        \item 各設定に対して, 1,000回の試行を行う
        \item 評価指標として, 正規化された最悪メイクスパンを用いる
              \begin{block}{最悪メイクスパン}
                  全てのノードがWCETで動作するとしてスケジューリングシミュレーションを行った時のスケジュール長
              \end{block}
    \end{itemize}
\end{frame}

\begin{frame}{結果}
    \fitimage{
        \begin{itemize}
            \item コア数7, 8の時, rta-cpf が平均 $15.7 \%$,$16.2 \%$既存手法を上回る
            \item これは多くのワークロードがクリティカルパスと並行して実行できるようになり, rta-cpf の利点が顕著になるため
        \end{itemize}
    }{eva1}
\end{frame}

% \begin{frame}{}
%     \begin{itemize}
%         \item 同様の結果はrta-cpf-eoとHe2019の比較でも得られ, rta-cpf-eoは $m \geq 4$ で, 例えば $11.1 \%$ と $12.0 \%$ まで, それぞれ $m=7$ と $m=8$ で最悪のレイテンシ近似を短くできる
%         \item 両手法におけるノードの実行順序も分析的な最悪境界値に影響を与える可能性があることに注意セクションVII-Cでは, スケジューリングと分析の方法を別々に比較する
%         \item さらに, $m=7$ では, rta-cpf (ランダムな実行順序) は $\mathrm{He}$ 2019と同様の結果をもたらし, $m=8$ ではHe2019を上回ることを観察している
%         \item この観測は, 提案する分析の有効性をさらに実証している
%     \end{itemize}
% \end{frame}


\subsection{Sensitivity of DAG properties on the evaluated methods}
\label{ssec: Sensitivity of DAG properties on the evaluated methods}

\begin{frame}{実験方法}
    \begin{itemize}
        \item この実験では, 分析がDAGの特性に対してどのように敏感であるかを評価する
        \item 本実験では, コア数を固定し, 以下のパラメータを使用する
              \begin{block}{DAG並列度 $p$}
                  ランダムDAG生成時の1層内の最大ノード数
              \end{block}
              \begin{block}{クリティカルパス比率 $\% L$}
                  全ワークロードに対するクリティカルパスの比率
                  \[
                      \% L=L / W \times 100 \%
                  \]
              \end{block}
    \end{itemize}
\end{frame}

\begin{frame}{DAG並列度を変化させた場合の最悪メイクスパン}
    \fullimage{eva2}
\end{frame}

\begin{frame}{結果の考察}
    \begin{itemize}
        \item rta-cpfはclassicを常に上回るが, $p$ を増加させると性能差は小さくなる
              \begin{block}{原因の考察}
                  同時実行ノード数が増加すると, プロバイダの干渉要因が増えるため
              \end{block}
              \vspace{5mm}
        \item rta-cpf-eoは$p$の影響を受けず, 全て設定において他の手法を上回った
              \begin{block}{原因の考察}
                  実行順序が明示されている場合, 同時実行ノードが増加してもノードの終了時間に大きな影響を与えず, 優先度の高いノードはすぐに実行できるため
              \end{block}
    \end{itemize}
\end{frame}

\begin{frame}{クリティカルパス比率を変化させた場合の最悪メイクスパン}
    \fullimage{eva3}
\end{frame}

\begin{frame}{結果の考察}
    \begin{itemize}
        \item rta-cpfは, $\% L$ が小さいと, 生成されるDAGの内部構造が変化するため, 最悪メイクスパンが変化する
        \item しかし, $\% L$ をさらに増加させると, 全ての非クリティカルワークロードがクリティカルパスと並列に実行できるため, 一定のメイクスパンが実現される
        \item この場合, メイクスパンは, クリティカルパスの長さに直接等しくなる
        \item rta-cpf-eoは全ての実験設定において一定のメイクスパンが得られている
    \end{itemize}
\end{frame}

\begin{frame}{得られた示唆}
    \begin{itemize}
        \item 提案手法は古典的手法と最先端技術を凌駕する性能を持つ
        \item テストした全てのパラメータ $m, p, \% L$ が提案手法の性能に影響を与える
        \item rta-cpfは, $m, p$ との関係に敏感であり, $m$ が低いか, $p$ が高いと, 手法の有効性が損なわれる
        \item rta-cpf と rtacpf-eoは $\% L$ の増加に伴い, より良い性能を示す
        \item rta-cpf-eoは, その明示的な実行順序により, rta-cpfよりはるかに優れた性能を示し, $p$ の影響を受けない
    \end{itemize}
\end{frame}


% \subsection{Effectiveness of the proposed schedule and analysis}
% \label{ssec: Effectiveness of the proposed schedule and analysis}

% \begin{frame}{実験方法}
%     \begin{itemize}
%         \item 本実験では, 提案する優先度割り当てをHe2019と比較する
%         \item 評価指標は以下2つ
%         \begin{enumerate}
%             \item 提案するrta-cpf-eo分析がHe2019よりも優れている割合
%             \item 改善されたケース内の正規化されたメイクスパン削減量
%         \end{enumerate}
%     \end{itemize}
% \end{frame}

% \begin{frame}{}
%     \begin{itemize}
%         \item 図6は, 提案する順序付け方法と $\mathrm{He} 2019$ の方法を, コア数を変化させて比較した結果を報告したものである
%         \item 頻度」という用語は, 提案するスケジュールがHe2019よりも短い (赤色) または長い (青色) メイクスパンを持つケースの数を示す
%         % \item 公平性を保つため, 提案した明示的順序の最悪メイクスパン分析 (セクションV-C) を両方の順序に適用しているので, 性能の違いは全て順序のポリシーに起因するものである
%     \end{itemize}
% \end{frame}

% \begin{frame}{}
%     \begin{itemize}
%         \item この結果から, 提案手法は一般に周波数が高いほど, 特にコア数が少ない場合, 例えば $m=2$ と $m=3$ では周波数600付近でHe2019を上回る性能を発揮することが分かりました
%         \item $m$ の増加に伴い, 各手法の周波数の差は徐々に小さくなり, $m=7,8$ では見分けがつかなくなる
%         \item このような場合, ほとんどのノードが並列に実行できるため, 実行順序の違いは最終的なメイクスパンに影響を与えるほど大きくはならない
%     \end{itemize}
% \end{frame}

% \begin{frame}{}
%     \begin{itemize}
%         \item 表IIIは, 両手法の有利なケースでの詳細な比較を, 改善率で示したものである
%         \item EO $\succ \mathrm{He} 2019$ (すなわち, 提案スケジュールが He2019 を上回る) の場合, 全てのケースで $5.4 \%$ ($7.89 \%$ まで) よりも高い平均改善度 (最悪メイクスパン) が観測される
%         \item $\mathrm{EO} \prec \mathrm{He} 2019$ の場合(すなわち, He2019の方が良い), 改善度は $\mathrm{EO} \succ \mathrm{He} 2019$ の場合よりも常に低くなる
%     \end{itemize}
% \end{frame}

% \begin{frame}{}
%     \begin{itemize}
%         \item 表 IV は, EO $\succ$ He2019 と EO $\prec \mathrm{He} 2019$ の両方において, 改善された利点の件数とその科学的意義を報告したものである
%         \item 表IVの大きさは, 科学的意義を反映するために,  (無視できる効果, 小さな効果, 中程度の効果, 大きな効果) のカテゴリー値である[22]
%         \item すなわち, 科学的有意性とは, 何らかの違いがランダムな偶然以上のものであるかどうか, そしてその違いの大きさを知らせるものである
%         \item データの列は, あるアプローチが他のアプローチより短いmakespanを持つ回数を示している
%         \item 全ての場合において, 我々のアプローチは, 最新技術のHe2019を上回っている
%         \item 例えば, $m=4$ では, EOがHe2019を上回った場合の効果量は, He2019が $\mathrm{EO}$ を上回った場合は小さく, $m=8$ では, データが同様の値であっても, 小さく, 無視できる程度になる
%     \end{itemize}
% \end{frame}

% \begin{frame}{}
%     \begin{itemize}
%         \item したがって, 提案するスケジューリングとアナライジングは有効であり, 一般的なケースにおいて最新技術を凌駕すると結論付けている
%     \end{itemize}
% \end{frame}


\subsection{Schedulability test with multi-DAGs}
\label{ssec: Schedulability test with multi-DAGs}

\begin{frame}{セクションサマリ}
    \begin{itembox}[l]{\textbf{目的}}
        ランダムに生成された複数DAGタスクセットのスケジューラビリティをテストする
    \end{itembox}
\end{frame}

\begin{frame}{実験設定}
    \begin{itemize}
        \item コア数は $m=6$ に固定
        \item DAGタスクの総利用率は, コアごとの平均で $0.1$ から $1.0$ の範囲で変動させる
        \item 1つのタスクセットには10個のDAGタスクをランダムに生成する
        \item DAGタスクの周期は $T_{i} \in(1000,2000)$の範囲でランダムに生成され, デッドラインは周期と等しい
        \item DAGタスク内のノードの実行時間は, そのワークロード $W_{i}=U_{i} / T_{i}$ に基づいて生成される
        \item 優先度はデッドラインモノトニックポリシーに基づいてDAGに割り当てられる
    \end{itemize}
\end{frame}

\begin{frame}{結果}
    \fitimage{
        提案するスケジューリング手法と分析手法は, ほとんどの場合において, 最新技術よりも優れたスケジューラビリティを提供する
    }{eva4}
\end{frame}
