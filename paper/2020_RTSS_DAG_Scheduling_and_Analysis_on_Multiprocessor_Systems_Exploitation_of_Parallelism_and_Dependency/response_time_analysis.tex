% !TeX root = main.tex

\section{$(\alpha, \beta)$-PAIR RESPONSE TIME ANALYSIS}
\label{sec: RESPONSE TIME ANALYSIS}

\begin{frame}{}
    \begin{itemize}
        \item 本資料では, 分析方法のみを示す
        \item 証明は論文を参照
    \end{itemize}
\end{frame}

% \begin{frame}{}
%     \begin{itemize}
%         \item 提案されたスケジュールとCPCモデルにより, このセクションでは, 並列ワークロード (すなわち $\alpha$)を明示的に考慮し, $\lambda^{*}$ を遅らせる干渉ワークロードに対する安全な削減として $\alpha$ を適用する新しい応答時間分析を提示する
%         \item さらに, 提案されたスケジュールは明示的なノード優先度を割り当てているが, クリティカルパス優先実行 (すなわち, CPFE) は並列度を最大化するための基本的な特性であり (Theorem1参照) , 多くの既存の手法で採用されていることを強調する[15], [11], [14]
%         \item 一般性のために, 提案する分析ではCPFEを仮定し, 非クリティカルノードのスケジューリング順序を任意に許容する
%         \item すなわち, 従来の分析[12], [13]と比較して, 本分析では, CPFEに基づく全てのスケジュールに対して改善された境界を提供する
%         \item 本分析では, 明示的な実行順序が事前に分かっていることは想定していない
%         \item セクションV-Cでは, 明示的な実行順序が事前に分かっているスケジューリング手法 (例えば, [17], [11], 提案スケジュール) に対して, 提案分析を若干の修正で拡張している
%         \item 表IIは構成された分析で導入された表記をまとめたものである
%     \end{itemize}
% \end{frame}


% \subsection{The $(\alpha, \beta)$-pair analysis formulation}
% \label{ssec: a}

% \begin{frame}{}
%     \begin{itemize}
%         \item CPC モデルでは, DAG タスクのクリティカルパスは, 連続し たプロバイダ $\Theta^{*}$ の集合に転送される
%         \item プロバイダ $\theta_{i}^{*} \in \Theta^{*}$ は, 前のプロバイダ $\theta_{i-1}^{*}$ とそのコンシューマ $F\left(\theta_{i-1}^{*}\right)$ が実行を終了した場合にのみ開始できる (図2 (b) )
%         \item また, $F\left(\theta_{i-1}^{*}\right)$ は $G\left(\theta_{i-1}^{*}\right)$  (すなわち, $F\left(\theta_{i-1}^{*}\right)$ と同時実行可能な早期リリースされたコンシューマ) からのレイテンシを受けることができ, その結果, $\theta_{i}^{*}$ の開始がレイテンシする (図2 (c) )
%     \end{itemize}
% \end{frame}

% \begin{frame}{}
%     \begin{itemize}
%         \item 定義 1 および 2 に基づき, $\theta_{i}^{*}$ の並列ワークロード $\alpha_{i}$ は, $m-1$ コアの $f\left(\theta_{i}^{*}\right)$ よりも遅く終了しない
%         \item $\theta_{i}^{*}$ が完了した後, 干渉ワークロード (存在する場合) が全ての $m$ コアで実行され, $F\left(\theta_{i}^{*}\right)$ の最新終了ノードが, 次のプロバイダ (存在する場合) に最も早い開始時刻を提供する
%     \end{itemize}
% \end{frame}

% \begin{frame}{}
%     そのため, 以下の上界が必要
%     \begin{enumerate}
%         \item  は, 並列ワークロードの境界 (すなわち, $\alpha_{i}$ )である

%         \item  $F\left(\theta_{i}^{*}\right)$ において, $f\left(\theta_{i}^{*}\right)$ よりも後に実行される (すなわち, 干渉するワークロードにおいて) 最長の実行 シーケンスに対する境界($\beta_{i}$ と表記する)

%     \end{enumerate}
% \end{frame}

\begin{frame}{DAGタスクの応答時間分析}
    DAGタスクの最悪応答時間は以下の式で導出可能
    \[
        \min \left\{R, L+\left\lceil\frac{1}{m}(W-L)\right\rceil\right\}
    \]
    ここで,
    \begin{equation*}
        R=\sum_{\theta_{i}^{*} \in \Theta^{*}}\left\{L_{i}+\left\lceil\frac{1}{m}\left(W_{i}-L_{i}-\alpha_{i}-\beta_{i}\right)\right\rceil+\beta_{i}\right\}
    \end{equation*}
    % 提案する分析は必ずしも従来の境界を支配しているわけではないため, $\min \left\{R, L+\left\lceil\frac{1}{m}(W-L)\right\rceil\right\}$ を最終的な分析的な境界とする
\end{frame}

\begin{frame}{$\alpha, \beta$計算方法}
    \fullimage{a_b}
\end{frame}
