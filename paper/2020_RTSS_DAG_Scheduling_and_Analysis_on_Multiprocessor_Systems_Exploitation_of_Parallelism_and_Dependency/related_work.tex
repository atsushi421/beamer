% !TeX root = main.tex

\section{RELATED WORK}
\label{sec: rw}

\begin{frame}{既存研究との比較表}
\full{
\begin{table}[tb]
\adjustbox{max width=\textwidth, max height=\slideheight}{
\centering\begin{tabular}{|c|c|} \hline
& XXX \\\hline
本論文 & \ch \\\hline
\end{tabular}
}
\end{table}
}
\end{frame}

\begin{frame}{}
    \begin{itemize}
        \item グローバルスキームを持つホモジニアスマルチプロセッサの場合, 既存のスケジューリング (およびその分析) 手法は, メイクスパンの短縮と最悪の分析的境界の厳密化を目的としている
        \item これらはスライスベース[15], [16]またはノードベース[11], [17]のいずれかに分類される
        \item スライスベースのスケジュールは, ノードレベルの先取りを強制し, 各ノードを多数の小さな計算ユニット (例えば, [15]ではWCETが1のユニット) に分割するものである
        \item このようにすることで, スライスベースの手法はノードレベルの並列性を向上させることができるが, 向上を実現するためには, プリエンプションとマイグレーションの回数を制御する必要がある
    \end{itemize}
\end{frame}

\begin{frame}{}
    \begin{itemize}
        \item ノードベースの手法は, DAGの空間的特性 (例えば, ノードの後続ノード数 [18]やノードのトポロジカルオーダー [11]) または時間的特性 (ノードの実行時間 [17], [2], [19])から得られるヒューリスティクスに基づいて, 明示的にノードの実行順序を生成することによってより汎用的なソリューションを提供する
        \item 以下では, 最も新しい2つのノードベースの手法を説明する
    \end{itemize}
\end{frame}

\begin{frame}{}
    \begin{itemize}
        \item 17]では, 単一周期DAGに対して, 常に最長のWCETを持つ準備ノードが実行され, 並列性を向上させるアノマリーフリーのノンプリエンプティブスケジューリング方式が提案されている
        \item [17]では, ノードの実行時間がWCETに満たない場合に, スケジュールと異なる実行順序になる異常が発生することを防ぐ
        \item これは, オフラインシミュレーションと同じ順序でノードが実行されることを保証することで実現されている
        \item しかし, ノード間の依存関係を考慮しないこのスケジュールでは, DAGの完成にかかるレイテンシを最小化することはできない
        \item 図1の例では, この方法では, ノンクリティカルノード $v_{6}$ の開始が遅れたためにDAGの完成が遅れ, メイクスパンが14となるシナリオが導き出される
    \end{itemize}
\end{frame}

\begin{frame}{}
    \begin{itemize}
        \item 11]では, 新しい応答時間分析が提示され, 明示的なノード実行順序が事前にわかっている場合, 従来の境界[13], [12]を支配している
        \item すなわち, ノード $v_{j}$ は, $v_{j}$ より前にスケジューリングされた同時実行ノードからしかレイテンシを受けられない
        \item そこで, i)クリティカルパスを先に実行し, ii)即時干渉ノード (現在検討されているパス上で最も即時的なレイテンシを引き起こす可能性のあるノード) を先に実行するスケジューリング方法を提案する
    \end{itemize}
\end{frame}

\begin{frame}{}
    \begin{itemize}
        \item 11]の新規性は, DAGのトポロジーとパスの長さの両方を考慮することであり, 我々のアプローチと比較される最先端の分析を提供するものである
        \item しかし, [11]のHeらは, 最長の完全パス (ソースからシンクノードへのパス) の長さに基づいて, 同時実行ノードをスケジュールする, すなわち, 最長の完全パスのノードを最初にスケジュールする
        \item セクションIVで示すように, このヒューリスティックは依存性を考慮しないため, 並列性が低下し, 最終的なクリティカルパスが長くなる可能性がある
    \end{itemize}
\end{frame}
