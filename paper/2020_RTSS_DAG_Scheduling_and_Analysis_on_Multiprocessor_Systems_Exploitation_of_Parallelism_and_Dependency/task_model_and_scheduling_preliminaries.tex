% !TeX root = main.tex

\section{TASK MODEL AND SCHEDULING PRELIMINARIES}
\label{sec: t}

\begin{frame}{}
    \begin{itemize}
        \item 始めに, 本論文で登場する表記法・用語の表を示す
        \item 基本的な表記法・用語は資料中で説明無しで使用する
        \item 別ファイルで開く・印刷するなどして, 常に参照できる状態にしておくことを推奨する
    \end{itemize}
\end{frame}

% !TeX root = main.tex

\begin{frame}{表記法・用語 1}
    \full{
        \begin{table}[tb]
            \adjustbox{max width=\textwidth, max height=\slideheight}{
                \centering\begin{tabular}{|c|l|} \hline
                    $m$                                                                                             & スレッド数                                                                                        \\\hline
                    $\Gamma=\left\{\mathcal{C}_{1}, \mathcal{C}_{2}, \cdots, \mathcal{C}_{|\Gamma|}\right\}$        & チェインのセット                                                                                  \\\hline
                    $|\Gamma|$                                                                                      & $\Gamma$内のチェインの数                                                                          \\\hline
                    $\mathcal{C}_{i}=\left\{c_{i, 1}, c_{i, 2}, \cdots, c_{i,\left|\mathcal{C}_{i}\right|}\right\}$ & チェイン                                                                                          \\\hline
                    $c_{i,j}$                                                                                       & $\mathcal{C}_{i}$の$j$番目のコールバック                                                          \\\hline
                    $\left|\mathcal{C}_{i}\right|$                                                                  & $\mathcal{C}_{i}$内のコールバックの数                                                             \\\hline
                    先行要素                                                                                        & $J_{i}^{k}$ 内の連続する要素 $c_{i, j}^{k}$ と $c_{i, j+1}^{k}$ の各ペアにおける $c_{i, j}^{k}$   \\\hline
                    後続要素                                                                                        & $J_{i}^{k}$ 内の連続する要素 $c_{i, j}^{k}$ と $c_{i, j+1}^{k}$ の各ペアにおける $c_{i, j+1}^{k}$ \\\hline
                    ソースコールバック                                                                              & $\mathcal{C}_{i}$ の最初のコールバック                                                            \\\hline
                    シンクコールバック                                                                              & $\mathcal{C}_{i}$ の最後のコールバック                                                            \\\hline
                \end{tabular}
            }
        \end{table}
    }
\end{frame}

\begin{frame}{表記法・用語 2}
    \full{
        \begin{table}[tb]
            \adjustbox{max width=\textwidth, max height=\slideheight}{
                \centering\begin{tabular}{|c|l|} \hline
                    $T_i$                 & \tabml{$\mathcal{C}_{i}$の周期                \\\underline{周期}: 2 つの連続するチェーンインスタンスのリリース時刻の間の最小間隔}                       \\\hline
                    $D_i$                 & \tabml{$\mathcal{C}_{i}$の相対デッドライン    \\\underline{相対デッドライン}: 時間 $r$ でリリースされた $\mathcal{C}_{i}$ の各チェーンインスタンスは, \\その絶対デッドライン $r+D_{i}$ までに終了する必要がある}           \\\hline
                    $e_{i,j}$             & $c_{i,j}$の最悪実行時間 (WCET)                \\\hline
                    $E_i$                 & $\mathcal{C}_{i}$内のコールバックのWCETの合計 \\\hline
                    $U_{i}=E_{i} / T_{i}$ & $\mathcal{C}_{i}$の利用率                     \\\hline
                \end{tabular}
            }
        \end{table}
    }
\end{frame}

\begin{frame}{表記法・用語 3}
    \full{
        \begin{table}[tb]
            \adjustbox{max width=\textwidth}{
                \centering\begin{tabular}{|c|l|} \hline
                    $J_{i}^{k}$                & $\mathcal{C}_{i}$ の $k$番目のチェーンインスタンス             \\\hline
                    $c_{i, j}^{k}$             & $J_{i}^{k}$ に含まれる$c_{i, j}$ のコールバックインスタンス    \\\hline
                    $R\left(J_{i}^{k}\right)$  & $J_{i}^{k}$の応答時間                                          \\\hline
                    $\mathcal{R}_i^{wc}$       & $\mathcal{C}_{i}$の最悪応答時間                                \\\hline
                    $\Omega$                   & ready セット                                                   \\\hline
                    $h p\left(c_{i, j}\right)$ & コールバック $c_{i, j}$ よりも優先度の高いコールバックのセット \\\hline
                \end{tabular}
            }
        \end{table}
    }
\end{frame}

\begin{frame}{表記法・用語 4}
    \full{
        \begin{table}[tb]
            \adjustbox{max width=\textwidth, max height=\slideheight}{
                \centering\begin{tabular}{|c|l|} \hline
                    updated  & $\Omega$ に新しい要素が追加されること                                                      \\\hline
                    バッチ   & \tabml{複数のコールバックインスタンスが同じポーリングポイントで $\Omega$ に追加された場合, \\これらのインスタンスは同じバッチである} \\\hline
                    ビジー   & スレッドがコールバックインスタンスが実行している状態                                       \\\hline
                    アイドル & スレッドがコールバックインスタンスを実行していない状態                                     \\\hline
                \end{tabular}
            }
        \end{table}
    }
\end{frame}

\begin{frame}{表記法・用語 5}
    \full{
        \begin{table}[tb]
            \adjustbox{max width=\textwidth, max height=\slideheight}{
                \centering\begin{tabular}{|c|l|} \hline
                    $J$   & 分析対象のチェーン                     \\\hline
                    $r$   & $J$のリリース時刻                      \\\hline
                    $f$   & $J$の終了時刻                          \\\hline
                    $c_i$ & $J$の$i$番目のコールバックインスタンス \\\hline
                    $r_i$ & $c_i$がリリースされる時刻              \\\hline
                    $s_i$ & $c_i$が実行開始する時刻                \\\hline
                    $|J|$ & $J$内のコールバックの数                \\\hline
                \end{tabular}
            }
        \end{table}
    }
\end{frame}


\begin{frame}{表記法・用語 6}
    \full{
        \begin{table}[tb]
            \adjustbox{max width=\textwidth, max height=\slideheight}{
                \centering\begin{tabular}{|c|l|} \hline
                    $\mathcal{S}_{k, i}=\left\langle e_{k, a}^{\prime}, e_{k, b}^{\prime}, \cdots\right\rangle$                                                         & $c_i$に対する$\mathcal{C}_{k}$のサブ干渉シーケンス                                            \\\hline
                    $e_{k, a}^{\prime}$                                                                                                                                 & コールバックインスタンス $c_{k, a}$ が $\left[r_{i}, s_{i}\right)$ の間に実行された時間の長さ \\\hline
                    $\mathcal{S}_{k}=\left\{\mathcal{S}_{k, 1}, \mathcal{S}_{k, 2}, \cdots, \mathcal{S}_{k,|\mathcal{C}|}\right\}$                                      & $J$に対する$\mathcal{C}_{k}$の干渉シーケンス                                                  \\\hline
                    $\mathcal{I}_{k,i}$                                                                                                                                 & $c_i$に対する$\mathcal{C}_{k}$の干渉作業                                                      \\\hline
                    $\mathcal{I}_{k}$                                                                                                                                   & $J$に対する$\mathcal{C}_{k}$の干渉作業                                                        \\\hline
                    $\mathcal{I}_{k,i}^\mathcal{P} $                                                                                                                    & \tabml{$c_i$がブロックされている間に, $c_i$と同じコールバックグループに属す                   \\$\mathcal{C}_k$のコールバックインスタンスが実行した時間の総和} \\\hline
                    $\mathcal{I}_{k,i}^\mathcal{E} $                                                                                                                    & \tabml{$c_i$がブロックされている間に, $c_i$と異なるコールバックグループに属す                 \\$\mathcal{C}_k$のコールバックインスタンスが実行した時間の総和} \\\hline
                    $\mathcal{I}_{k,i}^\mathcal{B}  $                                                                                                                   & \tabml{$[r_i, s_i)$の間に, $c_i$と同じmutually exclusiveコールバックグループに属す            \\$\mathcal{C}_k$のコールバックインスタンスが実行した時間の総和} \\\hline
                    $\mathcal{Q}_{k}=\sum_{i=1}^{|\mathcal{C}|}\left(\mathcal{I}_{k, i}+(m-1) \mathcal{I}_{k, i}^{\mathcal{B}}-\mathcal{I}_{k, i}^{\mathcal{E}}\right)$ & $\mathcal{C}_{k}$の実行が$J$の終了時間に与える影響を特徴付けるために開発した項                \\\hline
                    $\Phi_{k,i}$                                                                                                                                        & $\mathcal{Q}_k$に対する $\mathcal{S}_{k,i}$の寄与                                             \\\hline
                \end{tabular}
            }
        \end{table}
    }
\end{frame}

\begin{frame}{表記法・用語 7}
    \full{
        \begin{table}[tb]
            \adjustbox{max width=\textwidth, max height=\slideheight}{
                \centering\begin{tabular}{|c|l|} \hline
                    $L$                                    & 問題ウィンドウの長さ                                                                                         \\\hline
                    $n_{k}(L)$                             & $J$ の問題ウィンドウ中に実行できる $\mathcal{C}_{k}$ のチェーンインスタンスの最大数                          \\\hline
                    $\overrightarrow{\mathcal{M}}_{k}$     & $J$に対する$\mathcal{C}_{k}$の超干渉シーケンス                                                               \\\hline
                    $ \hat{\Phi} $                         & $\Phi_{k,i}$の上界                                                                                           \\\hline
                    $\mathcal{G}(c_{i,j}) $                & \tabml{$c_{i,j}$が属すmutually exclusiveコールバックグループのインデックス}                                  \\\hline
                    $\theta_i$                             & \tabml{$\mathcal{C}_{i}$ の各コールバックが属すmutually exclusiveコールバックグループの集合                  \\ $\theta_{i}=\cup_{\forall c_{i, j} \in \mathcal{C}_{i}}\left\{\mathcal{G}\left(c_{i, j}\right)\right\}$} \\\hline
                    $\mathcal{I}_{k, i}^{\mathcal{E}^{*}}$ & $\mathcal{S}_{k, i}$ 内の $c_{i}$ とは異なるコールバックグループに属すコールバックインスタンスの合計実行時間 \\\hline
                    $\mathcal{Q}_{k}(\mathcal{Y})$         & 各サブ干渉シーケンス $\mathcal{S}_{k, i}$ に関する $\hat{\Phi}_{k, i}$ の合計                                \\\hline
                \end{tabular}
            }
        \end{table}
    }
\end{frame}

\begin{frame}{表記法・用語 8}
    \full{
        \begin{table}[tb]
            \adjustbox{max width=\textwidth, max height=\slideheight}{
                \centering\begin{tabular}{|c|l|} \hline
                    $\Phi_{k, i}^{p, q}$                            & \tabml{$\overrightarrow{\mathcal{M}}_{k}$ の$ p $番目のコールバックインスタンスの開始時刻から \\$ q $番目のコールバックインスタンスの開始時刻までの \\範囲内にある任意のウィンドウによって発生しうる最大の $\hat{\Phi}_{k, i}$} \\\hline
                    $\left|\overrightarrow{\mathcal{M}}_{k}\right|$ & $\overrightarrow{\mathcal{M}}_{k}$ のコールバックインスタンスの数                             \\\hline
                    $\Theta_{i, p}$ ($i \in[1,|\mathcal{C}|])$      & \tabml{$ p $番目のコールバックインスタンスの開始時刻から                                      \\$\overrightarrow{\mathcal{M}}_{k}$ の最後のコールバックインスタンスの終了時刻までの \\範囲に収まる任意のウィンドウによって発生しうる最大の $\sum_{j=i}^{|\mathcal{C}|} \hat{\Phi}_{k, j}$} \\\hline
                \end{tabular}
            }
        \end{table}
    }
\end{frame}


\begin{frame}{研究対象のモデル}
    単一周期ノンプリエンプティブDAGをホモジニアスマルチコアプラットフォーム上で実行する
\end{frame}

\subsection{Task Model}
\label{ssec: ta}

\begin{frame}{基本の定義1}
    \begin{itemize}
        \item DAGタスク $\tau_{x}$ は, $\left\{T_{x}, D_{x}, \mathcal{G}_{x}=\left(V_{x}, E_{x}\right)\right\}$ で定義され, $T_{x}$ はその最小到着間時間, $D_{x}$ は制約付き相対デッドライン, すなわち $D_{x} \leq T_{x}$, $\mathcal{G}_{x}$ はタスクを形成するアクティビティの集合を定義するグラフとする
        \item グラフは $\mathcal{G}_{x}=\left(V_{x}, E_{x}\right)$ と定義され, $V_{x}$ はノードの集合を示し, $E_{x} \subseteq\left(V_{x} \times V_{x}\right)$ は任意の2つのノードを結ぶエッジの集合を与える
        \item 各ノード $v_{x, j} \in V_{x}$ は, 順次実行されなければならない計算ユニットを表し, その最悪実行時間 (WCET)  $C_{x, j}$ によって特徴付けられる
              \notes{簡単のため, DAGタスクが1つの場合は, DAGタスクの添え字($x$, $\tau_{x}$ など) を省略する}
    \end{itemize}
\end{frame}

\begin{frame}{基本の定義2}
    \begin{itemize}
        \item エッジで結ばれた任意の2つのノード $v_{j}$ と $v_{k}$ に対して, $v_{k}$ は $v_{j}$ が実行を終了している場合にのみ実行を開始できる
        \item $v_{j}$ は $v_{k}$ の先行ノードであり, $v_{k}$ は $v_{j}$ の後続ノードである
        \item ノード $v_{j}$ は, 少なくとも一つの先行ノード $\operatorname{pre}\left(v_{j}\right)$ と少なくとも一つの後続ノード $\operatorname{suc}\left(v_{j}\right)$ を持ち, それぞれ正式には $\operatorname{pre}\left(v_{j}\right)=\left\{v_{k} \in V \mid\left(v_{k}, v_{j}\right) \in E\right\}$ と $\operatorname{suc}\left(v_{j}\right)=\left\{v_{k} \in V \mid\left(v_{j}, v_{k}\right) \in E\right\}$ と定義される
    \end{itemize}
\end{frame}

\begin{frame}{基本の定義3}
    \begin{itemize}
        \item ノード $v_{j}$ の直接または推移的に先行ノードおよび後続ノードであるノードを, それぞれ祖先 $\operatorname{anc}\left(v_{j}\right)$ および子孫 $\operatorname{des}\left(v_{j}\right)$ と呼ぶ
        \item $\operatorname{pre}\left(v_{j}\right)=\varnothing$ または $\operatorname{suc}\left(v_{j}\right)=\varnothing$ を持つノード $v_{j}$ を, それぞれソース $v_{s r c}$, シンク $ v_{s i n k}$ と呼ぶ
        \item 各DAGは1つのソースノードと1つのシンクノードを持つと仮定する
        \item $v_{j}$ と同時実行可能なノードは $\mathcal{C}\left(v_{j}\right)=\left\{v_{k} \mid v_{k} \notin\left(\operatorname{anc}\left(v_{j}\right) \cup \operatorname{des}\left(v_{j}\right)\right), \forall v_{k} \in V\right\}$ で与えられる
    \end{itemize}
\end{frame}

\begin{frame}{パス}
    \begin{block}{パス $\lambda_{a}=\left\{v_{s}, \cdots, v_{e}\right\}$}
        $V$ 内のエッジで接続されたノード列
    \end{block}
    \begin{block}{完全パス}
        ソース $v_{s r c}$ とシンク $v_{s i n k}$を含むパス
    \end{block}
    \begin{block}{パスの長さ $\operatorname{len}\left(\lambda_{a}\right)=\sum_{\forall v_{k} \in \lambda_{a}} C_{k}$}
        パス内のノードのWCETの合計
    \end{block}
    % \item ローカルパスは, タスク内のサブパスであり, ソース $v_{s r c}$ とシンク $v_{s i n k}$ の両方を特徴としていない
\end{frame}

\begin{frame}{クリティカルパス}
    \begin{block}{クリティカルパス $\lambda^{*}$}
        最長の完全パス
    \end{block}
    \begin{block}{クリティカルノード}
        クリティカルパスに含まれるノード
    \end{block}
    \begin{block}{非クリティカルノード $V^{\urcorner}=V \backslash \lambda^{*}$}
        クリティカルノード以外のノード
    \end{block}
    \begin{block}{ワークロード $W=$  $\sum_{\forall v_{k} \in V} C_{k}$}
        DAGタスクの WCET の合計
    \end{block}
    % \item 全ての非クリティカルノードのワークロードを, 非クリティカルワークロードと呼ぶ
\end{frame}


\begin{frame}{DAGタスクの例}
    \begin{columns}
        \begin{column}{0.4\textwidth}
            \begin{itemize}
                \item $\operatorname{pre}\left(v_{7}\right)=\left\{v_{5}, v_{6}\right\}$
                \item $\operatorname{anc}\left(v_{7}\right)=\left\{v_{1}, v_{5}, v_{6}\right\}$
                \item $\operatorname{suc}\left(v_{7}\right)=\operatorname{des}\left(v_{7}\right)=\left\{v_{8}\right\}$
                \item $\mathcal{C}\left(v_{7}\right)=\left\{v_{2}, v_{3}, v_{4}\right\}$
                \item $L=10, W=24$
                \item $\lambda^{*}=\left\{v_{1}, v_{5}, v_{7}, v_{8}\right\}$
                \item $v_{s r c}=v_{1}$, $v_{s i n k}=v_{8}$
            \end{itemize}
        \end{column}
        \begin{column}{0.6\textwidth}
            \fullimage{dag_exam}
        \end{column}
    \end{columns}
\end{frame}


\subsection{Work-Conserving schedule and analysis}
\label{ssec: wc}

\begin{frame}{作業保存型スケジューラ}
    DAGタスクのスケジューリングに関する既存の研究の大部分は, 作業保存型スケジューラを想定する
    \begin{block}{作業保存型スケジューラ}
        保留中のワークロードが存在するときにプロセッサを決してアイドル状態にしないスケジューラ
    \end{block}
\end{frame}

\begin{frame}[label=oldRes]{作業保存型スケジューラの最悪応答時間}
    任意の作業保存型スケジューラでグローバルにスケジューリングされたタスクの最悪応答時間が知られている
    \begin{block}{DAGタスク$\tau_x$の既知の最悪応答時間}
        \begin{equation*}
            R_{x}=L_{x}+\left\lceil\frac{1}{m}\left(W_{x}-L_{x}\right)\right\rceil+\sum_{\tau_{y} \in h p(x)} I_{x, y}
        \end{equation*}
        \begin{itemize}
            \item \desc{$m$}{プロセッサ数}
            \item \desc{$I_{x, y}$}{高優先度DAGタスク $\tau_{y}$ から$\tau_{x}$への干渉}
            \item \desc{$h p(x)$}{$\tau_{x}$以上の優先度を持つDAGタスクセット}
        \end{itemize}
    \end{block}
\end{frame}

\begin{frame}{既知の応答時間の悲観性}
    \begin{itemize}
        \item 既知の分析は, ノード $v_{j}$ が全ての同時実行ノードによって干渉されると仮定しているため悲観的
        \item そこで, 単一DAGタスクの実行時メイクスパンを減らし, 分析的な最悪応答時間を厳しくするための新しい方法を提案する
    \end{itemize}
\end{frame}

% \begin{frame}{}
%     \begin{itemize}
%         \item 図 1(b)は, デュアルコアシステムにおける, 例の DAG の可能な実行シナリオを示したものである
%         \item ノードをランダムにスケジューリングした場合, 合計240の異なる実行シナリオが可能であり, メイクスパンは13から17の範囲である
%         \item 上記の分析により, $R=L+\frac{1}{m}$ ( $W-$  $L)=10+\frac{1}{2}(24-10)=17$ )と安全な境界が得られる
%         \item しかし, 17よりはるかに低いメイクスパンを持つスケジューリングオーダーが存在する
%         \item ジョブ保存スケジュールと古典的な分析に基づいて, 単一の再帰的DAGタスクの実行時メイクスパンを減らし, 分析的な境界を厳しくするための新しい方法を提案する
%     \end{itemize}
% \end{frame}
