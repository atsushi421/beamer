% !TeX root = main.tex

\begin{frame}{表記法・用語 1 (添え字あり版)}
    \full{
        \begin{table}[tb]
            \adjustbox{max width=\textwidth, max height=\slideheight}{
                \centering\begin{tabular}{|c|l|} \hline
                    $\tau_x = \left\{T_{x}, D_{x}, \mathcal{G}_{x}=\left(V_{x}, E_{x}\right)\right\}$ & DAGタスク              \\\hline
                    $T_x$                                                                             & $\tau_x$の最小到着間隔 \\\hline
                    $D_x$                                                                             & 相対デッドライン       \\\hline
                    $\mathcal{G}_{x}=\left(V_{x}, E_{x}\right)$                                       & $\tau_x$を定義するDAG  \\\hline
                    $V_x$                                                                             & ノード集合             \\\hline
                    $E_x$                                                                             & エッジ集合             \\\hline
                    $v_{x,j} \in V_x$                                                                 & ノード                 \\\hline
                    $C_{x,j}$                                                                         & $v_{x,j}$のWCET        \\\hline
                \end{tabular}
            }
        \end{table}
    }
\end{frame}

\begin{frame}{表記法・用語 2 (添え字省略版)}
    \full{
        \begin{table}[tb]
            \adjustbox{max width=0.8\textwidth, max height=0.6\slideheight}{
                \centering\begin{tabular}{|c|l|} \hline
                    $\operatorname{pre}\left(v_{j}\right)=\left\{v_{k} \in V \mid\left(v_{k}, v_{j}\right) \in E\right\} $  & $v_j$の先行ノード                  \\\hline
                    $\operatorname{suc}\left(v_{j}\right)=\left\{v_{k} \in V \mid\left(v_{j}, v_{k}\right) \in E\right\}$   & $v_j$の後続ノード                  \\\hline
                    $anc(v_j)$                                                                                              & $v_J$の祖先                        \\\hline
                    $des(v_j)$                                                                                              & $v_J$の子孫                        \\\hline
                    $\mathcal{C}\left(v_{j}\right)$                                                                         & $v_{j}$ と同時実行可能なノード集合 \\\hline
                    $\lambda_{a}=\left\{v_{s}, \cdots, v_{e}\right\}$                                                       & パス                               \\\hline
                    $\Lambda_{V}$                                                                                           & $V$ のパスの集合                   \\\hline
                    $\operatorname{len}\left(\lambda_{a}\right)$                                                            & $\lambda_{a}$の長さ                \\\hline
                    $L=\max \left\{\operatorname{len}\left(\lambda_{a}\right), \forall \lambda_{a} \in \Lambda_{V}\right\}$ & クリティカルパスの長さ             \\\hline
                    $V^{\urcorner}$                                                                                         & 非クリティカルノード               \\\hline
                    $W$                                                                                                     & ワークロード                       \\\hline
                \end{tabular}
            }
        \end{table}
    }
\end{frame}

\begin{frame}{表記法・用語 3}
    \full{
        \begin{table}[tb]
            \adjustbox{max width=\textwidth, max height=\slideheight}{
                \centering\begin{tabular}{|c|l|} \hline
                    $m$        & プロセッサ数                                        \\\hline
                    $I_{x, y}$ & 高優先度DAGタスク $\tau_{y}$ から$\tau_{x}$への干渉 \\\hline
                    $h p(x)$   & $\tau_{x}$以上の優先度を持つDAGタスクセット         \\\hline
                \end{tabular}
            }
        \end{table}
    }
\end{frame}

\begin{frame}{表記法・用語 4}
    \full{
        \begin{table}[tb]
            \adjustbox{max width=0.8\textwidth, max height=\slideheight}{
                \centering\begin{tabular}{|c|l|} \hline
                    $\Theta^*$                 & プロバイダ集合                                                                                     \\\hline
                    $\Theta$                   & コンシューマ集合                                                                                   \\\hline
                    $\theta_i^*$               & $i$番目のプロバイダ                                                                                \\\hline
                    $p_j$                      & $\tau_j$の優先度                                                                                   \\\hline
                    $L_i$                      & $\theta_i^*$の長さ                                                                                 \\\hline
                    $W_i$                      & $F\left(\theta_i^*\right) \text { and } G\left(\theta_i^*\right)$内のノードの総ワークロード        \\\hline
                    $\alpha_i$                 & $\theta_i^*$と並列に実行できる$F\left(\theta_i^*\right), G\left(\theta_i^*\right)$内のワークロード \\\hline
                    $F\left(\theta_i^*\right)$ & $\theta_i^*$のコンシューマ集合                                                                     \\\hline
                    $G\left(\theta_i^*\right)$ & $\theta_i^*$と並列に実行できる後続プロバイダのコンシューマ集合                                     \\\hline
                    $f(\cdot)$                 & プロバイダまたはコンシューマの終了時間を返す関数                                                   \\\hline
                    $l_j(\cdot)$               & $v_j$を含むクリティカルパスの長さを返す関数                                                        \\\hline
                \end{tabular}
            }
        \end{table}
    }
\end{frame}
