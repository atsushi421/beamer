% !TeX root = main.tex

\section{INTRODUCTION}
\label{sec: introduction}

\begin{frame}{}
    \begin{itemize}
        \item ROS (ロボットオペレーティングシステム) [1] は, 現在最も人気のあるフレームワークであり, ロボットソフトウェア開発のデファクトスタンダードである2010 年にバージョン 1.0 がリリースされて以来, ROS は産業界と学界の両方で何十万人もの開発者と研究者によって使用され, 膨大な数のさまざまな種類のロボットシステムを動かしてきた [2], [3]
        \item ROS の主要な哲学は, ソフトウェアのモジュール性と構成可能性を促進することによって, ロボットソフトウェア開発の生産性を促進することであるが, これはいくつかの根本的な欠点にもつながる
        \item 主な欠点は, リアルタイム機能が不足していることである
        \item ロボットでの計算は通常, タイミングの制約を受けるため, ROS の広範な普及が制限される [4]
    \end{itemize}
\end{frame}

\begin{frame}{}
    \begin{itemize}
        \item 2017年からリリースされた第2世代のROSであるROS 2[5]は, ROSの成功したコンセプトを継承し, 改良された基盤の上に載せています[6].
        \item ROS 2における主な検討事項は, 強力なリアルタイム性をサポートすることです.
        \item 例えば, ROS 2は, リアルタイムデータ交換のためのプロセス間通信ミドルウェアとしてDDS (Data Distribution Service) [7]を採用し,  (Linux上でのみ動作するのではなく) リアルタイムOS上に展開でき, 制御指向コードパス上のリアルタイムセーフなリソース事前割り当てをサポートします.
    \end{itemize}
\end{frame}

\begin{frame}{}
    \begin{itemize}
        \item ROS 2の新しいアーキテクチャは良い基礎を提供しますが, それだけではハードリアルタイムロボティックソフトウェアをサポートするには不十分です.
        \item ハードリアルタイムシステムの場合, 設計者はシステムのタイミング特性を検証し, 実行時にいかなる状況下でもタイミング制約が守られることを保証しなければならない.
        \item 最近, Casini ら [6] は, ROS 2 におけるプロセッサスケジューリングの形式的なモデリングと分析に関する先駆的な研究を行った.
        \item 特に[6]は, 計算タスクを多重化するROS 2の中核コンポーネントであるROS 2エグゼキュータにおけるワークロード構造とスケジューリングポリシーをモデル化し, ROS 2エグゼキュータ上の処理チェーンの応答時間を上界する技術を開発した.
    \end{itemize}
\end{frame}

\begin{frame}{}
    \begin{itemize}
        \item その結果, ROS 2エグゼキュータのスケジューリング動作は, これまでのリアルタイムスケジューリング研究で研究されてきたものとは大きく異なり, 新たな分析技術が必要であることが判明しました.
        \item 6]のハイレベルなビジョンと具体的な成果は, ROSコミュニティ[8], [9]に直ちに影響を与え, ROS 2の分析的ハードリアルタイム保証をサポートするために, リアルタイムスケジューリングとROSコミュニティ間の緊密な相互作用を引き起こす可能性があります.
    \end{itemize}
\end{frame}


\begin{frame}{}
    \begin{itemize}
        \item その重要性にも関わらず, [6] の作業にはいくつかの問題がある
        \item ROS 2 エグゼキュータ上の処理チェーン ([6] ではサブチェーンと呼ばれる) の応答時間は, 楽観的 (境界は実際の最悪のケースよりも小さい可能性がある) と悲観的の両方である
        \item (一般に, 境界は実際の最悪のケースよりも不必要に大きくなる)
        \item これらの問題については, セクション IV で詳しく説明する
    \end{itemize}
\end{frame}

\begin{frame}{}
    \begin{itemize}
        \item 本論文の最初の貢献は, ROS 2 エグゼキュータ上でチェーンを処理するための新しい応答時間分析技術を開発することによって, [6] の上記の問題に対処することである
        \item 本論文の新しい分析手法は, ROS 2 エグゼキュータのスケジューリング動作に関する深い洞察を調査することで, 分析精度を大幅に向上させます
        \item ランダムに生成されたワークロードを使用した経験的評価では, 新しい最悪応答時間が, さまざまなパラメーター設定の下で大幅なマージンを持って [6] よりも一貫して優れていることが示されている
    \end{itemize}
\end{frame}

\begin{frame}{}
    \begin{itemize}
        \item 本論文の 2 番目の貢献は, コールバックの優先度の割り当てが ROS 2 エグゼキュータの処理チェーンの応答時間にどのように影響するかを研究することである
        \item 処理チェーンの応答時間は, この処理チェーンの最後のコールバックの優先度のみに依存することを自明にした
        \item 最後のコールバックに割り当てられた優先度が高いほど, 処理チェーンの応答時間が短くなる可能性がある
        \item このプロパティは, ROS 2 アプリケーション開発者がシステムの応答性を改善するのに役立つ
        \item これは, ハードリアルタイムシステムの最悪応答時間だけでなく, ソフトリアルタイムシステムと一般システムの実際の最悪のケースと平均応答時間も改善する
        \item 余分な設計費もないランダムに生成されたワークロードを使用したシミュレーション実験と, 現実的なROS 2 ベースのロボットソフトウェアシステムを使用したケーススタディの両方を使用して, 調査結果の使用法を示す
    \end{itemize}
\end{frame}
