% !TeX root = main.tex

\section{SYSTEM MODEL}
\label{sec: system model}

\begin{frame}{}
    \fitimage{
        \begin{itemize}
            \item 初に, ROS2 に関する簡単な背景とシステムモデルのコンテキストを提供する (ROS2 のより詳細な紹介は [6] にある)
            \item 図 1-(a) に ROS2 のアーキテクチャを示す
        \end{itemize}
    }{ROS.png}
\end{frame}

\begin{frame}{}
    \begin{itemize}
        \item ROS2 アプリケーションは通常, publishsubscribe パラダイムを使用して相互に通信する個々のノードの構成で構成される
        \item ノードはトピックにメッセージをpublishし, トピックにsubscribeしているノードにメッセージをブロードキャストする
        \item ノードは, コールバックをアクティブにして各メッセージを処理することにより, publish メッセージに反応する
        \item ROS アプリケーションを展開するには, 個々のノードがホストに分散され, オペレーティングシステムプロセスにマッピングされる
    \end{itemize}
\end{frame}

\begin{frame}{}
    \begin{itemize}
        \item ROS2 クライアントライブラリの Executor は, オペレーティングシステムプロセス内のノードのコールバックの実行を調整する
        \item ROS2 には 2 つの組み込みエグゼキュータが用意されている1 つのスレッドでコールバックを実行するシーケンシャルエグゼキュータと, 保留中のコールバックの処理を複数のスレッドに分散する並列エグゼキュータである
        \item [6] と同様に, 本論文では, ワークロードをシーケンシャルに実行する必要があるシングルスレッドエグゼキュータで実行される処理チェーン (図 1-(b) に示すように) として実装されるアプリケーションを検討する
    \end{itemize}
\end{frame}

\begin{frame}{}
    \begin{itemize}
        \item チェーン内の各頂点は, コールバック (*1): を表す
        \item 処理チェーンは, 通常, システムタイマによってアクティベーションされたタイマコールバックで始まり, 複数のエグゼキュータにまたがる可能性があるいくつかのレギュラーコールバックが続きます
        \item 本論文では, 単一のエグゼキュータ上のチェーンにバインドされた応答時間を計算することに注意を限定する
        \item それにも関わらず, [6] と同じように, 構成パフォーマンス分析 (CPA) アプローチ [26]-[29] によって, 複数のエグゼキュータにまたがるチェーンのエンドツーエンドの応答時間分析に結果を簡単に拡張できる
    \end{itemize}
\end{frame}
