% !TeX root = main.tex

\section{RELATED WORK}
\label{sec: Related Work}

\begin{frame}{}
    \begin{itemize}
        \item 本節では, 主に並列タスクに焦点を当て, 実時間スケジューリングに関する密接な関連研究をレビューする
    \end{itemize}
\end{frame}

\begin{frame}{}
    \begin{itemize}
        \item リアルタイムマルチプロセッサスケジューリングは, 複数のプロセッサまたはコアを持つコンピュータ上でシーケンシャルなタスクをスケジューリングすることを考慮し, 広範囲に研究されている (サーベイとして[10, 23]を参照)
\item また, Litmus $^{R T}[17$, $19]$ のようなプラットフォームはこれらのタスクセットをサポートするように設計されている
    \end{itemize}
\end{frame}

\begin{frame}{}
    \begin{itemize}
        \item ここでは, 関連するいくつかの理論的な結果をレビューする
\item 研究者たちは, リソース増強境界, 利用境界, 容量増強境界を証明してきた
\item マルチプロセッサ上のシーケンシャルタスクに対するG-EDFの最もよく知られたリソース境界は2 [7]であり, 小さな $\epsilon$ に対する $2-\frac{1}{m}+\epsilon$ のキャパシティ拡張境界は[14]である
\item また, 分割EDFとバージョン分割静的優先度スケジューラは, $2[3,36]$ の利用率境界を提供する
\item G-RMは暗黙的デッドラインタスクに対して3[2]の容量増強境界を提供する
    \end{itemize}
\end{frame}

\begin{frame}{}
    \begin{itemize}
        \item リアルタイム並列タスクの場合, 初期の研究の多くは, 可鍛型タスク[22, 29, 32]や可成型タスク[38]などの限られたタスクモデルのタスク内並列性を考慮していた
\item 加藤ら[29]は, 成形可能な並列タスクシステムのギャングEDFスケジューリングを研究している
    \end{itemize}
\end{frame}

\begin{frame}{}
    \begin{itemize}
        \item その後, Cilkファミリー[13, 21]やOpenMP[41], インテルのThread Building Blocks[43]などの一般的な並列プログラミング言語によって生成される, より現実的なタスクモデルが検討されている
\item これらの言語やライブラリは, 並列for-loopやfork/join, spawn/syncなどのプリミティブをサポートし, プログラム内に並列性を持たせている
\item これらの構成要素を使用することで, 異なるタイプのDAGで表現可能な構造を持つタスクが生成される
    \end{itemize}
\end{frame}

\begin{frame}{}
    \begin{itemize}
        \item リアルタイムコミュニティでは, 並列同期タスクを持つタスクが他よりも多く研究されている
\item これらのタスクは, 並列forループのみを使用して並列性を生成する場合である
\item Lakshmananら[31]は制限付き同期タスクモデルに対して $3.42$ の (容量) 増大境界を証明したが, これはタスク内の各並列forループが同じ反復回数を持つように制限したときに生成されるものである
\item 一般的な同期タスク (parallel-forループの反復回数に制限なし) についても $[4,30,40,44]$ で研究されている
\item (これらの結果の詳細は第I節で示した) Chwaら[20]は応答時間の分析を行っている
    \end{itemize}
\end{frame}

\begin{frame}{}
    \begin{itemize}
        \item もし, 使用するプリミティブを並列-forループに限定しないならば, より一般的なタスクモデルが得られ, 一般的な有向非巡回グラフで最も簡単に表現される
\item G-EDFに対するリソース増大境界 $2-\frac{1}{m}$ は, 任意のデッドラインを持つ単一のDAGに対して証明され [8], また複数のDAGに対しても証明された [15, 34]
\item 暗黙的デッドラインを持つタスクに対して, $4-\frac{2}{m}$ のキャパシティ拡張境界が[34]で証明された
\item Liuら[35]は, G-EDFの応答時間に関する分析を行った
    \end{itemize}
\end{frame}

\begin{frame}{}
    \begin{itemize}
        \item 非リアルタイム並列システム $[5,6,24-26,42]$ のスケジューリングについては, 重要な研究がなされている
\item この文脈では, 一般にスループットを最大化することが目標である
\item リストスケジューリング[18, 27]やワークスティアリング[12]など, 証明可能な良いスケジューリング戦略がいろいろと設計されてきた
\item また, これらの成果をもとに, 多くの並列言語やエグゼキュータが構築されている
\item 単一プラットフォーム上の複数タスクはリソース配分の公平性の観点から検討されているが[1], 実時間制約を考慮したものはない
    \end{itemize}
\end{frame}
