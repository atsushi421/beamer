% !TeX root = main.tex

\section{SYSTEM MODEL}
\label{sec: system model}

\begin{frame}{}
    \begin{itemize}
        \item 始めに, 本論文で登場する表記法・用語の表を示す
        \item 基本的な表記法・用語は資料中で説明無しで使用する
        \item 別ファイルで開く・印刷するなどして, 常に参照できる状態にしておくことを推奨する
    \end{itemize}
\end{frame}

% !TeX root = main.tex

\begin{frame}{表記法・用語 1}
    \full{
        \begin{table}[tb]
            \adjustbox{max width=\textwidth, max height=\slideheight}{
                \centering\begin{tabular}{|c|l|} \hline
                    $N$      & タスクの数                 \\\hline
                    $M$      & プロセッサの個数           \\\hline
                    $\tau_i$ & タスク                     \\\hline
                    $T_i$    & $\tau_i$の周期             \\\hline
                    $D_i$    & $\tau_i$の相対デッドライン \\\hline
                    $J_{i}$  & $\tau_i$のジョブ           \\\hline
                \end{tabular}
            }
        \end{table}
    }
\end{frame}

\begin{frame}{表記法・用語 2}
    \full{
        \begin{table}[tb]
            \adjustbox{max width=\textwidth, max height=\slideheight}{
                \centering\begin{tabular}{|c|l|} \hline
                    $G\left(J_{i}\right)=\langle V, E\rangle$ & $J_{i}$のDAG                                 \\\hline
                    $V$                                       & 頂点のセット                                 \\\hline
                    $E$                                       & エッジのセット                               \\\hline
                    $v$                                       & ワークロードの一部                           \\\hline
                    $c(v)$                                    & $v$のWCET                                    \\\hline
                    $(u, v) \in E$                            & $u$ と $v$ の間の優先関係                    \\\hline
                    先行頂点                                  & エッジ$(u, v)$があるとき, $u$は$v$の先行頂点 \\\hline
                    後続頂点                                  & エッジ$(u, v)$があるとき, $v$は$u$の後続頂点 \\\hline
                    ソース頂点                                & 先行頂点を持たない頂点                       \\\hline
                    シンク頂点                                & 後続頂点を持たない頂点                       \\\hline
                \end{tabular}
            }
        \end{table}
    }
\end{frame}

\begin{frame}{表記法・用語 3}
    \full{
        \begin{table}[tb]
            \adjustbox{max width=\textwidth, max height=\slideheight}{
                \centering\begin{tabular}{|c|l|} \hline
                    適格     & 頂点は, その先行頂点が全て終了している場合に適格である \\\hline
                    完全パス & ソース頂点で始まり, シンク頂点で終わるパス             \\\hline
                    $C_i$    & $G(J_i)$内の全ての頂点の合計WCET                       \\\hline
                    $L_i$    & $G(J_i)$内のクリティカルパス上の全ての頂点の合計 WCET  \\\hline
                \end{tabular}
            }
        \end{table}
    }
\end{frame}

\begin{frame}{表記法・用語 4}
\full{
\begin{table}[tb]
\adjustbox{max width=\textwidth, max height=\slideheight}{
\centering\begin{tabular}{|c|l|} \hline
    $\Theta=\left\{\theta_{1}, \cdots, \theta_{|\Theta|}\right\}$ & アクティブVPのセット \\\hline
   $\theta_{z}$ & \tabml{$\Theta$ 内の $z$番目のアクティブVPの初期バジェット \\コンテキストから明白な場合は$\Theta$ 内の $z$番目のアクティブVPも示す} \\\hline

\end{tabular}
}
\end{table}
}
\end{frame}


\begin{frame}{タスクモデル}
    \begin{itemize}
        \item $n$ 個の独立な散発的実時間タスク $\left\{\tau_{1}, \tau_{2}, \ldots, \tau_{n}\right\}$ の集合 $\tau$ を考える
        \item タスク $\tau_{i}$ はジョブの無限のシーケンスを生成する
        \item タスク $\tau_{i}$ に対して, 周期 $T_{i}$ はタスクインスタンスの連続到着間の最小間隔を表し, 相対デッドライン $D_{i}$ はジョブ実行の時間制約を表す
        \item 各タスクは暗黙的デッドラインを持つ
    \end{itemize}
\end{frame}

\begin{frame}{プラットフォームモデル}
    タスクセットのスケジューラビリティを, $m$ 個のホモジニアスマルチコアプラットフォーム上で検討する
\end{frame}

\begin{frame}{DAGモデル}
    \begin{itemize}
        \item 各タスク $\tau_{i} \in \tau$ は並列タスクであり, 有向非巡回グラフ (DAG) として特徴付けられる
        \item DAGの各ノード (サブタスク) は命令列 (スレッド) を表し, 各エッジはノード間の依存関係を表す
        \item ノードは, その先行タスクが全て実行されたときに実行可能な状態になる
    \end{itemize}
\end{frame}

\begin{frame}{DAGのパラメータ}
    本論文はDAGの具体的な構造に基づいた分析を行う必要がないため, 以下2つのパラメータのみを定義する
    \begin{block}{$\tau_{i}$ の総ワークロード $C_{i}$}
        $\tau_{i}$ の全てのサブタスクの最悪実行時間の合計
    \end{block}
    \begin{block}{$\tau_{i}$ のクリティカルパス長 $L_{i}$}
        コアが無限にあると想定した場合のタスクの最悪実行時間
    \end{block}
\end{frame}

\begin{frame}{利用率}
    \begin{block}{$\tau_{i}$ の利用率 $u_i$}
        $u_i = \frac{C_{i}}{T_{i}}=\frac{C_{i}}{D_{i}}$
    \end{block}
    \begin{block}{タスクセットの総利用率 $U_{\Sigma}$}
        $U_{\Sigma}=\sum_{\tau_{i} \in \tau} u_{i}$
    \end{block}
\end{frame}
