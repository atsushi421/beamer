% !TeX root = main.tex

\section{PRACTICAL CONSIDERATIONS}
\label{sec: Practical Considerations}

\begin{frame}{セクションサマリ}
    \begin{itembox}[l]{\textbf{目的}}
        実用的な観点から, フェデレートスケジューリングの実行効率と有効性を検討する
    \end{itembox}
\end{frame}

% \begin{frame}{}
%     \begin{itemize}
%         \item 前節で示したように, フェデレートスケジューリング, G-EDF, G-RMの容量増大限界はそれぞれ, 2, $2.618$, 3.732である
%         \item 本節では, 実用的な観点から, それらの実行効率と有効性を検討する
%         \item 静的優先度と動的優先度, および, グローバルスケジューリングとパーティショニングスケジューリング, スケジューリングと同期のオーバヘッドによるオーバヘッド, および, 作業保存型と非保存型の4つの次元を考慮する
%     \end{itemize}
% \end{frame}

\begin{frame}{実装のしやすさ}
    \begin{itemize}
        \item 一般に, 動的優先度スケジューラよりも固定優先度スケジューラの方が実装しやすい
        \item フェデレートスケジューリングでは, 高利用率タスクには優先度の割り当てが不要で, 低利用率タスクには固定または動的優先度スケジューラを使用できる
        \item そのため, フェデレートスケジューリングは比較的容易に実装できる
    \end{itemize}
\end{frame}

\begin{frame}{プリエンプション・マイグレーションによるオーバヘッド}
    以下の理由により, フェデレートスケジューリングのオーバヘッドはグローバルスケジューリングより小さいと予想できる
    \begin{itemize}
        \item 各高利用率タスクに最小限の専用コアを割り当てるので, 高利用率タスクのプリエンプションはなく, マイグレーションの回数も最小に抑えられる
        \item シーケンシャル低利用率タスクの場合, プリエンプションとマイグレーションは, より高い優先度を持つ新しいジョブがリリースされたときにのみ発生する
    \end{itemize}
\end{frame}

\begin{frame}{同期やスケジューリングによるオーバヘッド}
    \begin{itemize}
        \item 同期やスケジューリングのオーバヘッドは通常, 各タスクに割り当てられたコア数に対してほぼ線形になる
        \item フェデレートスケジューリングの下では, 最小限のコア数が割り当てられるので, このオーバヘッドは小さい
    \end{itemize}
\end{frame}

\begin{frame}{フェデレートスケジューリングの欠点}
    \begin{itemize}
        \item 多くの実システムでは, 最悪実行時間は悲観的
        \item フェデレートスケジューリングでは, 最悪実行時間に基づいてタスクに割り当てられたコアは, リソースの過剰供給により, ほとんどの時間アイドル状態になる可能性がある
        \item これに対し, グローバルEDFやグローバルRMは, スレッドマイグレーションにより利用可能なコアを動的に活用できる
    \end{itemize}
\end{frame}
