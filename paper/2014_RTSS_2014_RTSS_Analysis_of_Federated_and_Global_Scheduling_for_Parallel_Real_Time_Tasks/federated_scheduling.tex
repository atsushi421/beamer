% !TeX root = main.tex

\section{FEDERATED SCHEDULING}
\label{sec: federated scheduling}

\begin{frame}{セクションサマリ}
    \begin{itembox}[l]{\textbf{目的}}
        暗黙的デッドラインを持つ並列タスクのためのフェデレートスケジューリング戦略を示す
    \end{itembox}
\end{frame}

\begin{frame}{タスクの分類}
    \begin{itemize}
        \item 入力はタスクセット $\tau$
        \item まず, タスクを以下2種類に分類する
              \begin{definition}[高利用率タスク $\tau_{\text {high }}$]
                  利用率が1以上 $\left(u_{i} \geq 1\right)$ であるタスク
              \end{definition}
              \begin{definition}[低利用率タスク $\tau_{\text {low }}$]
                  利用率が1未満 $\left(u_{i} < 1\right)$ であるタスク
              \end{definition}
    \end{itemize}
\end{frame}

\begin{frame}{高利用率タスクの専用コア数の決定}
    \begin{itemize}
        \item 各高利用率タスク $\tau_i$ に $n_{i}$ 個の専用コアを割り当てる
        \item $n_{i}$ は以下の式で計算する
              \begin{equation*}
                  n_{i}=\left\lceil\frac{C_{i}-L_{i}}{D_{i}-L_{i}}\right\rceil
              \end{equation*}
    \end{itemize}
\end{frame}

\begin{frame}{フェデレートスケジューリングのスケジュール成功条件}
    \begin{itemize}
        \item 高利用率タスク $\tau_{\text {high }}$ に割り当てられたコアの総数を $n_{\text {high }}=\sum_{\tau_{i} \in \pi_{\text {high }}} n_{i}$ とする
        \item 残りのコアを全ての低利用率タスク $\tau_{\text {low }}$ に割り当て, $n_{\text {low }}=m-n_{\text {high }}$ とする
        \item フェデレートスケジューリングアルゴリズムは以下の式が満たされる場合, タスクセット $\tau$ をスケジュール可能
              \[
                  n_{\text {low }} \geq 2 \sum_{\tau_{i} \in \tau_{\text {low }}} u_{i}
              \]
              \notes{証明略}
    \end{itemize}
\end{frame}

\begin{frame}{ランタイムスケジューリングの選択肢}
    フェデレートスケジューリングでは, 以下の条件を満たす任意のランタイムスケジューリングアルゴリズムを使用できる
    \begin{block}{高利用率タスク}
        貪欲 (作業保存型) スケジューラ
    \end{block}
    \begin{block}{低利用率タスク}
        $1 / 2$ の利用率境界を持つスケジューリングアルゴリズム
        \notes{この時, $C_{i} \leq D_{i}$ で並列実行が必要ないため, 低利用率タスクは安全にシーケンシャルタスクとして扱える}
    \end{block}
\end{frame}
