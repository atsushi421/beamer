% !TeX root = main.tex

\section{INTRODUCTION}
\label{sec: introduction}

\begin{frame}{DAG}
    \begin{itemize}
        \item サイバーフィジカルシステム (CPS) はとコンポーネントの機能間の複雑な相互作用のために, 有向非巡回グラフ(DAG)で表現される
        \item ヘテロジニアスコンピュータプラットフォームが与えられた場合, リソースと優先順位の仕様をすべて満たしながら CPS のアプリケーションタスクの実行を成功させるために, DAGスケジューリング問題を解く必要がある
    \end{itemize}
\end{frame}

\begin{frame}{背景と提案}
    \begin{itemize}
        \item DAGスケジューリング問題はNP-完全問題であるため, ヒューリスティックの設計に重点が置かれている
        \item リストベーススケジューリングは効率的なヒューリスティックとして知られている
        \item 本論文では,「Heterogeneous Makespan minimizing DAG Scheduler(HMDS)」を開発する
    \end{itemize}
\end{frame}

\begin{frame}{貢献}
    \begin{itemize}
        \item 優れた性能を示すリストベースの貪欲なオフラインスケジューリングアルゴリズム HMDS-Bl を提案
        \item HMDS-Bl に枝切りメカニズムを適用し, 低オーバヘッドに拡張した HMDS を提案
    \end{itemize}
\end{frame}
