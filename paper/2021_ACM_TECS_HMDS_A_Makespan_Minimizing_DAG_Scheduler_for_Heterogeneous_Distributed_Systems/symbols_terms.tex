% !TeX root = main.tex

\begin{frame}{表記法・用語 1}
    \full{
        \begin{table}[tb]
            \adjustbox{max width=\textwidth, max height=\slideheight}{
                \centering\begin{tabular}{|c|l|} \hline
                    $G(V, E)$                               & DAG                                      \\\hline
                    $V = \{\tau_1, \tau_2,
...  , \tau_{|V|}\}$ & タスクセット                             \\\hline
                    $E$                                     & タスク間の優先度制約                   \\\hline
                    $e_{i,j} \in E$                         & $\tau_i, \tau_j$ 間の優先度制約        \\\hline
                    $data_{i,j}$                            & $\tau_i$から$\tau_j$に送信されるデータ量 \\\hline
                    $pred(\tau_j)$                          & $\tau_j$の直接先行タスク                 \\\hline
                    $succ(\tau_j)$                          & $\tau_j$の直接後続タスク                 \\\hline
                    $\tau_{exit}$                           & 出口ノード                               \\\hline
                \end{tabular}
            }
        \end{table}
    }
\end{frame}

\begin{frame}{表記法・用語 2}
    \full{
        \begin{table}[tb]
            \adjustbox{max width=\textwidth, max height=\slideheight}{
                \centering\begin{tabular}{|c|l|} \hline
                    $P = \{p_1, p_2,
...  , p_{|P|}\}$                     & ヘテロジニアスプロセッサセット                                               \\\hline
                    行列 $B$ (サイズ $|P|\times|P|$)                   & 全てのプロセッサ間のデータ転送帯域幅を格納した行列                           \\\hline
                    $b_{m,n}$                                          & $p_m$と$p_n$間のデータ転送帯域幅                                             \\\hline
                    $L_m$                                              & $p_m$の通信起動コスト                                                        \\\hline
                    $c_{i,j}^{m,n} = L_m + \frac{data_{i,j}}{b_{m,n}}$ & $p_m$上で実行される$\tau_i$と$p_n$上で実行される$\tau_j$間のデータ通信コスト \\\hline
                    行列 $W$ (サイズ $|V|\times|P|$)                   & タスクのプロセッサの全てのペアの実行時間行列                                 \\\hline
                    $\omega_{j,n} \in W$                               & $p_n$上で実行された場合の$\tau_j$のWCET                                      \\\hline
                \end{tabular}
            }
        \end{table}
    }
\end{frame}

\begin{frame}{表記法・用語 3}
    \full{
        \begin{table}[tb]
            \adjustbox{max width=\textwidth, max height=\slideheight}{
                \centering\begin{tabular}{|c|l|} \hline
                    $initT$            & 現在のCPUクロック時間                                                                    \\\hline
                    $msl$              & 現在の最短スケジュール長                                                                 \\\hline
                    $ops$              & プロセッサ選択肢の上界 (枝切りメカニズム2)                                               \\\hline
                    $\lambda$          & OEFTの許容増加率  (枝切りメカニズム3)                                                    \\\hline
                    $capT$             & Run-timeの上界                                                                           \\\hline
                    $BAST, BAFT, BPRO$ & \tabml{最適なスケジュールを保持する配列                                                  \\スケジュール内の全てのタスクについて, BAST は開始時刻, \\BAFT は終了時刻, BPRO はプロセッサの割り当てを格納}  \\\hline
                    $p'$               & \tabml{タスクに対するプロセッサ選択の優先度を表すプロセッサ ID を含むサイズ OPS の配列 \\OEFT の降順にソートされている} \\\hline
                    $taskListID$       & 現在処理中のタスクのタスクリストにおけるインデックス                                     \\\hline
                    $FSTime$           & HMDSの最初の解を生成するのにかかる時間                                                   \\\hline
                \end{tabular}
            }
        \end{table}
    }
\end{frame}
