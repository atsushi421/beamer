% !TeX root = main.tex

\section{PERFORMANCE EVALUATION}
\label{sec: performance evaluation}


\subsection{Experimental Setup}
\label{ssec: Experimental Setup}

\begin{frame}{ランダム DAG 生成のパラメータ1}
    \begin{block}{$|V|$}
        $|V| = \{10, 20, 30, 40, 50\}$
    \end{block}
    \begin{block}{$\alpha$}
        \setlength{\linewidth}{0.98\columnwidth}
        \begin{itemize}
            \item DAGの高さと幅を定義するパラメータ
            \item $\alpha = \{0.5, 1, 2\}$
        \end{itemize}
    \end{block}
    \begin{block}{$|P|$}
        $|P| = \{4, 8, 16, 32\}$
    \end{block}
\end{frame}

\begin{frame}{使用ベンチマーク}
    \fullimage{benchmark}
\end{frame}


\subsection{Performance Metrics}
\label{ssec: Performance Metrics}

\begin{frame}{Schedule Length Ratio (SLR)}
    \fitimage{
        \begin{itemize}
            \item SLR は最短スケジュール長に対する比率
            \item SLR の値が小さいほど, スケジューリングアルゴリズムの性能が高い
        \end{itemize}
    }{slr}
\end{frame}

\begin{frame}{Run-time}
    一定のパラメータ値で得られたデータセットに対するスケジューリング戦略の平均解生成時間
\end{frame}


\subsection{Experimental Results}
\label{ssec: Experimental Results}

\begin{frame}{メイクスパンの比較}
    \fitimage{
        優れた性能を発揮したテストケース, 同等の性能を発揮したテストケース, 劣った性能を発揮したテストケースの割合
    }{eva1}
\end{frame}

\begin{frame}{パラメータを変動させた場合のSLRの比較}
    \fitimage{
        HMDSは全てのアルゴリズムを総合的に上回っている
    }{eva2}
\end{frame}

\begin{frame}{パラメータを変動させた場合のRun-timeの比較}
    \fitimage{
        Run-timeは $PALG, HEFT < HMDS-Bl < PEFT < PPTS < PSLS$
    }{eva3}
\end{frame}

\begin{frame}{HMDSとHMDS-Blの比較指標}
    \begin{block}{Average Improvement in Schedule length ratios (AIS)}
        \begin{equation*}
            A I S=\frac{S L R_{H M D S-B l}-S L R_{H M D S}}{S L R_{H M D S}} \times 100
        \end{equation*}
    \end{block}
    \begin{block}{Average Slow Down (ASD)}
        \begin{equation*}
            A S D=\frac{\text { run-time }_{H M D S}}{\text { run-time }_{H M D S-B l}}
        \end{equation*}
    \end{block}
\end{frame}

\begin{frame}{$\lambda$を変動させた場合のHMDSとHMDS-Blの比較}
    \fitimage{
        全てのケースでHMDSがHMDS-Blよりも優れた性能を示す
    }{eva4}
\end{frame}

\begin{frame}{許容Run-timeを変動させた場合の結果}
    \fitimage{
        \begin{itemize}
            \item HMDS はより良い解を探索するために時間をかけることで, 着実に性能を向上させている
            \item 当然のことながら, 実行時間に制限がある場合には, タスク数やプロセッサ数の増加に伴って性能が低下する
        \end{itemize}
    }{eva5}
\end{frame}
