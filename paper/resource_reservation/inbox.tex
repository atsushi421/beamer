% !TeX root = main.tex

% (参考)  Hard Real-Time Computing Systems

\section{RESOURCE RESERVATION}
\label{sec: resource reservation}

\begin{frame}{}
\begin{itemize}
\item 資源予約はリアルタイムシステムにおいて, 計算時間が変動するタスクのオーバーランの影響を抑えるための一般的な手法です [MST93,MST94b, AB98,AB04].この手法によれば, 各タスクはタイミング制約を満たすのに十分なプロセッサバンド幅の一部分を割り当てられます.しかし, カーネルはシステム内の他のタスクを保護するために, 各タスクが要求された量以上を消費しないようにしなければなりません (時間的保護)
\item このように, 全プロセッサ帯域幅の何分の一かのUiを受け取ったタスクは, 全速度のUi倍に等しい速度を持つ低速プロセッサ上で単独で実行するように動作する.この方法の利点は, 各タスクが他のタスクの動作とは無関係に, 単独で保証されることである.
\end{itemize}
\end{frame}

\begin{frame}{}
\begin{itemize}
\item 実時間システムにおいて資源予約を実現するための簡単で効果的なメカニズムは, 各タスクnに間隔Piごとに指定された量のCPU時間Qiを予約することである.このような一般的なアプローチは, CPU以外の他の資源にも適用できるが, ここでは, CPUのスケジューリングが主題であるため, 主にCPUに焦点を当てることにする.
\end{itemize}
\end{frame}

\begin{frame}{}
\begin{itemize}
\item いくつかの著者 [RJM098] は, ハード予約とソフト予約を区別する傾向がある.この分類法によれば, ハード予約は予約されたタスクがPi毎に最大でQi時間単位で実行されることを保証し, ソフト予約はタスクがPi毎に少なくともQt時間単位で実行され, アイドル時間があればより多く実行されることを保証している.
\end{itemize}
\end{frame}

\begin{frame}{}
\begin{itemize}
\item 固定優先度スケジューリングのための資源予約技法は, Mercer, Savage, and Tokudaによって最初に提示された[MST94a].この手法によれば, タスクTiはまずペア(Qi, Pi)を割り当てられ (CPU能力予約と表記) , PiごとにQi単位の時間だけリアルタイムタスクとして実行可能になる.タスクが予約された量子Qjを消費すると, 予約がハードであれば次の期間までブロックされ, 予約がソフトであれば非リアルタイムタスクとしてバックグラウンドでスケジューリングされる.タスクが終了していない場合, 次の期間の最初に別の時間量子Qjが割り当てられ, バジェットが切れるまでリアルタイムタスクとしてスケジューリングされる, といった具合です
\end{itemize}
\end{frame}

\begin{frame}{}
\begin{itemize}
\item このようにして, タスクは既知のパラメータ (Qi,Pi) を持つ周期的なリアルタイムタスクのように振る舞い, 古典的なリアルタイムスケジューラで適切にスケジュールできるように再形成される.
\end{itemize}
\end{frame}

\begin{frame}{}
\begin{itemize}
\item このような方法は, 実行時間が変動するタスクが存在する場合に予測可能性を得るために不可欠ですが, システム全体の性能は, 正しいリソース割り当てに大きく依存します.例えば, あるタスクに割り当てられたCPU帯域がその平均要求値よりかなり小さい場合, そのタスクの速度が落ちすぎてシステムの性能が低下する可能性があります.一方, 割り当てられた帯域幅が実際のニーズよりもはるかに大きい場合, システムは低効率で動作し, 利用可能なリソースを浪費することになります.
\end{itemize}
\end{frame}

\begin{frame}{}
\begin{itemize}
\item EDFスケジューリング下で時間的保護を強制する簡単なカーネルメカニズムは, 6章 で説明するConstant Bandwidth Server (CBS) [AB98, AB04]である.しかし, 時間的保護を適切に実装するためには, 計算時間が可変の各タスクnは, バンド幅U.nを持つ専用のCBSによって処理され, USi以上の時間他のタスクに干渉しないようにする必要があります.図 9.16 は, 2 つのタスク (ti と r2) を帯域幅 USl = 0.15 と U "3 = 0.1 の 2 つの専用 CBS で処理し, 2 つのタスク群 (r3, r4) を帯域幅 C/.S3 = 0.25 の単一の CBS で処理, 3 つのハード周期タスク (r5, r6, r7) を実行時間が大きく変動しないのでサーバーを介在せずに EDF で直接スケジュールする例である.
\end{itemize}
\end{frame}

\begin{frame}{}
\fitimage{
    この例では, 図 9.17 に示すように, 全プロセッサの帯域をタスク間で共有します.
}{cbs_example}
\end{frame}

\begin{frame}{}
\begin{itemize}
\item CBSの特性は, Aの場合に限り, ハードな周期タスクの集合 (利用率Up) がEDFでスケジューリング可能であることを保証している.
\end{itemize}
\end{frame}
