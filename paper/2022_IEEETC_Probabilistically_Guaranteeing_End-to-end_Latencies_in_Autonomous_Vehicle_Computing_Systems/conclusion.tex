% !TeX root = main.tex

\section{CONCLUSION}
\label{sec: conclusion}

\begin{frame}{}
    \begin{itemize}
        \item 本論文では, マルチコアシステム上で動作するAVソフトウェアスタックのエンドツーエンドレイテンシを解析する確率的解析を提案した
\item 提案する解析は, 各CPUコアの最大コア利用率が $1.0$ より大きい可能性のあるタスクグラフを扱う点で優れており, タスク実行時間が独立しているという仮定の下, 解析したレイテンシ分布が実際の分布を上回ることが証明された
    \end{itemize}
\end{frame}

\begin{frame}{}
    \begin{itemize}
        \item この解析をAutowareと呼ばれるAVスタックに適用するため, タスク間の実行時間依存の影響を緩和するためにタスクのグループ化を行い, タスクグループのグラフ上で解析を行った
\item 実験により, 我々の解析結果は, 実際のセンサワークロードでカスタマイズしたAutowareから観測されたレイテンシ分布をほぼ上回り, 各タスクパスのレイテンシ分布が得られることが示された
\item 今後は, 共有リソースの影響をさらに軽減し, 任意のタスクグラフに対して最適なリソーススケジュールを合成する方法を検討する予定である
    \end{itemize}
\end{frame}
