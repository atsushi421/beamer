% !TeX root = main.tex

\section{INTRODUCTION}
\label{sec: introduction}

% \begin{frame}{}
%     \begin{itemize}
%         \item ロボット工学のような, 様々な分野の深い専門知識を必要とする学際的で複雑なアプリケーション領域では, 通常, 全ての, あるいはほとんどのソフトウェアをゼロから書くという選択肢はあり得ない
%         \item その代わりに, ロボット工学者は, ROSのような一般的なロボット工学フレームワークで容易に利用できる, 標準機能を提供する既存のサードパーティコンポーネントの統合を採用するのが一般的である
%         \item その利点は数多く, 簡単に理解できる
%         \item 例えば, 複数の最新パス計画アルゴリズムと3D可視化サポートを備えた完全なナビゲーションスタックがたった1回のダウンロードで手に入るなら, なぜ新しいナビゲーションサブシステムを苦労して開発する必要があるか?
%     \end{itemize}
% \end{frame}

% \begin{frame}{}
%     \begin{itemize}
%         \item 完全なロボットシステムを構築するためには, 多くの相互作用するコンポーネントを統合する必要がある
%         \item ROS開発プロセスの分散型オープンソースの性質により, これらのコンポーネントは通常, 必ずしもお互いを知らない複数の独立したコンポーネント開発者によって分離して開発される
%         \item 同様に, システムインテグレータは, アプリケーションおよびミッション固有のロジックと「グルーコード」で展開プラットフォーム上で選択したコンポーネントを構成するが, 通常, それぞれのコンポーネント開発者と密接に連携することはない
%     \end{itemize}
% \end{frame}

% \begin{frame}{}
%     \begin{itemize}
%         \item コンポーネントの統合を可能な限りシンプルに保つために, ROSはコンポーネントの疎結合を可能にする古典的なトピックベースのpublish/subscribeパラダイムを採用している
%         \item 概念的には, 各コンポーネントは, 特定のトピックをsubscribeする多数のコールバックを含む「ブラックボックス」として理解できる
%         \item 与えられたトピックに関連するメッセージがpublishされるたびに, 全てのsubscribeコールバックが呼び出され, 何らかの計算を実行し, 次に他のトピックに後続のメッセージをpublishすることができ, これがさらにコールバックをトリガするというように, 繰り返す
%     \end{itemize}
% \end{frame}

% \begin{frame}{}
%     \begin{itemize}
%         \item インテグレータは, あるコンポーネントの「入力コールバック」を別のコンポーネントの「出力トピック」に接続することによってコンポーネントを構成する
%         \item ROSシステムは, このように相互接続されたトピックとコールバックの複雑なネットワークを形成し, データ (環境刺激など) は, イベント駆動型の方法でネットワークを通じてcause-effectチェーンに沿って伝播し, インテグレータが望むように透過的にコンポーネント境界を交差させることができる
%     \end{itemize}
% \end{frame}

\begin{frame}{cause-effectチェーン}
    \begin{itemize}
        \item 進路上の障害物を検知して反応する必要のある移動ロボットのセンシング-計算-行動パイプラインはcause-effectチェーンと呼ばれる
        \item ハードウェアドライバコンポーネントがレーザースキャナから新しいサンプルを取得し (cause) , それが複数のマッピング, 座標変換, パス計画, 車輪制御コンポーネントを経て, 最終的に車輪速度の変化 (effect) をもたらす
        \item cause-effectチェーンにおけるcauseからeffectまでの最大レイテンシは, ロボットが正しく機能するために重要な役割を果たす
    \end{itemize}
\end{frame}

% \begin{frame}{}
%     \begin{itemize}
%         \item 重要なのは, システムインテグレータにできるだけ多くの展開の選択肢を残し, コンポーネントの再利用の機会を最大化するために, ROSの実行管理層と基礎となるオペレーティングシステムは, 意図的にコンポーネント開発者に公開されないことである
%         \item むしろ, ROSの中心的なコールバック抽象化は, コールバック手続きがいつどのようにスケジュールされるか, コールバックの実行がスレッドまたはプロセスにわたってどのように組織されるか, またはネットワーキング層がメッセージの送受信をどのように処理するかを全く意識せずに, 実行から完了までセマンティクスを持つ単なる手続きである
%     \end{itemize}
% \end{frame}


% \begin{frame}{}
%     \begin{itemize}
%         \item ROSはオープンソースソフトウェアであるため, 原理的にはシステムの実行と通信の挙動を完全に理解し制御することが可能である
%         \item このため, リアルタイムシステムの専門家から見れば, ROSにリアルタイムシステム研究でよく知られた技術を導入することは論理的なステップであるように思える可能性がある
%         \item しかし, 一見したところ, これを難しくしているハードルがいくつかある
%     \end{itemize}
% \end{frame}

\begin{frame}{ROSアプリケーションにリアルタイム分析技術を導入する際の課題}
    \begin{block}{課題1}
        インテグレータは低レベルシステムの詳細な情報を持たない
    \end{block}
    \begin{block}{課題2}
        システムの詳細な情報は静的に決定されない
    \end{block}
    \begin{block}{課題3}
        最悪のケースを想定した静的リソース配置は現実的ではない
    \end{block}
\end{frame}

\begin{frame}{インテグレータは低レベルシステムの詳細な情報を持たない}
    \begin{itemize}
        \item ほとんどのリアルタイム分析では, 同時実行タスクの数, それらの起動セマンティクスや機能的相互作用, メッセージの到着パターン, 最悪実行時間など, 多くの低レベルシステムの詳細に関する深い知識が前提となっている
        \item ROSコンポーネントは, この種の情報を提供するマニフェストと一緒に来ることはない
              % \item さらに悪いことに, リアルタイム分析は, 欠陥のある情報や不完全な情報にうまく対処できない
              % \item モデリング目的でサードパーティコンポーネントを手作業でリバースエンジニアリングしているときに, たった一つのミスや見落としがあれば, その取り組み全体を無効にしてしまう
    \end{itemize}
\end{frame}

\begin{frame}{システムの詳細な情報は静的に決定されない}
    多くのロボット工学アルゴリズムが, ユースケースやプラットフォーム固有の側面に依存し, 実行時間や起動パターンが大きく変化する
    \begin{block}{例}
        \setlength{\linewidth}{0.98\columnwidth}
        \begin{itemize}
            \item ビデオストリーム中の物体を識別し, その軌跡を推測する一般的な物体追跡コンポーネントを考える
            \item この機能の実行時間は, ビデオストリームのフレームレート, 解像度, コーデックなど様々なパラメータに依存する
            \item これらのパラメータは, 一般的なオブジェクトトラッキングコンポーネントの開発者が前もって知っていたり, 決まっていたりしない
        \end{itemize}
    \end{block}
\end{frame}

% \begin{frame}{}
%     \begin{itemize}
%         \item このようなユースケース特有の情報は, 特定のロボットを構築するインテグレーターにしか分からない
%         \item インテグレーターは, 必ずしもオブジェクトトラッキングやリアルタイムシステムの専門家ではないため, 特定の構成を選択した場合の影響を常に予測できるわけではない
%         \item したがって, コンポーネントのリソース要求とリアルタイム動作は, 常に特定の展開で使用するという文脈で評価されなければならない
%         \item これは, ROSフレームワークの人気の根底にある「ブラックボックス」コンポーネントのモジュール式再利用と相容れるものではない
%     \end{itemize}
% \end{frame}

\begin{frame}{最悪のケースを想定した静的リソース配置は現実的ではない}
    \begin{block}{例}
        \setlength{\linewidth}{0.98\columnwidth}
        \begin{itemize}
            \item 物体追跡コンポーネントと, 自己位置推定コンポーネントに依存するロボットを考える
            \item 人口が少ない田舎町よりも, にぎやかな街中を移動する方が, 物体追跡装置の処理時間はずっと長くなる
            \item 一方, 認識可能なランドマークが多い都市部では, ほぼ一様な風景よりも自己位置推定がはるかに容易
            \item どちらの状況でも十分なリソースを確保するためには, 不毛の土地からなる賑やかな都市を想定したシステムを用意しなければならない
        \end{itemize}
    \end{block}
\end{frame}

% \begin{frame}{}
%     \begin{itemize}
%         \item ロボット工学では, このような悲観的なシステム設計を行うと, すぐに現実的な限界に直面することになる
%         \item その代わりに, 実用的で費用対効果の高いシステムを維持するためには, 各コンポーネントのピーク需要の合計ではなく, 予想されるジョイントリソースのピーク需要に対してプロビジョニングを行う必要がある
%     \end{itemize}
% \end{frame}

\begin{frame}{貢献}
    \begin{itemize}
        \item これらの課題を克服するために, 実行時に動的にタイミングを考慮した方法でROSシステムをプロビジョニングするための ROS Live latency manager (ROS-Llama) を提案する
        \item ROS-Llama は重要なcause-effectチェーンのレイテンシを, 既存のリアルタイム機構を使用して制御することを可能にする
        \item ROS-Llamaは, 実行時に必要なパラメータを自動的に推定し, 状況の変化に応じてスケジューリングパラメータを動的に調整する
        \item 指定されたレイテンシ目標が全て同時に達成できない場合, ROS-Llamaは制御された緩やかなデグレードプロセスを開始し, システムインテグレーターが純粋に宣言的な方法で cause-effectチェーンの重要性を指定できるようにする
    \end{itemize}
\end{frame}

% \begin{frame}{本論文の貢献}
%     \begin{itemize}
%         \item  ロボティクス領域における動的レイテンシ管理問題を探求し, 実用的なソリューションが満たさなければならない制約と要件を文書化する (第III章)

%         \item  ROSのための最初の自動レイテンシマネージャであるROS-Llamaの設計と実装を紹介する (セクションIV)

%         \item 標準的な Linux システム上の ROS コンポーネントを用いて, ROS-Llama が移動ロボットのcause-effectチェーンのレイテンシをうまく制御できることを示す評価について報告する (セクションVI)

%     \end{itemize}
% \end{frame}

% \begin{frame}{}
%     \begin{itemize}
%         \item ROS-Llamaは, 数年にわたる研究とエンジニアリングの努力の結果であり, その間, 我々は多くの課題や技術的な限界に遭遇した
%         \item セクションVIIでは, 以下の点を強調する
%               \begin{itemize}
%                   \item  ROS-Llamaをより効果的かつ正確にするための分析改善の機会
%                   \item  ROSとLinuxのプラットフォームには, システムのさらなる改良の妨げとなる大きな限界がある
%               \end{itemize}
%     \end{itemize}
% \end{frame}
