% !TeX root = main.tex

\section{OPEN PROBLEMS AND PRAGMATIC WORKAROUNDS}
\label{sec: open problems and pragmatic workarounds}

\begin{frame}{}
    \begin{itemize}
        \item ROS-Llamaを開発する中で, 本論文は, 今後の研究のための多くの機会や, 当初は明らかではなかったROSやLinuxの制限に気づくことができた
    \end{itemize}
\end{frame}


\subsection{Open Analysis Problems}
\label{ssec: open analysis problems}

\begin{frame}{}
    \begin{itemize}
        \item ROS-Llamaは, そのプラットフォームである分析プラットフォームをいくつかの方向から改善することで恩恵を受けることができる
    \end{itemize}
\end{frame}


\subsubsection{Complex activations}
\label{sssec: complex activations}

\begin{frame}{}
    \begin{itemize}
        \item メッセージの公開時に無条件にアクティブになるレギュラーコールバックに加えて, ROSはメッセージフィルタの形で高度なアクティベーションセマンティクスも提供する
        \item 特に興味深いのは, ナビゲーションスタックで使用されるTF関連のメッセージフィルターである
        \item 概念的には, フィルターは, 複雑なルールに基づいて共同でダウンストリーム処理を行うために,  (別々のメッセージから) 複数の受信データ項目を選択し, 単一のメッセージにまとめる
        \item 表現力の面では, 古典的な「と」起動のセマンティクスをはるかに超え, 実際, 結合されるメッセージの内容にも依存することがある
    \end{itemize}
\end{frame}

\begin{frame}{}
    \begin{itemize}
        \item 例えば, ナビゲーションスタックでは, ローカリゼーションコンポーネントがメッセージフィルターを使用して, レーザースキャナーコールバックが, 最新のレーザースキャンサンプルより古くないオドメトリ測定値が利用できる場合にのみトリガされることを保証する
        \item このような複雑なアクティベーションセマンティクスは, 現在のモデリングアプローチでは表現できない
        \item 我々のケーススタディでは, メッセージフィルタの前後の部分を別々のチェーン (すなわち, レーザースキャナ, オドメトリロク, ローカリゼーションチェーン) として宣言し, メッセージフィルタの前後のチェーン間でレイテンシ目標を手動で分散させることによってこの問題を回避している
        \item より表現力の高いモデルを用いれば, このような手動による設定調整は不要となり, ROS-Llamaの使いやすさと効率性の双方に貢献できる
    \end{itemize}
\end{frame}


\subsubsection{In want of stochasticity}
\label{sssec: in want of stochasticity}

\begin{frame}{}
    \begin{itemize}
        \item 第V章で述べたように, 有用な分析精度を得るためには, スカラーWCETモデルではなく, 実行時間曲線の使用が不可欠である
        \item それにも関わらず, 我々の実験では, 観測されたレイテンシと予測された境界の間に大きなギャップがあることに気づくことがあった
    \end{itemize}
\end{frame}

\begin{frame}{}
    \begin{itemize}
        \item 理由は明白で, 実行時間曲線は, 悲観的ではないものの, 依然として最悪の要求のモデルであり, 実際には, ROSシステムは一貫して最悪の動作を示すことは極めてまれである
        \item 概念的には, 確率的タイミング分析は, 事実上常にコモディティプラットフォームに展開され, 静的WCET分析に従わないROSシステムにはるかによくマッチする可能性がある
    \end{itemize}
\end{frame}

\begin{frame}{}
    \begin{itemize}
        \item 例えば, 最近の注目の的である確率論的WCETの概念に類似した, 追跡された「確率論的最悪実行時間曲線」 $p W C E T(n)$ の概念に根ざした仮想的な分析で, より厳しい最悪応答時間といくつかの分析の確実性を喜んで交換する[24-30]
        \item ROSシステムとその動的環境における固有の不確実性を考慮すると, この方向性には大きな期待が持てると考えている
    \end{itemize}
\end{frame}


\subsubsection{DDS}
\label{sssec: dds}

\begin{frame}{}
    \begin{itemize}
        \item DDSミドルウェアは, 全ての通信がそれを通過する必要があるため, ROSコンポーネント間の伝送レイテンシに大きく影響する
        \item このレイテンシを調整するために, DDS は多数の QoS オプションを提供している
        \item ある実装のドキュメント [31] には, 特に, 作成されるスレッドの数, 複数の DDS サポートスレッドのスケジューリング優先度, またはメッセージ送信順序に影響する 53 のパラメータだけがリストアップされている
        \item 実装に依存しないQoSオプションの分析に関する先行研究がある一方で [32, 33], DDSスレッドのスケジューリングは, DDS標準によって意図的に「実装定義」として残され, 今日までほとんど理解されていないままとなっている
        \item 本論文の知る限り, スケジューリングの決定がDDSのレイテンシに与える影響を予測する原理的な方法は, 今のところない
    \end{itemize}
\end{frame}

\begin{frame}{}
    \begin{itemize}
        \item そこで, 本論文では, DDSインフラストラクチャをブラックボックスとして扱い, 全てのDDSスレッドを高いスケジューリング優先度で別のシステムコアに分離することで, そのレイテンシの影響を低減した
        \item さらに, 応答時間の分析では, DDSメッセージの伝搬レイテンシを無視した (無視できないほどのレイテンシであることは間違いないのであるが)
        \item 今後, スレッドスケジューリングがDDSのレイテンシに与える影響を自明にすることで, ROS-Llamaが自動的に適切なDDS QoSオプションを選択し, 最悪伝搬レイテンシを予測し考慮することが可能になるかもしれない
    \end{itemize}
\end{frame}


\subsection{Linux Platform Limitations}
\label{ssec: linux platform limitations}

\begin{frame}{}
    \begin{itemize}
        \item ROS-Llamaは, Linuxのリアルタイム機能, 特にSCHED\_DEADLINE [8]から大きな恩恵を受けている
        \item それにも関わらず, ある種の問題は意外なほど困難でもありました
    \end{itemize}
\end{frame}


\subsection{High-latency I/O}
\label{ssec: high-latency i/o}

\begin{frame}{}
    \begin{itemize}
        \item 本論文の実験では, レーザースキャナーとオドメトリのデータが過度のレイテンシを経てからTurtlebotのドライバースレッドに到着することがありました
        \item これは, USB シリアルポートに到着したデータが TTY レイヤを通過するためで, CFS スケジューリングされたカーネルスレッド (PREEMPT-RT カーネルであっても) はリアルタイムプロセスによって簡単に停止させられてしまうことがわかり ました
        \item 最終的には, Linuxの "unbound kworker threads "をシステムプロセッサに強制することでこの問題を回避することができたが, これは, リアルタイムI/Oが依然として頻繁に見落とされ, 研究されていない問題であることを思い出させるものとなっている
    \end{itemize}
\end{frame}


\subsection{Scheduler inversion}
\label{ssec: scheduler inversion}

\begin{frame}{}
    \begin{itemize}
        \item SCHED\_DEADLINEスレッドは常にSCHED-FIFOおよびSCHED\_RRスレッドより優先される
        \item これは分析を単純化するが, 多くの実用的な問題を引き起こする
        \item 割り当てられた予約帯域幅によってこのレイテンシはいくらか制限されるが, それでもかなりのレイテンシ (すなわち, EDFの最大ビジーウィンドウ長) になる可能性がある
        \item 多くのシステムクリティカルなカーネルスレッド (ディスクドライバなど) は, SCHED-FIFOまたはSCHED\_RRの優先度でスケジュールされているので, この設計は特に残念なことである
    \end{itemize}
\end{frame}

\begin{frame}{}
    \begin{itemize}
        \item 予約に「過剰な」帯域幅を割り当てると, 重要なカーネルスレッドが飢餓状態になり, 実際にカーネルパニックに陥る可能性がある
        \item 原理的な解決策は, SCHED\_DEADLINEスケジュールにおいて, SCHED-FIFO, SCHED\_RR, およびCFS用のプロセッサ時間を明示的に予約する「スケジューリングクラスリザベーション」を導入することかもしれない
    \end{itemize}
\end{frame}



\subsection{Threads vsreservations}
\label{ssec: threads vs reservations}

\begin{frame}{}
    \begin{itemize}
        \item SCHED\_DEADLINEは, 予約パラメータを個々のスレッドに結びつけます
        \item そのため, 複数のスレッド間でバジェットを共有することができず, スレッド間で動的に作業を分配するマルチスレッドアプリケーション (例えば, 事実上全てのDDSミドルウェア) に予約を適用することは非常に非効率的である
        \item boost::asioや $\mathrm{C}++$ std::async APIなどの非同期プログラミング用の一般的なライブラリも提供することができない
        \item 複数のクライアントのスレッドをサポートするファーストクラスの予約は, 救済される
    \end{itemize}
\end{frame}
