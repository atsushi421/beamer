% !TeX root = main.tex

\section{PRACTICAL CONSIDERATIONS}
\label{sec: practical considerations}

\begin{frame}{セクションサマリ}
    \begin{itembox}[l]{\textbf{目的}}
        ROS-Llamaを実装する際に遭遇した予期せぬ課題を説明し, その対処法を示す
    \end{itembox}
\end{frame}


\subsubsection{Activation cycles}
\label{sssec: activation cycles}

\begin{frame}{サイクル発生ケース}
    \begin{itemize}
        \item 自己位置特定コンポーネントAMCLにサイクルが発生した
        \item AMCLコンポーネントは, ロボットの推定位置をpublishするだけでなく, その推定位置を生成するためにオドメトリ更新を消費するため, /tfトピックへのsubscribeとpublishの両方を行う
        \item AMCLの位置更新は, オドメトリ座標フレームを更新するメッセージによってのみトリガされ, AMCL自身がpublishする位置更新ではトリガされない
    \end{itemize}
\end{frame}

\begin{frame}{対処法}
    \begin{itemize}
        \item /tfトピックのsubscriberセットから, サイクルを引き起こすコンポーネントを取り除く
        \item このモデリング調整により, Casiniらの応答時間分析[6]が適用可能になる
        \item より複雑なサイクルを実現するには, より洗練されたモデルの調整, または手動での介入が必要となる
    \end{itemize}
\end{frame}


\subsubsection{Scalar WCETs are pessimistic}
\label{sssec: scalar wcets are pessimistic}

\begin{frame}{スカラーWCETの悲観性}
    \begin{columns}
        \begin{column}{0.5\textwidth}
            \begin{itemize}
                \item 図は, AMCLノードの/tfコールバックの実行時間を観測したもの
                \item 1回の起動にかかる最大コストはおよそ $56 \mathrm{~ms}$だが, そのケースは非常にまれ
                \item コールバックの全てのインスタンスが $56 \mathrm{~ms}$ を必要とするという仮定は, 恐ろしく悲観的
            \end{itemize}
        \end{column}
        \begin{column}{0.5\textwidth}
            \fullimage{tf_observe}
        \end{column}
    \end{columns}
\end{frame}

\begin{frame}{対処法}
    \begin{itemize}
        \item このようなコールバックの実行時間要件をより正確に記述するために, ROS-Llamaは累積実行時間曲線 [17] を使用する
        \item Casiniらの応答時間解析[6]を修正し, $r b f_{c}(\Delta)=E T^{+}\left(\eta_{c}(\Delta)\right)$ をコールバックのRBFとして使用するようにした
    \end{itemize}
    \begin{block}{累積実行時間曲線 $E T^{+}(n)$}
        実行時間曲線 $E T^{+}(n)$ は, 任意の $n$ 連続起動の最大累積実行時間の上界を示す
        \notes{スカラーWCETは $E T^{+}(1)$ と等価}
    \end{block}
\end{frame}

\begin{frame}{スカラーWCETと累積実行時間曲線の比較}
    \begin{columns}
        \begin{column}{0.5\textwidth}
            \begin{itemize}
                \item $E T^{+}(n)$ 曲線は, 1回の起動に $56 \mathrm{~ms}$ までかかる可能性があることを正しく表しているが, 40回連続実行の累積実行時間が $E T^{+}(40)=$  $122 \mathrm{~ms}$ を超えないことも示している
                \item 一方, スカラーWCETモデルでは $40 \times E T^{+}(1)=2240 \mathrm{~ms}$ のコストが予測される
            \end{itemize}
        \end{column}
        \begin{column}{0.5\textwidth}
            \fullimage{et}
        \end{column}
    \end{columns}
\end{frame}


\subsubsection{Rare events}
\label{sssec: rare events}

\begin{frame}{稀なイベントの到着曲線の悲観性}
    到着曲線は, 稀な非周期的事象に対して非常に不正確な場合がある
    \begin{block}{例}
    \setlength{\linewidth}{0.98\columnwidth}
    \begin{itemize}
        \item システム起動数分後に2回連続して入力されるインタラクティブな操作を考える
        \item 到着曲線推定器では, 観測された2つのイベントのみを考慮し, 短い周期を持つ周期的到着パターンに従っていると悲観的に誤判定してしまう
    \end{itemize}
    \end{block}
\end{frame}

\begin{frame}{対処法}
    \begin{itemize}
        \item システム起動以来2回しか観測されていないことを組み込むために, システム起動をコールバックの仮想起動として扱う
        \item これにより, 上記の例では, 2つの事象が連続して発生することはあっても, 3つ以上の事象を観測するには, はるかに時間がかかるとモデル抽出器が判断できる
    \end{itemize}
\end{frame}
