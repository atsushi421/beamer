% !TeX root = main.tex

\section{DESIGN OF ROS-Llama}
\label{sec: design of ros-llama}

\begin{frame}{ROS-Llamaでユーザが設定する項目}
    \begin{block}{レイテンシ管理が必要なチェーンのレイテンシ目標}
        例えば, ユーザは, レーザースキャナーコールバックに新しい測定値が到着してから, 検出された障害物をマップに登録するコールバックが完了するまで, 最大で $200 \mathrm{~ms}$ 経過すると設定できる
    \end{block}
    \begin{block}{過負荷時に参照されるチェーンのデグレード順位}
        \setlength{\linewidth}{0.98\columnwidth}
        \begin{itemize}
            \item 全てのレイテンシ目標を同時に保証できないと判断した場合, いくつかのチェーンをベストエフォートモードにデグレードする
            \item これにより, 過負荷時の動作を予測可能にする
        \end{itemize}
    \end{block}
\end{frame}

\begin{frame}{ROS-Llamaが自動解決する問題}
    \begin{enumerate}
        \item[(1)] 実行中のROSシステムのモデル (全てのトピック, コールバック, エグゼキュータ, それらのリソース要求など) を抽出
        \item[(2)] 設定されたレイテンシ目標を可能な限り満たすように全てのスレッドをプロビジョニングし, いずれかのチェーンをベストエフォートモードに落とす必要があるかを決定
        \item[(3)] 全てのスレッドを (2) でプロビジョニングした方法に従ってスケジュール
    \end{enumerate}
\end{frame}

\begin{frame}{ROS-Llama全体象}
    \fitimage{
        ROS-Llamaは (1) に対応するモデル抽出器と (2) に対応するバジェットマネージャからなり,  (3) にはLinuxのスケジューラSCHED\_DEADLINEを使用
    }{overview}
\end{frame}

\begin{frame}{要求の変化への対応方法}
    \fitimage{
        \begin{itemize}
            \item モデル抽出器は実行時にモデルを継続的に更新する
            \item バジェットマネージャは定期的にモデルを取得し, 新しいスケジューリングパラメータを計算して, SCHED\_DEADLINEで実行する
        \end{itemize}
    }{overview}
\end{frame}


\subsection{The Model Extractor}
\label{ssec: the model extractor}

\begin{frame}{モデル抽出器の役割}
    \begin{block}{目的}
        システム内の全ての関連スレッドを識別し, コールバックグラフ構造を構築し, 到着時間と実行時間を測定する
    \end{block}
    \begin{block}{方法}
        \setlength{\linewidth}{0.98\columnwidth}
        \begin{itemize}
            \item モデル抽出器は, ROSコアライブラリ $\mathrm{rclcpp}$ および $\mathrm{rcl}$ を分析する
            \item これらのライブラリは, 全ての (C++ベースの) ROSノードで使用されているため, サードパーティのROSコンポーネントに変更を加えることなく分析可能
        \end{itemize}
    \end{block}
\end{frame}

\begin{frame}{分析に使用するトレースポイント}
    \vspace{1mm}
    \begin{block}{コールバックに関するトレースポイント}
        \setlength{\linewidth}{0.98\columnwidth}
        \begin{itemize}
            \item コールバック登録
            \item コールバック起動
            \item publish
            \item コールバック完了時
        \end{itemize}
    \end{block}
    \begin{block}{その他のトレースポイント}
        \setlength{\linewidth}{0.98\columnwidth}
        \begin{itemize}
            \item スレッドがコールバック処理ループを開始してエグゼキュータになる時
            \item サービスなど特定のAPIの使用を監視する
        \end{itemize}
    \end{block}
\end{frame}

\begin{frame}{コールバックグラフ構築方法}
    モデル抽出器はコールバックがpublishされるたびに, コールバックとトピックの間にエッジを追加する (トピックがまだ存在しない場合)
    % \begin{block}{静的プロパティ}
    %     \setlength{\linewidth}{0.98\columnwidth}
    %     \begin{itemize}
    %         \footnotesize
    %         \item 新しいコールバックが登録されると, 対応するトレースイベントは, タイプ・関連するトピック・識別子を報告する
    %         \item これらはコールバックのライフタイム中に変更されない
    %     \end{itemize}
    % \end{block}
    % \begin{block}{動的プロパティ}
    %     \setlength{\linewidth}{0.98\columnwidth}
    %     \begin{itemize}
    %         \footnotesize
    %         \item 動的プロパティは, コールバックの実行開始, 完了, またはpublish時に対応するトレースイベントから派生し, 継続的に更新される
    %         \item 各トレースイベントは, 識別子, publish の場合はpublish先のトピックを含む
    %     \end{itemize}
    % \end{block}
\end{frame}

\begin{frame}{到着/実行時間曲線の更新}
    モデル抽出器は以下のタイミングで到着/実行時間曲線を更新する
    \begin{itemize}
        \item コールバックが完了するたびに, コールバックの実行時間曲線を更新する
        \item publishイベントと開始イベントの発生に基づいて到着曲線を更新する
    \end{itemize}
\end{frame}

% \begin{frame}{コールバックグラフ構築方法1}
%     \begin{itemize}
%         \item モデル抽出器は, トレースイベントに基づいて, コールバックアクティベーショングラフを構築する
%         \item 各トレースイベントは, そのタイプ, 発信元スレッド, タイムスタンプ, およびイベント固有の追加データから構成される
%         \item 2 つのタイムスタンプは, ウォールタイマの時間を示すシステム全体のモノトニッククロックと, スレッドが受け取ったプロセッササービスの量を追跡するスレッドごとの $C P U$ タイムクロックという異なるクロックに従って時間を測定する
%         \item 抽出器は, モノトニックタイムスタンプを使用して到着時刻を推測し, CPU時間クロックを使用して実行時間を測定する
%     \end{itemize}
% \end{frame}

% \begin{frame}{イベントソースの管理}
%     \begin{itemize}
%         \item モデル抽出器は全てのイベントソース, すなわち, ROSシステムと相互作用するがエグゼキュータ自身ではないスレッドも管理する
%         \item このようなスレッドは, コールバックを開始することなくpublishされるため, 容易に識別できる
%         \item ROS-Llamaは, データ取得から始まるチェーンのレイテンシを制御するために, このようなスレッドを管理する
%     \end{itemize}
% \end{frame}

% \begin{frame}{サービスの対応方法}
%     \begin{itemize}
%         \item コールバックに加え, ROS APIはコールリターンセマンティクスを実現するサービスの概念も提供していることは言及に値する:コールバックは一見ブロックする方法でサービスを呼び出し, 応答で応答を受け取ることができる
%         \item しかし, その裏側では, サービスは継続パッシングアプローチを用いたレギュラーコールバックとして実装されている (クライアントは特別なリクエストトピックにメッセージを投稿することでサービスを呼び出し, どのトピックで応答を受け取りたいかを指定する)
%         \item したがって, モデル抽出器は, サービスを検出し, 追跡し, ROS-Llamaの残りの部分に対して透過的な方法でそれらを表現できる
%     \end{itemize}
% \end{frame}

\begin{frame}{モデルの提供}
    推論された最新のコールバックグラフとタイミングモデルは, 定期的にバジェットマネージャに提供され, バジェットマネージャはそれを使って最新のプロビジョニングを決定する
\end{frame}


\subsection{Predictable Thread Scheduling}
\label{ssec: predictable thread scheduling}

\begin{frame}{セクションサマリ}
    \begin{itembox}[l]{\textbf{目的}}
        プロビジョニングされたスレッドがROS-Llamaによってどのようにスケジューリングされるかを説明する
    \end{itembox}
\end{frame}

\begin{frame}{使用するスケジューラ}
    ROS-LlamaはSCHED\_DEADLINEスケジューラを使用する
    \begin{block}{設計理由}
        \setlength{\linewidth}{0.98\columnwidth}
        \begin{itemize}
            \item ROS処理チェーンに対する\ourl{Casiniらの応答時間分析}{https://drive.google.com/file/d/1sHujFqbmgCoJbC6g6KdC7ihua4Jqddju/view?usp=share_link}が, リソース予約でスケジュールされたエグゼキュータスレッドに直接適用できる
            \item CBSの機能により, プロセッサ要求が予想外に急増したスレッドが, 他のスレッドのバジェット受け取りを妨げない
        \end{itemize}
    \end{block}
\end{frame}

\begin{frame}{使用するスケジューリング手法}
    ROS-Llamaはパーティショニングスケジューリングを使用する
    \begin{block}{設計理由}
        \setlength{\linewidth}{0.98\columnwidth}
        \begin{itemize}
            \item パーティショニングスケジューリングは, 経験的にほとんどのワークロードでより高いスケジューラビリティを達成することが示されている [11]
            \item ROS-Llamaは, 適切なマッピングを自動的に決定するのに十分な情報を持っているので, システムインテグレータに追加の負担がない
        \end{itemize}
    \end{block}
\end{frame}

\begin{frame}{スレッドの分割例}
    \begin{itemize}
        \item ROS-Llamaは自分自身と, 様々なカーネルやDDSミドルウェアのスレッドのような雑多なシステムインフラを, 予約したシステムコアに分離することも可能
        \item 残りのコアは, ROSスレッドのプロビジョニングのためにバジェットマネージャが使用できる
    \end{itemize}
\end{frame}

\begin{frame}{ベストエフォートモードのチェーン処理方法}
    ベストエフォートモードでチェーンを処理するエグゼキュータは, ROS-Llamaによってプロビジョニングせず, デフォルトのCFSによってスケジューリングする
\end{frame}


\subsection{The Budgeting Algorithm}
\label{ssec: the budgeting algorithm}

\begin{frame}{セクションサマリ}
    \begin{itembox}[l]{\textbf{目的}}
        抽出されたモデルに基づいて, バジェットマネージャは各予約のバジェットと周期, および予約とコアの実現可能なマッピングを見つけ, 構成されたチェーンを時間内に完了させる
    \end{itembox}
\end{frame}

\begin{frame}{予約の周期の決定方法}
    ROS-Llamaは全ての予約に, 最も厳しいレイテンシ目標よりも短いかつ, 過度のコンテキストスイッチングオーバヘッドを避ける, 均一な周期を割り振る
    \begin{block}{設計理由}
        エグゼキュータは異なる周期を持つ複数のチェーンに影響を与えるため, タスクの周期性が不明確
    \end{block}
    % 本論文のケーススタディでは, 一様な予約期間は $10 \mathrm{~ms}$ でした
\end{frame}

\begin{frame}{予約のバジェットの決定方法}
    ROS-Llamaは, 各チェーンに対して反復的なヒューリスティック駆動探索により予約のバジェットを決定する
    \begin{block}{設計理由}
        コールバックグラフの相互接続性やバジェットの相互依存性がある複雑な最適化問題を解くことは, 実行時にコストがかかり過ぎる
    \end{block}
\end{frame}


\subsubsection{Finding a Starting Point}
\label{sssec: finding a starting point}

\begin{frame}{初期バジェット割り当てアルゴリズム全体像}
    \fitimage{
        各エグゼキュータの長期的な最大プロセッサ要求を満たす初期バジェット割り当てを見つける
    }{alg1}
\end{frame}

% \begin{frame}{}
%     \begin{itemize}
%         \item アルゴリズム1は, 各エグゼキュータの長期的な最大プロセッサ要求を反映する初期バジェット割り当てを見つけるために使用される
%         \item 後のステップでは帯域幅を追加するだけで, 削除することはないので, 有限の最悪応答時間をまだ保証する最小の推定値から始めることが望ましい
%     \end{itemize}
% \end{frame}

\begin{frame}{初期バジェット設定}
    \fullimage{alg1_sup1}
\end{frame}

\begin{frame}{初期バジェット設定後}
    \begin{itemize}
        \item 初期バジェットを割り当てたので, 他のエグゼキュータが $100 \%$ の帯域幅を持つという仮定を取り除く
        \item その結果, 一部のコールバックは, ジッタ効果の増加により, スケジューリング不能になる可能性がある
        \item アルゴリズム4行目からはこれに対処する
    \end{itemize}
\end{frame}

\begin{frame}{エグゼキュータの帯域幅の調整}
    \fitimage{
        horizonにおけるプロセッサ要求が満たされるか, エグゼキュータの帯域幅が $100 \%$ を超えるまで, 対応エグゼキュータの帯域幅を繰り返し増加させる
    }{alg2_sup2}
\end{frame}

\begin{frame}{予約のプロセッサへの割り当て}
    \fitimage{
        \begin{itemize}
            \item ワーストフィット, ファーストフィットの順で予約の割り当てを試行する
                  \begin{block}{ワーストフィット}
                      最も利用率が低いプロセッサに割り当てる
                  \end{block}
                  \begin{block}{ファーストフィット}
                      利用率が1を超えない最初に見つかったプロセッサに割り当てる
                  \end{block}
                  \vspace{5mm}
            \item それでも割り当てられるプロセッサが無い場合, チェーンをデグレードする
        \end{itemize}
    }{alg1_3}
\end{frame}


\subsubsection{Refining the Budget}
\label{sssec: refining the budget}

\begin{frame}{バジェットの精緻化}
    \fitimage{
        処理チェーンの最悪応答時間がチェーンのレイテンシ目標を満たすまで, エグゼキュータのバジェットを追加する
    }{alg2}
\end{frame}

\begin{frame}{バジェット追加の方針}
    \fitimage{
        \begin{itemize}
            \item バジェット追加は, バジェット不足レイテンシ $d(e)$ に依存する
            \item バジェット不足レイテンシが大きいということは, このエグゼキュータのバジェットを増やすとシステムの応答時間に正の効果がある可能性が高い
            \item \desc{$R T(c)$}{現在の最悪応答時間}
            \item \desc{$R T^{100 \%}(c)$}{100\%の帯域幅を仮定して得られる最悪応答時間}
        \end{itemize}
    }{d_e}
\end{frame}

\begin{frame}{バジェット追加アルゴリズム}
    \fullimage{alg2_sup1}
\end{frame}
