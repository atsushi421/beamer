% !TeX root = main.tex

% \begin{frame}{提案の概要}
%     \begin{itemize}
%         \item ロボットシステムは, 一般的にリアルタイムの制約を受けることがある
%         \item しかし, オープンソースのロボットソフトウェアで最も人気のあるROSエコシステムでは, 最悪の応答時間を制限したり制御するためにリアルタイム理論を使用した証明がほとんどない
%         \item リアルタイムスケジューリングメカニズムを正しく使用するために必要な専門知識の量と, 静的プロビジョニングを拒む典型的なロボティクスワークロードの固有の予測不可能性が, 採用へのハードルとなっている
%     \end{itemize}
% \end{frame}

\begin{frame}{提案の概要}
    \begin{itemize}
        \item ROS 2用自動レイテンシマネージャである ROS Live latency manager (ROS-Llama) を提案する
        \item ROS-Llamaはわずかな努力とリアルタイムコンセプトの知識だけで使用できる
        \item ROS 2の関連する特性とロボット工学領域の必須要件が特定され, このような自動ツールを開発する際の概念的および実際的な課題を議論する
        % \item モバイルロボット上の実験は, アプローチの実行可能性を実証し, ROS-LlamaがデフォルトのLinuxスケジューラと比較して, 負荷の下で観測された最大レイテンシを短縮することを示す
        % \item 最後に, 基礎となるリアルタイム分析における未解決の問題と, さらなる改善を妨げるLinuxとROS 2の主要なプラットフォームの制限を特定する
    \end{itemize}
\end{frame}
