% !TeX root = main.tex

\begin{frame}{提案の概要}
    \begin{itemize}
        \item  ロボット システムは通常, リアルタイムの制約を受ける.それでも, オープンソースのロボティクス ソフトウェアの最も人気のあるリポジトリである ROS エコシステムは, 最悪の場合の応答時間を制限または制御するためにリアルタイム理論を使用したという証明をほとんど示していない.導入のハードルは, リアルタイム スケジューリング メカニズムを正しく使用するために必要な専門知識の量と, 静的なプロビジョニングを無視する一般的なロボット工学のワークロードに固有の予測不可能性である.これらのハードルを克服するために, ROS 2 の自動レイテンシ マネージャーである ROS-Llama が提案されている.重要なことに, ROS-Llama の使用には, リアルタイムの概念に関するわずかな労力と知識しか必要ない. ROS 2 の関連するプロパティとロボティクス ドメインの必須要件が特定され, そのようなほぼ自動化されたツールを開発する際の概念的および実際的な課題が議論される.モバイル ロボットでの実験は, このアプローチの実行可能性を実証し, ROS-Llama がデフォルトの Linux スケジューラと比較して, 負荷がかかった状態で観察された最大レイテンシを削減することを示している.最後に, 基礎となるリアルタイム分析の未解決の問題と, Linux および ROS 2 の主要なプラットフォームの制限により, さらなる改善が妨げられていることが特定されている.
    \end{itemize}
\end{frame}
