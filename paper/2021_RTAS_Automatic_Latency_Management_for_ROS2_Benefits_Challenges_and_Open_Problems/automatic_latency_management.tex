% !TeX root = main.tex

\section{AUTOMATIC LATENCY MANAgEMENT}
\label{sec: automatic latency management}

\begin{frame}{}
    \begin{itemize}
        \item セクションで主張したように
\item ロボティクスの領域には, これまでリアルタイムシステムコミュニティであまり注目されていなかった特定の要件と制約が伴いる
\item ロボット工学全般, 特に ROS のコンテキストにおける自動レイテンシ管理の問題の困難な性質を文書化するために, ROS-Llama の設計を導いた ROS ワークロードの最も注目すべき側面を強調する
    \end{itemize}
\end{frame}

\begin{frame}{形式は機能に従わない}
    \begin{itemize}
        \item 従来のリアルタイムの文献では, システムの機能と要件が, OS レベルで実行可能なタスクのセットとして実装にきちんと反映されていると想定するのが一般的である
\item 同様に, 応答時間, 優先度, 重要度などの中心的な概念は通常, 特定のタスクに関連付けられているため, 特定の機能の正確なタイミングを確保する問題は, 対応するタスクを適切にプロビジョニングする問題と同等であると暗黙のうちに理解されている
    \end{itemize}
\end{frame}

\begin{frame}{}
    \begin{itemize}
        \item Secs から明らかなように
\item I と II のように, これは ROS ではほとんど当てはまらない: レイテンシが重要な機能が単一のコンポーネントに含まれることはめったになく, 原因と結果のチェーンは通常, 多くのエグゼキュータ (したがってスレッド) にまたがり, 共有エグゼキュータは, 非常に異なるレイテンシを持つ複数のチェーンを頻繁に提供する
\item ニーズとアクティベーションパターンしたがって, レイテンシマネージャーは ROS システムを全体的に考慮する必要があり, 個々のタスク, スレッド, または他の OS エンティティを分離してプロビジョニングすることはできない
    \end{itemize}
\end{frame}

\begin{frame}{害を与えない}
    \begin{itemize}
        \item ROS が人気があるのは, 経験的に, 多くのワークロードに対して (十分に) うまく機能するからである
\item デフォルトでは, ROS は Linux のベストエフォート CFS スケジューラに依存しており, 設定は全く必要ない
\item 明白なことを述べると, アクティブなレイテンシ管理は, CFS で観察されるよりもレイテンシ目標の遵守を悪化させるべきではない
\item しかし, 構成が不十分なリアルタイムスケジューラは, デフォルトの CFS スケジューラよりもはるかにパフォーマンスが低下するため, これは思ったほど簡単ではない
\item したがって, レイテンシマネージャーは自己認識し, 不確実な利益の構成変更を実行することを控える必要がある
    \end{itemize}
\end{frame}

\begin{frame}{エキゾチックなカーネルパッチがない}
    \begin{itemize}
        \item 一般に, ロボット工学エンジニアは, より優れたリアルタイムサポートが約束されているという理由だけで, 公式にサポートされている「実戦テスト済み」のプラットフォームから切り替えることを望んでいない
\item 予測可能性の向上が, ツールの不足, サポートの取得の難しさ, または (認識された) コードの成熟度の欠如を上回ることはめったにない
\item これにより, カーネルのリアルタイム機能を強化する特注のパッチの使用が除外される
\item したがって, 実用的な解決策は, ストック Linux カーネル (およびその広く使用されている PREEMPT\_RT バリアント) にある機能に限定される
    \end{itemize}
\end{frame}

\begin{frame}{普遍的なバイインなし}
    \begin{itemize}
        \item セクションで議論されているように
\item I
\item ROS エコシステムは本質的にヘテロジニアス混合であり, 開発は非同期でアジャイルな方法で進行し, 頻繁なコンポーネントの更新が特徴である
\item したがって, 全ての (または任意の) コンポーネント開発者が特定のレイテンシ管理アプローチをサポートするために努力することを期待することは現実的ではない
\item 特に, これは, 実用的なソリューションがソースレベルのアノテーションに依存したり, カスタム API の使用を前提にしたり, ROS の基本的な動作方法を変更したりできないことを意味する
    \end{itemize}
\end{frame}

\begin{frame}{採用の容易さ}
    \begin{itemize}
        \item 同様に, レイテンシマネージャーは, システムインテグレーターが被る事前の構成と継続的なメンテナンスの負担を最小限に抑える必要がある
\item これは, ベースラインの選択 (デフォルトの CFS スケジューラ) がセットアップを全く必要としないことを考えると特に当てはまります
\item システムインテグレーターは, 通常, ロボットおよびミッション固有のエンドツーエンドレイテンシ要件について高レベルの理解を持っているが, さまざまな ROS コンポーネントがどのように正確に相互作用するか, それらがどのくらいの頻度で相互作用するかなど, 低レベルのシステム内部を知ることは合理的に期待できない
    \end{itemize}
\end{frame}

\begin{frame}{}
    \begin{itemize}
        \item エグゼキュータの数, ROS エグゼキュータによるコールバックのスケジュール方法, または Linux のリアルタイムスケジューリング機能の詳細な動作方法
\item したがって, 採用への障壁を最小限に抑えるために, 実用的なレイテンシマネージャーは, 事前の構成 (またはコストのかかる静的分析) ではなく動的なイントロスペクションに可能な限り依存し, メカニズム固有のオプションよりも宣言的な目標による構成を優先する必要がある
    \end{itemize}
\end{frame}

\begin{frame}{予測不可能な環境}
    \begin{itemize}
        \item 動的な環境ではリソースのニーズが本質的に不確実で変化するため, 動的な内省ベースのアプローチも推奨される
\item I
\item さらに, ミッションプロファイルが進化し, ロボットがその動作に適応するにつれて, レイテンシの目標が変化する可能性があるため, 高レベルで目標指向のアプローチの必要性が高まります
    \end{itemize}
\end{frame}

\begin{frame}{あると便利なペイロード}
    \begin{itemize}
        \item 前のポイントと密接に関連しているが, ロボットが実際に全てのソフトウェア機能をあらゆる状況で維持するのに十分なコンピューティングリソースを備えていると仮定するのは, 素朴で誤った方向に進んでいる
\item それどころか, 特に, スペース, 重量, 電力 (SWaP) の制約を受けるモバイルロボットでは, 「ほとんどの場合」動作するはずの「あると便利な」機能を含めることは珍しくないが, 厳密には必須ではなく, 条件が厳しくなった場合 (例えば, ミッションクリティカルだがセーフティクリティカルではないペイロード), 低下したレベルで動作する (または全く動作しない) ことが完全に予想される
\item 実用的なレイテンシマネージャーは, このような重要でない機能の意図的な過少プロビジョニングを認識している必要がある
    \end{itemize}
\end{frame}

\begin{frame}{当然の過負荷動作}
    \begin{itemize}
        \item 逆に, ロボットが一時的な過負荷の期間を経験することは異常ではない
\item その性質上, このような期間は自動レイテンシマネージャーにとって最も困難な状況であり, 全てのレイテンシ目標を同時に満たすことはできないため, 難しい選択が必要になる
\item 実用的なレイテンシマネージャーは, 過負荷下での不安定な意思決定や不安定な動作に陥ってはならない
\item むしろ, Cause-effect chainのレイテンシを予測可能かつ適切に低下させることにより, 「驚き」を回避する必要がある
    \end{itemize}
\end{frame}

\begin{frame}{維持する}
    \begin{itemize}
        \item 最後になったが, レイテンシマネージャーで費やされる全てのプロセッササイクルは, 特に過負荷の期間中は, ワークロードで費やされないサイクルであることを強調する価値がある
\item 実用的に言えば, CFS でのレイテンシの問題は, 追加のリソースを利用可能にするだけで軽減できることが多いため, アクティブなレイテンシマネージャーの存在がレイテンシ目標の遵守に関して実際に有益であるとは限りません
\item 言い換えれば, レイテンシマネージャーは, そもそもそれを実行するコストを補うのに十分な利益を生み出す必要がある.したがって, 実装効率と, 採用されている分析の実行時間は非常に重要である
    \end{itemize}
\end{frame}
