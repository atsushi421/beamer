% !TeX root = main.tex

\section{AUTOMATIC LATENCY MANAGEMENT}
\label{sec: automatic latency management}

\begin{frame}{}
    \begin{itemize}
        \item セクション I で論じたように, ロボット工学の領域には, これまでリアルタイムシステムのコミュニティであまり注目されてこなかった特定の要件と制約がある
        \item 一般的なロボット工学, 特にROSのコンテキストにおける自動レイテンシ管理問題の困難な性質を文書化するために, ROS-Llamaの設計を導いたROSワークロードの最も注目すべき側面を強調する
    \end{itemize}
\end{frame}

\begin{frame}{形は機能に従うべきではない}
    \begin{itemize}
        \item 古典的なリアルタイム技術では, システムの機能性や要求がOSレベルの実行可能なタスクの集合として実装にきちんと反映されると仮定するのが一般的である
        \item そのため, 応答時間, 優先度, クリティカリティなどの中心的な概念は, 通常, 特定のタスクと関連付けられており, 与えられた機能に対する正しいタイミングの確保という問題は, 対応するタスクを適切に準備するという問題と暗黙のうちに等価であると理解されている
    \end{itemize}
\end{frame}

\begin{frame}{}
    \begin{itemize}
        \item レイテンシが重要な機能が単一のコンポーネントに含まれることはほとんどなく, cause-effectチェーンは通常多くのエグゼキュータ (したがってスレッド) にまたがり, 共有エグゼキュータは, 大きく異なるレイテンシニーズと起動パターンを持つ複数のチェーンに頻繁に対応している
        \item したがって, レイテンシマネージャは, ROSシステムを全体的に考慮しなければならず, 個々のタスク, スレッド, または他のOSエンティティを分離してプロビジョニングすることはできない
    \end{itemize}
\end{frame}

\begin{frame}{害を与えないこと}
    \begin{itemize}
        \item ROSが人気なのは, 経験的に, 多くのワークロードに対してうまく (十分) 機能するからである
        \item デフォルトでは, ROSはLinuxのベストエフォート型CFSスケジューラに依存し, 何の設定も必要ない
        \item 明白なことは, 能動的なレイテンシ管理は, CFSの下で観察されるよりもレイテンシ目標への準拠が悪くなってはいけないということである
        \item 設定 の甘いリアルタイムスケジューラは, CFS スケジューラよりはるかに悪い結果を出すことがある
        \item したがって, レイテンシマネージャは自己認識し, 不確実な利益のために設定を変更することは避けなければならない
    \end{itemize}
\end{frame}

\begin{frame}{エキゾチックなカーネルパッチがないこと}
    \begin{itemize}
        \item ロボット工学のエンジニアは一般的に, より良いリアルタイムサポートが約束されているからといって, 公式にサポートされ, 「戦場でテストされた」プラットフォームから切り替えようとはしない
        \item ツールやサポートの入手の難しさ, コードの成熟度の低さ, 稀なバグや未検証のコーナーケースのリスクなどを差し引いても, 予測可能性が向上することはほとんどない
        \item このため, カーネルのリアルタイム機能を拡張する特注のパッチを使用することはできない
        \item したがって, 実用的な解決策は, 標準的なLinuxカーネル (および広く使われているPREEMPT-RT) に見られる機能に限定される
    \end{itemize}
\end{frame}

\begin{frame}{普遍的な賛同が得られない}
    \begin{itemize}
        \item セクション I で議論したように, ROS のエコシステムは本質的にヘテロジニアスであり, 開発は頻繁なコンポーネントの更新によって特徴付けられる, 非同期でアジャイルな方法で進行する
        \item したがって, 全てのコンポーネント開発者が, 特定のレイテンシ管理アプローチのサポートに労力を費やすことを期待するのは現実的ではないし, システムインテグレータが不足している部分を補うことを期待するのも妥当ではない
        \item 特に, これは実用的なソリューションが, ソースレベルのアノテーションに依存したり, カスタムAPIの使用を前提にしたり, ROSの動作を根本的に変更したりできないことを意味する
    \end{itemize}
\end{frame}

\begin{frame}{採用のしやすさ}
    \begin{itemize}
        \item レイテンシマネージャは初期設定と継続的なメンテナンスの負担を最小にする必要がある
        \item 特に, デフォルトスケジューラであるCFS のベースラインは, 全く設定する必要がない
        \item システムインテグレータは, 通常, ロボットやミッション固有のエンドツーエンドレイテンシ要件に関するハイレベルな理解を持っているが, 低レベルのシステム内部を知ることは合理的に期待されるものではない
              \forme{\item 様々なROSコンポーネントがどのように正確に相互作用し, どの程度の頻度で相互作用し, 何個のエグゼキュータがあり, ROSエグゼキュータでコールバックがスケジュールされ, Linuxのリアルタイムスケジュール機能がどのように動作するのか}
        \item 採用の障壁を最小にするために, 実用的なレイテンシマネージャは, アップフロント設定 (または高価な静的分析) よりもむしろ動的イントロスペクションにできるだけ依存し, メカニズム固有のオプションよりも宣言的目標による設定を優先すべきである
    \end{itemize}
\end{frame}

\begin{frame}{予測不可能な環境}
    \begin{itemize}
        \item 動的環境におけるリソースニーズが本質的に不確実であり, 変化するため, 動的なイントロスペクションに基づくアプローチも推奨される
        \item さらに, ミッションプロファイルが進化し, ロボットがその行動を適応させると, レイテンシの目標が変わる可能性があるため, 高レベルの目標指向のアプローチの必要性が強まる
    \end{itemize}
\end{frame}

\begin{frame}{必要なペイロード}
    \begin{itemize}
        \item 前項と密接に関連するが, ロボットがあらゆる状況下で全てのソフトウェア機能を維持するのに十分な計算リソースを実際に備えていると仮定するのは見当違いである
        \item それどころか, 特にスペース, 重量, 電力 (SWaP) の制約を受けるモバイルロボットでは, 「ほとんどの場合」動作するはずであるが, 厳密には必須ではなく, 条件が厳しくなるとデグレードしたレベルで動作する (または全く動作しない) ことが十分に予想される「いいとこ取り」の機能 (例えば, ミッションは重要だが安全性が重要ではない積載物) を含むことが珍しくない
        \item 実用的なレイテンシマネージャは, このような非重要機能の意図的な過小提供を認識する必要がある
    \end{itemize}
\end{frame}

\begin{frame}{意外と知られていない過負荷の挙動}
    \begin{itemize}
        \item 逆に, ロボットが一時的な過負荷の期間を経験することは, 決して珍しいことではない
        \item そのため, このような期間は自動レイテンシ管理者にとって最も困難な状況であり, 全てのレイテンシ目標を同時に満たすことができないため, 難しい選択を迫られることになる
        \item 実用的なレイテンシマネージャは, 過負荷時に不規則な意思決定や不安定な動作に陥ってはならない
        \item むしろ, cause-effectチェーンのレイテンシを予測可能かつ優雅に低下させることによって, 「驚き」を回避する必要がある
    \end{itemize}
\end{frame}

\begin{frame}{Earn your keep}
    \begin{itemize}
        \item 最後になるが, レイテンシマネージャに費やすプロセッササイクルは, 特に過負荷の期間中は, ワークロードに使われないサイクルとなることを強調する価値がある
        \item 現実的には, CFS のレイテンシ問題は, 単に追加リソースを利用可能にすることで軽減できることが多いので, レイテンシマネージャがアクティブであることが, レイテンシ目標の遵守という点で, 実際に有益であるとは言えない
        \item 言い換えれば, レイテンシマネージャは, それを実行するコストを補うに足る十分な利点をもたらさなければならないのである
        \item したがって, 実装の効率と, 採用する分析の実行時間は非常に重要である
    \end{itemize}
\end{frame}
