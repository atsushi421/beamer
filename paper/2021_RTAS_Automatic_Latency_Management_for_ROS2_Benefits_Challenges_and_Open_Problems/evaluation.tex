% !TeX root = main.tex

\section{EVALUATION}
\label{sec: evaluation}

% \begin{frame}{実験環境}
%     \begin{itemize}
%         \item ROS-Llama を Raspberry Pi4B で制御された Turtlebot 3 "Burger" で評価した
%         \item Raspberry Piは, $600 \mathrm{MHz}$ で動作する4つのコアを持つARM A72 CPUを搭載し, システムはPREEMPT-RTパッチ付きの標準Linuxカーネル (バージョン4.19.71-rt24-raspi2) で, EclipseのCyclone DDS (バージョン0.5.1-1) によるROS 2 "Dashing Diademata" が動作している
%               % \item プロセッサは $1.2 \mathrm{GHz}$ の設定もサポートしている
%               % \item しかし, この周波数はオーバーヒートにより連続動作が維持できず, すぐにコアの予測できないサーマルスロットリングにつながる
%               % \item 動的な周波数スケーリングは本論文の範囲外であるため, ここでは安定した $600 \mathrm{MHz}$ 設定に焦点を当てる
%     \end{itemize}
% \end{frame}


% \subsubsection{ROS components}
% \label{sssec: ros components}

\begin{frame}{ROSコンポーネントの配置}
    \begin{block}{Turtlebot 用ドライバ}
        ロボットのハードウェア (レーザースキャナ, オドメータ, 電気モーターで動く車輪) にROSインタフェースを提供する Turtlebot 用ドライバ
    \end{block}
    \begin{block}{ROSナビゲーションスタック}
        ナビゲーション計画, パス追従, および自己位置特定を含む, 車輪付き移動ロボットのための一般的なナビゲーションプリミティブを実装
    \end{block}
    \begin{block}{オブジェクトトラッカーペイロード}
        ビデオシーケンスを通して指定されたオブジェクトを追跡し, 計算負荷の高いミッションクリティカル (セーフティクリティカルではない) ペイロードの典型例として機能する
    \end{block}
\end{frame}

% \begin{frame}{カメラのシミュレート方法}
%     \begin{itemize}
%         \item 本論文のセットアップでは, VOT 2018チャレンジ[20, 21]から取得したcarlビデオを繰り返し再生してカメラをシミュレートした
%         \item このシナリオでは, オブジェクトトラッカーはシーンを通じて車を追跡することが課題となっている
%         \item Raspberry Piとパッケージ開発者が推奨するハードウェア($2.6 \mathrm{Ghz}$ でクロックされた4コアのIntel $i 7-6700 \mathrm{HQ}$ )の間の大きな性能差を補うために, 我々はそれに応じてビデオをダウンサンプリングした
%     \end{itemize}
% \end{frame}


\subsubsection{Latency goals}
\label{sssec: latency goals}

\begin{frame}{コールバックチェーン全体像}
    \fitimage{
        表に示すコールバックチェーンに対して, レイテンシ目標を設定した
    }{input_chains}
\end{frame}

% \begin{frame}{}
% \begin{itemize}
%     \item ハートビートチェーンは, $100 \mathrm{~ms}$ の周期を持つウォッチドッグタイマを管理することで, ハートビートの発生を防ぐだけである
% \end{itemize}
% \end{frame}

% \begin{frame}{}
%     \begin{itemize}
%         \item 最初の 2 つの機能チェーンは, ロボットの車輪の動きに関するものである
%         \item パイロットチェーンは, 次のモーターコマンドの計算を担当する計算量の多いローカルプランナーコールバックと, その後に続く, 電気モーターに送信するコマンドをエンコードする短いコールバックで構成されている
%         \item $125 \mathrm{~ms}$ のレイテンシ目標は, モーターがローカルプランナーのコマンドを1周期に1回受信することを保証する
%         \item オドメトリナビチェーンは, $50 \mathrm{~ms}$ ごとに計測された車輪の動きをプランナーに報告する
%         \item オドメトリの更新が毎周期届くように, すなわち, 最大 $50 \mathrm{~ms}$ のサンプリングレイテンシを含む2つの計測値の間隔が $125 \mathrm{~ms}$ 以下に収まるように, レイテンシ目標を $75 \mathrm{~ms}$ に設定した
%     \end{itemize}
% \end{frame}

% \begin{frame}{}
%     \begin{itemize}
%         \item 次の3つのチェーンは, ロボットのセルフローカライゼーションをカバーしている
%         \item ローカライゼーションでは, レーザースキャンと内部マップを利用して, ロボットの現在位置を推定する
%         \item レーザーはロボットと一緒に動くので, スキャンを解釈するためにはロボットの動きと向きに関する情報, すなわちオドメトリも必要である
%         \item この2つの入力は, レーザースキャナーチェーンおよびオドメトリロケーションチェーンによって提供される
%         \item ローカライゼーションチェーンでは, これらの入力を統合し, 位置推定を行う
%         \item このチェーンについては, 第VII章で再度述べる
%     \end{itemize}
% \end{frame}

% \begin{frame}{}
%     \begin{itemize}
%         \item この推定値は, レーザースキャンの撮影時刻から数えて1秒後に失効する
%         \item このため, ローカライゼーションコンポーネントにはタイミングの制約がある
%         \item ローカライゼーション推定値がデッドライン切れになると, ロボットはナビゲートできなくなり, 緊急停止を行う
%         \item そのため, レーザースキャナーと最終的な位置特定コールバックの間のエンドツーエンドレイテンシを最大1秒にする必要がある
%     \end{itemize}
% \end{frame}

% \begin{frame}{}
%     \begin{itemize}
%         \item レーザースキャナは $5 \mathrm{~Hz}$ で回転し, $200 \mathrm{~ms}$ ごとに1回のスキャンが可能である
%         \item 実際には, 個々のスキャンがハードウェアによって不完全に送信され, 解釈できないことがあることがわかった
%         \item このようなスキャンを考慮すると, 最悪でも $400 \mathrm{~ms}$ のサンプリングレイテンシが発生し, $600 \mathrm{~ms}$ が処理に残される
%         \item $150 \mathrm{~ms}$ はレーザースキャナーチェーンに, $450 \mathrm{~ms}$ はローカライズチェーンに割り当てた
%         \item オドメトリについては, さらに $50 \mathrm{~ms}$ のサンプリングレイテンシを考慮する必要があり, $100 \mathrm{~ms}$ をオドメトリロックチェーンに残した
%     \end{itemize}
% \end{frame}

% \begin{frame}{}
%     \begin{itemize}
%         \item 最後に, tracker chainはイメージトラッカーをカバーする
%         \item このチェーンは, 定期的に (ディスクから) フレームを取得し, /rgbトピックに送信する (シミュレートされた) カメラと, マークされたオブジェクトを追跡し, 最新のフレームでその位置を出力するトラッカーをカバーする
%         \item 割り当てられたレイテンシゴールは, 全てのフレームが次のフレームが到着する前に処理されるようにし, トラッカーが通常の条件下で遅れをとらないようにする
%         \item しかし, トラッカーチェーンはデグレードの順番も1番目で, これはその出力が「あると便利」だが, ロボットの正しい動作には必須ではないことを反映している
%     \end{itemize}
% \end{frame}

% \begin{frame}{}
%     \begin{itemize}
%         \item 我々は, 全てのレイテンシの目標は, システムインテグレータが知っている高レベルの機能的な考慮事項とハードウェアの特性から派生していることを強調する
%     \end{itemize}
% \end{frame}


\subsubsection{Baselines}
\label{sssec: baselines}

\begin{frame}{比較対象1}
    \begin{block}{比較対象1}
        標準的なLinuxのセットアップで, スレッドはCFSを使用して4つのコア全てにグローバルにスケジューリングされる
    \end{block}
    % \vspace{5mm}
    % \begin{itemize}
    %     \item これはROSのデフォルトの設定であり, システムインテグレータやコンポーネント開発者に代わってリアルタイムの専門知識を必要としない, 唯一の利用可能な選択肢
    %     \item ROS-Llamaのオーバヘッドがないため, ROS-Llamaの実行コストがその利点を上回るかどうかという問題に関して, 公平なベースラインを提供する
    % \end{itemize}
\end{frame}

\begin{frame}{比較対象2}
    \begin{block}{比較対象2}
        \begin{itemize}
            \item ROS-Llamaの原型で, 応答時間分析を用いずに, レイテンシが重要なエグゼキュータを SCHED\_RRでスケジューリングする
            \item デグレードの順序に従って優先度を割り当て, デグレードの順序が遅いエグゼキュータを, 順序が早いチェーンにのみ対応するエグゼキュータより優先させる
        \end{itemize}
    \end{block}
    % \vspace{5mm}
    % \begin{itemize}
    %     \item このベースラインを使用して, 応答時間分析の使用によりROS-Llamaの判断がどの程度改善されるかを評価する
    % \end{itemize}
\end{frame}


% \subsubsection{Scenario}
% \label{sssec: scenario}

\begin{frame}{実験シナリオ}
    \begin{itemize}
        \item ロボットは映像中の多数の物体を追いかけながら, 2つの固定された場所の間を巡回する
        \item 実験は, 無負荷, 通常負荷, 高負荷の3つのフェーズに分けられる
        \item これにより, 追跡するオブジェクトの数を増加させた場合の, 他のチェーンに与える影響を観察する
    \end{itemize}
\end{frame}


\subsection{Evaluation Results: Latency Goal Compliance}
\label{ssec: evaluation results: latency goal compliance}

\begin{frame}{セクションサマリ}
    \begin{itembox}[l]{\textbf{目的}}
        ROS-Llamaがどの程度レイテンシ目標を達成できるかを評価する
    \end{itembox}
\end{frame}

\begin{frame}{結果}
    \fitimage{
        エンドツーエンドレイテンシが設定した目標を超えて観測されたチェーン完了の数
    }{eval_goal}
\end{frame}

\begin{frame}{ROS-Llamaの優位性}
    \begin{itemize}
        \item トラッカーチェーンは高負荷の場合, 頻繁に目標を超える
        \item 2つのベースラインでは, トラッカーチェーン以外のチェーンがレイテンシ目標を超過している
        \item 一方, ROS-Llamaは, より重要なチェーンをトラッカーチェーンによる干渉から守り, システムをデグレードさせることに成功した
    \end{itemize}
\end{frame}

% \begin{frame}{パイロットチェーンレイテンシの分析}
%     \begin{columns}
%         \begin{column}{0.5\textwidth}
%             \begin{itemize}
%                 \item 図はパイロットチェーンで観測されたエンドツーエンドレイテンシのCDFを, フェーズごとに分けて示している
%                 \item 曲線のポイントが灰色の線の左側にある場合, 目標は常に満たされている
%             \end{itemize}
%         \end{column}
%         \begin{column}{0.5\textwidth}
%             \fullimage{pilot_chain}
%         \end{column}
%     \end{columns}
% \end{frame}

% \begin{frame}{結果からわかること}
%     \begin{itemize}
%         \item 最初のフェーズでは, 3つのアプローチとも, レイテンシ目標に対して少なくとも $25 \mathrm{~ms}$ のマージンを確保する
%         \item 負荷が増加するにつれて, テールが長くなり, 最終的にベースラインはレイテンシ目標を超える
%         \item トラッカーチェーンは, クリティカルなパイロットチェーンとは機能的に全く無関係であるが, トラッカーチェーンの過負荷は, クリティカルチェーンのエンドツーエンドレイテンシに大きな影響を与え, 最も厳しいケースでは $25 \mathrm{~ms}$ 以上増加させる
%     \end{itemize}
% \end{frame}

% \begin{frame}{CFSにおけるパイロットチェーンレイテンシ増加原因}
%     \begin{block}{原因}
%         CFSのデフォルトのタイムシェアリングポリシーでは, 両方のリソースが等しく使用されるため
%     \end{block}
%     \begin{block}{示唆}
%         それぞれのレイテンシ要件と, ロボットの全体的な正しい動作に対する相対的な重要性を無視しているため, 過負荷状態ではCFSは適切なポリシーとは言えない
%     \end{block}
% \end{frame}

% \begin{frame}{SCHED\_RRにおけるパイロットチェーンレイテンシ増加原因}
%     \begin{block}{原因}
%         トラッカーチェーンはリアルタイム優先で過負荷になっているため, DDS層にもバースト的な過負荷を引き起こし, それがパイロットチェーンのチェーン内通信レイテンシに悪影響を与えている
%     \end{block}
% \end{frame}

% \begin{frame}{SCHED\_RRにおけるodometry-locチェーンレイテンシ増加}
%     \begin{columns}
%         \begin{column}{0.5\textwidth}
%             SCHED\_RRベースラインでは, odometry-locチェーンで深刻なレイテンシスパイクが観測され, 高負荷時のテールレイテンシは無負荷シナリオで観測された最大レイテンシをほぼ $100 \mathrm{~ms}$ も上回った
%         \end{column}
%         \begin{column}{0.5\textwidth}
%             \fullimage{odometry_loc_chain}
%         \end{column}
%     \end{columns}
% \end{frame}

% \begin{frame}{SCHED\_RRにおけるodometry-locチェーンレイテンシ増加原因}
%     \begin{block}{原因}
%         \setlength{\linewidth}{0.98\columnwidth}
%         \begin{itemize}
%             \item odometry-locチェーンがデグレードオーダーの初期にあり, そのコールバックに対応するエグゼキュータに低いスケジューリング優先度が割り当てられていることが原因となっている
%             \item これは, 緊急度ではなく, クリティカリティモノトニックに優先度を割り当てる場合の典型的な問題[22]
%         \end{itemize}
%     \end{block}
% \end{frame}

% \begin{frame}{}
%     \begin{itemize}
%         \item 全体として, ここで報告された特定の数字に過度の重きを置かないように注意する必要がある
%         \item なぜなら, 評価の設定 (実際の物理環境内を自律的に動くロボットで, ノイズの多いセンサ入力に基づいて決定を行う複雑なナビゲーションヒューリスティックスの制御下) は, 実行ごとにかなりの変動を許容するからである (例えば, 表IIの様々なチェーンアクティベーションカウントに示されているように)
%         \item しかし, このケーススタディでは, 実験を何度も繰り返すうちに, 明確な傾向が見られるようになった
%     \end{itemize}
% \end{frame}

% \begin{frame}{まとめ}
%     \begin{itemize}
%         \item デフォルトのCFS, SCHED\_RRのベースラインは満足できる結果にはつながらない
%         \item ROS-Llamaは, LinuxのSCHED\_DEADLINEスケジューラによって提供される既存の機能を利用し, 応答時間分析によって導かれる動的なシステムプロビジョニングを行い, 特にシステムインテグレータにリアルタイムスケジューリングの詳細を負担させないようにしている
%     \end{itemize}
% \end{frame}


% \subsection{Evaluation Results: ROS-Llama Runtime Costs}
% \label{ssec: evaluation results: ros-llama runtime costs}

% \begin{frame}{セクションサマリ}
%     \begin{itembox}[l]{\textbf{目的}}
%         ROS-Llamaの実行に関わるコストを調査する
%     \end{itembox}
% \end{frame}

% \begin{frame}{結果}
%     \fitimage{
%         ROS-Llamaの主要部分の平均的な呼び出しごとのコスト
%     }{overhead}
% \end{frame}

% \begin{frame}{結果から分かること}
%     \begin{itemize}
%         % \item ROS-LlamaのメモリフットプリントはROSのフットプリントと比較してごくわずかであるため, 我々はプロセッサ時間に注目した
%         \item ROS-Llamaは6秒ごとにバジェットを再計算するように設定されており, 全体としてROS-Llamaは1CPUの30--40\% を消費する
%         \item モデル抽出器は, システムとともに継続的に実行され, ROS-Llamaの総コストの約 $60 \%$ を占める
%         \item 分析のためにタイミングモデルを準備するコストは, オーバヘッドの約 $15 \%$
%         \item 残りの実行時コストは, バジェット選択プロセスによるもので, バジェット編成ヒューリスティック $(\approx 5 \%)$ と応答時間分析 $(\approx 20 \%)$ に分かれている
%     \end{itemize}
% \end{frame}

% \begin{frame}{示唆}
%     \begin{itemize}
%         \item ROS-Llamaは顕著な実行時オーバヘッドをもたらす
%         \item しかし, ROS-Llamaはレイテンシ目標への適合性が高く, デグレードの挙動も緩やか
%         \item すなわち, ROS-Llamaは大きなコストを伴うが, 性能と予測性の良好なトレードオフを実現するのに十分な利点を提供する
%     \end{itemize}
% \end{frame}

% \begin{frame}{オーバヘッド改善可能性}
%     \begin{itemize}
%         \item 現在のプロトタイプでは, 最適化されていないトレース実装のため, モデル抽出器がほとんどのオーバヘッドを発生させている
%         \item ROS-LlamaをLTTng [23]のような成熟したトレースシステムと統合することで, ROS-Llamaの実行コストを下げることができると考えられる
%     \end{itemize}
% \end{frame}
