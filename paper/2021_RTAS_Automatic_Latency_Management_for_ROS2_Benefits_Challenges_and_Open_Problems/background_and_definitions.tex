% !TeX root = main.tex

\section{BACKGROUND AND DEFINITIONS}
\label{sec: background and definitions}

% \begin{frame}{セクションサマリ}
%     \begin{itembox}[l]{\textbf{目的}}
%         ROS-Llamaの基礎となる必要な背景知識と主な分析コンセプトを簡単にまとめる
%     \end{itembox}
% \end{frame}

% \begin{frame}{ROS}
%     \begin{itemize}
%         \item ROSフレームワークは, ロボットアプリケーションのための一般的なオープンソースのミドルウェアとコンポーネントリポジトリである[1]
%         \item 本論文は, 特に, 第一世代のROSフレームワークの最近の主要なリファクタリングであるROS 2 [2]に関連する
%         \item 簡潔さを保つため, 本論文ではバージョン番号を省略する
%     \end{itemize}
% \end{frame}

% \begin{frame}{}
%     \begin{itemize}
%         \item ROSは, 通常Linux上に展開される成熟した機能豊富なミドルウェアである
%         \item 我々は, レイテンシ管理に不可欠なランタイムの主要要素であるトピック, コールバック, エグゼキュータに着目している
%         \item すでに述べたように, ROSはpublish/subscribeインフラを中心に構築されており, デプロイの選択にはほとんど無頓着である
%         \item したがって, ROSアプリケーションは, 複数のホスト, コア, プロセス, およびスレッドにまたがることができる
%         \item 自動レイテンシ管理の目的では, Linuxを実行している共有メモリマルチコアシステムが自然な範囲であり, 我々はここに注意を制限する
%     \end{itemize}
% \end{frame}

% \begin{frame}{}
%     \begin{itemize}
%         \item 実装レベルでは, ROSアプリケーションは, それぞれが1つ以上のスレッドからなる1つ以上のプロセスで構成され, 順番にエグゼキュータ (すなわち, コールバックを呼び出すROSライブラリ機能) を実行する
%         \item 各コールバックは, 特定のエグゼキュータと関連付けられている
%         \item 新しいメッセージがトピックに投稿されると, ROSミドルウェアは, メッセージのコピーがトピックのsubscriberにサービスを提供するエグゼキュータをホストする全てのスレッドに配布されることを保証する
%         \item 標準的な構成では, ROSは $D D S$ ミドルウェアに依存して, エグゼキュータ間でメッセージを仲介しており, 適切なDDS実装が複数存在する[3-5]
%     \end{itemize}
% \end{frame}

% \begin{frame}{}
%     \begin{itemize}
%         \item エグゼキュータが新しいメッセージが利用可能であることを通知されると, 対応するコールバックがアクティブになり, 実行のためにキューに入れられる
%         \item したがって, コールバックがメッセージを処理するレイテンシは, 2つの主要な要因によって決定される
%         \item (i) OSがそれぞれのエグゼキュータをホストするスレッドに割り当てるプロセッサ時間, および (ii) エグゼキュータが保留中のコールバックの起動をどのようにシーケンスするかによって発生する待ち行列のレイテンシ時間である
%     \end{itemize}
% \end{frame}

% \begin{frame}{}
%     \begin{itemize}
%         \item 我々の目的では, 側面(i)-エグゼキュータスレッドのスケジューリング-は, 自動レイテンシマネージャが実行時に制御できる主要な要因であるため, 主な関心事である
%         \item (ii)のキューイングレイテンシは, コールバックレイテンシに大きな影響を与え, 決して分析が容易ではないが[6], (現在のROSバージョンでは)ランタイム管理には適していない
%     \end{itemize}
% \end{frame}

\begin{frame}{応答時間分析}
    ROSのデフォルトのエグゼキュータによって管理されるコールバックの最悪応答時間は, Casiniらによる先行研究~\cite{casini2019response}で計算する
    \begin{block}{先行研究~\cite{casini2019response}}
        \begin{itemize}
            \item ROSシステムを, アクティベーション関係によって接続されたコールバックの有向非巡回グラフ (DAG) としてモデル化
            \item 処理チェーンのためのエグゼキュータを考慮した最悪応答時間, すなわちコールバックDAG内の任意のパスのエンドツーエンドレイテンシ境界を提供
        \end{itemize}
    \end{block}
\end{frame}

\begin{frame}{供給境界関数 (SBF)}
    各スレッドについて供給境界関数 (SBF) が既知であるとする
    \begin{block}{供給境界関数 (SBF)}
        あるスレッドがある区間にOSのスケジューラからどれだけの処理時間を割り当てられることが保証されているかを特徴づける
    \end{block}
\end{frame}

\begin{frame}{SBF保証を得る方法}
    SBF保証を得るための標準的な方法は, LinuxではSCHED\_DEADLINEポリシーによって実現されている予約ベーススケジューリングを使用すること
    \begin{block}{予約ベーススケジューリング}
        \setlength{\linewidth}{0.98\columnwidth}
        \begin{itemize}
            \item 各予約 $r$ はバジェット $\operatorname{budget}(r)$ と期間 $\operatorname{period}(r)$ で特徴付けられる
            \item スケジューラは各予約の各周期で少なくとも $\operatorname{budget}(r)$ ユニットのプロセッササービスを受けることを保証する
            \item この保証は帯域幅 $b w(r)=\frac{\operatorname{budget}(r)}{\text { period }(r)}$ で表記される
        \end{itemize}
    \end{block}
\end{frame}

\begin{frame}{到着曲線}
    外部イベントソースによってトリガーされたコールバックについて, 到達曲線が既知であると仮定する
    \begin{block}{外部イベントソース}
        \setlength{\linewidth}{0.98\columnwidth}
        \begin{itemize}
            \item それ自体がコールバックではなく, 1つ以上のトピックにメッセージを送信するもの
            \item 例えば, センサ値を取得してコールバックDAGに送り込むデバイスドライバ
        \end{itemize}
    \end{block}
    \begin{block}{到着曲線 $\eta_{c}(\Delta)$}
        到着曲線 $\eta_{c}(\Delta)$ は, 長さ $\Delta$ の任意の間隔でコールバック $c$ のアクティベーションの最大数を示す
    \end{block}
\end{frame}

\begin{frame}{要求境界関数}
    \begin{block}{要求境界関数 (RBF)}
        \setlength{\linewidth}{0.98\columnwidth}
        \begin{itemize}
            \item 要求境界関数は, 任意の間隔におけるコールバック$c$ の全てのアクティベーションの累積プロセッサ要求の最大値を示す
            \item $r b f_{c}(\Delta)=e_{c} \cdot \eta_{c}(\Delta)$
                  \begin{itemize}
                      \item \desc{$e_{c}$}{コールバック $c$ の最悪実行時間}
                  \end{itemize}
        \end{itemize}
    \end{block}
\end{frame}
