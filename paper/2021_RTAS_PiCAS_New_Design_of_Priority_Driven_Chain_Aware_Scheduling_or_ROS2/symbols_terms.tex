% !TeX root = main.tex

\begin{frame}{表記法・用語 1}
    \full{
        \begin{table}[tb]
            \adjustbox{max width=\textwidth, max height=\slideheight}{
                \centering\begin{tabular}{|c|l|} \hline
                    $M$                                                                                & システム内のコールバックの数                         \\\hline
                    $\tau_{i}:=\left(C_{i}, D_{i}, T_{i}, \pi_{i}\right)$                              & コールバック                                         \\\hline
                    $C_{i}$                                                                            & $\tau_{i}$ のジョブの最悪実行時間                    \\\hline
                    $D_{i}$                                                                            & コールバック $\tau_{i}$ の相対デッドライン       \\\hline
                    $T_{i}$                                                                            & コールバック $\tau_{i}$ の周期                       \\\hline
                    $\pi_{i}$                                                                          & エグゼキュータ内でのコールバック $\tau_{i}$ の優先度 \\\hline
                    $\mathcal{N}=:\left\{n_{1}, \ldots, n_{j}, \ldots, n_{N}\right\}$                  & ノードセット                                         \\\hline
                    $U\left(n_{j}\right)=\sum_{\forall t_{i}: \tau_{i} \in n_{j}} \frac{C_{i}}{T_{i}}$ & ノードの利用率                                       \\\hline
                    $\mathcal{E}=:\left\{e_{1}, \ldots, e_{j}, \ldots, e_{E}\right\}$                  & エグゼキュータセット                                 \\\hline
                    $\pi_{e_{j}}$                                                                      & $j$ 番目のエグゼキュータの優先度                     \\\hline
                \end{tabular}
            }
        \end{table}
    }
\end{frame}

\begin{frame}{表記法・用語 2}
    \full{
        \begin{table}[tb]
            \adjustbox{max width=\textwidth, max height=\slideheight}{
                \centering\begin{tabular}{|c|l|} \hline
                    $\Gamma^{c}:=\left[\tau_{s}, \tau_{m 1}, \tau_{m 2}, \ldots, \tau_{e}\right]$ & チェーン                                 \\\hline
                    $\tau_{s}$                                                                    & チェーン $\Gamma^{c}$ の開始コールバック \\\hline
                    $\tau_{m *}$                                                                  & チェーン $\Gamma^{c}$ の中間コールバック \\\hline
                    $\tau_{e}$                                                                    & チェーン $\Gamma^{c}$ の終了コールバック \\\hline
                    $\pi_{\Gamma^{c}}$                                                            & チェーン $\Gamma^{c}$ の優先度           \\\hline
                    $\mathcal{C}_{\Gamma^{c}}=\sum_{\forall i: \tau_{i} \in \Gamma^{c}} C_{i}$    & チェーン $\Gamma^{c}$ の合計 WCET        \\\hline
                \end{tabular}
            }
        \end{table}
    }
\end{frame}

\begin{frame}{表記法・用語 3}
    \full{
        \begin{table}[tb]
            \adjustbox{max width=\textwidth, max height=\slideheight}{
                \centering\begin{tabular}{|c|l|} \hline
                    $\mathcal{N} $                        & ノードセット                                          \\\hline
                    $\Gamma^{c}$                          & まだ割り当てられていない最高優先度のチェーン          \\\hline
                    $\mathbb{N}$ $(U_{\mathbb{N}} \ge 1)$ & チェーン $\Gamma^c$ のコールバックを含むノードセット  \\\hline
                    $n$ $(n \in \mathbb{N})$              & $\Gamma^c$ の最も低い優先度のコールバックを含むノード \\\hline
                    $e_e$                                 & 空のエグゼキュータ                                    \\\hline
                    $e_m$                                 & 空でないエグゼキュータ                                \\\hline
                    $M$                                   & $e_m$の数                                             \\\hline
                    $P_k$                                 & CPU コア                                              \\\hline
                    $U_{P_k}$                             & CPU コア $P_k$ の利用率                               \\\hline
                    $P$                                   & $P_k$の数                                             \\\hline
                \end{tabular}
            }
        \end{table}
    }
\end{frame}
