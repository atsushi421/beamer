% !TeX root = main.tex

\begin{frame}{提案手法の概要}
    \begin{itemize}
        \item ROS2フレームワーク用の新しい優先度駆動型チェーン考慮スケジューラ (PiCAS) を提案
        \item PiCASの最悪エンドツーエンドレイテンシ分析を示す
        \item PiCASでは対応するチェーンの重要度に基づいてコールバックに優先度が付けられるため, 重要なチェーンのエンドツーエンドレイテンシを改善できる
    \end{itemize}
\end{frame}

% \begin{frame}{Abstract 1}
%     \begin{itemize}
%         \item ROS (ロボットオペレーティングシステム) では, ほとんどのアプリケーションがデータ依存関係を持つコールバックチェーンの形式で構築される
%         \item リアルタイムサポートの欠点により, ROS はタイミング保証を提供せず, 悲惨な結果につながる可能性がある
%         \item ROS2 はリアルタイム機能を強化するが, 予測可能なエンドツーエンドチェーンレイテンシを確保することは依然として困難な問題である
%     \end{itemize}
% \end{frame}

% \begin{frame}{Abstract 2}
%     \begin{itemize}
%         \item 本論文では, ROS2フレームワーク用の新しい優先度駆動型チェーン考慮スケジューラを提案し, 提案されたスケジューラのエンドツーエンドのレイテンシ分析を提示する
%         \item このスケジューラを使用すると, 対応するチェーンの特定のタイミング要件に基づいてコールバックに優先度が付けられるため, 重要なチェーンのエンドツーエンドのレイテンシを予測可能な範囲で改善できる
%         \item 提案されたスケジューリング設計には, ROS2 スケジューリング関連の全ての要素 (コールバック, ノード, エグゼキュータなど) を考慮した優先度の割り当てとリソース割り当てが含まれる
%     \end{itemize}
% \end{frame}

% \begin{frame}{Abstract 3}
%     \begin{itemize}
%         \item 本論文は, 新しいスケジューラ設計を提案することによってエンドツーエンドのレイテンシにおける ROS2 固有の制限に対処する最初の作業である
%         \item NVIDIA Xavier NX 上で動作する ROS2 にスケジューラを実装した
%         \item ケーススタディとスケジューラビリティの実験を実施した結果は, 提案されたスケジューラがデフォルトの ROS2 スケジューラおよび実際のシナリオでの最新の作業よりもエンドツーエンドのレイテンシを大幅に改善することを示した
%     \end{itemize}
% \end{frame}
