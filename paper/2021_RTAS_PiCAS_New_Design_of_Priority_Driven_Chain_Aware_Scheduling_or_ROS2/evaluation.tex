% !TeX root = main.tex

\section{EVALUATION}
\label{sec: evaluation}

\begin{frame}{セクションサマリ}
    \begin{itembox}[l]{\textbf{目的}}
        デフォルトの ROS2 スケジューラおよび最先端の分析作業と比較することにより, 提案された PiCAS フレームワークを評価する
    \end{itembox}
    \begin{itembox}[l]{\textbf{流れ}}
        \begin{itemize}
            \item 実際のプラットフォームでケーススタディを実施し, PiCAS の実際の有効性を特定
            \item ランダムに生成されたワークロードを使用してスケジューリングシミュレーションを行い, PiCAS とデフォルトの ROS2 スケジューラのパフォーマンス特性を調査
        \end{itemize}
    \end{itembox}
\end{frame}


\subsection{Implementation}
\label{ssec: implementation}

\begin{frame}{実装の詳細1}
    \begin{itemize}
        \item 提案されたスケジューラを, 6 コア ARMv8.2 $1.4 \mathrm{GHz}$ プロセッサを搭載した NVIDIA Xavier NX プラットフォームの Ubuntu $18.04$ 上で動作する Eloquent Elusor バージョンの ROS2 に実装した
        \item ROS2スケジューラは, 独自のコールバックスケジューリングポリシーを備えているため, 本実装では主に以下2つを変更した
              \begin{itemize}
                  \item 実行キューの準備完了コールバックの更新条件
                  \item 個々のコールバックに対する優先度割り当て
              \end{itemize}
    \end{itemize}
\end{frame}

\begin{frame}{実装の詳細2}
    \begin{itemize}
        \item ROS2 では, エグゼキュータが通信層と対話して, readyコールバックをキューにフェッチする
        \item スケジューラはコールバックが完了するたびにキューを更新する
        \item もし 1 つ以上のコールバックがキューに準備されている場合, PiCASは, \hlink{alg1}{提案優先度割り当てアルゴリズム} により決定された最も高い優先度を持つコールバックを選択する
    \end{itemize}
\end{frame}


\subsection{Case study}
\label{ssec: case study}

\begin{frame}{セクションサマリ}
    \begin{itembox}[l]{\textbf{流れ}}
        \begin{itemize}
            \item 最先端の分析作業と比較評価するための単純なユニプロセッサシステムに焦点を当てる
            \item 非過負荷および過負荷の状況を考慮して, より実用的で複雑なマルチコアシステムを評価する
        \end{itemize}
    \end{itembox}
\end{frame}

\begin{frame}{アプローチの比較}
    次の 3 つの手法を比較する
    \begin{block}{ROS2}
        分析なしの ROS2 デフォルトスケジューラ
    \end{block}
    \begin{block}{ROS2-SD}
        リソース予約と最悪応答時間分析を備えた ROS2 デフォルトスケジューラ~\cite{casini2019response}
        \notes{リソース予約は, Linux システムで SCHED\_DEADLINE ポリシーを使用して実装できる}
    \end{block}
    \begin{block}{ROS2-PiCAS}
        エンドツーエンド分析を備えた提案スケジューラ
    \end{block}
\end{frame}

\begin{frame}{コールバックの WCET計測方法}
    組み込みプラットフォームでの 5,000 回の実行から各コールバックの実行時間を測定し, そのコールバックの観測された最大の実行時間を WCET として選択
\end{frame}

\begin{frame}{ケーススタディ I (ユニプロセッサシステム)}
    \fitimage{
        \begin{itemize}
            \item ユニプロセッサ環境における, 特定の利用可能なバジェットに対するチェーンレイテンシを評価する
            \item 図はケーススタディ I のチェーン構成を示す
        \end{itemize}
    }{case_study1}
\end{frame}

\begin{frame}{ケーススタディ I 実験環境}
    \begin{itemize}
        \item コールバックは固有のバジェットで同じ予約に割り当てられ, コールバックの優先度は ROS2-SD のコールバックのインデックスの降順で与えられる
        \item チェーン $1\left(\Gamma^{1}\right)$ はチェーン $2\left(\Gamma^{2}\right)$ よりも優先度が高いとする
        \item ROS2PiCAS でのバジェットの影響を評価するために, 2 つのチェーンと同じ周期で実行され, システムで最も高い優先度を持つ定期的なスレッドを作成
              \begin{itemize}
                  \item このスレッドの実行時間を調整することで, ROS2-PiCAS の利用可能なバジェットを制限する
              \end{itemize}
    \end{itemize}
\end{frame}

\begin{frame}{ケーススタディ I 結果}
    \fitimage{
        2 つのアプローチでのチェーンのエンドツーエンドレイテンシ
    }{case_study1_result}
\end{frame}

\begin{frame}{ケーススタディ I 考察1}
    \begin{itemize}
        \item 利用可能なバジェットが増加するにつれて, 両方のアプローチのレイテンシが減少
        \item ROS2-PiCAS のチェーン 1 は ROS2-SD よりもレイテンシが低くなる
    \end{itemize}
\end{frame}

\begin{frame}{ケーススタディ I 考察2}
    \fitimage{
        \begin{itemize}
            \item また, ROS2-PiCAS の 2 つのチェーンのエグゼキュータの優先度を周期的なスレッドよりも高くして, 別の実験を行った
            \item 図の黄色の線で示されているように, チェーンは定期的なスレッドによって干渉されないため, 利用可能なバジェットに関係なくレイテンシが一貫している
        \end{itemize}
    }{case_study1_result}
\end{frame}

\begin{frame}{ケーススタディ II (マルチコアシステム)}
    \fitimage{
        \begin{itemize}
            \item 実際のシナリオでのチェーン考慮スケジューリングの有効性を評価する
            \item 図はケーススタディ II のチェーン構成を示す
        \end{itemize}
    }{case_study2}
\end{frame}

\begin{frame}{ケーススタディ II 実験環境1}
    \begin{itemize}
        \item チェーンのインデックスが小さい程, 高い優先度を持つとする
        \item 最初にチェーン 1 からチェーン 4 をそれぞれ異なる CPU コアに割り当て, 次に負荷分散方式でその他を分散する
        \item 割り当て後の各 CPU コアの利用率は, 平均で $0.97$
        \item デフォルトの ROS2 スケジューラと以前の研究 [11] では割り当て方式が提供されていないため, ROS2 と ROS2-SD では, 同じノードからコアへの割り当てと, コアごとに 1 つのエグゼキュータを使用する
        \item ROS2-SD では, 各コアのリソース予約額を $100 \%$ に設定する
    \end{itemize}
\end{frame}

\begin{frame}[label=latency]{ケーススタディ II 実験環境2}
    実験で観測されたレイテンシは, タイマコールバックの実行開始時刻を記録することで得られたものであり, リリース時刻ではない
    \notes{ROS2が最初のリリースオフセットを設定するためのサポートを欠いているため}
    \forme{
        \begin{itemize}
            \item しかし, ROS2-PiCASでは, チェーン1--4はそれぞれ異なるCPUコアの最も優先度が高いエグゼキュータに割り当てられているため, その開始時刻とリリース時刻は全く同じである
            \item チェーン5とチェーン6がそれぞれチェーン1とチェーン4と同じCPUコアに割り当てられていても, ROS2-PiCAS下では, それらのエグゼキュータの優先度と周期を考慮すれば, リリース時間ベースのレイテンシを容易に見積もることができる
            \item しかし, ROS2およびROS2-SDでは, リリース時間ベースのレイテンシを直感的に見積もることができない
        \end{itemize}
    }
\end{frame}

\begin{frame}{ケーススタディ II 結果1}
    \fitimage{
        3 つのアプローチで観察されたチェーンのエンドツーエンドレイテンシ
    }{case_study2_result_a}
\end{frame}

\begin{frame}{ケーススタディ II 結果1}
    \fitimage{
        測定から観測された最大レイテンシと分析されたレイテンシの比較
    }{case_study2_result_b}
\end{frame}

\begin{frame}{ケーススタディ II PiCASの優れている点}
    \begin{itemize}
        \item ROS2-PiCAS はほとんどのチェーンで他のチェーンよりも優れている
        \item レイテンシ分析では, チェーンの上界がより厳しくなっている
    \end{itemize}
\end{frame}

\begin{frame}{ケーススタディ II PiCASの問題点}
    \begin{columns}
        \begin{column}{0.8\textwidth}
            \begin{itemize}
                \item ROS2-PiCAS でのチェーン 6 の観測および分析されたレイテンシは, 他の手法より悪い
                \item この現象を次の 2 つの理由から説明する
            \end{itemize}
        \end{column}
        \begin{column}{0.2\textwidth}
            \fullimage{chain6_latency}
        \end{column}
    \end{columns}
\end{frame}

\begin{frame}{PiCASの問題点の理由}
    \vspace{1mm}
    \begin{block}{理由1}
        \setlength{\linewidth}{0.98\columnwidth}
        \begin{itemize}
            \item ROS2-PiCAS では, 優先度の高いチェーン 4 がチェーン 6 に複数回干渉する可能性がある
            \item 対してデフォルト ROS2 スケジューラでは, ノンプリエンプティブコールバック実行および公平性に基づくスケジューリングにより, チェイン 4 とチェイン 6 が互いに 1 回 だけブロックできる
        \end{itemize}
    \end{block}
    \begin{block}{理由2}
        ROS2およびROS2-SDにおけるチェーン6の観測レイテンシは, タイマコールバックの開始時刻を用いて得られるため, チェーン4からのブロック時間は含まれない
    \end{block}
\end{frame}

\begin{frame}{ケーススタディ II 考察結論}
    PiCAS はエンドツーエンドのチェーンレイテンシを大きく改善し, 優先度を尊重してチェーンをスケジュールできる
\end{frame}


\begin{frame}{ケーススタディ III (過負荷シナリオ)}
    \begin{itemize}
        \item 過負荷のシステム設定でのレイテンシの動作を評価する
        \item ケーススタディ II にさらに 4 つのチェーンを追加することで, 各 CPU コアの利用率を平均で $1.25$ にした
    \end{itemize}
\end{frame}

\begin{frame}{ケーススタディ III 結果}
    \fullimage{case_study3_result}
\end{frame}

\begin{frame}{ケーススタディ III 考察}
    \begin{itemize}
        \item ROS2-PiCAS は, 最も重要なチェーン, すなわちチェーン 1 とチェーン 2 の平均エンドツーエンドレイテンシをそれぞれ $85 \%$ と $90 \%$ まで削減
        \item ROS2-PiCAS の分析ではチェーンの厳密な上界が示されたが, ROS2-SD の分析ではこの過負荷システムのテストに失敗した
    \end{itemize}
\end{frame}


\subsection{Schedulability experiments}
\label{ssec: schedulability experiments}

\begin{frame}{セクションサマリ}
    \begin{itembox}[l]{\textbf{目的}}
        \begin{itemize}
            \item ROS2-SD および ROS2-PiCAS でのチェーンのスケジューラビリティを評価する
            \item これらのアプローチの分析実行時間も評価する
        \end{itemize}
    \end{itembox}
\end{frame}

\begin{frame}{実験マシン}
    全ての実験は, デュアル AMD EPYC 7452 $2.35 \mathrm{GHz}$ プロセッサを搭載したマシンで行う
\end{frame}

\begin{frame}{ワークロードの生成方法}
    \begin{itemize}
        \item ワークロードセットの利用率を, 4 コア環境の $\{2.5,3.0,3.5\}$ から選択
        \item 利用率ごとにランダムに生成された 1,000 個のワークロードセットを使用
        \item ワークロードセットごとに各5つのコールバックを持つ 9 つのチェーンを使用
        \item チェーン周期は $[50,1000]$ ms の範囲でランダムに選択され, デッドラインは周期に等しく設定される
        \item 各チェーンの利用率は 1.0 を超えない
        \item 各ワークロードセットの生成後, インデックスの小さいチェーンほど周期が短くなるようにチェーンの順序を変更し, インデックスの小さいチェーンにはより高い優先度を割り当てる
    \end{itemize}
\end{frame}

\begin{frame}{スケジューラビリティテストの比較 [結果]}
    \fullimage{schedulability_result1}
\end{frame}

\begin{frame}{スケジューラビリティテストの比較 [結果考察]}
    \begin{itemize}
        \item どちらのアプローチでも, 利用率が高くなるとスケジューラビリティは低下
        \item ROS2-PiCAS は, 全ての利用設定で ROS2-SD よりも大幅に優れている
        \item ROS2-PiCAS では, チェーン 1--4 は, それぞれ CPU コア上の最も優先度の高いエグゼキュータに割り当てられるため, 常にスケジュール可能
        \item ROS2-PiCAS ではチェーンの優先度が低くなると, スケジューラビリティが低下
        \item ROS2-SD では, チェーンデッドラインが短くなるにつれてスケジューラビリティが減少
    \end{itemize}
\end{frame}

\begin{frame}{分析実行時間の比較 [結果]}
    \fitimage{
        ROS2-SD と ROS2-PiCAS について, 各ワークロードセットに対するエンドツーエンドレイテンシ分析の実行時間を比較
    }{analysys_time_result}
\end{frame}

\begin{frame}{分析実行時間の比較 [結果考察]}
    \begin{itemize}
        \item どちらのアプローチでも利用率が上がると分析時間は長くなるが, ROS2-SD は ROS2-PiCAS よりもはるかに遅い
              \forme{\item ROS2-SD 分析の探索空間は可能な限り長いビジーウィンドウであり, 分析は全ての応答時間が収束するグローバルな固定点を繰り返し探索する~\cite{casini2019response}}
        \item 提案された分析は既存のアプローチよりもはるかに高速であり, 複雑なシステムへの適用やランタイム制御が可能
    \end{itemize}
\end{frame}
