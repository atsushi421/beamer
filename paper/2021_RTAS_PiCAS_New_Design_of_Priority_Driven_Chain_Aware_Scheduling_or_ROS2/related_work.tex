% !TeX root = main.tex

\section{RELATED WORK}
\label{sec: related_work}

\begin{frame}{既存研究との比較表}
    \full{
        \begin{table}[tb]
            \adjustbox{max width=\textwidth, max height=\slideheight}{
                \centering\begin{tabular}{|c|c|}      \\\hline
                           & XXX \\\hline
                    本論文 & \ch \\\hline
                \end{tabular}
            }
        \end{table}
    }
\end{frame}

\begin{frame}{}
    \begin{itemize}
        \item ROS に関するほとんどの作業は, リアルタイム機能の改善に焦点を当ててきた [26, 27, 29]
\item [26]では, 斎藤ら
\item publisherが優先度に基づいてデータを送信できるようにすることで, 優先度に基づくメッセージ送信メカニズムを提案した
\item 魏ら
\item [29] は, 2 つのオペレーティングシステムを実行することによって, リアルタイム ROS ノード (Nuttx 上) と非リアルタイム ROS ノード (Linux 上) を別々に実行するハイブリッド OS プラットフォームを提案した
\item [27] では, ロード可能なカーネルフレームワークである ROSCH-G が, CPU/GPU 調整メカニズムを備えた ROS のリアルタイム拡張として提案されている
\item しかし, これらの研究は, リアルタイムのタイミング制約を保証する分析方法を提供していないか, ROS の第 1 世代にのみ適用されている
    \end{itemize}
\end{frame}

\begin{frame}{}
    \begin{itemize}
        \item ROS2 については, 丸山ら
\item [23] は, さまざまなベンダー固有の DDS 実装の下でさまざまな QoS 構成を使用して経験的評価を実施した
\item [16] では, 2 つのノード間の最悪の場合のレイテンシが測定され, PREEMPT-RT パッチが適用された Linux カーネルシステムでデッドラインミスの動作が観察される
\item どちらの研究も, 測定ベースのアプローチを使用して ROS2 のパフォーマンスを評価したが, 正式なモデリングや分析は提供していない
    \end{itemize}
\end{frame}

\begin{frame}{}
    \begin{itemize}
        \item publisher/subscriber モデルまたは readexecute-write セマンティクスでのエンドツーエンドチェーンレイテンシの分析について, 多くの研究が行われてきた
\item パレンシア等マルチコアシステムで優先順位の制約があるタスクを分析するための提案されたアプローチ [24], 25]
\item [15, 28] では, 最悪の場合の応答時間に基づいて, タスクのエンドツーエンドレイテンシの上界を把握する方法が提示されている
\item クローダ等
\item [21], アブドラら
\item [7], およびベッカーら
\item [9] は, 固定優先度スケジューリングの下でチェーンのエンドツーエンドのレイテンシーを制限するための分析方法を提示した
\item 崔らによる最新の論文
\item [12]は, チェーンのエンドツーエンドのレイテンシを改善するために, チェーンベースの固定優先度スケジューリングを提案した
\item しかし, これらの分析アプローチはいずれも, スケジューリングモデルの違いにより, ROS2 に直接適用することはできない
    \end{itemize}
\end{frame}

\begin{frame}{}
    \begin{itemize}
        \item ROS2 処理チェーンのレイテンシーに関する文献は非常に限られている
\item 本論文の知る限り, [11] は ROS2 スケジューラーのモデル化とチェーンの応答時間分析の提供に関する唯一の研究である
\item その論文の著者は, エグゼキュータ内のコールバックスケジューリング動作を調査し, リソース予約を使用して, 特定のエグゼキュータのリソースの可用性をモデル化した
\item チェーンのエンドツーエンドのレイテンシーは, 最初にエグゼキュータ内のコールバックで構成される各サブチェーンの応答時間を計算し, 次に構成パフォーマンス分析に基づいてエグゼキュータ全体にまたがるサブチェーンの応答時間を追加することによって分析される (CPA) [18]
\item 彼らのアプローチは, デフォルトの ROS2 スケジューラーを分析するための土台を築きます
\item しかし, リソースを割り当てて, クリティカルチェーンのエンドツーエンドのレイテンシをさらに改善する方法については, 答えが出ていない
    \end{itemize}
\end{frame}
