% !TeX root = main.tex

\section{INTRODUCTION}
\label{sec: introduction}

\begin{frame}{ROS 2 アプリケーションのリアルタイム保証が困難な理由}
    \begin{enumerate}
        \item アプリケーションはデータ依存関係を持つコールバックのセットで構成されるチェーンを形成するため, システム設計者は, チェーンのエンドツーエンドレイテンシの安全な上界を知る必要がある
        \item 多くの先行研究がエンドツーエンドチェーンレイテンシの分析手法を提案しているが, それらは ROS 2 フレームワークに直接適用できない
    \end{enumerate}
\end{frame}

\begin{frame}{ROS 2 アプリケーションの分析に関する先行研究}
    ROS 2 スケジューリングアーキテクチャのリアルタイム性を正式に分析した研究は \cite{casini2019response} のみ
    \begin{block}{先行研究~\cite{casini2019response}の課題}
        エグゼキュータ内のスケジューリング動作にのみ焦点を当てており, マルチコアシステムのエグゼキュータと CPU コアへのスケジューリングエンティティの割り当ては考慮されていない
    \end{block}
\end{frame}

\begin{frame}{本論文の提案}
    \begin{itemize}
        \item マルチコア環境における ROS 2用 の優先度駆動型チェーン考慮スケジューラである PiCAS を提案
        \item 目的は, 重要度とタイミングの要件に基づいてチェーンのコールバック実行に優先度を付けることにより, エンドツーエンドのチェーンレイテンシを最小限に抑えること
        \item ROS 2「Eloquent Elusor」バージョンに PiCAS を実装し, 実際の組込みプラットフォームでそのパフォーマンスを評価する
    \end{itemize}
\end{frame}

\begin{frame}{貢献}
    \begin{itemize}
        \item  コールバックの優先度の割り当てと, マルチコアプラットフォームの CPU コアへのエグゼキュータ割り当て及び, エグゼキュータへのノードの割り当てを含む PiCAS の設計を提示
        \item PiCAS は, チェーンのエンドツーエンドのタイミング要件を尊重しながら, コールバックとエグゼキュータを実行する
        \item  PiCASフレームワークの下で, チェーンのエンドツーエンドレイテンシの安全な境界を提供する分析を開発
    \end{itemize}
\end{frame}
