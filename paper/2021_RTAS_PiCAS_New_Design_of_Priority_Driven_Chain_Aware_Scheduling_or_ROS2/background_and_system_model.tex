% !TeX root = main.tex

\section{BACKGROUND AND SYSTEM MODEL}
\label{sec: background_and_system_model}

\subsection{System model}
\label{ssec: system model}

\begin{frame}{}
    \begin{itemize}
        \item 始めに, 本論文で登場する表記法・用語の表を示す
        \item 基本的な表記法・用語は資料中で説明無しで使用する
        \item 別ファイルで開く・印刷するなどして, 常に参照できる状態にしておくことを推奨する
    \end{itemize}
\end{frame}

% !TeX root = main.tex

\begin{frame}{表記法・用語 1}
    \full{
        \begin{table}[tb]
            \adjustbox{max width=\textwidth, max height=\slideheight}{
                \centering\begin{tabular}{|c|l|} \hline
                    $N$      & タスクの数                 \\\hline
                    $M$      & プロセッサの個数           \\\hline
                    $\tau_i$ & タスク                     \\\hline
                    $T_i$    & $\tau_i$の周期             \\\hline
                    $D_i$    & $\tau_i$の相対デッドライン \\\hline
                    $J_{i}$  & $\tau_i$のジョブ           \\\hline
                \end{tabular}
            }
        \end{table}
    }
\end{frame}

\begin{frame}{表記法・用語 2}
    \full{
        \begin{table}[tb]
            \adjustbox{max width=\textwidth, max height=\slideheight}{
                \centering\begin{tabular}{|c|l|} \hline
                    $G\left(J_{i}\right)=\langle V, E\rangle$ & $J_{i}$のDAG                                 \\\hline
                    $V$                                       & 頂点のセット                                 \\\hline
                    $E$                                       & エッジのセット                               \\\hline
                    $v$                                       & ワークロードの一部                           \\\hline
                    $c(v)$                                    & $v$のWCET                                    \\\hline
                    $(u, v) \in E$                            & $u$ と $v$ の間の優先関係                    \\\hline
                    先行頂点                                  & エッジ$(u, v)$があるとき, $u$は$v$の先行頂点 \\\hline
                    後続頂点                                  & エッジ$(u, v)$があるとき, $v$は$u$の後続頂点 \\\hline
                    ソース頂点                                & 先行頂点を持たない頂点                       \\\hline
                    シンク頂点                                & 後続頂点を持たない頂点                       \\\hline
                \end{tabular}
            }
        \end{table}
    }
\end{frame}

\begin{frame}{表記法・用語 3}
    \full{
        \begin{table}[tb]
            \adjustbox{max width=\textwidth, max height=\slideheight}{
                \centering\begin{tabular}{|c|l|} \hline
                    適格     & 頂点は, その先行頂点が全て終了している場合に適格である \\\hline
                    完全パス & ソース頂点で始まり, シンク頂点で終わるパス             \\\hline
                    $C_i$    & $G(J_i)$内の全ての頂点の合計WCET                       \\\hline
                    $L_i$    & $G(J_i)$内のクリティカルパス上の全ての頂点の合計 WCET  \\\hline
                \end{tabular}
            }
        \end{table}
    }
\end{frame}

\begin{frame}{表記法・用語 4}
\full{
\begin{table}[tb]
\adjustbox{max width=\textwidth, max height=\slideheight}{
\centering\begin{tabular}{|c|l|} \hline
    $\Theta=\left\{\theta_{1}, \cdots, \theta_{|\Theta|}\right\}$ & アクティブVPのセット \\\hline
   $\theta_{z}$ & \tabml{$\Theta$ 内の $z$番目のアクティブVPの初期バジェット \\コンテキストから明白な場合は$\Theta$ 内の $z$番目のアクティブVPも示す} \\\hline

\end{tabular}
}
\end{table}
}
\end{frame}


\begin{frame}{コールバックモデル}
    \begin{itemize}
        \item システムには $M$ 個のコールバックがある
        \item 各コールバックはタイマコールバック・レギュラーコールバックのいずれか
              \begin{block}{タイマコールバック}
                  周期的にトリガされるコールバック
              \end{block}
              \begin{block}{レギュラーコールバック}
                  別のコールバックからのイベントによってトリガされるコールバック
              \end{block}
              \vspace{3mm}
        \item 各コールバックは 1 つのチェーンに含まれる
        \item コールバックの相対デッドライン・周期はチェーンの相対デッドライン・周期と同じ
    \end{itemize}
\end{frame}

\begin{frame}{ノードモデル}
    \begin{itemize}
        \item  $\mathcal{N}$ を使用して一連のノードを表す
              \vspace{-2mm}
              \begin{equation*}
                  \mathcal{N}=:\left\{n_{1}, \ldots, n_{j}, \ldots, n_{N}\right\}
              \end{equation*}
        \item 各ノード $n_{j}$ の利用率は次の式で与えられる
              \vspace{-2mm}
              \begin{equation*}
                  U\left(n_{j}\right)=\sum_{\forall t_{i}: \tau_{i} \in n_{j}} \frac{C_{i}}{T_{i}}
              \end{equation*}
        \item 同じノード内のコールバックを 2 つ以上のエグゼキュータに個別に割り当てることはできない
    \end{itemize}
\end{frame}

\begin{frame}{エグゼキュータモデル}
    \begin{itemize}
        \item エグゼキュータのセットを以下のように表す
              \vspace{-2mm}
              \begin{equation*}
                  \mathcal{E}=:\left\{e_{1}, \ldots, e_{j}, \ldots, e_{E}\right\}
              \end{equation*}

        \item $\mathcal{E}$ は優先度の降順, すなわち $\pi_{e_{j}}>\pi_{e_{j+1}}$ でソートされる
        \item 本論文では, 各エグゼキュータを 1 つの CPU コアに割り当て, 各コアのエグゼキュータを SCHED\_FIFO でスケジュールする
              \begin{block}{SCHED\_FIFO}
                  Linux の固定優先度プリエンプティブリアルタイムスケジューリングポリシーであり, 優先度の範囲は 1 から 99
              \end{block}
              \vspace{5mm}
        \item したがって, エグゼキュータの最大数は 99
    \end{itemize}
\end{frame}

\begin{frame}{チェーンモデル}
    \begin{itemize}
        \item 各チェーンは 1 つ以上のコールバックで構成される
        \item チェーン $\Gamma^{c}$ は次のように表される
              \vspace{-2mm}
              \begin{equation*}
                  \Gamma^{c}:=\left[\tau_{s}, \tau_{m 1}, \tau_{m 2}, \ldots, \tau_{e}\right]
              \end{equation*}
        \item チェーンの開始コールバックはタイマコールバックであり, その他はレギュラーコールバック
              \notes{イベントによってトリガされるチェーンの場合, チェーンの最初のレギュラーコールバックを, イベントの到着周期を持つタイマコールバックとしてモデル化できる~\cite{casini2019response}}
              \forme{\item 従来のタスクモデルでの周期的なリアルタイムタスクは, 本論文のモデルでは単一のタイマコールバックチェーンとして表すことができる}
    \end{itemize}
\end{frame}

\begin{frame}{ROS 2 における2重スケジューリング}
    \fullimage{ros2_scheduling}
\end{frame}

\forme{
    \begin{frame}{チェーン優先度の意味合い}
        \begin{itemize}
            \item チェーンの優先度 $\pi_{\Gamma^{c}}$ は, システム設計者がシステム内での重要性または重要性に基づいて指定すると仮定する
            \item これは元の ROS 2 フレームワークの一部ではないが, アプリケーションレベルの要件を満たすために, コールバックとエグゼキュータの優先度の割り当てを調整する必要があるため, セマンティクス優先度と呼ぶ
        \end{itemize}
    \end{frame}
}
