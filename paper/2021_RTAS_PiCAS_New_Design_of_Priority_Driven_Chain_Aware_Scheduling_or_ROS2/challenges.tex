% !TeX root = main.tex

\section{CHALLENGES}
\label{sec: challenges}

\begin{frame}{セクションサマリ}
    \begin{itembox}[l]{\textbf{目的}}
        組み込みプラットフォームを使用した実験から観察したスケジューリング動作に基づいて, 現在の ROS2 フレームワークの課題について詳しく説明する
    \end{itembox}
\end{frame}

\begin{frame}{実験環境1}
    \begin{itemize}
        \item 実験は, NVIDIA Xavier NX 上で動作する ROS2 "Eloquent Elusor" バージョンを使用して実施する
        \item ユニプロセッサシステムを想定する
    \end{itemize}
\end{frame}

\begin{frame}[label=exampleChain]{実験環境2}
    \fitimage{
        \begin{itemize}
            \item 10 個のコールバックで構成される 2 つのチェーンを使用する
            \item チェーン 1 はチェーン 2 よりも重要であるとする
        \end{itemize}
    }{challenge_model.png}
\end{frame}

\begin{frame}{実験環境3}
    次の 2 つのケースで実験を行う
    \begin{itemize}
        \item 全てのコールバックが単一のエグゼキュータにある
        \item 各チェーンごとに 1 つのエグゼキュータが割り当てられる
    \end{itemize}
\end{frame}

\begin{frame}{[課題 1] エグゼキュータ内の公平性に基づくスケジューリング}
    \begin{block}{エグゼキュータ内の公平性に基づくスケジューリング}
      \setlength{\linewidth}{0.98\columnwidth}
      \begin{itemize}
        \item タイマコールバックを常に最初にスケジュールし, 他のレギュラーコールバックの優先度割り当てを無効にする
        \item チェーンによってコールバックを区別しないため, 全てのコールバックはチェーンレベルの重要度を考慮せずにスケジュールされる
      \end{itemize}
    \end{block}
\end{frame}

\begin{frame}{[課題 1] 例}
    \fitimage{
        \begin{itemize}
            \item 図は, 2 つのチェーンが一緒に実行されるときの単一のエグゼキュータ内のコールバックのスケジューリングタイムライン
            \item ROS2 は両方のチェーンを公平に処理することが分かる
        \end{itemize}
    }{challenge1_example}
\end{frame}

\begin{frame}{[課題 1] 問題点}
    \fitimage{
        公平性に基づくスケジューリングは高いレイテンシが発生するため, セーフティクリティカルチェーンのリアルタイム性を危険にさらす可能性がある
    }{challenge1_result.png}
\end{frame}

\begin{frame}{[課題2] エグゼキュータの優先度割り当て}
    \begin{block}{エグゼキュータの優先度割り当て}
      \setlength{\linewidth}{0.98\columnwidth}
      \begin{itemize}
        \item デフォルトでは, エグゼキュータは Linux カーネルの Completely Fair Scheduler (CFS)~\cite{wong2008fairness} によってスケジュールされる
        \item CFSの下では, 重要度が高いチェーンからのコールバックを実行しているエグゼキュータに予測可能な優先度を付けることは困難
        \item 開発者は OS レベルの優先度をエグゼキュータに手動で割り当てることができるが, ROS2 は公式のインターフェイスを提供していない
        \item ROS2 エグゼキュータの優先度の割り当てに関する一般的なガイドラインは無い
      \end{itemize}
    \end{block}
\end{frame}

\begin{frame}{[課題2] 直観的な対策}
    \fitimage{
        \begin{block}{直観的な対策}
            重要なチェーンを実行しているエグゼキュータに最高の優先度を割り当てる
        \end{block}
        \vspace{5mm}
        チェーン1の全てのコールバックを含むエグゼキュータの優先度を99とし, チェーン2のエグゼキュータの優先度98とした場合のスケジュールタイムライン
    }{challenge2_timeline.png}
\end{frame}

\begin{frame}{[課題2] 直観的な方法の問題点}
    \fitimage{
        \begin{itemize}
            \item 自己干渉によるチェーン 2 の長いレイテンシの問題は解決されない
            \item 異なる重要度を持つチェーンからのコールバックが 1 つのノードに混在している場合, コールバックを異なるエグゼキュータに割り当てることができないため, エグゼキュータの優先度の割り当てはより困難になる
        \end{itemize}
    }{challenge2_result.jpg}
\end{frame}
