% !TeX root = main.tex

\section{CHALLENGES}
\label{sec: challenges}

\begin{frame}{セクションサマリ}
    \begin{itembox}[l]{\textbf{目的}}
        組み込みプラットフォームを使用した実験から観察したスケジューリング動作に基づいて, 現在の ROS2 フレームワークの課題について詳しく説明する
    \end{itembox}
\end{frame}

\begin{frame}{実験環境1}
    \begin{itemize}
        \item 実験は, NVIDIA Xavier NX 上で動作する ROS2 "Eloquent Elusor" バージョンを使用して実施する
        \item ユニプロセッサシステムを想定する
    \end{itemize}
\end{frame}

\begin{frame}{実験環境2}
    \fitimage{
        \begin{itemize}
            \item 表 に示すように, 10 個のコールバックで構成される 2 つのセーフティクリティカルチェーンを使用する
            \item チェーン 1 はチェーン 2 よりも重要であると想定する
        \end{itemize}
    }{challenge_model.png}
\end{frame}

\begin{frame}{実験環境3}
    次の 2 つのケースで実験を行う
    \begin{itemize}
        \item 全てのコールバックが単一のエグゼキュータにある
        \item 各チェーンのコールバックが別のエグゼキュータに割り当てられる, すなわち, チェーンごとに 1 つのエグゼキュータが割り当てられる
    \end{itemize}
\end{frame}

\begin{frame}{[課題 1] エグゼキュータ内の公平性に基づくスケジューリング}
    \begin{itemize}
        \item エクゼキュータ特有のスケジューリング動作は, タイマコールバックを常に最初にスケジュールし, 他のレギュラーコールバックの優先度割り当てを無効にするというものである
        \item さらに, ROS2スケジューラは, チェーンによってコールバックを区別しないため, 全てのコールバックは, チェーンレベルのタイミング要件の概念を考慮せずにスケジュールされる.
    \end{itemize}
\end{frame}

\begin{frame}{[課題 1] 例}
    \fitimage{
        \begin{itemize}
            \item 図は, 2 つのチェーンが一緒に実行されるときの単一のエグゼキュータ内のコールバックのスケジューリングタイムラインを示している
            \item ROS2 は両方のチェーンを公平に処理することが分かる
        \end{itemize}
    }{challenge1_example.png}
\end{frame}

\begin{frame}{[課題 1] 問題点}
    \fitimage{
        公平性に基づくスケジューリングは, 表に示すように非常に高いレイテンシが発生するため, セーフティクリティカルチェーンの適時性を危険にさらす可能性がある
    }{challenge1_result.png}
\end{frame}

\begin{frame}{[課題2] エグゼキュータの優先順位割り当て}
    \begin{itemize}
        \item デフォルトでは, エグゼキュータは Linux カーネルの Completely Fair Scheduler (CFS)~\cite{wong2008fairness} によってスケジュールされる
        \item このスケジューラの下では, クリティカルチェーンからのコールバックを実行しているエグゼキュータに予測可能な優先順位を付けることは困難である
        \item システム開発者は OS レベルの優先度をエグゼキュータに手動で割り当てることができるが, ROS2 はそれを構成するための公式のインターフェイスを提供していない
        \item さらに, ROS2 エグゼキュータ の優先順位の割り当てに関する一般的なガイドラインはない
    \end{itemize}
\end{frame}

\begin{frame}{[課題2] 直観的な方法}
    \fitimage{
        \begin{itemize}
            \item 最も直感的な方法は, 最も重要なチェーンを実行しているエグゼキュータに最高の優先度を割り当てること
            \item チェーン1の全てのコールバックを含むエグゼキュータの優先度を99とし, チェーン2の他のエグゼキュータの優先度98とした場合のスケジュールタイムラインを図に示す
        \end{itemize}
    }{challenge2_timeline.png}
\end{frame}

\begin{frame}{[課題2] 直観的な方法の問題点}
    \fitimage{
        \begin{itemize}
            \item 表に示すように, インスタンス間の自己干渉によって発生するチェーン 2 の長いレイテンシの問題は解決されない
            \item さらに, 異なる重要度を持つチェーンからのコールバックが 1 つのノードに混在している場合, コールバックを異なるエグゼキュータに割り当てることができないため, エグゼキュータの優先順位の割り当てはより困難になる
        \end{itemize}
    }{challenge2_result.jpg}
\end{frame}
