% !TeX root = main.tex

\section{CONCLUSION}
\label{sec: conslusion}

\begin{frame}{本論文の提案}
    \begin{itemize}
        \item 優先度駆動型のチェーン考慮スケジューリングと, そのエンドツーエンドレイテンシ解析フレームワークを提案した
        \item チェーンの重要度に基づいてエンドツーエンドレイテンシを削減するために, エグゼキュータ内およびエグゼキュータ間のコールバックのスケジューリング戦略を提案した
        \item スケジューリング戦略を実現するコールバック優先度割り当てとチェーン考慮ノード割り当てアルゴリズムを提示した
        \item NVIDIA Xavier NX プラットフォーム上で動作する ROS2 の Eloquent Elusor バージョンにスケジューラを実装した
    \end{itemize}
\end{frame}

\begin{frame}{評価結果}
    ケーススタディの結果は以下を示した
    \begin{itemize}
        \item 提案されたスケジューラが実際のシナリオでのエンドツーエンドレイテンシに関して, 既存の ROS2 スケジューリングよりも優れている
        \item 分析技術がレイテンシの上界を厳密に制限し, 最先端技術と比較してスケジューラビリティが向上する
    \end{itemize}
\end{frame}

\begin{frame}{Future work}
    \begin{itemize}
        \item Autoware.Auto などの ROS2 上に構築された自動運転ソフトウェアなど, より複雑で実用的なシナリオに作業を展開する
        \item DDS 通信の QoS の構成を含む, さまざまな ROS2 設定の作業を拡張する
    \end{itemize}
\end{frame}
