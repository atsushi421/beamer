% !TeX root = main.tex

\section{Conclusion}
\label{sec: conslusion}

\begin{frame}{}
    \begin{itemize}
        \item 本論文では, 優先度駆動型のチェーンを意識したスケジューリングと, ROS2 向けのそのエンドツーエンドのレイテンシ解析フレームワークを提案する
\item チェーンの重要度に基づいてエンドツーエンドのレイテンシを削減するために, エグゼキュータ内およびエグゼキュータ間のコールバックのスケジューリングポリシーを提案する
\item 次に, これらの戦略を実現するコールバック優先度割り当てとチェーン対応ノード割り当てアルゴリズムを提示する
\item NVIDIA Xavier NX プラットフォーム上で動作する ROS2 の Eloquent Elusor バージョンにチェーンウェアスケジューラを実装した
\item ケーススタディの結果は, 提案されたスケジューラが, 実際のシナリオでのエンドツーエンドのレイテンシに関して, 既存の ROS2 スケジューリングよりも優れていることを示している
\item また, 本論文の分析技術がレイテンシーの上界を厳密に制限し, 最先端技術と比較してスケジューリング可能率が向上することも示されている
    \end{itemize}
\end{frame}

\begin{frame}{}
    \begin{itemize}
        \item 次のステップとして, autoware.auto などの ROS2 上に構築された自動運転ソフトウェアなど, より複雑で実用的なシナリオに作業を展開する
\item また, DDS 通信の QoS の構成を含む, さまざまな ROS2 設定の作業を拡張する予定である
    \end{itemize}
\end{frame}
