% !TeX root = main.tex

\section{TASKS EXECUTED ON PASSIVE-VP}
\label{sec: tasks executed on passive-vp}

\begin{frame}{パッシブ VP}
    \begin{itemize}
        \item プロセッサ上のアクティブ VP によって残された未使用の処理能力は, パッシブ VP として他のタスクに割り当てられる
        \item $\tau_{i}$ は作業保存型方法で実行され, プリエンプションとマイグレーションの両方が許可される
        \item $\tau_{i}$ の現在のジョブの実行可能な頂点を実行するために, 任意のパッシブ VP を使用できる
        \item パッシブ VP はそのプロセッサでアクティブ VP が実行されていない場合にのみ実行できる
    \end{itemize}
\end{frame}

\begin{frame}{ビジー/使用不可/アイドル}
    \begin{block}{ビジー}
        パッシブ VP が何らかのワークロードを実行している場合はビジー
    \end{block}
    \begin{block}{使用不可}
        パッシブ VPはそのプロセッサ上のアクティブ VP が実行されている場合は使用不可
    \end{block}
    \begin{block}{アイドル}
        パッシブ VPは, 使用不可でなく, 何のワークロードも実行していない場合はアイドル
    \end{block}
\end{frame}

\begin{frame}{パッシブVPの例}
    \fitimage{
        \begin{itemize}
            \item ジョブ $J_{i}$ はアクティブ VP グループ $\Theta=\left\{\theta_{1}=\right.$  $\left.6, \theta_{2}=2\right\}$ によって処理され, 未使用の処理能力は, $J_{k}$ を処理するパッシブ VP グループを形成する
            \item $J_{k}$ はアクティブ VP がビジーでない場合にパッシブ VP で実行される

        \end{itemize}
    }{passive_vp}
\end{frame}

\forme{
    \begin{itemize}
        \item 一般に, パッシブ VP グループ内のパッシブ VP は, アクティブ VP グループ内の全てのアクティブ VP に対して必ずしも補完的ではないことに注意
        \item すなわち, パッシブ VP グループ内のパッシブ VP は, 異なるアクティブ VP グループのアクティブ VP を補完することができ, その逆も可能である
    \end{itemize}
}

\begin{frame}{供給境界関数}
    各パッシブ VP $\pi_{x}$ について, 供給境界関数 $s b f_{\pi_{x}}(\Delta)$ を導き出すことができる
    \begin{block}{供給境界関数 $s b f_{\pi_{x}}(\Delta)$}
        長さ $\Delta$ の任意の時間間隔で $\pi_{x}$ によって提供される処理能力の最小量を示す
    \end{block}
\end{frame}

\begin{frame}[label=lemma3]{Lemma 3}
    \begin{lemma}[]
        $\pi_{x}$ がアクティブ VP と対になるパッシブ VP であり, 初期バジェット $\theta_{z}$ がタスク $\tau_{i}$ に割り当てられ, タスク $\tau_{i}$ がデッドラインに間に合う場合, $\pi_{x}$ の供給境界関数は次のように計算される

        \begin{equation*}
            s b f_{\pi_{x}}(\Delta)= \begin{cases}0, & \Delta<\theta_{z} \\ \alpha(\Delta)+\gamma(\Delta), & \Delta \geq \theta_{z}\end{cases}
        \end{equation*}
    \end{lemma}
\end{frame}

\begin{frame}{Lemma 3 続き}
    $\alpha(\Delta)=\left\lfloor\frac{\Delta-\theta_{z}}{T_{i}}\right\rfloor\left(T_{i}-\theta_{z}\right)$,\\
    $\beta(\Delta)=\left(\Delta-\theta_{z}\right) \bmod T_{i}$,
    \begin{equation*}
        \gamma(\Delta)=\left\{\begin{array}{lr}
            \beta(\Delta),            & \beta(\Delta) \leq T_{i}-D_{i}                        \\
            T_{i}-D_{i},              & T_{i}-D_{i}<\beta(\Delta) \leq T_{i}-D_{i}+\theta_{z} \\
            \beta(\Delta)-\theta_{z}, & \beta(\Delta)>T_{i}-D_{i}+\theta_{z}
        \end{array}\right.
    \end{equation*}
\end{frame}

\begin{frame}{Lemma 3 例}
    \fullimage{lemma3_example}
\end{frame}

\begin{frame}{キーパス}
    \begin{definition}[キーパス]
        \setlength{\linewidth}{0.98\columnwidth}
        \begin{itemize}
            \item タスク $\tau_{i}$ のジョブ $J_{i}$のキーパスを $\lambda=\left\{v_{1}, \cdots, v_{|\lambda|}\right\}$ と表記する
            \item キーパスは以下の条件を満たす
            \begin{itemize}
                \item $\forall k: 1<k \leq|\lambda|$, $v_{k-1}$ は $v_{k}$ の先行頂点であり, $v_{k}$ の全ての先行頂点の中で最も遅い終了時刻を持つ
                \item 最後の頂点 $v_{|\lambda|}$ は $G\left(J_{i}\right)$ の全ての頂点の中で最も遅い終了時間を持つ
            \end{itemize}
        \end{itemize}
    \end{definition}
\end{frame}

\begin{frame}{キーパスの例}
    \fitimage{
        $J_{k}$ のキーパスは $\left\{u_{1}, u_{3}\right\}$ であり, これは $\left\{u_{1}, u_{2}\right\}$ のクリティカルパスとは異なる
    }{key_path_example}
\end{frame}

\begin{frame}[label=lemma4]{Lemma 4}
    \begin{lemma}[]
        \begin{itemize}
            \item タスク $\tau_{i}$ がパッシブ VP グループ $\Pi$ によって処理されるとする
            \item $J_{i}$ が $r$ の時点でリリースされ, $f$ の時点で終了した $\tau_{i}$ のジョブであるとする
            \item $\lambda$ を $J_{i}$ のキーパスとする
            \item $\lambda$ が実行されていない $[r, f)$ の任意の時点で, $\Pi$ のパッシブ VP がアイドル状態になることはない
        \end{itemize}
    \end{lemma}
\end{frame}

\begin{frame}[label=theorem2]{Theorem 2}
    \begin{theorem}[]
        次の場合, タスク $\tau_{i}$ はパッシブ VP グループ $\Pi=$  $\left\{\pi_{1}, \cdots, \pi_{|\Pi|}\right\}$ でスケジュール可能

        \begin{equation*}
            C_{i} \leq L_{i}+\sum_{\pi_{x} \in \Pi} \max \left(s b f_{\pi_{x}}\left(D_{i}\right)-L_{i}, 0\right)
        \end{equation*}
    \end{theorem}
\end{frame}
