% !TeX root = main.tex

\section{TASKS EXECUTED ON PASSIVE-VP}
\label{sec: tasks executed on passive-vp}

\begin{frame}{}
    \begin{itemize}
        \item プロセッサ上のアクティブ VP によって残された未使用の処理能力は, パッシブ $V P$ として他のタスクに割り当てられる
\item このパッシブ VP は, そのアクティブ VP を補完するものである
\item パッシブ VP グループ $\Pi=\left\{\pi_{1}, \cdots, \pi_{|\Pi|}\right\}$ をタスク $\tau_{i}$ に割り当て, $\tau_{i}$ のジョブを $\Pi$ で実行する場合, $\Pi$ が $\tau_{i}$ にサービスを提供していると言えます
\item パッシブ VP は, 何らかのワークロードを実行している場合はビジーであり, そのプロセッサ上のアクティブ VP が実行されている場合は利用できず, そのプロセッサ上のアクティブ VP が実行されておらず, パッシブ VP も実行されていない場合はアイドル状態である
\item $\tau_{i}$ は作業保存型方法で $\Pi$ で実行される
\item すなわち, $\tau_{i}$ の現在のジョブに適格な頂点があるが現在実行されていない場合, $\Pi$ のパッシブ VP はアイドル状態になることはない
\item アクティブ VP でのスケジューリングと同様に, プリエンプションと移行の両方が許可される
\item $\tau_{i}$ が実行のために $\Pi$ でパッシブ VP を選択するための優先度はない
\item すなわち, 現在のジョブの適格な頂点を実行するために, 利用可能な任意のパッシブ VP を使用できる
\item パッシブ VP は優先度の低いプロセッサで実行される
\item すなわち, そのプロセッサでアクティブ VP が実行されていない場合にのみ実行できる
    \end{itemize}
\end{frame}

\begin{frame}{}
    \begin{itemize}
        \item 図 5 では, ジョブ $J_{i}$ はアクティブ VP グループ $\Theta=\left\{\theta_{1}=\right.$  $\left.6, \theta_{2}=2\right\}$ によって処理され, これら 2 つのプロセッサの未使用の処理能力は, $J_{k}$ を処理するパッシブ VP グループを形成する
\item 図 5-(b) は実行シーケンスを示しており, $J_{k}$ は, その相補的なアクティブ VP がビジーでない場合にのみパッシブ VP で実行される
\item 一般に, パッシブ VP グループ内のパッシブ VP は, アクティブ VP グループ内の全てのアクティブ VP に対して必ずしも補完的ではないことに注意
\item すなわち, パッシブ VP グループ内のパッシブ VP は, 異なるアクティブ VP グループのアクティブ VP を補完することができ, その逆も可能である
    \end{itemize}
\end{frame}

\begin{frame}{}
    \begin{itemize}
        \item 各パッシブ VP $\pi_{x}$ について, $s b f_{\pi_{x}}(\Delta)$ で示されるその供給境界関数 [13] を導き出すことができる
\item これは, サイズ $\Delta$ の任意の時間間隔で $\pi_{x}$ によって提供される処理能力の最小量を特徴付ける
    \end{itemize}
\end{frame}

\begin{frame}[label=lemma3]{Lemma 3}
    \begin{lemma}[]
        $\pi_{x}$ がアクティブ VP を補完するパッシブ VP であり, 初期予算 $\theta_{z}$ がタスク $\tau_{i}$ に割り当てられ, タスク $\tau_{i}$ は常にそのデッドラインに間に合うと仮定すると, $\pi_{x}$ の供給境界関数は次のように計算される

        \begin{equation*}
            s b f_{\pi_{x}}(\Delta)= \begin{cases}0, & \Delta<\theta_{z} \\ \alpha(\Delta)+\gamma(\Delta), & \Delta \geq \theta_{z}\end{cases}
        \end{equation*}

        ここで $\alpha(\Delta)=\left\lfloor\frac{\Delta-\theta_{z}}{T_{i}}\right\rfloor\left(T_{i}-\theta_{z}\right), \beta(\Delta)=\left(\Delta-\theta_{z}\right) \bmod T_{i}$

        \begin{equation*}
            \gamma(\Delta)=\left\{\begin{array}{lr}
                \beta(\Delta),            & \beta(\Delta) \leq T_{i}-D_{i}                        \\
                T_{i}-D_{i},              & T_{i}-D_{i}<\beta(\Delta) \leq T_{i}-D_{i}+\theta_{z} \\
                \beta(\Delta)-\theta_{z}, & \beta(\Delta)>T_{i}-D_{i}+\theta_{z}
            \end{array}\right.
        \end{equation*}
    \end{lemma}
\end{frame}

\begin{frame}{}
    \begin{itemize}
        \item 例えば, Lemma 3 によると, 図 5 の 2 つのプロセッサ上のパッシブ VP の sbf を図 7 に示す
    \end{itemize}
\end{frame}

\begin{frame}{}
    \begin{itemize}
        \item 次に, 重要な概念キーパスを紹介する
    \end{itemize}
\end{frame}

\begin{frame}{}
    \begin{definition}[キーパス]
        タスク $\tau_{i}$ のジョブ $J_{i}$ のキーパスは, $G\left(J_{i}\right)$ 内の完全なパス $\lambda=\left\{v_{1}, \cdots, v_{|\lambda|}\right\}$ であり, $\forall k: 1<k \leq|\lambda|$, $v_{k-1}$ は $v_{k}$ の先行であり, $v_{k}$ の全ての先行の中で最も遅い終了時刻を持ち, 最後の頂点 $v_{|\lambda|}$ は最新である $G\left(J_{i}\right)$ の全ての頂点の終了時間
    \end{definition}
\end{frame}

\begin{frame}{}
    \begin{itemize}
        \item ジョブ $J_{i}$ のキーパスは, $G\left(J_{i}\right)$ の最長パスとは異なる可能性があることに注意
\item $J_{i}$ のキーパスは, そのスケジューリングシーケンスがわかっている場合にのみ決定されるが, $G\left(J_{i}\right)$ の最長パスは, ジョブの実際の実行シーケンスとは無関係の静的情報である
\item 例えば, 図 5-(b) に示す実行シーケンスの $J_{k}$ のキーパスは $\left\{u_{1}, u_{3}\right\}$ であり ($u_{3}$ は終了する最後の頂点であるため), これは $\left\{u_{1}, u_{2}\right\}$ の最長パスとは異なる
    \end{itemize}
\end{frame}

\begin{frame}[label=lemma4]{Lemma 4}
    \begin{lemma}[]
        \begin{itemize}
            \item タスク $\tau_{i}$ がパッシブ VP グループ $\Pi$ によってサービスされ, $J_{i}$ が $r$ の時点でリリースされ, $f$ の時点で終了した $\tau_{i}$ のジョブであるとする
\item $\lambda$ を $J_{i}$ のキーパスとする
\item $\lambda$ が実行されていない $[r, f)$ の任意の時点で, $\Pi$ のパッシブ VP がアイドル状態になることはない
        \end{itemize}
    \end{lemma}
\end{frame}

\begin{frame}[label=theorem2]{Theorem 2}
    \begin{theorem}[]
        次の場合, タスク $\tau_{i}$ はパッシブ VP グループ $\Pi=$  $\left\{\pi_{1}, \cdots, \pi_{|\Pi|}\right\}$ でスケジュール可能である

        \begin{equation*}
            C_{i} \leq L_{i}+\sum_{\pi_{x} \in \Pi} \max \left(s b f_{\pi_{x}}\left(D_{i}\right)-L_{i}, 0\right)
        \end{equation*}
    \end{theorem}
\end{frame}
