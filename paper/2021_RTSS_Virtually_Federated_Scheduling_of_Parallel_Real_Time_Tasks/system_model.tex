% !TeX root = main.tex

\section{SYSTEM MODEL}
\label{sec: system model}

\begin{frame}{}
    \begin{itemize}
        \item 始めに, 本論文で登場する表記法・用語の表を示す
        \item 基本的な表記法・用語は資料中で説明無しで使用する
        \item 別ファイルで開く・印刷するなどして, 常に参照できる状態にしておくことを推奨する
    \end{itemize}
\end{frame}

% !TeX root = main.tex

\begin{frame}{表記法・用語 1}
    \full{
        \begin{table}[tb]
            \adjustbox{max width=\textwidth, max height=\slideheight}{
                \centering\begin{tabular}{|c|l|} \hline
                    $N$      & タスクの数                 \\\hline
                    $M$      & プロセッサの個数           \\\hline
                    $\tau_i$ & タスク                     \\\hline
                    $T_i$    & $\tau_i$の周期             \\\hline
                    $D_i$    & $\tau_i$の相対デッドライン \\\hline
                    $J_{i}$  & $\tau_i$のジョブ           \\\hline
                \end{tabular}
            }
        \end{table}
    }
\end{frame}

\begin{frame}{表記法・用語 2}
    \full{
        \begin{table}[tb]
            \adjustbox{max width=\textwidth, max height=\slideheight}{
                \centering\begin{tabular}{|c|l|} \hline
                    $G\left(J_{i}\right)=\langle V, E\rangle$ & $J_{i}$のDAG                                 \\\hline
                    $V$                                       & 頂点のセット                                 \\\hline
                    $E$                                       & エッジのセット                               \\\hline
                    $v$                                       & ワークロードの一部                           \\\hline
                    $c(v)$                                    & $v$のWCET                                    \\\hline
                    $(u, v) \in E$                            & $u$ と $v$ の間の優先関係                    \\\hline
                    先行頂点                                  & エッジ$(u, v)$があるとき, $u$は$v$の先行頂点 \\\hline
                    後続頂点                                  & エッジ$(u, v)$があるとき, $v$は$u$の後続頂点 \\\hline
                    ソース頂点                                & 先行頂点を持たない頂点                       \\\hline
                    シンク頂点                                & 後続頂点を持たない頂点                       \\\hline
                \end{tabular}
            }
        \end{table}
    }
\end{frame}

\begin{frame}{表記法・用語 3}
    \full{
        \begin{table}[tb]
            \adjustbox{max width=\textwidth, max height=\slideheight}{
                \centering\begin{tabular}{|c|l|} \hline
                    適格     & 頂点は, その先行頂点が全て終了している場合に適格である \\\hline
                    完全パス & ソース頂点で始まり, シンク頂点で終わるパス             \\\hline
                    $C_i$    & $G(J_i)$内の全ての頂点の合計WCET                       \\\hline
                    $L_i$    & $G(J_i)$内のクリティカルパス上の全ての頂点の合計 WCET  \\\hline
                \end{tabular}
            }
        \end{table}
    }
\end{frame}

\begin{frame}{表記法・用語 4}
\full{
\begin{table}[tb]
\adjustbox{max width=\textwidth, max height=\slideheight}{
\centering\begin{tabular}{|c|l|} \hline
    $\Theta=\left\{\theta_{1}, \cdots, \theta_{|\Theta|}\right\}$ & アクティブVPのセット \\\hline
   $\theta_{z}$ & \tabml{$\Theta$ 内の $z$番目のアクティブVPの初期バジェット \\コンテキストから明白な場合は$\Theta$ 内の $z$番目のアクティブVPも示す} \\\hline

\end{tabular}
}
\end{table}
}
\end{frame}


\begin{frame}{タスクセットの定義1}
    \begin{itemize}
        \item $N$ 個の散発的なタスク $\left\{\tau_{1}, \tau_{2}, \ldots, \tau_{N}\right\}$ が $M$ 個の同質プロセッサにスケジュールされているタスクセットを考える
        \item 各タスク $\tau_{i}$ は, ジョブの無限シーケンスをリリースする
        \item $\tau_{i}$ の $T_{i}$ の周期は, $\tau_{i}$ の 2 つの連続するジョブのリリース時刻の間の最小間隔
        \item 各タスク $\tau_{i}$ には相対デッドライン $D_{i}$ がある
              \begin{itemize}
                  \item 時間 $r$ でリリースされたタスク $\tau_{i}$ のジョブは, その絶対デッドライン $r+D_{i}$ までに終了する必要がある
              \end{itemize}
        \item 全てのタスク $\tau_{i}$ が $D_{i} \leq T_{i}$ を満たすとする
    \end{itemize}
\end{frame}

\begin{frame}{タスクセットの定義2}
    \begin{itemize}
        \item 各ジョブには, $G\left(J_{i}\right)=\langle V, E\rangle$ で示される有向非巡回グラフ (DAG) によってモデル化できる並列ワークロード構造がある
        \item 頂点 $v \in V$ は, 順次実行されるワークロードの一部をモデル化し, 最悪実行時間 (WCET) $c(v)$ によって特徴付けられる
        \item エッジ $(u, v) \in E$ は, $u$ と $v$ の間の優先関係を表す
        \item 各 DAG には, 一つのソース頂点と一つのシンク頂点があると仮定する
    \end{itemize}
\end{frame}

\begin{frame}{アプローチに必要な情報}
    \begin{itemize}
        \item 各ジョブのワークロードは DAG によってモデル化できると想定するが, 実際にはジョブの正確な DAG 構造を知る必要はない
        \item ボリューム $C_{i}$ とクリティカルパス長 $L_{i}$ の 2 つのパラメーターに関する制約を満たすタスクのジョブのみが必要
    \end{itemize}
\end{frame}

\begin{frame}{ジョブのDAG構造の例}
    \fullimage{dag_example}
\end{frame}

\begin{frame}{タスクの前提}
    本論文では各タスク $\tau_i$に対して以下を仮定する
    \begin{itemize}
        \item $C_{i}>L_{i}$, すなわち, ヘビータスクのみを対象とする
        \item $C_{i}>D_{i}$, すなわち, デッドラインに間に合わせるために $\tau_{i}$ を並行して実行する必要がある
        \notes{本論文で開発された方法は, $C_{i} \leq D_{i}$ を使用したタスク (ライトタスク) を含む場合にも簡単に拡張できる}
    \end{itemize}
\end{frame}
