% !TeX root = main.tex

\section{RELATED WORK}
\label{sec: related work}

\begin{frame}{}
    \begin{itemize}
        \item 並列タスクのリアルタイムスケジューリングに関する初期の研究では, 制限されたタスク構造を想定する [23]-[31]
\item 例えば, Gang EDF スケジューリングアルゴリズムは, [24] で成形可能な並列タスク用に提案された
\item 並列同期タスクモデルは, [25]-[31] で研究された
\item 近年, 多くの並列ソフトウェアシステムのワークロードを DAG タスクモデルまたはその変形としてモデル化できるため, DAG ベースの並列リアルタイムタスクがますます注目を集めている
    \end{itemize}
\end{frame}

\begin{frame}{}
    \begin{itemize}
        \item DAG リアルタイムタスクの既存のスケジューリングアルゴリズムは, 4 つの主要なパラダイム (グローバルスケジューリング, 分割スケジューリング, フェデレートスケジューリング, および分解ベースのスケジューリング) に分類できる
    \end{itemize}
\end{frame}

\begin{frame}{}
    \begin{itemize}
        \item グローバルスケジューリングでは, 全てのタスクタスクが全てのプロセッサで一緒にスケジュールされる
\item [32] と [6] では, グローバル EDF (GEDF) とグローバル RM (GRM) スケジューリングの両方に対して線形時間テストが提案されている
\item 疑似多項式の時間十分なスケジューラビリティテストも [32] で提示された
\item これは後に一般化され, [33] によって制限されたデッドライン DAG タスクに対して支配された
\item [7] では, GEDF スケジューリングの下で暗黙のデッドラインを持つ DAG タスクのスケジューラビリティ分析を実行するために, 利用率の限界が提案された
\item GEDF または GRM スケジューリング [2], [16]-[19], [34] の下で DAG タスクをスケジューリングするための応答時間分析手法も提案されている
    \end{itemize}
\end{frame}

\begin{frame}{}
    \begin{itemize}
        \item パーティショニングスケジューリングでは, DAGタスクの各頂点が静的にプロセッサに割り当てられ, 各プロセッサのワークロードは自己中断を伴うリアルタイムタスクとしてモデル化され, 分析される[10], [20]
\item 連合スケジューリングでは, 各タスクは専用のプロセッサ群を割り当てられ, 実行時にその上で排他的に実行される.連合スケジューリングはLi.et.al.によって最初に提案され[1], 暗黙のデッドラインを持つタスクに対して, (容量拡張境界の観点から)最高の定量的最悪ケース性能保証を提供するものであった.その後, 連合スケジューリングは制約付き締切タスク[35], 任意締切タスク[36], さらに条件分岐を持つDAG[37]へと一般化された.連合スケジューリングは, タスク間で一定の資源共有を可能にするいくつかのバリエーションに拡張される.セミフェデレートスケジューリングアプローチ[4]は, 異なる重いタスクが1つまたは2つのプロセッサを他のタスクと共有することを可能にします.予約ベースの連合型アプローチ[5]は, 従来のシーケンシャルタスクのためのアルゴリズムによってスケジューリングされた予約の集合に, 各タスクをディスパッチさせるものである.セクションIで述べたように, [4]と[5]はともに連合スケジューリングの資源浪費の問題を限定的にしか解決していない.
    \end{itemize}
\end{frame}

\begin{frame}{段落続き}
    \begin{itemize}
        \item 本論文で開発された仮想的フェデレーションアプローチは, タスク間でより積極的にプロセッサを共有することができます.分解ベースのスケジューリングでは, 各DAGは, 独自のリリース時間と相対デッドラインを持つシーケンシャルなサブタスクの集合に変換され, シーケンシャルタスクのための従来のマルチプロセッサスケジューリングアルゴリズムによってスケジューリングされる.分解はタスクの内部構造に従ってオフラインで行われるため, 例えば条件分岐を持つDAGタスクのように, 実行中にDAG構造が変化する可能性がある状況では, 実装が困難である.いくつかの異なる分解方法とその基礎となるスケジューラビリティテストが開発された [11], [12], [38], [39]
    \end{itemize}
\end{frame}
