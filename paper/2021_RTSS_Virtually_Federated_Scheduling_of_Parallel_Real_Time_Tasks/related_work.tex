% !TeX root = main.tex

\section{RELATED WORK}
\label{sec: related work}

\begin{frame}{}
    \begin{itemize}
        \item 並列タスクのリアルタイム スケジューリングに関する初期の研究では, 制限されたタスク構造を想定している [23]-[31].たとえば, Gang EDF スケジューリング アルゴリズムは, [24] で成形可能な並列タスク用に提案された.並列同期タスク モデルは, [25]-[31] で研究された.近年, 多くの並列ソフトウェア システムのワークロードを DAG タスク モデルまたはその変形としてモデル化できるため, DAG ベースの並列リアルタイム タスクがますます注目を集めている.
    \end{itemize}
\end{frame}

\begin{frame}{}
    \begin{itemize}
        \item DAG リアルタイム タスクの既存のスケジューリング アルゴリズムは, 4 つの主要なパラダイム (グローバル スケジューリング, 分割スケジューリング, フェデレートスケジューリング, および分解ベースのスケジューリング) に分類できる.
    \end{itemize}
\end{frame}

\begin{frame}{}
    \begin{itemize}
        \item グローバル スケジューリングでは, すべてのタスク タスクがすべてのプロセッサで一緒にスケジュールされる. [32] と [6] では, グローバル EDF (GEDF) とグローバル RM (GRM) スケジューリングの両方に対して線形時間テストが提案されている.疑似多項式の時間十分なスケジューリング可能性テストも [32] で提示された.これは後に一般化され, [33] によって制限されたデッドライン DAG タスクに対して支配された. [7] では, GEDF スケジューリングの下で暗黙のデッドラインを持つ DAG タスクのスケジューリング可能性分析を実行するために, 使用率の限界が提案された. GEDF または GRM スケジューリング [2], [16]-[19], [34] の下で DAG タスクをスケジューリングするための応答時間分析手法も提案されている.
    \end{itemize}
\end{frame}

\begin{frame}{}
    \begin{itemize}
        \item パーティショニングスケジューリングでは、DAGタスクの各頂点が静的にプロセッサに割り当てられ、各プロセッサの作業負荷は自己中断を伴うリアルタイムタスクとしてモデル化され、分析されます[10], [20].連合スケジューリングでは、各タスクは専用のプロセッサ群を割り当てられ、実行時にその上で排他的に実行される。連合スケジューリングはLi.et.al.によって最初に提案され[1]、暗黙のデッドラインを持つタスクに対して、(容量拡張境界の観点から)最高の定量的最悪ケース性能保証を提供するものであった。その後、連合スケジューリングは制約付き締切タスク[35]、任意締切タスク[36]、さらに条件分岐を持つDAG[37]へと一般化された。連合スケジューリングは、タスク間で一定の資源共有を可能にするいくつかのバリエーションに拡張される。セミフェデレートスケジューリングアプローチ[4]は、異なる重いタスクが1つまたは2つのプロセッサを他のタスクと共有することを可能にします。予約ベースの連合型アプローチ[5]は、従来のシーケンシャルタスクのためのアルゴリズムによってスケジューリングされた予約の集合に、各タスクをディスパッチさせるものである。セクションIで述べたように、[4]と[5]はともに連合スケジューリングの資源浪費の問題を限定的にしか解決していない。
    \end{itemize}
\end{frame}

\begin{frame}{段落続き}
    \begin{itemize}
        \item 本論文で開発された仮想的フェデレーションアプローチは、タスク間でより積極的にプロセッサを共有することができます。分解ベースのスケジューリングでは、各DAGは、独自のリリース時間と相対的なデッドラインを持つシーケンシャルなサブタスクの集合に変換され、シーケンシャルタスクのための従来のマルチプロセッサスケジューリングアルゴリズムによってスケジューリングされます。分解はタスクの内部構造に従ってオフラインで行われるため、例えば条件分岐を持つDAGタスクのように、実行中にDAG構造が変化する可能性がある状況では、実装が困難である。いくつかの異なる分解方法とその基礎となるスケジューリング可能性テストが開発された [11], [12], [38], [39].
    \end{itemize}
\end{frame}
