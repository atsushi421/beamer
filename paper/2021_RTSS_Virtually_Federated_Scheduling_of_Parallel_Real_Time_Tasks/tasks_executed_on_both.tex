% !TeX root = main.tex

\section{TASKS EXECUTED ON BOTH}
\label{sec: tasks executed on both}

\begin{frame}{アクティブ/パッシブ VP両方で処理するケース}
    \begin{itemize}
        \item タスク $\tau_{i}$ をアクティブ VP グループ $\Theta$ で部分的に実行し, パッシブ VP グループで部分的に実行することも可能
        \item タスクが実行可能なアクティブ VP とパッシブ VP を選択する順序は以下
        \begin{enumerate}
            \item リーディングアクティブ VP
            \item 非リーディングアクティブ VP
            \item パッシブ VP
        \end{enumerate}
    \end{itemize}
\end{frame}

\begin{frame}{両方での処理例}
    \fitimage{
        $J_{i}$ は初期バジェット 5 と 2 の 2 つのアクティブ VPで実行され, $J_{k}$ は初期バジェット 6 の 1 つのアクティブ VP と $J_{i}$ を処理する 2 つのアクティブ VP の対の 2 つのパッシブ VP で実行されるとする
    }{both_executed}
\end{frame}

\begin{frame}[label=theorem3]{Theorem 3}
    \begin{theorem}[]
        \setlength{\linewidth}{0.98\columnwidth}
        \begin{itemize}
            \item タスク $\tau_{i}$ が $\langle\Theta, \Pi\rangle$ で実行されると仮定する
            \begin{itemize}
                \item \desc{$\Theta$}{$C_{i}$ より小さい総初期バジェットを持つアクティブ VP グループ}
                \item \desc{$\Pi$}{パッシブ VP グループ}
            \end{itemize}
        \end{itemize}
        \begin{itemize}
            \item 次の条件が全て満たされている場合, $\tau_{i}$ はスケジュール可能
            \begin{itemize}
                \item $C_{i} \leq \sum_{\theta_{z} \in \Theta} \theta_{z}+\sum_{\pi_{x} \in \Pi} \max \left(s b f_{\pi_{x}}\left(D_{i}\right)-L_{i}, 0\right)$
                \item $\theta_{1} \leq D_{i}$
                \item $\forall \theta_{z} \in \Theta \backslash\left\{\theta_{1}\right\}: \theta_{z} \leq D_{i}-L_{i}$
            \end{itemize}
        \end{itemize}
    \end{theorem}
\end{frame}

\forme{
    \begin{frame}{}
        \begin{itemize}
            \item 最後に、定理1、2、3のスケジューラビリティテスト条件は、CiとLiに関して持続可能であることを指摘したい。つまり、あるジョブのCiと/またはLiが小さくなっても(またはあるジョブが実行時に最悪のCiと/またはLiまで実行しなくても)これらの定理の条件は、タスクのスケジューラビリティを保証するために有効であるということである。一般にDAGタスクのスケジューリングはタイミング異常、すなわちタスクが最悪負荷まで実行されたときに実際の最悪応答時間が発生するとは限らないが、定理1、2、3の持続的スケジューラビリティ試験条件を用いる場合はタイミング異常を気にする必要はない。
        \end{itemize}
    \end{frame}
}
