% !TeX root = main.tex

\section{INTRODUCTION}
\label{sec: introduction}

\begin{frame}{}
    \begin{itemize}
        \item マルチコア プロセッサは, リアルタイム システムでますます広く使用され, 高性能と低消費電力に対する急速に高まる要件に対応している.マルチコアの計算能力を最大限に活用するには, ソフトウェアを並列化する必要がある.シーケンシャル リアルタイム タスクの既存のスケジューリングおよび分析手法は, 並列ワークロード設定に移行するのが困難である.マルチコアに並列リアルタイム タスク システムを効率的に展開するには, 新しいスケジューリングと分析の手法が必要である
    \end{itemize}
\end{frame}

\begin{frame}{}
    \begin{itemize}
        \item リアルタイム並列タスクのスケジューリングには様々なアプローチがあるが、その中でもフェデレーテッドスケジューリング[1]はリアルタイム性能と高い柔軟性を両立する有望なアプローチである。フェデレーテッドスケジューリングでは、各タスクは専用のプロセッサのサブセット上で排他的に実行されます(ここでは、納期を守るために並列実行が必要な密度1以上の「重い」タスクを意味し、密度1以下の「軽い」タスクは、他のプロセッサ上で従来の逐次リアルタイムタスクスケジューリングアルゴリズムを用いてスケジュールすれば、その上に処理することが可能です)。連合スケジューリングは、他の手法ではスケジューリングできない並列リアルタイムタスクシステムの大部分をうまくスケジューリングできるだけでなく、定量的に最も優れた最悪ケース性能保証(容量拡張境界の観点)を提供することができます[1]。一方、連合スケジューリングは、タスクグラフの最長経路長や総作業量といった抽象的な情報のみを必要とする解析技術であるため、柔軟な作業量指定が可能であり、条件付きDAGタスク[2]やOpenMPタスクモデル[3]など、より表現力の高い並列実時間タスクモデルに拡張する可能性も十分に秘めています。
    \end{itemize}
\end{frame}

\begin{frame}{}
    \begin{itemize}
        \item ただし, 各タスクは独自のプロセッサで排他的に実行され, これらのプロセッサの未使用の処理能力は他のタスクで使用できないため, フェデレート スケジューリングは依然としてかなりのリソースの浪費に悩まされている.セクション III で示すように, 無駄な処理能力の比率は, 極端なケースでは任意に $100 \%$ に近くなる可能性がある.この問題に対処するために, 以前の作業が行われました. [4] は, セミフェデレート スケジューリング アプローチを提案した.これにより, タスクは未使用の処理能力を他のタスクと共有できる.ただし, セミフェデレート スケジューリングでの共有は非常に限られている (タスクに割り当てられるプロセッサの数に関係なく, 共有される処理能力の量は最大で 1 つのプロセッサである). [5] は, ジョブの絶対デッドラインと次のジョブのリリース時刻の間に未使用の処理能力を再利用できる, 予約ベースのスケジューリング アプローチを開発した.ただし, タスクの相対的なデッドラインがその期間に近い場合, パフォーマンスの向上はわずかである.要約すると, [4] と [5] はどちらも, フェデレート スケジューリングの処理能力浪費の問題を限られた範囲で解決するだけである.
    \end{itemize}
\end{frame}

\begin{frame}{}
    \begin{itemize}
        \item グローバル スケジューリング [6]-[8], 分割スケジューリング [9], [10], および分解ベースのスケジューリング [11], [12] など, タスクが互いにプロセッサを共有できるようにする他のタイプのスケジューリング アルゴリズムがある. .ただし, 分析的に保証されたスケジューリング可能性は, 実際にはフェデレート スケジューリングよりも劣っている.これは主に, フェデレート スケジューリングでの排他的なプロセッサの所有権により, 各タスクのスケジューリングと分析が大幅に簡素化され, 他のタイプのスケジューリング アルゴリズムよりも優れたスケジューリング可能性が分析的に保証されるためである.
    \end{itemize}
\end{frame}

\begin{frame}{}
    \begin{itemize}
        \item 本論文では, フェデレート スケジューリングの優れた分析可能性を享受し, タスクが互いにプロセッサを効率的に共有できるようにする仮想フェデレート スケジューリング アプローチを紹介する.主なアイデアは, 物理プロセッサ上に仮想プロセッサを構築し, 各タスクを専用の仮想プロセッサ セットで排他的に実行することである.物理プロセッサが仮想プロセッサによって共有されるため, タスクは互いにプロセッサを効果的に共有し, その結果, システム全体のリソース使用率が向上する.一方, 各タスクは独自の仮想プロセッサ セットで排他的に実行されるため, フェデレート スケジューリングの優れた分析可能性は, 仮想プロセッサを使用した新しいアプローチに持ち込むことができる.包括的なパフォーマンス評価を実施して, アプローチを既存のさまざまなタイプのスケジューリング/分析手法と比較する.実験結果は, 仮想フェデレーション スケジューリング アプローチが, さまざまなパラメーター設定の下で, さまざまな種類の既存のすべての方法よりも一貫して優れていることを示している.
    \end{itemize}
\end{frame}
