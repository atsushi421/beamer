% !TeX root = main.tex

\section{TASKS EXECUTED ON ACTIVE-VP}
\label{sec: tasks executed on active-vp}

\begin{frame}{アクティブ VP グループ}
    \begin{itemize}
        \item アクティブ VP グループは, 異なるプロセッサ上の複数のアクティブ VP で構成される
        \item 各アクティブ VP グループはただ 1 つのタスクを処理する
        \item 各物理プロセッサはアクティブ VP を最大 1 つホストする
        \item アクティブVPはそのプロセッサ上で最高の優先度で実行される
    \end{itemize}
\end{frame}

\begin{frame}{アクティブ VP 上でのジョブの実行}
    \begin{block}{アクティブ VP グループ $\Theta$ がタスク $\tau_{i}$ を処理する場合のフロー}
        \setlength{\linewidth}{0.98\columnwidth}
        \begin{enumerate}
            \item $\tau_{i}$ がジョブ $J_{i}$ をリリースするたびに, $\Theta$ 内の全てのアクティブ VP の残りのバジェットが初期バジェットにリセットされる
            \item $J_{i}$ が $\Theta$ のアクティブ VP で実行されるたびに, それらのバジェットを1消費する
            \item アクティブVPの残バジェットが0になると, $J_{i}$はアクティブVP上で実行できなくなる
        \end{enumerate}
    \end{block}
\end{frame}

\begin{frame}{ビジー/アイドル/失効}
    \begin{definition}[ビジー]
        アクティブ VP の残りのバジェットがゼロでなく, 対象タスクのワークロードを実行している場合, アクティブ VP はビジー
    \end{definition}
    \begin{definition}[アイドル]
        アクティブ VP の残りのバジェットがゼロ以外であるが, ワークロードを実行していない場合, アクティブ VP はアイドル
    \end{definition}
    \begin{definition}[失効]
        アクティブ VP は, 残りのバジェットが 0 になると失効
    \end{definition}
\end{frame}

\begin{frame}{タスクの実行方法}
    \begin{itemize}
        \item タスクはアクティブ VP グループ上で作業保存型でワークロードを実行する
              \begin{block}{作業保存型}
                  アイドル状態のアクティブ VP がある場合は, 必ず実行可能な頂点を実行する
              \end{block}
              \vspace{5mm}
        \item 異なるアクティブ VP 間のプリエンプションとマイグレーションの両方が許可される
        \item 同じタスク内の頂点間にタスク内優先度がないと仮定し, スケジューラが実行可能な頂点を選択する順序を制限しない
    \end{itemize}
\end{frame}

\begin{frame}{リーディングアクティブVP}
    \begin{itemize}
        \item $\Theta$ の最初のアクティブ VP をリーディングアクティブVPとする
              \begin{definition}[リーディングアクティブ VP]
                  \begin{itemize}
                      \item 対象タスクのワークロードを実行する特権を持つアクティブVP
                      \item $\tau_{i}$ の現在のジョブに実行可能な頂点がある場合, $\Theta$ のリーディングアクティブアクティブ VP を使用する必要がある
                  \end{itemize}
              \end{definition}
              \vspace{5mm}
        \item 非リーディングアクティブ VP は, リーディングアクティブ VP がビジーの場合にのみ, $\tau_{i}$ のワークロードを実行できる
    \end{itemize}
\end{frame}

\begin{frame}{実行シーケンス例}
    \fitimage{
        \begin{itemize}
            \item $\Theta=\left\{\theta_{1}=8, \theta_{2}=3\right\}$ とする
            \item $\tau_{i}$ のジョブがこのアクティブ VP グループで実行される場合の実行シーケンス
        \end{itemize}
    }{active_vp_executed}
\end{frame}

\forme{
    \begin{frame}{残りのグラフ}
        \begin{definition}[残りのグラフ]
            \setlength{\linewidth}{0.98\columnwidth}
            \begin{itemize}
                \item ジョブ $J_{i}$ の残りのグラフは, まだ実行されていないワークロード
                \item 残りのグラフは, 完了した頂点とそれらに接続するエッジを削除し, 未完了の各頂点の WCET を現在の残りのワークロードで置き換えることにより, $G\left(J_{i}\right)$ から導き出せる
            \end{itemize}
        \end{definition}
    \end{frame}

    \begin{frame}{残りのグラフ例}
        \fullimage{remanning_graph}
    \end{frame}

    \begin{frame}[label=lemma1]{Lemma 1}
        \begin{lemma}[]
            \begin{itemize}
                \item タスク $\tau_{i}$ のジョブ $J_{i}$ がアクティブ VP グループ $\Theta$ によって処理されるとする
                \item $\Theta$ の一部のアクティブ VP がアイドル状態かつ, $J_{i}$ が完了していない任意の時間単位で,  $J_{i}$ の残りのグラフのクリティカルパス長が 1 減る
            \end{itemize}
        \end{lemma}
    \end{frame}

    \begin{frame}[label=lemma2]{Lemma 2}
        \begin{lemma}[]
            \begin{itemize}
                \item タスク $\tau_{i}$ がアクティブ VP グループ $\Theta=$  $\left\{\theta_{1}, \cdots, \theta_{|\Theta|}\right\}$ によって処理されると仮定する
                      \begin{itemize}
                          \item $\theta_{1} \leq D_{i}$
                          \item $\forall \theta_{z} \in \Theta \backslash\left\{\theta_{1}\right\}: \theta_{z} \leq D_{i}-L_{i}$
                      \end{itemize}
                \item $r$ でリリースされた $\tau_{i}$ のジョブ $J_{i}$ が $r+D_{i}$ のデッドラインに間に合わない場合, $r+D_{i}$ で $\Theta$ のアクティブ VP の残りバジェットは 0
            \end{itemize}
        \end{lemma}
    \end{frame}
}

\begin{frame}[label=theorem1]{Theorem 1}
    \begin{theorem}[]
        次の全ての条件が満たされる場合, タスク $\tau_{i}$ はアクティブ VP グループ $\Theta=$  $\left\{\theta_{1}, \cdots, \theta_{|\Theta|}\right\}$ でスケジュール可能
        \begin{enumerate}
            \item $\sum_{\theta_{z} \in \Theta} \theta_{z}=C_{i}$
            \item $\theta_{1} \leq D_{i}$
            \item $\forall \theta_{z} \in \Theta \backslash\left\{\theta_{1}\right\}: \theta_{z} \leq D_{i}-L_{i}$
        \end{enumerate}
    \end{theorem}
\end{frame}
