% !TeX root = main.tex

\section{EVALUATIONS}
\label{sec: evaluations}

\begin{frame}{比較対象}
    \fullimage{compared_method}
\end{frame}

\begin{frame}{DAGの生成方法}
    \begin{itemize}
        \item 各 DAG は, エッジで接続された 2 つの頂点から生成される
        \item 再帰手順が確率的に実行され, 次のパラメーターを使用して頂点を並列サブグラフに展開する
              \begin{itemize}
                  \item  $p_{p a r}$ : 頂点を並列サブグラフに拡張する確率

                  \item  $n_{par}$ : 頂点を並列サブグラフに展開する場合, $\left[2, n_{p a r}\right]$ で枝の数を一律に選択する

                  \item  $p_{a d d}$ : 結果のタスクは, 確率 $p_{a d d}=0.1$ で頂点間にエッジを追加することによって DAG に変換される

              \end{itemize}
        \item 実験では $p_{\text {par }}=0.8$, $n_{\text {par }}=8$, および $p_{a d d}=0.1$ を設定した
    \end{itemize}
\end{frame}

\begin{frame}{タスクのパラメータ設定方法}
    DAG 構造を生成した後, 次のようにタスク $\tau_{i}$ を構築する

    \begin{itemize}
        \item  各頂点 $v$ の WCET $c(v)$ は, [1, 100] の範囲で一様に選択された整数であり, それに応じてタスクの $L_{i}$ および $C_{i}$ が計算される

        \item $D_{i}$ は, $\left[L_{i}, L_{i}+\alpha\left(C_{i}-L_{i}\right)\right]$ の範囲の整数として一様に選択される
              \begin{itemize}
                  \item デフォルトでは, $\alpha$ は $0.6$ に設定される (一部の実験では異なる)
              \end{itemize}
        \item $T_{i}$ は $D_{i}$ と同じ
              \begin{itemize}
                  \item デッドラインが制約されたタスクで評価するために, 後に $T_{i}$ を別の方法で設定する
              \end{itemize}

    \end{itemize}
\end{frame}

\begin{frame}{タスクセット生成方法}
    \begin{itemize}
        \item 各タスクセットに対して, $N$ 個のタスクを生成する
              \begin{itemize}
                  \item $N$ はデフォルトで 5  (一部の実験では異なる)
              \end{itemize}
        \item タスクセットが生成された後, 合計利用率 $U_{\sum}=\sum_{i=1}^{N} C_{i} / T_{i}$ が計算される
        \item 生成されたタスクセットごとに, $U_{\text {norm }}$ で示される目標均一利用率を設定する
        \item タスクセットをスケジュールするプロセッサの総数は, $m=\left\lceil\frac{U_{\Sigma}}{U_{\text {norm }}}\right\rceil$ によって計算される
    \end{itemize}
\end{frame}

\begin{frame}{評価指標}
    \begin{itemize}
        \item 各手法の受入率を比較する
              \begin{definition}[受入率]
                  手法によってスケジュール可能となるタスクセットの数と, 実験におけるタスクセットの総数との比率
              \end{definition}
        \item 受入率が高いほど手法のスケジュール能力が高い
    \end{itemize}
\end{frame}

\begin{frame}{異なる $U_{\text {norm }}$ の実験結果}
    \fitimage{
        提案手法は, 他の手法を常に凌駕する
    }{u_norm}
\end{frame}

\begin{frame}{異なる $\alpha$ の実験結果}
    \fitimage{
        提案手法が比較された全ての既存手法よりも優れている
    }{alpha}
\end{frame}

\begin{frame}{異なる $N$ の実験結果}
    \fitimage{
        提案手法が比較された全ての既存手法よりも優れている
    }{N}
\end{frame}

\begin{frame}{デッドライン付きのタスクセットを使用した実験結果}
    \fitimage{
        制約付きデッドライン付きタスクセットにおいても, 提案手法は全ての既存手法よりも一貫して優れている
    }{deadline}
\end{frame}

\begin{frame}{予約ベースフェデレートスケジューリングとの比較}
    提案手法は予約ベースのフェデレートスケジューリング\cite{ueter2018reservation}より自明に優れており, $L_{i}$ が大きいほど, この差は顕著になる
    \begin{block}{理由}
        \setlength{\linewidth}{0.98\columnwidth}
        \begin{itemize}
            \item 提案手法では, 各周期にアクティブ VP グループによって $\tau_{i}$ に提供される合計処理能力が丁度 $C_{i}$ であり, $\tau_{i}$ によって完全に利用される
            \item 予約ベースのアプローチで各周期に $\tau_{i}$ を処理するために割り当てられる予約の合計バジェットは $C_{i}+L_{i} \cdot\left(m_{i}-1\right)$ 以下
                  \begin{itemize}
                      \item \desc{$m_{i}$}{$\tau_{i}$ に割り当てられる予約サーバの数}
                  \end{itemize}
            \item $m_{i}>1$ の場合, 予約の合計バジェットが $C_{i}$ を超える
        \end{itemize}
    \end{block}
\end{frame}
