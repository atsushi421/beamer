% !TeX root = main.tex

\section{OVERVIEW}
\label{sec: overview}

\begin{frame}{セクションサマリ}
    \begin{itembox}[l]{\textbf{目的}}
        フェデレートスケジューリングアプローチを簡単に確認し, そのリソース浪費の問題について説明する
    \end{itembox}
\end{frame}

\begin{frame}{フェデレートスケジューリングの概要}
    \begin{itemize}
        \item フェデレートスケジューリングでは, 各タスク $\tau_{i}$ に$m_{i}$個のプロセッサが割り当てられ, 排他的に実行される

              \begin{equation*}
                  m_{i}=\left\lceil\frac{C_{i}-L_{i}}{D_{i}-L_{i}}\right\rceil
              \end{equation*}

        \item これにより, $\tau_{i}$ は作業保存型スケジューラによってスケジュール可能であることが保証される
    \end{itemize}
\end{frame}

\forme{
    \begin{frame}{フェデレートスケジューリングの問題点}
        \begin{itemize}
            \item $\tau_{i}$ が実際に $m_{i}$ プロセッサで実行されると, これらのプロセッサの処理能力のかなりの部分が浪費される可能性がある
            \item タスクの周期が相対デッドラインよりもはるかに長い場合, ジョブの絶対デッドラインと次のジョブのリリース時刻の間の処理能力は無駄になる
            \item 暗黙のデッドラインタスクの場合でも, 処理能力の浪費が非常に大きくなる可能性がある
        \end{itemize}
    \end{frame}
}

\begin{frame}{フェデレートスケジューリングのリソース浪費の例}
    \fitimage{
        \begin{itemize}
            \item $\tau_{i}$ が実際に $m_{i}$ プロセッサで実行されると, これらのプロセッサの処理能力のかなりの部分が浪費される可能性がある
            \item 極端なケースでは, 無駄な処理能力の割合が $100 \%$ に限りなく近くなる
        \end{itemize}
    }{resource_wasting}
\end{frame}

\begin{frame}{リソース浪費の例補足}
    \fitimage{
        \begin{itemize}
            \item ジョブがデッドラインに間に合うように $\frac{k}{2}$ 個のプロセッサが必要
            \item プロセッサが 1周期に提供する処理能力の合計量は $\left(1+\frac{2}{k}\right) \frac{k}{2}=$  $\frac{k}{2}+1$
            \item ジョブが使用していない処理容量は $\frac{k}{2}+1-\left(1+k \times \frac{1}{k}\right)=\frac{k}{2}-1$
            \item したがって, リソースの浪費の比率は $\frac{k / 2-1}{k / 2+1}$ であり, $k$ が無限に近づくにつれて, $100 \%$ に近づく
        \end{itemize}
    }{resource_wasting}
\end{frame}

\begin{frame}{本論文のハイレベルなアイデア}
    フェデレートスケジューリングで浪費された処理能力を再利用し, それを他のタスクの実行に使用する
\end{frame}

\begin{frame}{単純な方法}
    \begin{itemize}
        \item フェデレートスケジューリングと同じ方法でプロセッサをタスクに割り当て, 割り当てられたプロセッサでこれらの各タスクを高い優先度で実行する
        \item 割り当てられたプロセッサで優先度の高いタスクによって使用されていない処理能力を, 優先度の低い他のタスクを実行するために使用する
        \item 優先度の高いタスクの実行は優先度の低いタスクの影響を受けないため, フェデレートスケジューリングと同様に排他的に実行されているかのように扱える
    \end{itemize}
\end{frame}

\begin{frame}{単純な方法の問題点}
    単純な方法ではスケジューラビリティを保証することは困難
    \begin{block}{理由}
        \setlength{\linewidth}{0.98\columnwidth}
        \begin{itemize}
            \item 各周期に優先度の高いタスク $\tau_{i}$ によって残された未使用の処理能力の合計量は明らかだが, それらがプロセッサにどのように分散されているかの情報が無い
            \item そのため, 一定時間内にどれだけの処理能力が保証されるかが不明
            \item すなわち, 未使用処理能力を各低優先度タスクに分割して割り当て, スケジューラビリティを保証する方法が不明確
        \end{itemize}
    \end{block}
\end{frame}

\begin{frame}{提案手法の概要}
    \begin{itemize}
        \item 各物理プロセッサ上に,  高優先度で実行される 1 つのアクティブ VP と 低優先度で実行される 1 つのパッシブ VP の 2 つの仮想プロセッサ (VP) を構築する
        \item 2 種類の仮想プロセッサを扱えるようにフェデレートスケジューリングを一般化し, 各タスクに専用のアクティブVPまたはパッシブVPのセットを割り当て, スケジューラビリティを保証する
    \end{itemize}
\end{frame}
