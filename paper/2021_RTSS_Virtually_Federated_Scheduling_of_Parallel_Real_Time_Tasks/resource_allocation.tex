% !TeX root = main.tex

\section{RESOURCE ALLOCATION}
\label{sec: resource allocation}

\begin{frame}{セクションサマリ}
    \begin{itembox}[l]{\textbf{目的}}
        アクティブ VP グループとパッシブ VP グループを作成してタスクに割り当てるヒューリスティックアルゴリズムを開発する
    \end{itembox}
\end{frame}

\subsection{Design Choices}
\label{ssec: design choices}

\forme{
    \begin{frame}{セクションサマリ}
        \begin{itembox}[l]{\textbf{目的}}
            スケジューリング条件のいくつかのプロパティと, これらのプロパティに基づくヒューリスティックアルゴリズムの設計上の選択について説明する
        \end{itembox}
    \end{frame}

    \begin{frame}{タスクのアクティブ VP グループを構築する方針}
        \begin{itemize}
            \item タスクのアクティブ VP グループを構築する方針は以下2つが考えられる
                  \begin{enumerate}
                      \item 初期バジェットが小さいアクティブ VP を増やしてアクティブ VP グループを構成する
                      \item 初期バジェットができるだけ大きいアクティブ VP をできるだけ少なく構成する
                  \end{enumerate}
            \item 本論文の割り当てヒューリスティックは, 2を選択する
        \end{itemize}
    \end{frame}

    \begin{frame}{2の選択理由}
        \begin{itemize}
            \item \hlink{thorem1}{Theorem 1} により, タスク $\tau_{i}$ は, サービスを提供するアクティブ VP グループによって提供される処理能力を完全に利用する
            \item パッシブ VP の処理能力は以下の2つの理由で十分に活用されていない
                  \begin{itemize}
                      \item 時間間隔で保証されている最小使用可能処理能力が, 一般に, 時間間隔で提供される実際の処理能力よりも小さい
                      \item  パッシブ VP $\pi_{x}$ から $\tau_{i}$ に提供される処理能力は, $s b f_{\pi_{x}}\left(D_{i}\right)$ から $L_{i}$ を差し引く必要がある
                  \end{itemize}
                  \forme{
            \item したがって, 一般的には, パッシブ VP よりもアクティブ VP を介して処理能力を提供する方が適切である
            \item 多くの「小さな」アクティブ VP を構築すると, これらのプロセッサに残っている大量の処理能力をパッシブ VP として非効率的に使用する必要がある
                  }
            \item したがって, \hlink{theorem1}{Theorem 1} の条件が満たされる限り, 可能な限り大きなアクティブ VP を作成し, アクティブ VP グループ内のアクティブ VP の数を最小限に抑えることを選択する
        \end{itemize}
    \end{frame}

    \begin{frame}{}
        \begin{itemize}
            \item アクティブ VP グループの合計初期バジェットが正確に $C_{i}$ である場合, 全てのアクティブ VP が条件 (5) および (6) で指定された最大許容初期バジェットに到達できるわけではない
            \item これは, $C_{i}-D_{i}$ を $D_{i}-L_{i}$ で正確に割り切れない場合に発生する
            \item この場合, 最初のバジェットを, 主要なアクティブ VP に対して $D_{i}$ に設定し, 最後のアクティブ VP を除く非主要アクティブ VP に対して $D_{i}-L_{i}$ に設定し,

                  \begin{equation*}
                      C_{i}-D_{i}-\left\lfloor\frac{C_{i}-D_{i}}{D_{i}-L_{i}}\right\rfloor\left(D_{i}-L_{i}\right)
                  \end{equation*}
        \end{itemize}
    \end{frame}
}

\forme{
    \begin{frame}{}
        \begin{itemize}
            \item 最後の非主要アクティブ VP の場合
            \item いずれの場合も, $\Theta$ の先頭以外のアクティブ VP の数は $\left\lceil\frac{C_{i}-D_{i}}{D_{i}-L_{i}}\right\rceil$ であるため, $\Theta$ のアクティブ VP の数は $\left\lceil\frac{C_{i}-D_{i}}{D_{i}-L_{i}}\right]+1=\left[\frac{C_{i}-D_{i}}{D_{i}-L_{i}}+1\right]=\left\lceil\frac{C_{i}-L_{i}}{D_{i}-L_{i}}\right\rceil$ である
            \item 要約すると, タスク $\tau_{i}$ に割り当てられたアクティブ VP グループ内のアクティブ VP の数は, 次のように計算される

                  \begin{equation*}
                      m_{i}=\left\lceil\frac{C_{i}-L_{i}}{D_{i}-L_{i}}\right\rceil
                  \end{equation*}

            \item $m_{i}$ は, フェデレートスケジューリングで $\tau_{i}$ をスケジュールするプロセッサの最小数 ((1) で計算) と同じであることに注意
        \end{itemize}
    \end{frame}
}

\begin{frame}{初期バジェットの決め方}
    以下のように初期バジェットを設定
    \begin{block}{リーディングアクティブVP}
        $D_i$
    \end{block}
    \begin{block}{最後以外の非リーディングアクティブVP}
        $D_i - L_i$
    \end{block}
    \begin{block}{最後の非リーディングアクティブVP}
        $C_{i}-D_{i}-\left\lfloor\frac{C_{i}-D_{i}}{D_{i}-L_{i}}\right\rfloor\left(D_{i}-L_{i}\right)$
    \end{block}
\end{frame}

\begin{frame}{アクティブVPの数の求め方}
    タスク $\tau_{i}$ に割り当てるアクティブ VP グループ内のアクティブ VP の数は, 次のように計算
    \begin{equation*}
        m_{i}=\left\lceil\frac{C_{i}-L_{i}}{D_{i}-L_{i}}\right\rceil
    \end{equation*}
\end{frame}

\begin{frame}{パッシブVPが有用である条件}
    \hlink{theorem2}{Theorem 2} と \hlink{theorem3}{Theorem 3} の条件より, パッシブ VP $\pi_{x}$ によって提供される処理能力がタスク $\tau_{j}$ にとって有用であるのは, 以下の場合である
    \notes{有用である:スケジュール可能になりうる}
    \begin{equation*}
        \operatorname{sbf}_{\pi_{x}}\left(D_{j}\right)>L_{j}
    \end{equation*}

\end{frame}

\begin{frame}[label=lemma6]{Lemma 2}
    Lemma 2 は対のパッシブVPが $\tau_j$ に有用であるための必要条件は $D_{i}-L_{i}<D_{j}-L_{j}$ であることを示す
    \begin{lemma}[]
        \setlength{\linewidth}{0.98\columnwidth}
        \begin{itemize}
            \item 2つのタスク $\tau_{i}$ と $\tau_{j}$ を考える
            \item $\pi_{x}$ が, 初期バジェット $\theta_{z}=D_{i}-L_{i}$ で $\tau_{i}$ に割り当てられたアクティブ VP の対のパッシブ VP である場合, 以下が成り立つ
        \end{itemize}
        \begin{equation*}
            s b f_{\pi_{x}}\left(D_{j}\right)>L_{j} \Rightarrow D_{i}-L_{i}<D_{j}-L_{j}
        \end{equation*}
    \end{lemma}
\end{frame}

\begin{frame}{アクティブ/パッシブVPの選択戦略}
    \hlink{lemma6}{Lemma 2} より, 割り当てヒューリスティックは, 小さい $D_{i}-L_{i}$ を持つタスクを選択してアクティブ VP で実行し, 大きい $D_{i}-L_{i}$ を持つタスクをパッシブ VP で実行するようにする
\end{frame}

\forme{
    \begin{frame}{}
        \begin{itemize}
            \item 条件 $D_{i}-L_{i}<$  $D_{j}-L_{j}$ は必要なだけであり, 結果のパッシブ VP が $\tau_{j}$ にとって有用であるために十分ではないことに注意
            \item それにも関わらず, アクティブ VP で実行するために小さい $D_{i}-L_{i}$ を使用するタスクを選択することは, 逆の方法で実行するよりもパッシブ VP が役立つ可能性がはるかに高いため, 一般的には依然として良い戦略である
        \end{itemize}
    \end{frame}
}


\subsection{Allocation Algorithm}
\label{ssec: allocation algorithm}

\begin{frame}{セクションサマリ}
    \begin{itembox}[l]{\textbf{目的}}
        リソース割り当てヒューリスティックアルゴリズムを提案する
    \end{itembox}
\end{frame}

\begin{frame}{アルゴリズムで登場する表記}
    \full{
        \begin{table}[tb]
            \adjustbox{max width=\textwidth, max height=\slideheight}{
                \centering\begin{tabular}{|c|l|} \hline
                    $Q$ & $D_{i}-L_{i}$ の昇順で全てのタスクが格納されるキュー \\\hline
                    $P$ & 全てのプロセッサ                                     \\\hline
                    $V$ & 生成されたパッシブVPのセット                         \\\hline
                \end{tabular}
            }
        \end{table}
    }
\end{frame}

\begin{frame}{割り当てアルゴリズム全体像}
    \fullimage{alg1}
\end{frame}

\begin{frame}{割り当てアルゴリズム補足 [アクティブVPグループの割り当て]}
    \fullimage{alg1_sup1}
\end{frame}

\begin{frame}{割り当てアルゴリズム補足 [パッシブVPグループの割り当て]}
    \fullimage{alg1_sup2}
\end{frame}

\forme{
    \begin{frame}{}
        \begin{itemize}
            \item 繰り返しになるが, アルゴリズム 1 の割り当て戦略は, 最適というよりは単なるヒューリスティックである
            \item セクション VIII で示されるように, アルゴリズム 1 を使用すると, 仮想連合アプローチはすでに既存の方法よりも大幅に優れている
            \item それにも関わらず, より洗練されたリソース割り当て戦略を使用して, 仮想フェデレーションスケジューリングアプローチのパフォーマンスをさらに改善する余地がある可能性がある
            \item これについては, 今後の作業で調査する
        \end{itemize}
    \end{frame}
}
