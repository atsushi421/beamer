% !TeX root = main.tex

\section{Case Study}
\label{sec: case study}

\begin{frame}{ケーススタディの概要}
    \fitimage{
        \begin{itemize}
            \item 本論文のアプローチの適合性を評価するために, 車輪付きロボットの ROS ナビゲーションスタックの中心部分である一般的な move\_base パッケージのセーフティクリティカルな処理チェーンを, センサレートと 観測された最大実行時間を使用して分析した
            \item move\_base コールバックグラフを図に示す
        \end{itemize}
    }{case_study}
\end{frame}

\begin{frame}{実験環境1}
    \begin{itemize}
        \item 2 つの予約を使用して, タイムクリティカルなローカルプランニングコールバックからグローバルプランニングコールバックを分離する
              \begin{itemize}
                  \item より予測不可能なコンポーネントをクリティカルパスから分離するだけでなく, ROS executor スケジューリングポリシーの影響を制限するため
              \end{itemize}
        \item ローカルプランナーが $12.5 \mathrm{~Hz}$ の固定センサレートと同期して実行され, グローバルプランナーが $1 \mathrm{~Hz}$ で実行されるように設定する
        \item 比較のために, 元の時間駆動型バージョンと, 内部トピックを使用してアクティベーションの依存関係を明示的にモデル化するイベント駆動型の代替設計の 2 つのバリアントをモデル化した
    \end{itemize}
\end{frame}

\begin{frame}{エンドツーエンドレイテンシの定義}
    セットアップに応じて, エンドツーエンドレイテンシを以下のように定義する
    \begin{block}{イベント駆動型のセットアップ}
        アクティベーションからチェーンの完了までの最悪応答時間
    \end{block}
    \begin{block}{時間駆動型のセットアップ}
        ローカルプランナーの応答時間を計算し, センサ入力にジッターがある場合は, アクティベーション周期を最悪サンプリングレイテンシとして追加する
    \end{block}
\end{frame}

\begin{frame}{実験環境2}
    \begin{itemize}
        \item イベント駆動型セットアップの場合, セクション 5.4 で説明したチェーン全体の分析を無効にしたときの分析結果も含める
        \item 簡単にするために, 残りのコアをグローバルプランナー予約に割り当てる
              \notes{グローバルプランナーの予約を通じて処理チェーンを分析していないため, この予約専用の帯域幅の正確な量は, 応答時間に影響しない}
    \end{itemize}
\end{frame}

\begin{frame}{[実験1] 結果}
    \fitimage{
        グラフの横軸は, 総コア帯域幅に対するローカルプランナーの予算の割合
    }{result1}
\end{frame}

\begin{frame}{[実験1] 結果の考察}
    \begin{itemize}
        \item 時間駆動型システムとイベント駆動型システムで, 予算の効果が似ている
        \item 最悪サンプリングレイテンシにより, 時間駆動型セットアップは, イベント駆動型セットアップよりも $80 \mathrm{~ms}$ サンプリング周期が高くなる
        \item この場合, イベント駆動セットアップが望ましいことは明らかで, システムは, 古い値に基づいて計画を開始するのではなく, センサーの結果が到着するまで待つことができる
        \item チェーン全体の分析を無効にすると, チェーンの自己干渉により, 予測される応答時間境界が大幅に増加する
        \item 分析では, 全てのコールバックが他の全てのコールバックによってブロックされると保守的に想定しているため, 実際の干渉は 4 倍過大評価される
    \end{itemize}
\end{frame}

\begin{frame}{[実験2] 結果}
    \fitimage{
        ローカルプランナーの予算を45\%として, 入力ジッターが増加したときの予測されるエンドツーエンドレイテンシ
    }{result2}
\end{frame}

\begin{frame}{[実験2] 結果の考察}
    \begin{itemize}
        \item 時間駆動型システムはバーストの影響を受けないため, 入力ジッタに対して非常に堅牢
        \item イベントドリブンシステムの場合, ほぼ $20 \mathrm{~ms}$ ごとに大幅な上昇が見られる
              \begin{itemize}
                  \item これらは, 処理チェーンの実行中にもう 1 つのイベントが到達できるポイント
              \end{itemize}
        \item イベント駆動型システムは, ジッターが $40 \mathrm{~ms}$ 未満 (すなわち, 同時に最大 1 つのイベント) では優れた性能を発揮するが, それ以上のジッターでは自己干渉に陥る
        \item 体系的な分析によって, このようなトレードオフを予測できる
    \end{itemize}
\end{frame}

\begin{frame}{ケーススタディまとめ}
    \begin{itemize}
        \item 自動化された応答時間分析によって, ROS コードを 1 行も実装しなくても, 2 つの全く異なる move\_base 設計の最悪レイテンシを推論し, さまざまなシナリオでの長所と短所を観察できる
        \item これにより, 応答時間を測定可能な設計制約として扱うことができ, ROS 開発者にとって大きな助けとなる
    \end{itemize}
\end{frame}
