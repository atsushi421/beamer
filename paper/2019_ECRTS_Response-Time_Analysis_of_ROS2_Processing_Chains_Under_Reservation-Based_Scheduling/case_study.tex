% !TeX root = main.tex

\section{Case Study}
\label{sec: case study}

\begin{frame}{ケーススタディの概要}
    \fitimage{
        \begin{itemize}
            \item 本論文のアプローチの適合性を評価するために, 車輪付きロボットの ROS ナビゲーションスタックの中心部分である一般的な move\_base パッケージのセーフティクリティカルな処理チェーンを, センサレートと 観測された最大実行時間を使用して分析した
            \item move\_base コールバックグラフを図に示す
        \end{itemize}
    }{case_study}
\end{frame}

\begin{frame}{実験環境1}
    \begin{itemize}
        \item 2 つの予約を使用して, タイムクリティカルなローカルプランニングコールバックからグローバルプランニングコールバックを分離する
              \begin{itemize}
                  \item より予測不可能なコンポーネントをクリティカルパスから分離するだけでなく, ROS executor スケジューリングポリシーの影響を制限するため
              \end{itemize}
        \item ローカルプランナーが $12.5 \mathrm{~Hz}$ の固定センサレートと同期して実行され, グローバルプランナーが $1 \mathrm{~Hz}$ で実行されるように設定する
    \end{itemize}
\end{frame}

\begin{frame}{実験環境2}
    \begin{itemize}
        \item 比較のために, 元の時間駆動型バージョンと, 内部トピックを使用してアクティベーションの依存関係を明示的にモデル化するイベント駆動型の代替設計をモデル化した
        \item イベント駆動型セットアップの場合, \hlink{ssec: analysis for processing chains}{サブチェーンの分析} を使用しない場合の分析も行う (図中の no chains)
    \end{itemize}
\end{frame}

\begin{frame}{エンドツーエンドレイテンシの定義}
    セットアップに応じて, エンドツーエンドレイテンシを以下のように定義する
    \begin{block}{イベント駆動型のセットアップ}
        アクティベーションからチェーンの完了までの最悪応答時間
    \end{block}
    \begin{block}{時間駆動型のセットアップ}
        ローカルプランナーの応答時間を計算し, センサ入力にジッタがある場合は, アクティベーション周期を最悪サンプリングレイテンシとして追加する
    \end{block}
\end{frame}


\begin{frame}{[実験1] 結果}
    \fitimage{
        グラフの横軸は, 総コア帯域幅に対するローカルプランナーのバジェットの割合
    }{result1}
\end{frame}

\begin{frame}{[実験1] 結果の考察1}
    \begin{itemize}
        \item 時間駆動型システムとイベント駆動型システムで, 同様のバジェットの効果が観察できる
        \item 最悪サンプリングレイテンシにより, 時間駆動型セットアップは, イベント駆動型セットアップよりも $80 \mathrm{~ms}$ サンプリング周期が高くなる
        \item この場合, イベント駆動セットアップが自明に望ましいと判断できる
    \end{itemize}
\end{frame}

\begin{frame}{[実験1] 結果の考察2}
    \begin{itemize}
        \item no chains は, チェーンの自己干渉により, 予測される最悪応答時間が大幅に増加する
        \item これは, 全てのコールバックが他の全てのコールバックによってブロックされると悲観的に想定するため
    \end{itemize}
\end{frame}

\begin{frame}{[実験2] 結果}
    \fitimage{
        ローカルプランナーのバジェットを45\%として, 入力ジッタが増加したときのエンドツーエンドレイテンシ
    }{result2}
\end{frame}

\begin{frame}{[実験2] 結果の考察}
    \begin{itemize}
        \item 時間駆動型システムはバーストの影響を受けないため, 入力ジッタに対して非常に堅牢
        \item イベントドリブンシステムの場合, ほぼ $20 \mathrm{~ms}$ ごとに大幅な上昇が見られる
              \begin{itemize}
                  \item これらは, 処理チェーンの実行中にもう 1 つのイベントが到達できるポイント
              \end{itemize}
        \item イベント駆動型システムは, ジッタが $40 \mathrm{~ms}$ 未満 (すなわち, 同時に最大 1 つのイベント) では優れた性能を発揮するが, それ以上のジッタでは自己干渉に陥る
        \item 体系的な分析によって, このようなトレードオフを予測できる
    \end{itemize}
\end{frame}

\begin{frame}{ケーススタディまとめ}
    \begin{itemize}
        \item 自動化された応答時間分析によって, ROS コードを 1 行も実装しなくても, 2 つの全く異なる move\_base 設計の最悪レイテンシを推論し, さまざまなシナリオでの長所と短所を観察できる
        \item これにより, 応答時間を測定可能な設計制約として扱うことができ, ROS 開発者にとって大きな助けとなる
    \end{itemize}
\end{frame}
