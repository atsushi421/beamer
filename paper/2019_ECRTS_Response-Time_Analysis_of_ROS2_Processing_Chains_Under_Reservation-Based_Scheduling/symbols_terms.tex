% !TeX root = main.tex

% \begin{frame}{表記法・用語 1}
%     \full{
%         \begin{table}[tb]
%             \adjustbox{max width=\textwidth, max height=\slideheight}{
%                 \centering\begin{tabular}{|c|l|} \hline
%                     $r_i$   & 予約サーバ                                                             \\\hline
%                     $Q_i$   & $r_i$のバジェット                                                            \\\hline
%                     $P_i$   & $r_i$の周期                                                            \\\hline
%                     $sbf()$ & ある長さの区間において予約が提供するサービスの最小量を返す供給境界関数 \\\hline
%                 \end{tabular}
%             }
%         \end{table}
%     }
% \end{frame}

\begin{frame}{表記法・用語 1}
    \full{
        \begin{table}[tb]
            \adjustbox{max width=\textwidth, max height=\slideheight}{
                \centering\begin{tabular}{|c|l|} \hline
                    $\mathcal{D}=\{\mathcal{C}, \mathcal{E}\}$             & ROSシステムのDAG                 \\\hline
                    $\mathcal{C}=\left\{c_{1}, \ldots, c_{n}\right\}$      & コールバックセット               \\\hline
                    $\mathcal{E} \subseteq \mathcal{C} \times \mathcal{C}$ & エッジセット                     \\\hline
                    $\mathcal{C}^{\text {tmr}}$                            & タイマコールバックのセット       \\\hline
                    $\mathcal{C}^{\text {srv}}$                            & サービスコールバックのセット     \\\hline
                    $\mathcal{C}^{\text {clt}}$                            & クライアントコールバックのセット \\\hline
                \end{tabular}
            }
        \end{table}
    }
\end{frame}

\begin{frame}{表記法・用語 2}
    \full{
        \begin{table}[tb]
            \adjustbox{max width=\textwidth, max height=\slideheight}{
                \centering\begin{tabular}{|c|l|} \hline
                    $\operatorname{pred}\left(c_{i}\right)=\left\{c_{j} \in \mathcal{C}: \exists\left(c_{j}, c_{i}\right) \in \mathcal{E}\right\}$ & $c_i$の先行コールバックのセット        \\\hline
                    $\operatorname{succ}\left(c_{i}\right)=\left\{c_{j} \in \mathcal{C}: \exists\left(c_{i}, c_{j}\right) \in \mathcal{E}\right\}$ & $c_i$の後続コールバックのセット        \\\hline
                    $c_i$                                                                                                                          & コールバック                           \\\hline
                    $e_i$                                                                                                                          & $c_i$のWCET                            \\\hline
                    $\pi_i$                                                                                                                        & $c_i$の優先度                          \\\hline
                    ソースコールバック                                                                                                             & 先行コールバックを持たないコールバック \\\hline
                    シンクコールバック                                                                                                             & 後続コールバックを持たないコールバック \\\hline
                \end{tabular}
            }
        \end{table}
    }
\end{frame}

\begin{frame}{表記法・用語 3}
    \full{
        \begin{table}[tb]
            \adjustbox{max width=\textwidth, max height=\slideheight}{
                \centering\begin{tabular}{|c|l|} \hline
                    $c_s$                                                                              & ソースコールバック                                                        \\\hline
                    $\eta_{s}^{e}(\Delta)$                                                             & 長さ $\Delta$ の任意の間隔でリリースできる $c_{s}$ のインスタンスの最大数 \\\hline
                    $\gamma^{x}=\left(c_{s}, \ldots, c_{e}\right)$                                     & コールバックチェーン                                                      \\\hline
                    $\operatorname{chains}(\mathcal{D})=\left\{\gamma^{1}, \ldots, \gamma^{s}\right\}$ & チェーンセット                                                            \\\hline
                    $\gamma^{x, y}$                                                                    & $\gamma^{x}$ の $y$ 番目のサブチェーン                                    \\\hline
                    $R_{x, y}$                                                                         & $\gamma^{x, y}$ の最悪応答時間                                            \\\hline
                    $R_{j}$                                                                            & $c_{j}$ の最悪応答時間                                                    \\\hline
                \end{tabular}
            }
        \end{table}
    }
\end{frame}

\begin{frame}{表記法・用語 4}
    \full{
        \begin{table}[tb]
            \adjustbox{max width=\textwidth, max height=\slideheight}{
                \centering\begin{tabular}{|c|l|} \hline
                    $r_k$                                             & 予約サーバ                                                  \\\hline
                    $\mathcal{C}_k$                                   & $r_k$ 内のコールバックのセット                              \\\hline
                    $\mathcal{R}=\left\{r_{1}, \ldots, r_{w}\right\}$ & 予約のセット                                                \\\hline
                    $w$                                               & システム内の予約の数                                        \\\hline
                    $\mathcal{C}_{k}^{\mathrm{tmr}}$                  & $r_k$に割り当てられたタイマコールバックのセット             \\\hline
                    $\mathcal{C}_{k}^{\mathrm{sub}}$                  & $r_k$に割り当てられたサブスクリプションコールバックのセット \\\hline
                    $\mathcal{C}_{k}^{\mathrm{clt}}$                  & $r_k$に割り当てられたクライアントコールバックのセット       \\\hline
                    $\mathcal{C}_{k}^{\mathrm{srv}}$                  & $r_k$に割り当てられたサービスコールバックのセット           \\\hline
                    $\mathcal{C}^{\text {evt}}$                       & イベントソースのセット                                      \\\hline
                    $\mathcal{C}_{k}^{\mathrm{pp}}$                   & $r_{k}$ に割り当てられた pp ベースのコールバックのセット    \\\hline
                \end{tabular}
            }
        \end{table}
    }
\end{frame}

\begin{frame}{表記法・用語 5}
    \full{
        \begin{table}[tb]
            \adjustbox{max width=\textwidth, max height=\slideheight}{
                \centering\begin{tabular}{|c|l|} \hline
                    $\mathcal{P}=\left\{p_{1}, \ldots, p_{m}\right\}$ & プロセッサセット                                                                   \\\hline
                    $m$                                               & プロセッサの個数                                                                   \\\hline
                    $PP_{n}$                                          & $n$ 番目のポーリングポイント                                                       \\\hline
                    $PW_{n}$                                          & $n$ 番目の処理ウィンドウ ($PP_{n}$ から $PP_{n+1}$ までの範囲)                     \\\hline
                    $lp_{k}\left(c_{i}\right)$                        & $\pi_{i}$より低い優先度を持つ$\mathcal{C}_{k}$内のコールバックのセット             \\\hline
                    $hp_{k}\left(c_{i}\right)$                        & $\pi_{i}$より高い優先度を持つ$\mathcal{C}_{k}$内のコールバックのセット             \\\hline
                    $sbf_{k}(\Delta)$                                 & 長さ $\Delta$ の任意の間隔で予約 $r_{k}$ によって提供される最小サービス            \\\hline
                    $rbf_{i}(\Delta)$                                 & 長さ $\Delta$ の間隔でリリースされたコールバックインスタンスが要求する最大サービス \\\hline
                    $RBF\left(\mathcal{C}^{*}, \Delta\right)$         & コールバックセット $\mathcal{C}^{*}$ の要求境界関数の合計                          \\\hline
                    $\delta_{i,j}$                                    & $c_i$から$c_j$への最悪通信レイテンシ                                               \\\hline
                \end{tabular}
            }
        \end{table}
    }
\end{frame}

\begin{frame}{表記法・用語 6}
    \full{
        \begin{table}[tb]
            \adjustbox{max width=\textwidth, max height=\slideheight}{
                \centering\begin{tabular}{|c|l|} \hline
                    $A$                                                      & ビジーウィンドウのオフセット                          \\\hline
                    $R_i^{*}(A)$                                             & $A$に対する$c_i$の応答時間計算式における最小の正の解  \\\hline
                    $L^{*}$                                                  & 最長のビジーウィンドウ                                \\\hline
                    $A_i$                                                    & $c_i$に関する制限された$A$の探索空間                  \\\hline
                    $rbf^{x, y}(\Delta)=\eta_{s}^{a}(\Delta) \cdot e^{x, y}$ & サブチェーン $\gamma^{x, y}$ に拡張された要求境界関数 \\\hline
                    $e^{x, y}$                                               & サブチェーンの最悪累積実行時間                        \\\hline
                    $\Gamma_{k}$                                             & $r_{k}$ に割り当てられたサブチェーンのセット          \\\hline
                \end{tabular}
            }
        \end{table}
    }
\end{frame}
