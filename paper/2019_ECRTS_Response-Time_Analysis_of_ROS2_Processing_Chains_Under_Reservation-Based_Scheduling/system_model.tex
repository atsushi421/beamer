% !TeX root = main.tex

\section{SYSTEM MODEL}
\label{sec: system model}

% \begin{frame}{セクションサマリ}
%     \begin{itembox}[l]{\textbf{目的}}
%         ROS システムのタイミング関連の側面のモデル, そのコールバック, およびそれらのアクティベーション関係を紹介する
%     \end{itembox}
% \end{frame}

\begin{frame}{}
    \begin{itemize}
        \item 始めに, 本論文で登場する表記法・用語の表を示す
        \item 基本的な表記法・用語は資料中で説明無しで使用する
        \item 別ファイルで開く・印刷するなどして, 常に参照できる状態にしておくことを推奨する
    \end{itemize}
\end{frame}

% !TeX root = main.tex

\begin{frame}{表記法・用語 1}
    \full{
        \begin{table}[tb]
            \adjustbox{max width=\textwidth, max height=\slideheight}{
                \centering\begin{tabular}{|c|l|} \hline
                    $N$      & タスクの数                 \\\hline
                    $M$      & プロセッサの個数           \\\hline
                    $\tau_i$ & タスク                     \\\hline
                    $T_i$    & $\tau_i$の周期             \\\hline
                    $D_i$    & $\tau_i$の相対デッドライン \\\hline
                    $J_{i}$  & $\tau_i$のジョブ           \\\hline
                \end{tabular}
            }
        \end{table}
    }
\end{frame}

\begin{frame}{表記法・用語 2}
    \full{
        \begin{table}[tb]
            \adjustbox{max width=\textwidth, max height=\slideheight}{
                \centering\begin{tabular}{|c|l|} \hline
                    $G\left(J_{i}\right)=\langle V, E\rangle$ & $J_{i}$のDAG                                 \\\hline
                    $V$                                       & 頂点のセット                                 \\\hline
                    $E$                                       & エッジのセット                               \\\hline
                    $v$                                       & ワークロードの一部                           \\\hline
                    $c(v)$                                    & $v$のWCET                                    \\\hline
                    $(u, v) \in E$                            & $u$ と $v$ の間の優先関係                    \\\hline
                    先行頂点                                  & エッジ$(u, v)$があるとき, $u$は$v$の先行頂点 \\\hline
                    後続頂点                                  & エッジ$(u, v)$があるとき, $v$は$u$の後続頂点 \\\hline
                    ソース頂点                                & 先行頂点を持たない頂点                       \\\hline
                    シンク頂点                                & 後続頂点を持たない頂点                       \\\hline
                \end{tabular}
            }
        \end{table}
    }
\end{frame}

\begin{frame}{表記法・用語 3}
    \full{
        \begin{table}[tb]
            \adjustbox{max width=\textwidth, max height=\slideheight}{
                \centering\begin{tabular}{|c|l|} \hline
                    適格     & 頂点は, その先行頂点が全て終了している場合に適格である \\\hline
                    完全パス & ソース頂点で始まり, シンク頂点で終わるパス             \\\hline
                    $C_i$    & $G(J_i)$内の全ての頂点の合計WCET                       \\\hline
                    $L_i$    & $G(J_i)$内のクリティカルパス上の全ての頂点の合計 WCET  \\\hline
                \end{tabular}
            }
        \end{table}
    }
\end{frame}

\begin{frame}{表記法・用語 4}
\full{
\begin{table}[tb]
\adjustbox{max width=\textwidth, max height=\slideheight}{
\centering\begin{tabular}{|c|l|} \hline
    $\Theta=\left\{\theta_{1}, \cdots, \theta_{|\Theta|}\right\}$ & アクティブVPのセット \\\hline
   $\theta_{z}$ & \tabml{$\Theta$ 内の $z$番目のアクティブVPの初期バジェット \\コンテキストから明白な場合は$\Theta$ 内の $z$番目のアクティブVPも示す} \\\hline

\end{tabular}
}
\end{table}
}
\end{frame}


\begin{frame}{DAGへのモデル化}
    \begin{itemize}
        \item ROS システムを, 一連のコールバック $\mathcal{C}=\left\{c_{1}, \ldots, c_{n}\right\}$ と一連のエッジ $\mathcal{E} \subseteq \mathcal{C} \times \mathcal{C}$ で構成される有向非巡回グラフ (DAG) $\mathcal{D}=\{\mathcal{C}, \mathcal{E}\}$ としてモデル化
        \item 各コールバックは無限のインスタンスシーケンスをリリースする
    \end{itemize}
    \assume{$\mathcal{D}$ は固定されている, すなわち, コールバックは実行時にシステムに参加したりシステムから離れたりしない}
\end{frame}

\begin{frame}{離散時間モデル}
    本論文では離散時間モデルを想定する
    \begin{block}{離散時間モデル}
        全ての時間パラメーターが基本時間単位 (プロセッササイクルなど) の整数倍
    \end{block}
\end{frame}

% \begin{frame}{pending, ready}
%     コールバックはDDS レイヤがメッセージを受信したとき, またはタイマ周期でアクティベーションされる
%     \begin{definition}[pending]
%         コールバックのインスタンスがアクティベーションされてから完了するまでの間, そのインスタンスは pending である
%     \end{definition}
%     \begin{definition}[ready]
%         コールバックインスタンスが pending だが実行されていない場合, そのインスタンスは ready である
%     \end{definition}
% \end{frame}

\begin{frame}{エッジ}
    \begin{itemize}
        \item 各エッジ $\left(c_{i}, c_{j}\right) \in \mathcal{E}$ は, $c_{i}$ から $c_{j}$ へのアクティベーション関係を示す
        \item $c_{i}$ のインスタンスの実行中に, $c_{j}$ のインスタンスを 1 つまでアクティベーションする
        \item コールバックは, 複数のpublisherがあるトピックをsubscribeする場合, 複数のエッジを持つ
              \begin{itemize}
                  \item この場合, 全てのsubscriberは, トピックにpublishされたメッセージごとに 1 回トリガされる
              \end{itemize}
    \end{itemize}
\end{frame}

\begin{frame}{処理チェーン}
    \fitimage{
        \begin{itemize}
            \item ROS グラフ $\mathcal{D}$ は, 複数のソースコールバックとシンクコールバックを持つことができる
            \item 各ソースは, 1 つまたは複数のコールバックチェーン $\gamma^{x}=\left(c_{s}, \ldots, c_{e}\right)$ に含まれる
            \item コールバックは複数のチェーンで共有される可能性がある
        \end{itemize}
    }{chain_example}
\end{frame}

\begin{frame}{アクティベーションモデル}
    各ソースコールバック $c_{s}$ は, 外部イベント到着曲線 $\eta_{s}^{e}(\Delta)$ を持つ
    \begin{block}{外部イベント到着曲線 $\eta_{s}^{e}$}
        \setlength{\linewidth}{0.98\columnwidth}
        \begin{itemize}
        \item $\eta_{s}^{e}(\Delta)$は任意の時間間隔 $[t, t+\Delta)$ に到着するイベント数の上界
        \item 例えば, タスク $T_{x}$ が定期的にトリガされる場合, $\eta_{x}^{e}(\Delta)=\lceil\frac{\Delta}{\operatorname{period}(x)}\rceil$ として表せる
        \end{itemize}
    \end{block}
    % \begin{itemize}
    %     \item 各ソースコールバック $c_{s}$ は, 外部イベント到着曲線 $\eta_{s}^{e}(\Delta)$ \cite{henia2005system, leboudec2001theory, thiele2000real}に関連付けられている
    %     % \item 本論文では一般性を失わず, $\Delta>0$ かつ $\eta_{s}^{e}(\Delta)>0$ を想定する
    % \end{itemize}
\end{frame}

\begin{frame}{エグゼキュータのスケジューリング}
    \begin{itemize}
        \item 本論文では, 組み込みのシングルスレッドエグゼキュータを採用する
        \item 各エグゼキュータが単一の予約サーバに割り当てられ, 各予約サーバが単一のエグゼキュータを処理すると仮定する
        % \item したがって, エグゼキュータに割り当てられたコールバックは, 対応する予約サーバに割り当てられたと見なせる
    \end{itemize}
\end{frame}

\begin{frame}{予約}
    \begin{itemize}
        \item システムは $w$個の予約 $\mathcal{R}=\left\{r_{1}, \ldots, r_{w}\right\}$ のセットで構成される
        \item 予約は $m$個のプロセッサ $\mathcal{P}=\left\{p_{1}, \ldots, p_{m}\right\}$ のセットに分割される
        \item 各予約は静的にプロセッサに割り当てられる
    \end{itemize}
\end{frame}

\begin{frame}{供給境界関数}
    \begin{itemize}
        \item 本論文は, 長さ $\Delta$ の任意の間隔で予約 $r_{k}$ によって提供される最小サービスを示す供給境界関数 $sbf_{k}(\Delta)$ \cite{lipari2003resource, shin2003periodic} に依存する
        \item 複数の予約が同じプロセッサに割り当てられる場合は, 全ての予約がスケジュール可能である場合にのみ, 供給境界関数によって記述されるリソースの割り当てが保証される
    \end{itemize}
    \assume{予約はスケジュール可能であると想定する}
\end{frame}

\begin{frame}{最悪通信レイテンシ}
    予約 $\left(r_{x}, r_{y}\right)$ の各ペアは, 最悪通信レイテンシ $\delta_{i, j}$ によって特徴付けられる
    \begin{block}{最悪通信レイテンシ $\delta_{i, j}$}
        $r_x$ に割り当てられた送信コールバック $c_i$ のDDS層から, $r_y$ に割り当てられた受信コールバック $c_j$ のDDS層によって受信されるまでに送信されるメッセージが経験する最大レイテンシ
    \end{block}
    \assume{$r_{x}=r_{y}$ の時, $\delta_{i, j}$ は無視できると想定する}
\end{frame}

\begin{frame}{イベントソース}
    イベントソースを疑似コールバック $c_{i} \in \mathcal{C}^{\text {evt}}$としてモデル化する
    \begin{block}{イベントソース}
        散発的にアクティベーションされる OS レベルのスレッドであり, 1 つまたは複数のトピックにpublishすることで ROS と対話し, 外部イベントのインターフェイスまたは入口ポイントとして機能する
    \end{block}
    \assume{各イベントソースは専用の予約に排他的に割り当てられると想定する}
\end{frame}
