% !TeX root = main.tex

\section{SYSTEM MODEL}
\label{sec: system model}

% \begin{frame}{セクションサマリ}
%     \begin{itembox}[l]{\textbf{目的}}
%         ROS システムのタイミング関連の側面のモデル, そのコールバック, およびそれらのアクティベーション関係を紹介する
%     \end{itembox}
% \end{frame}

% \begin{frame}{}
%     \begin{itemize}
%         \item 始めに, 本論文で登場する表記法・用語の表を示す
%         \item 基本的な表記法・用語は資料中で説明無しで使用する
%         \item 別ファイルで開く・印刷するなどして, 常に参照できる状態にしておくことを推奨する
%     \end{itemize}
% \end{frame}

% % !TeX root = main.tex

\begin{frame}{表記法・用語 1}
    \full{
        \begin{table}[tb]
            \adjustbox{max width=\textwidth, max height=\slideheight}{
                \centering\begin{tabular}{|c|l|} \hline
                    $m$                                                                                             & スレッド数                                                                                        \\\hline
                    $\Gamma=\left\{\mathcal{C}_{1}, \mathcal{C}_{2}, \cdots, \mathcal{C}_{|\Gamma|}\right\}$        & チェインのセット                                                                                  \\\hline
                    $|\Gamma|$                                                                                      & $\Gamma$内のチェインの数                                                                          \\\hline
                    $\mathcal{C}_{i}=\left\{c_{i, 1}, c_{i, 2}, \cdots, c_{i,\left|\mathcal{C}_{i}\right|}\right\}$ & チェイン                                                                                          \\\hline
                    $c_{i,j}$                                                                                       & $\mathcal{C}_{i}$の$j$番目のコールバック                                                          \\\hline
                    $\left|\mathcal{C}_{i}\right|$                                                                  & $\mathcal{C}_{i}$内のコールバックの数                                                             \\\hline
                    先行要素                                                                                        & $J_{i}^{k}$ 内の連続する要素 $c_{i, j}^{k}$ と $c_{i, j+1}^{k}$ の各ペアにおける $c_{i, j}^{k}$   \\\hline
                    後続要素                                                                                        & $J_{i}^{k}$ 内の連続する要素 $c_{i, j}^{k}$ と $c_{i, j+1}^{k}$ の各ペアにおける $c_{i, j+1}^{k}$ \\\hline
                    ソースコールバック                                                                              & $\mathcal{C}_{i}$ の最初のコールバック                                                            \\\hline
                    シンクコールバック                                                                              & $\mathcal{C}_{i}$ の最後のコールバック                                                            \\\hline
                \end{tabular}
            }
        \end{table}
    }
\end{frame}

\begin{frame}{表記法・用語 2}
    \full{
        \begin{table}[tb]
            \adjustbox{max width=\textwidth, max height=\slideheight}{
                \centering\begin{tabular}{|c|l|} \hline
                    $T_i$                 & \tabml{$\mathcal{C}_{i}$の周期                \\\underline{周期}: 2 つの連続するチェーンインスタンスのリリース時刻の間の最小間隔}                       \\\hline
                    $D_i$                 & \tabml{$\mathcal{C}_{i}$の相対デッドライン    \\\underline{相対デッドライン}: 時間 $r$ でリリースされた $\mathcal{C}_{i}$ の各チェーンインスタンスは, \\その絶対デッドライン $r+D_{i}$ までに終了する必要がある}           \\\hline
                    $e_{i,j}$             & $c_{i,j}$の最悪実行時間 (WCET)                \\\hline
                    $E_i$                 & $\mathcal{C}_{i}$内のコールバックのWCETの合計 \\\hline
                    $U_{i}=E_{i} / T_{i}$ & $\mathcal{C}_{i}$の利用率                     \\\hline
                \end{tabular}
            }
        \end{table}
    }
\end{frame}

\begin{frame}{表記法・用語 3}
    \full{
        \begin{table}[tb]
            \adjustbox{max width=\textwidth}{
                \centering\begin{tabular}{|c|l|} \hline
                    $J_{i}^{k}$                & $\mathcal{C}_{i}$ の $k$番目のチェーンインスタンス             \\\hline
                    $c_{i, j}^{k}$             & $J_{i}^{k}$ に含まれる$c_{i, j}$ のコールバックインスタンス    \\\hline
                    $R\left(J_{i}^{k}\right)$  & $J_{i}^{k}$の応答時間                                          \\\hline
                    $\mathcal{R}_i^{wc}$       & $\mathcal{C}_{i}$の最悪応答時間                                \\\hline
                    $\Omega$                   & ready セット                                                   \\\hline
                    $h p\left(c_{i, j}\right)$ & コールバック $c_{i, j}$ よりも優先度の高いコールバックのセット \\\hline
                \end{tabular}
            }
        \end{table}
    }
\end{frame}

\begin{frame}{表記法・用語 4}
    \full{
        \begin{table}[tb]
            \adjustbox{max width=\textwidth, max height=\slideheight}{
                \centering\begin{tabular}{|c|l|} \hline
                    updated  & $\Omega$ に新しい要素が追加されること                                                      \\\hline
                    バッチ   & \tabml{複数のコールバックインスタンスが同じポーリングポイントで $\Omega$ に追加された場合, \\これらのインスタンスは同じバッチである} \\\hline
                    ビジー   & スレッドがコールバックインスタンスが実行している状態                                       \\\hline
                    アイドル & スレッドがコールバックインスタンスを実行していない状態                                     \\\hline
                \end{tabular}
            }
        \end{table}
    }
\end{frame}

\begin{frame}{表記法・用語 5}
    \full{
        \begin{table}[tb]
            \adjustbox{max width=\textwidth, max height=\slideheight}{
                \centering\begin{tabular}{|c|l|} \hline
                    $J$   & 分析対象のチェーン                     \\\hline
                    $r$   & $J$のリリース時刻                      \\\hline
                    $f$   & $J$の終了時刻                          \\\hline
                    $c_i$ & $J$の$i$番目のコールバックインスタンス \\\hline
                    $r_i$ & $c_i$がリリースされる時刻              \\\hline
                    $s_i$ & $c_i$が実行開始する時刻                \\\hline
                    $|J|$ & $J$内のコールバックの数                \\\hline
                \end{tabular}
            }
        \end{table}
    }
\end{frame}


\begin{frame}{表記法・用語 6}
    \full{
        \begin{table}[tb]
            \adjustbox{max width=\textwidth, max height=\slideheight}{
                \centering\begin{tabular}{|c|l|} \hline
                    $\mathcal{S}_{k, i}=\left\langle e_{k, a}^{\prime}, e_{k, b}^{\prime}, \cdots\right\rangle$                                                         & $c_i$に対する$\mathcal{C}_{k}$のサブ干渉シーケンス                                            \\\hline
                    $e_{k, a}^{\prime}$                                                                                                                                 & コールバックインスタンス $c_{k, a}$ が $\left[r_{i}, s_{i}\right)$ の間に実行された時間の長さ \\\hline
                    $\mathcal{S}_{k}=\left\{\mathcal{S}_{k, 1}, \mathcal{S}_{k, 2}, \cdots, \mathcal{S}_{k,|\mathcal{C}|}\right\}$                                      & $J$に対する$\mathcal{C}_{k}$の干渉シーケンス                                                  \\\hline
                    $\mathcal{I}_{k,i}$                                                                                                                                 & $c_i$に対する$\mathcal{C}_{k}$の干渉作業                                                      \\\hline
                    $\mathcal{I}_{k}$                                                                                                                                   & $J$に対する$\mathcal{C}_{k}$の干渉作業                                                        \\\hline
                    $\mathcal{I}_{k,i}^\mathcal{P} $                                                                                                                    & \tabml{$c_i$がブロックされている間に, $c_i$と同じコールバックグループに属す                   \\$\mathcal{C}_k$のコールバックインスタンスが実行した時間の総和} \\\hline
                    $\mathcal{I}_{k,i}^\mathcal{E} $                                                                                                                    & \tabml{$c_i$がブロックされている間に, $c_i$と異なるコールバックグループに属す                 \\$\mathcal{C}_k$のコールバックインスタンスが実行した時間の総和} \\\hline
                    $\mathcal{I}_{k,i}^\mathcal{B}  $                                                                                                                   & \tabml{$[r_i, s_i)$の間に, $c_i$と同じmutually exclusiveコールバックグループに属す            \\$\mathcal{C}_k$のコールバックインスタンスが実行した時間の総和} \\\hline
                    $\mathcal{Q}_{k}=\sum_{i=1}^{|\mathcal{C}|}\left(\mathcal{I}_{k, i}+(m-1) \mathcal{I}_{k, i}^{\mathcal{B}}-\mathcal{I}_{k, i}^{\mathcal{E}}\right)$ & $\mathcal{C}_{k}$の実行が$J$の終了時間に与える影響を特徴付けるために開発した項                \\\hline
                    $\Phi_{k,i}$                                                                                                                                        & $\mathcal{Q}_k$に対する $\mathcal{S}_{k,i}$の寄与                                             \\\hline
                \end{tabular}
            }
        \end{table}
    }
\end{frame}

\begin{frame}{表記法・用語 7}
    \full{
        \begin{table}[tb]
            \adjustbox{max width=\textwidth, max height=\slideheight}{
                \centering\begin{tabular}{|c|l|} \hline
                    $L$                                    & 問題ウィンドウの長さ                                                                                         \\\hline
                    $n_{k}(L)$                             & $J$ の問題ウィンドウ中に実行できる $\mathcal{C}_{k}$ のチェーンインスタンスの最大数                          \\\hline
                    $\overrightarrow{\mathcal{M}}_{k}$     & $J$に対する$\mathcal{C}_{k}$の超干渉シーケンス                                                               \\\hline
                    $ \hat{\Phi} $                         & $\Phi_{k,i}$の上界                                                                                           \\\hline
                    $\mathcal{G}(c_{i,j}) $                & \tabml{$c_{i,j}$が属すmutually exclusiveコールバックグループのインデックス}                                  \\\hline
                    $\theta_i$                             & \tabml{$\mathcal{C}_{i}$ の各コールバックが属すmutually exclusiveコールバックグループの集合                  \\ $\theta_{i}=\cup_{\forall c_{i, j} \in \mathcal{C}_{i}}\left\{\mathcal{G}\left(c_{i, j}\right)\right\}$} \\\hline
                    $\mathcal{I}_{k, i}^{\mathcal{E}^{*}}$ & $\mathcal{S}_{k, i}$ 内の $c_{i}$ とは異なるコールバックグループに属すコールバックインスタンスの合計実行時間 \\\hline
                    $\mathcal{Q}_{k}(\mathcal{Y})$         & 各サブ干渉シーケンス $\mathcal{S}_{k, i}$ に関する $\hat{\Phi}_{k, i}$ の合計                                \\\hline
                \end{tabular}
            }
        \end{table}
    }
\end{frame}

\begin{frame}{表記法・用語 8}
    \full{
        \begin{table}[tb]
            \adjustbox{max width=\textwidth, max height=\slideheight}{
                \centering\begin{tabular}{|c|l|} \hline
                    $\Phi_{k, i}^{p, q}$                            & \tabml{$\overrightarrow{\mathcal{M}}_{k}$ の$ p $番目のコールバックインスタンスの開始時刻から \\$ q $番目のコールバックインスタンスの開始時刻までの \\範囲内にある任意のウィンドウによって発生しうる最大の $\hat{\Phi}_{k, i}$} \\\hline
                    $\left|\overrightarrow{\mathcal{M}}_{k}\right|$ & $\overrightarrow{\mathcal{M}}_{k}$ のコールバックインスタンスの数                             \\\hline
                    $\Theta_{i, p}$ ($i \in[1,|\mathcal{C}|])$      & \tabml{$ p $番目のコールバックインスタンスの開始時刻から                                      \\$\overrightarrow{\mathcal{M}}_{k}$ の最後のコールバックインスタンスの終了時刻までの \\範囲に収まる任意のウィンドウによって発生しうる最大の $\sum_{j=i}^{|\mathcal{C}|} \hat{\Phi}_{k, j}$} \\\hline
                \end{tabular}
            }
        \end{table}
    }
\end{frame}


\begin{frame}{DAGへのモデル化}
    \begin{itemize}
        \item ROS システムを, 一連のコールバック $\mathcal{C}=\left\{c_{1}, \ldots, c_{n}\right\}$ と一連のエッジ $\mathcal{E} \subseteq \mathcal{C} \times \mathcal{C}$ で構成される有向非巡回グラフ (DAG) $\mathcal{D}=\{\mathcal{C}, \mathcal{E}\}$ としてモデル化
        \item 各コールバックは無限のインスタンスシーケンスをリリースする
    \end{itemize}
    \assume{$\mathcal{D}$ は固定されている, すなわち, コールバックは実行時にシステムに参加したりシステムから離れたりしない}
\end{frame}

\begin{frame}{離散時間モデル}
    本論文では離散時間モデルを想定する
    \begin{block}{離散時間モデル}
        全ての時間パラメーターが基本時間単位 (プロセッササイクルなど) の整数倍
    \end{block}
\end{frame}

% \begin{frame}{pending, ready}
%     コールバックはDDS レイヤがメッセージを受信したとき, またはタイマ周期でアクティベーションされる
%     \begin{definition}[pending]
%         コールバックのインスタンスがアクティベーションされてから完了するまでの間, そのインスタンスは pending である
%     \end{definition}
%     \begin{definition}[ready]
%         コールバックインスタンスが pending だが実行されていない場合, そのインスタンスは ready である
%     \end{definition}
% \end{frame}

\begin{frame}{エッジ}
    \begin{itemize}
        \item 各エッジ $\left(c_{i}, c_{j}\right) \in \mathcal{E}$ は, $c_{i}$ から $c_{j}$ へのアクティベーション関係を示す
        \item $c_{i}$ のインスタンスの実行中に, $c_{j}$ のインスタンスを 1 つまでアクティベーションする
        \item コールバックは, 複数のpublisherがあるトピックをsubscribeする場合, 複数のエッジを持つ
              \begin{itemize}
                  \item この場合, 全てのsubscriberは, トピックにpublishされたメッセージごとに 1 回トリガされる
              \end{itemize}
    \end{itemize}
\end{frame}

\begin{frame}{処理チェーン}
    \fitimage{
        \begin{itemize}
            \item ROS グラフ $\mathcal{D}$ は, 複数のソースコールバックとシンクコールバックを持つことができる
            \item 各ソースは, 1 つまたは複数のコールバックチェーン $\gamma^{x}=\left(c_{s}, \ldots, c_{e}\right)$ に含まれる
            \item コールバックは複数のチェーンで共有される可能性がある
        \end{itemize}
    }{chain_example}
\end{frame}

\begin{frame}{アクティベーションモデル}
    各ソースコールバック $c_{s}$ は, 外部イベント到着曲線 $\eta_{s}^{e}(\Delta)$ を持つ
    \begin{block}{外部イベント到着曲線 $\eta_{s}^{e}$}
        \setlength{\linewidth}{0.98\columnwidth}
        \begin{itemize}
        \item $\eta_{s}^{e}(\Delta)$は任意の時間間隔 $[t, t+\Delta)$ に到着するイベント数の上界
        \item 例えば, タスク $T_{x}$ が定期的にトリガされる場合, $\eta_{x}^{e}(\Delta)=\lceil\frac{\Delta}{\operatorname{period}(x)}\rceil$ として表せる
        \end{itemize}
    \end{block}
    % \begin{itemize}
    %     \item 各ソースコールバック $c_{s}$ は, 外部イベント到着曲線 $\eta_{s}^{e}(\Delta)$ \cite{henia2005system, leboudec2001theory, thiele2000real}に関連付けられている
    %     % \item 本論文では一般性を失わず, $\Delta>0$ かつ $\eta_{s}^{e}(\Delta)>0$ を想定する
    % \end{itemize}
\end{frame}

\begin{frame}{エグゼキュータのスケジューリング}
    \begin{itemize}
        \item 本論文では, 組み込みのシングルスレッドエグゼキュータを採用する
        \item 各エグゼキュータが単一の予約サーバに割り当てられ, 各予約サーバが単一のエグゼキュータを処理すると仮定する
        % \item したがって, エグゼキュータに割り当てられたコールバックは, 対応する予約サーバに割り当てられたと見なせる
    \end{itemize}
\end{frame}

\begin{frame}{予約}
    \begin{itemize}
        \item システムは $w$個の予約 $\mathcal{R}=\left\{r_{1}, \ldots, r_{w}\right\}$ のセットで構成される
        \item 予約は $m$個のプロセッサ $\mathcal{P}=\left\{p_{1}, \ldots, p_{m}\right\}$ のセットに分割される
        \item 各予約は静的にプロセッサに割り当てられる
    \end{itemize}
\end{frame}

\begin{frame}{供給境界関数}
    \begin{itemize}
        \item 本論文は, 長さ $\Delta$ の任意の間隔で予約 $r_{k}$ によって提供される最小サービスを示す供給境界関数 $sbf_{k}(\Delta)$ \cite{lipari2003resource, shin2003periodic} に依存する
        \item 複数の予約が同じプロセッサに割り当てられる場合は, 全ての予約がスケジュール可能である場合にのみ, 供給境界関数によって記述されるリソースの割り当てが保証される
    \end{itemize}
    \assume{予約はスケジュール可能であると想定する}
\end{frame}

\begin{frame}{最悪通信レイテンシ}
    予約 $\left(r_{x}, r_{y}\right)$ の各ペアは, 最悪通信レイテンシ $\delta_{i, j}$ によって特徴付けられる
    \begin{block}{最悪通信レイテンシ $\delta_{i, j}$}
        $r_x$ に割り当てられた送信コールバック $c_i$ のDDS層から, $r_y$ に割り当てられた受信コールバック $c_j$ のDDS層によって受信されるまでに送信されるメッセージが経験する最大時間, すなわち, $c_j$ によって起動される前に経験する最大レイテンシ
    \end{block}
    \assume{$r_{x}=r_{y}$ の時, $\delta_{i, j}$ は無視できると想定する}
\end{frame}

\begin{frame}{イベントソース}
    イベントソースを疑似コールバック $c_{i} \in \mathcal{C}^{\text {evt}}$としてモデル化する
    \begin{block}{イベントソース}
        散発的にアクティベーションされる OS レベルのスレッドであり, 1 つまたは複数のトピックにpublishすることで ROS と対話し, 外部イベントのインターフェイスまたは入口ポイントとして機能する
    \end{block}
    \assume{各イベントソースは専用の予約に排他的に割り当てられると想定する}
\end{frame}
