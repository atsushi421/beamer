% !TeX root = main.tex

\section{System Model}
\label{sec: system model}

\begin{frame}{セクションサマリ}
    \begin{itembox}[l]{\textbf{目的}}
        ROS システムのタイミング関連の側面のモデル, そのコールバック, およびそれらのアクティベーション関係を紹介する
    \end{itembox}
\end{frame}

\begin{frame}{}
    \begin{itemize}
        \item 始めに, 本論文で登場する表記法・用語の表を示す
        \item 基本的な表記法・用語は資料中で説明無しで使用する
        \item 別ファイルで開く・印刷するなどして, 常に参照できる状態にしておくことを推奨する
    \end{itemize}
\end{frame}

% !TeX root = main.tex

\begin{frame}{表記法・用語 1}
    \full{
        \begin{table}[tb]
            \adjustbox{max width=\textwidth, max height=\slideheight}{
                \centering\begin{tabular}{|c|l|} \hline
                    $N$      & タスクの数                 \\\hline
                    $M$      & プロセッサの個数           \\\hline
                    $\tau_i$ & タスク                     \\\hline
                    $T_i$    & $\tau_i$の周期             \\\hline
                    $D_i$    & $\tau_i$の相対デッドライン \\\hline
                    $J_{i}$  & $\tau_i$のジョブ           \\\hline
                \end{tabular}
            }
        \end{table}
    }
\end{frame}

\begin{frame}{表記法・用語 2}
    \full{
        \begin{table}[tb]
            \adjustbox{max width=\textwidth, max height=\slideheight}{
                \centering\begin{tabular}{|c|l|} \hline
                    $G\left(J_{i}\right)=\langle V, E\rangle$ & $J_{i}$のDAG                                 \\\hline
                    $V$                                       & 頂点のセット                                 \\\hline
                    $E$                                       & エッジのセット                               \\\hline
                    $v$                                       & ワークロードの一部                           \\\hline
                    $c(v)$                                    & $v$のWCET                                    \\\hline
                    $(u, v) \in E$                            & $u$ と $v$ の間の優先関係                    \\\hline
                    先行頂点                                  & エッジ$(u, v)$があるとき, $u$は$v$の先行頂点 \\\hline
                    後続頂点                                  & エッジ$(u, v)$があるとき, $v$は$u$の後続頂点 \\\hline
                    ソース頂点                                & 先行頂点を持たない頂点                       \\\hline
                    シンク頂点                                & 後続頂点を持たない頂点                       \\\hline
                \end{tabular}
            }
        \end{table}
    }
\end{frame}

\begin{frame}{表記法・用語 3}
    \full{
        \begin{table}[tb]
            \adjustbox{max width=\textwidth, max height=\slideheight}{
                \centering\begin{tabular}{|c|l|} \hline
                    適格     & 頂点は, その先行頂点が全て終了している場合に適格である \\\hline
                    完全パス & ソース頂点で始まり, シンク頂点で終わるパス             \\\hline
                    $C_i$    & $G(J_i)$内の全ての頂点の合計WCET                       \\\hline
                    $L_i$    & $G(J_i)$内のクリティカルパス上の全ての頂点の合計 WCET  \\\hline
                \end{tabular}
            }
        \end{table}
    }
\end{frame}

\begin{frame}{表記法・用語 4}
\full{
\begin{table}[tb]
\adjustbox{max width=\textwidth, max height=\slideheight}{
\centering\begin{tabular}{|c|l|} \hline
    $\Theta=\left\{\theta_{1}, \cdots, \theta_{|\Theta|}\right\}$ & アクティブVPのセット \\\hline
   $\theta_{z}$ & \tabml{$\Theta$ 内の $z$番目のアクティブVPの初期バジェット \\コンテキストから明白な場合は$\Theta$ 内の $z$番目のアクティブVPも示す} \\\hline

\end{tabular}
}
\end{table}
}
\end{frame}


\begin{frame}{}
    \begin{itemize}
        \item ROS システムを, 一連のコールバック $\mathcal{C}=\left\{c_{1}, \ldots, c_{n}\right\}$ と一連の有向辺 $\mathcal{E} \subseteq \mathcal{C} \times \mathcal{C}$ で構成される直接非巡回グラフ (DAG) $\mathcal{D}=\{\mathcal{C}, \mathcal{E}\}$ としてモデル化する
\item グラフ $\mathcal{D}$ は固定されていると仮定する
\item すなわち, コールバックは実行時にシステムに参加したりシステムから離れたりすることはできない
    \end{itemize}
\end{frame}

\begin{frame}{}
    \begin{itemize}
        \item 各コールバック $c_{i} \in \mathcal{C}$ には, 最悪実行時間 $e_{i}$, 一意の優先度 $\pi_{i}$ があり, 潜在的に無限のインスタンスシーケンスをリリースする
\item 離散時間モデルを想定しているすなわち, 全ての時間パラメーターは, 基本的な時間単位 (プロセッササイクルなど) の整数倍である
    \end{itemize}
\end{frame}

\begin{frame}{}
    \begin{itemize}
        \item そのタイプに応じて, DDS レイヤがメッセージを受信したとき, またはタイマがデッドライン切れになったときに, コールバックインスタンスがアクティベーションされる
\item コールバックのインスタンスがアクティベーションされると, それは保留中であると言われ, 完了するまで保留されたままになる
\item コールバックインスタンスは, 保留中であるが実行されていない場合, 準備ができていると言われる
\item 各エッジ $\left(c_{i}, c_{j}\right) \in \mathcal{E}$ は, コールバック $c_{i}$ からコールバック $c_{j}$ へのアクティベーション関係をエンコードする
\item すなわち, $c_{i}$ のインスタンスの実行中に, $c_{j}$ のインスタンスを 1 つまでアクティベーションする (例えば, $c_{j}$ がsubscribeしているトピックにメッセージをpublishすることによって)
    \end{itemize}
\end{frame}

\begin{frame}{処理チェーン}
    \begin{itemize}
        \item ROS グラフ $\mathcal{D}$ は, 複数のソースコールバックとシンクコールバックを持つことができる
\item 各ソースは, 1 つまたは複数のコールバックチェーン $\gamma^{x}=\left(c_{s}, \ldots, c_{e}\right)$, すなわちグラフ内の有向パスを生成する
\item ソースコールバックから他のコールバックへのグラフの全てのチェーンのセットは, $\operatorname{chains}(\mathcal{D})=\left\{\gamma^{1}, \ldots, \gamma^{s}\right\}$ で示される
\item コールバックは複数のチェーンで共有できる複数のチェーンを持つ ROS グラフの例を図 4 に示す
    \end{itemize}
\end{frame}

\begin{frame}{アクティベーションモデル}
    \begin{itemize}
        \item CPA と同様に, 各ソースコールバック $c_{s}$ は, 特定の外部イベント到着曲線 $\eta_{s}^{e}(\Delta)$ に関連付けられている
\item これは, 長さ $\Delta$ の任意の間隔でリリースできる $c_{s}$ のインスタンスの最大数を示す
\item アクティベーション曲線
\item w.l.o.gを想定している $\Delta>0$ の $\eta_{s}^{e}(\Delta)>0$
    \end{itemize}
\end{frame}

\begin{frame}{}
    \begin{itemize}
        \item セクション 2.1 で説明したように, 非タイマコールバックはデータ駆動型の方法でアクティベーションされる
\item このモデルでは, コールバックが単一のタイマ, トピック, またはサービスに属していると想定している
\item (2 つのコールバックが同じコードを実行する可能性があるため, これは制限ではない
\item ) したがって, コールバックは, 複数のpublisherを持つトピックにsubscribeする場合にのみ, 複数の着信エッジを持つことができる 他のジョブ[27])
\item この場合, 全てのsubscriberは, トピックにpublishされたメッセージごとに 1 回トリガされる
\item 複数の着信エッジを持つコールバックのアクティベーションカーブの導出については, セクション 5.1 で説明する
    \end{itemize}
\end{frame}

\begin{frame}{エグゼキュータのスケジューリング}
    \begin{itemize}
        \item セクション 3 で説明したように, 本論文では組み込みのシングルスレッドエグゼキュータを採用している
\item OS に ROS エグゼキュータへの予測可能なサービスを保証させるために, 各エグゼキュータ (すなわち, 各スレッド) が単一の予約サーバに割り当てられ, 各予約サーバが単一のエグゼキュータを処理すると仮定する
\item したがって, エグゼキュータに割り当てられたコールバックは, 対応する予約サーバに割り当てられたものと同等に見なすことができる
    \end{itemize}
\end{frame}

\begin{frame}{}
    \begin{itemize}
        \item システムは $w$ 予約 $\mathcal{R}=\left\{r_{1}, \ldots, r_{w}\right\}$ のセットで構成される
\item 予約は $m$ プロセッサ $\mathcal{P}=\left\{p_{1}, \ldots, p_{m}\right\}$ のセットに分割される
\item すなわち, 各予約は静的にプロセッサに割り当てられる
    \end{itemize}
\end{frame}

\begin{frame}{}
    \begin{itemize}
        \item 本論文で提示された結果は, 長さ $\Delta$ の任意の間隔で予約 $r_{k}$ によって提供される最小サービスを示す供給限界関数 $s b f_{k}(\Delta)$ の可用性に依存している (セクション 2.2 を思い出してください)
\item 複数の予約が同じプロセッサに割り当てられるときはいつでも, 全ての予約がスケジュール可能である場合にのみ, 供給限界関数によって記述されるリソースのプロビジョニングが保証される
\item 本論文では, 予約はスケジュール可能であると想定している
\item 予約がタイミング制約を満たすことを保証する問題は, 以前の研究ではグローバルスケジューラビリティとも呼ばれる (例えば, 階層型スケジューリング [9] のコンテキストで)
\item [37] など, グローバルなスケジューラビリティを確保するために, 多くの結果が利用可能である
    \end{itemize}
\end{frame}

\begin{frame}{伝搬レイテンシ}
    \begin{itemize}
        \item 送信コールバックによるメッセージのパブリッシュと, 関連付けられた受信コールバックのアクティベーションの間の伝播レイテンシは, 重大な場合がある
\item 実際, ROS システムは本質的に分散トポロジであるため, メッセージ交換にネットワークが関与し, 追加のレイテンシが発生する可能性がある
\item このようなレイテンシをモデル化するために, 予約 $\left(r_{x}, r_{y}\right)$ の各ペアは, (DDS に依存する) 最悪通信レイテンシ $\delta_{i, j}$ によって特徴付けられる
\item $r_{y}$ に割り当てられた受信コールバック $c_{j}$ の DDS 層によって受信された, すなわち, $c_{j}$ がアクティブになる前に経験した最大の追加レイテンシ
\item $r_{x}=r_{y}, \delta_{i, j}$ が無視できると想定される場合
\item このレイテンシは, ターゲットアプリケーションドメインの要件に応じて, さまざまな種類のネットワークに対して分析的に上界を設定するか (例えば,  $[17,19,51]$ を参照), 実用的に測定することができる
    \end{itemize}
\end{frame}

\begin{frame}{イベントソース}
    \begin{itemize}
        \item タイマを除いて, ROS によって提供される全てのコールバックタイプは, データ駆動型アクティベーションセマンティクスを実装する
\item したがって, ROS コールバックのみで構成される全てのチェーンは, 最初はタイマによってトリガされる
\item それにもかかわらず, アプリケーションは多くの場合, 割り込みを介して非同期に配信される外部イベントに対応する必要がある (例えば, 特定のセンサ, 監視コントローラーまたは人間のオペレーターからの入力を配信するネットワークパケットなど)
\item このようなイベントを ROS モデルに統合するために, 外部スレッドが ROS と対話し, 疑似コールバックとしてモデル化できるようにする
\item 具体的には, これらのスレッドのイベントソースに名前を付ける
\item イベントソースは, 散発的にアクティベーションされる通常の OS レベルのスレッドであり, 1 つまたは複数のトピックにpublishすることで ROS と対話し, 外部イベントのインターフェイスまたは入口ポイントとして機能する
\item Executor の場合と同様に, 各イベントソースが専用の予約に排他的に割り当てられると想定している
\item イベントソースをコールバック $c_{i} \in \mathcal{C}^{\text {evt }}$ と呼ぶ
    \end{itemize}
\end{frame}
