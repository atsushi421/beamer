% !TeX root = main.tex

\section{System Model}
\label{sec: system model}

\begin{frame}{セクションサマリ}
    \begin{itembox}[l]{\textbf{目的}}
        ROS システムのタイミング関連の側面のモデル, そのコールバック, およびそれらのアクティベーション関係を紹介する
    \end{itembox}
\end{frame}

\begin{frame}{}
    \begin{itemize}
        \item 始めに, 本論文で登場する表記法・用語の表を示す
        \item 基本的な表記法・用語は資料中で説明無しで使用する
        \item 別ファイルで開く・印刷するなどして, 常に参照できる状態にしておくことを推奨する
    \end{itemize}
\end{frame}

% !TeX root = main.tex

\begin{frame}{表記法・用語 1}
    \full{
        \begin{table}[tb]
            \adjustbox{max width=\textwidth, max height=\slideheight}{
                \centering\begin{tabular}{|c|l|} \hline
                    $m$                                                                                             & スレッド数                                                                                        \\\hline
                    $\Gamma=\left\{\mathcal{C}_{1}, \mathcal{C}_{2}, \cdots, \mathcal{C}_{|\Gamma|}\right\}$        & チェインのセット                                                                                  \\\hline
                    $|\Gamma|$                                                                                      & $\Gamma$内のチェインの数                                                                          \\\hline
                    $\mathcal{C}_{i}=\left\{c_{i, 1}, c_{i, 2}, \cdots, c_{i,\left|\mathcal{C}_{i}\right|}\right\}$ & チェイン                                                                                          \\\hline
                    $c_{i,j}$                                                                                       & $\mathcal{C}_{i}$の$j$番目のコールバック                                                          \\\hline
                    $\left|\mathcal{C}_{i}\right|$                                                                  & $\mathcal{C}_{i}$内のコールバックの数                                                             \\\hline
                    先行要素                                                                                        & $J_{i}^{k}$ 内の連続する要素 $c_{i, j}^{k}$ と $c_{i, j+1}^{k}$ の各ペアにおける $c_{i, j}^{k}$   \\\hline
                    後続要素                                                                                        & $J_{i}^{k}$ 内の連続する要素 $c_{i, j}^{k}$ と $c_{i, j+1}^{k}$ の各ペアにおける $c_{i, j+1}^{k}$ \\\hline
                    ソースコールバック                                                                              & $\mathcal{C}_{i}$ の最初のコールバック                                                            \\\hline
                    シンクコールバック                                                                              & $\mathcal{C}_{i}$ の最後のコールバック                                                            \\\hline
                \end{tabular}
            }
        \end{table}
    }
\end{frame}

\begin{frame}{表記法・用語 2}
    \full{
        \begin{table}[tb]
            \adjustbox{max width=\textwidth, max height=\slideheight}{
                \centering\begin{tabular}{|c|l|} \hline
                    $T_i$                 & \tabml{$\mathcal{C}_{i}$の周期                \\\underline{周期}: 2 つの連続するチェーンインスタンスのリリース時刻の間の最小間隔}                       \\\hline
                    $D_i$                 & \tabml{$\mathcal{C}_{i}$の相対デッドライン    \\\underline{相対デッドライン}: 時間 $r$ でリリースされた $\mathcal{C}_{i}$ の各チェーンインスタンスは, \\その絶対デッドライン $r+D_{i}$ までに終了する必要がある}           \\\hline
                    $e_{i,j}$             & $c_{i,j}$の最悪実行時間 (WCET)                \\\hline
                    $E_i$                 & $\mathcal{C}_{i}$内のコールバックのWCETの合計 \\\hline
                    $U_{i}=E_{i} / T_{i}$ & $\mathcal{C}_{i}$の利用率                     \\\hline
                \end{tabular}
            }
        \end{table}
    }
\end{frame}

\begin{frame}{表記法・用語 3}
    \full{
        \begin{table}[tb]
            \adjustbox{max width=\textwidth}{
                \centering\begin{tabular}{|c|l|} \hline
                    $J_{i}^{k}$                & $\mathcal{C}_{i}$ の $k$番目のチェーンインスタンス             \\\hline
                    $c_{i, j}^{k}$             & $J_{i}^{k}$ に含まれる$c_{i, j}$ のコールバックインスタンス    \\\hline
                    $R\left(J_{i}^{k}\right)$  & $J_{i}^{k}$の応答時間                                          \\\hline
                    $\mathcal{R}_i^{wc}$       & $\mathcal{C}_{i}$の最悪応答時間                                \\\hline
                    $\Omega$                   & ready セット                                                   \\\hline
                    $h p\left(c_{i, j}\right)$ & コールバック $c_{i, j}$ よりも優先度の高いコールバックのセット \\\hline
                \end{tabular}
            }
        \end{table}
    }
\end{frame}

\begin{frame}{表記法・用語 4}
    \full{
        \begin{table}[tb]
            \adjustbox{max width=\textwidth, max height=\slideheight}{
                \centering\begin{tabular}{|c|l|} \hline
                    updated  & $\Omega$ に新しい要素が追加されること                                                      \\\hline
                    バッチ   & \tabml{複数のコールバックインスタンスが同じポーリングポイントで $\Omega$ に追加された場合, \\これらのインスタンスは同じバッチである} \\\hline
                    ビジー   & スレッドがコールバックインスタンスが実行している状態                                       \\\hline
                    アイドル & スレッドがコールバックインスタンスを実行していない状態                                     \\\hline
                \end{tabular}
            }
        \end{table}
    }
\end{frame}

\begin{frame}{表記法・用語 5}
    \full{
        \begin{table}[tb]
            \adjustbox{max width=\textwidth, max height=\slideheight}{
                \centering\begin{tabular}{|c|l|} \hline
                    $J$   & 分析対象のチェーン                     \\\hline
                    $r$   & $J$のリリース時刻                      \\\hline
                    $f$   & $J$の終了時刻                          \\\hline
                    $c_i$ & $J$の$i$番目のコールバックインスタンス \\\hline
                    $r_i$ & $c_i$がリリースされる時刻              \\\hline
                    $s_i$ & $c_i$が実行開始する時刻                \\\hline
                    $|J|$ & $J$内のコールバックの数                \\\hline
                \end{tabular}
            }
        \end{table}
    }
\end{frame}


\begin{frame}{表記法・用語 6}
    \full{
        \begin{table}[tb]
            \adjustbox{max width=\textwidth, max height=\slideheight}{
                \centering\begin{tabular}{|c|l|} \hline
                    $\mathcal{S}_{k, i}=\left\langle e_{k, a}^{\prime}, e_{k, b}^{\prime}, \cdots\right\rangle$                                                         & $c_i$に対する$\mathcal{C}_{k}$のサブ干渉シーケンス                                            \\\hline
                    $e_{k, a}^{\prime}$                                                                                                                                 & コールバックインスタンス $c_{k, a}$ が $\left[r_{i}, s_{i}\right)$ の間に実行された時間の長さ \\\hline
                    $\mathcal{S}_{k}=\left\{\mathcal{S}_{k, 1}, \mathcal{S}_{k, 2}, \cdots, \mathcal{S}_{k,|\mathcal{C}|}\right\}$                                      & $J$に対する$\mathcal{C}_{k}$の干渉シーケンス                                                  \\\hline
                    $\mathcal{I}_{k,i}$                                                                                                                                 & $c_i$に対する$\mathcal{C}_{k}$の干渉作業                                                      \\\hline
                    $\mathcal{I}_{k}$                                                                                                                                   & $J$に対する$\mathcal{C}_{k}$の干渉作業                                                        \\\hline
                    $\mathcal{I}_{k,i}^\mathcal{P} $                                                                                                                    & \tabml{$c_i$がブロックされている間に, $c_i$と同じコールバックグループに属す                   \\$\mathcal{C}_k$のコールバックインスタンスが実行した時間の総和} \\\hline
                    $\mathcal{I}_{k,i}^\mathcal{E} $                                                                                                                    & \tabml{$c_i$がブロックされている間に, $c_i$と異なるコールバックグループに属す                 \\$\mathcal{C}_k$のコールバックインスタンスが実行した時間の総和} \\\hline
                    $\mathcal{I}_{k,i}^\mathcal{B}  $                                                                                                                   & \tabml{$[r_i, s_i)$の間に, $c_i$と同じmutually exclusiveコールバックグループに属す            \\$\mathcal{C}_k$のコールバックインスタンスが実行した時間の総和} \\\hline
                    $\mathcal{Q}_{k}=\sum_{i=1}^{|\mathcal{C}|}\left(\mathcal{I}_{k, i}+(m-1) \mathcal{I}_{k, i}^{\mathcal{B}}-\mathcal{I}_{k, i}^{\mathcal{E}}\right)$ & $\mathcal{C}_{k}$の実行が$J$の終了時間に与える影響を特徴付けるために開発した項                \\\hline
                    $\Phi_{k,i}$                                                                                                                                        & $\mathcal{Q}_k$に対する $\mathcal{S}_{k,i}$の寄与                                             \\\hline
                \end{tabular}
            }
        \end{table}
    }
\end{frame}

\begin{frame}{表記法・用語 7}
    \full{
        \begin{table}[tb]
            \adjustbox{max width=\textwidth, max height=\slideheight}{
                \centering\begin{tabular}{|c|l|} \hline
                    $L$                                    & 問題ウィンドウの長さ                                                                                         \\\hline
                    $n_{k}(L)$                             & $J$ の問題ウィンドウ中に実行できる $\mathcal{C}_{k}$ のチェーンインスタンスの最大数                          \\\hline
                    $\overrightarrow{\mathcal{M}}_{k}$     & $J$に対する$\mathcal{C}_{k}$の超干渉シーケンス                                                               \\\hline
                    $ \hat{\Phi} $                         & $\Phi_{k,i}$の上界                                                                                           \\\hline
                    $\mathcal{G}(c_{i,j}) $                & \tabml{$c_{i,j}$が属すmutually exclusiveコールバックグループのインデックス}                                  \\\hline
                    $\theta_i$                             & \tabml{$\mathcal{C}_{i}$ の各コールバックが属すmutually exclusiveコールバックグループの集合                  \\ $\theta_{i}=\cup_{\forall c_{i, j} \in \mathcal{C}_{i}}\left\{\mathcal{G}\left(c_{i, j}\right)\right\}$} \\\hline
                    $\mathcal{I}_{k, i}^{\mathcal{E}^{*}}$ & $\mathcal{S}_{k, i}$ 内の $c_{i}$ とは異なるコールバックグループに属すコールバックインスタンスの合計実行時間 \\\hline
                    $\mathcal{Q}_{k}(\mathcal{Y})$         & 各サブ干渉シーケンス $\mathcal{S}_{k, i}$ に関する $\hat{\Phi}_{k, i}$ の合計                                \\\hline
                \end{tabular}
            }
        \end{table}
    }
\end{frame}

\begin{frame}{表記法・用語 8}
    \full{
        \begin{table}[tb]
            \adjustbox{max width=\textwidth, max height=\slideheight}{
                \centering\begin{tabular}{|c|l|} \hline
                    $\Phi_{k, i}^{p, q}$                            & \tabml{$\overrightarrow{\mathcal{M}}_{k}$ の$ p $番目のコールバックインスタンスの開始時刻から \\$ q $番目のコールバックインスタンスの開始時刻までの \\範囲内にある任意のウィンドウによって発生しうる最大の $\hat{\Phi}_{k, i}$} \\\hline
                    $\left|\overrightarrow{\mathcal{M}}_{k}\right|$ & $\overrightarrow{\mathcal{M}}_{k}$ のコールバックインスタンスの数                             \\\hline
                    $\Theta_{i, p}$ ($i \in[1,|\mathcal{C}|])$      & \tabml{$ p $番目のコールバックインスタンスの開始時刻から                                      \\$\overrightarrow{\mathcal{M}}_{k}$ の最後のコールバックインスタンスの終了時刻までの \\範囲に収まる任意のウィンドウによって発生しうる最大の $\sum_{j=i}^{|\mathcal{C}|} \hat{\Phi}_{k, j}$} \\\hline
                \end{tabular}
            }
        \end{table}
    }
\end{frame}


\begin{frame}{}
    \begin{itemize}
        \item ROS システムを, 一連のコールバック $\mathcal{C}=\left\{c_{1}, \ldots, c_{n}\right\}$ と一連の有向辺 $\mathcal{E} \subseteq \mathcal{C} \times \mathcal{C}$ で構成される直接非巡回グラフ (DAG) $\mathcal{D}=\{\mathcal{C}, \mathcal{E}\}$ としてモデル化する
\item グラフ $\mathcal{D}$ は固定されていると仮定する
\item すなわち, コールバックは実行時にシステムに参加したりシステムから離れたりすることはできない
    \end{itemize}
\end{frame}

\begin{frame}{}
    \begin{itemize}
        \item 各コールバック $c_{i} \in \mathcal{C}$ には, 最悪実行時間 $e_{i}$, 一意の優先度 $\pi_{i}$ があり, 潜在的に無限のインスタンスシーケンスをリリースする
\item 離散時間モデルを想定しているすなわち, 全ての時間パラメーターは, 基本的な時間単位 (プロセッササイクルなど) の整数倍である
    \end{itemize}
\end{frame}

\begin{frame}{}
    \begin{itemize}
        \item そのタイプに応じて, DDS レイヤがメッセージを受信したとき, またはタイマがデッドライン切れになったときに, コールバックインスタンスがアクティベーションされる
\item コールバックのインスタンスがアクティベーションされると, それは保留中であると言われ, 完了するまで保留されたままになる
\item コールバックインスタンスは, 保留中であるが実行されていない場合, 準備ができていると言われる
\item 各エッジ $\left(c_{i}, c_{j}\right) \in \mathcal{E}$ は, コールバック $c_{i}$ からコールバック $c_{j}$ へのアクティベーション関係をエンコードする
\item すなわち, $c_{i}$ のインスタンスの実行中に, $c_{j}$ のインスタンスを 1 つまでアクティベーションする (例えば, $c_{j}$ がsubscribeしているトピックにメッセージをpublishすることによって)
    \end{itemize}
\end{frame}

\begin{frame}{処理チェーン}
    \begin{itemize}
        \item ROS グラフ $\mathcal{D}$ は, 複数のソースコールバックとシンクコールバックを持つことができる
\item 各ソースは, 1 つまたは複数のコールバックチェーン $\gamma^{x}=\left(c_{s}, \ldots, c_{e}\right)$, すなわちグラフ内の有向パスを生成する
\item ソースコールバックから他のコールバックへのグラフの全てのチェーンのセットは, $\operatorname{chains}(\mathcal{D})=\left\{\gamma^{1}, \ldots, \gamma^{s}\right\}$ で示される
\item コールバックは複数のチェーンで共有できる複数のチェーンを持つ ROS グラフの例を図 4 に示す
    \end{itemize}
\end{frame}

\begin{frame}{アクティベーションモデル}
    \begin{itemize}
        \item CPA と同様に, 各ソースコールバック $c_{s}$ は, 特定の外部イベント到着曲線 $\eta_{s}^{e}(\Delta)$ に関連付けられている
\item これは, 長さ $\Delta$ の任意の間隔でリリースできる $c_{s}$ のインスタンスの最大数を示す
\item アクティベーション曲線
\item w.l.o.gを想定している $\Delta>0$ の $\eta_{s}^{e}(\Delta)>0$
    \end{itemize}
\end{frame}

\begin{frame}{}
    \begin{itemize}
        \item セクション 2.1 で説明したように, 非タイマコールバックはデータ駆動型の方法でアクティベーションされる
\item このモデルでは, コールバックが単一のタイマ, トピック, またはサービスに属していると想定している
\item (2 つのコールバックが同じコードを実行する可能性があるため, これは制限ではない
\item ) したがって, コールバックは, 複数のpublisherを持つトピックにsubscribeする場合にのみ, 複数の着信エッジを持つことができる 他のジョブ[27])
\item この場合, 全てのsubscriberは, トピックにpublishされたメッセージごとに 1 回トリガされる
\item 複数の着信エッジを持つコールバックのアクティベーションカーブの導出については, セクション 5.1 で説明する
    \end{itemize}
\end{frame}

\begin{frame}{エグゼキュータのスケジューリング}
    \begin{itemize}
        \item セクション 3 で説明したように, 本論文では組み込みのシングルスレッドエグゼキュータを採用している
\item OS に ROS エグゼキュータへの予測可能なサービスを保証させるために, 各エグゼキュータ (すなわち, 各スレッド) が単一の予約サーバに割り当てられ, 各予約サーバが単一のエグゼキュータを処理すると仮定する
\item したがって, エグゼキュータに割り当てられたコールバックは, 対応する予約サーバに割り当てられたものと同等に見なすことができる
    \end{itemize}
\end{frame}

\begin{frame}{}
    \begin{itemize}
        \item システムは $w$ 予約 $\mathcal{R}=\left\{r_{1}, \ldots, r_{w}\right\}$ のセットで構成される
\item 予約は $m$ プロセッサ $\mathcal{P}=\left\{p_{1}, \ldots, p_{m}\right\}$ のセットに分割される
\item すなわち, 各予約は静的にプロセッサに割り当てられる
    \end{itemize}
\end{frame}

\begin{frame}{}
    \begin{itemize}
        \item 本論文で提示された結果は, 長さ $\Delta$ の任意の間隔で予約 $r_{k}$ によって提供される最小サービスを示す供給限界関数 $s b f_{k}(\Delta)$ の可用性に依存している (セクション 2.2 を思い出してください)
\item 複数の予約が同じプロセッサに割り当てられるときはいつでも, 全ての予約がスケジュール可能である場合にのみ, 供給限界関数によって記述されるリソースのプロビジョニングが保証される
\item 本論文では, 予約はスケジュール可能であると想定している
\item 予約がタイミング制約を満たすことを保証する問題は, 以前の研究ではグローバルスケジューラビリティとも呼ばれる (例えば, 階層型スケジューリング [9] のコンテキストで)
\item [37] など, グローバルなスケジューラビリティを確保するために, 多くの結果が利用可能である
    \end{itemize}
\end{frame}

\begin{frame}{伝搬レイテンシ}
    \begin{itemize}
        \item 送信コールバックによるメッセージのパブリッシュと, 関連付けられた受信コールバックのアクティベーションの間の伝播レイテンシは, 重大な場合がある
\item 実際, ROS システムは本質的に分散トポロジであるため, メッセージ交換にネットワークが関与し, 追加のレイテンシが発生する可能性がある
\item このようなレイテンシをモデル化するために, 予約 $\left(r_{x}, r_{y}\right)$ の各ペアは, (DDS に依存する) 最悪通信レイテンシ $\delta_{i, j}$ によって特徴付けられる
\item $r_{y}$ に割り当てられた受信コールバック $c_{j}$ の DDS 層によって受信された, すなわち, $c_{j}$ がアクティブになる前に経験した最大の追加レイテンシ
\item $r_{x}=r_{y}, \delta_{i, j}$ が無視できると想定される場合
\item このレイテンシは, ターゲットアプリケーションドメインの要件に応じて, さまざまな種類のネットワークに対して分析的に上界を設定するか (例えば,  $[17,19,51]$ を参照), 実用的に測定することができる
    \end{itemize}
\end{frame}

\begin{frame}{イベントソース}
    \begin{itemize}
        \item タイマを除いて, ROS によって提供される全てのコールバックタイプは, データ駆動型アクティベーションセマンティクスを実装する
\item したがって, ROS コールバックのみで構成される全てのチェーンは, 最初はタイマによってトリガされる
\item それにもかかわらず, アプリケーションは多くの場合, 割り込みを介して非同期に配信される外部イベントに対応する必要がある (例えば, 特定のセンサ, 監視コントローラーまたは人間のオペレーターからの入力を配信するネットワークパケットなど)
\item このようなイベントを ROS モデルに統合するために, 外部スレッドが ROS と対話し, 疑似コールバックとしてモデル化できるようにする
\item 具体的には, これらのスレッドのイベントソースに名前を付ける
\item イベントソースは, 散発的にアクティベーションされる通常の OS レベルのスレッドであり, 1 つまたは複数のトピックにpublishすることで ROS と対話し, 外部イベントのインターフェイスまたは入口ポイントとして機能する
\item Executor の場合と同様に, 各イベントソースが専用の予約に排他的に割り当てられると想定している
\item イベントソースをコールバック $c_{i} \in \mathcal{C}^{\text {evt }}$ と呼ぶ
    \end{itemize}
\end{frame}
