% !TeX root = main.tex

\section{INTRODUCTION}
\label{sec: introduction}

\begin{frame}{}
    \begin{itemize}
        \item ロボットオペレーティングシステム [43] である ROS は, Linux ベースのロボットを設計および開発するための最も一般的なフレームワークの 1 つである 100 を超えるさまざまなロボット設計を強化し, 産業界と学界の両方で何万人もの開発者と研究者によって使用されている [4, 22]
        \item しかし, 開発から 10 年以上が経過し, ますます要求の厳しいアプリケーションに直面した結果, ROS コミュニティは, フレームワークが, 後方互換性のある方法で修正できないいくつかの長年にわたる欠点とアーキテクチャ上の制限によって妨げられていることが自明になった
    \end{itemize}
\end{frame}

\begin{frame}{}
    \begin{itemize}
        \item これが ROS 2 の開発の動機となった
        \item これは ROS の完全なリファクタリングであり, 成功したコンセプトを最新化され改善されたプラットフォームに置きます
        \item 本論文にとって特に興味深いのは, ROS 2 の主な設計目標は, フレームワークのリアルタイム機能を向上させ, ROS 内でタイムクリティカルな制御パスの実装を可能にすることである [24]
    \end{itemize}
\end{frame}


\begin{frame}{}
    \begin{itemize}
        \item このような制御パスを安全に実装するには, タイムクリティカルな処理チェーンのエンドツーエンドレイテンシ (または応答時間) を予測する必要がある
        \item 例えば, このようなチェーンは, センサからコントローラーを介して最終的なアクチュエーターまでの全てのステップをカバーし, 複数のソフトウェアコンポーネント, 複数のコア, さらには複数のホストにまたがる場合がある
        \item エンドツーエンドレイテンシの問題は文献で十分に研究されているが, 既存の研究では多くの場合, 実際のシステムに直接適用できるとは限らない理想化されたスケジューリングモデルが想定されている
    \end{itemize}
\end{frame}

\begin{frame}{}
    \begin{itemize}
        \item 適切な例として, $\operatorname{ROS}^{1}$ スケジューリングアプローチ (セクション 3 で詳細に説明) は, エンドツーエンドレイテンシの制限に関する従来の結果のいずれとも一致しない
        \item したがって, ROS 開発者は, タイミング特性を測定するために, 完全なプロトタイピングと設計の展開に頼らなければならず, 実際に設計空間を調査できる程度が大幅に制限される
    \end{itemize}
\end{frame}


\begin{frame}{}
    \begin{itemize}
        \item Linux のようなオペレーティングシステムのスケジューリングモデルは広範囲に研究されているが, これは ROS には当てはまらない
        \item これはミドルウェアレイヤであり, 適切なオペレーティングシステムではないが, アプリケーションの実行時の動作に対する影響は大きく, プラットフォームとなる OS に匹敵するか, それを超えることさえある
        \item 例えば, ROS は, カスタムスケジューリングポリシーを使用して, 独立したメッセージハンドラーを共有スレッドに多重化する
        \item その結果, ROS 上で実行されるアプリケーションは, プラットフォームとなるオペレーティングシステムとミドルウェアレイヤのスケジューリング決定の影響を受け, 複雑で相互に依存するタイミングの影響を受ける
    \end{itemize}
\end{frame}

\begin{frame}{}
    \begin{itemize}
        \item 追加の複雑さは, ROS の重要な強みの 1 つであるモジュール構造に起因する
        \item ROS は, 各ロボットに共通のサブシステムをゼロから再実装するのではなく, 既存の戦闘でテスト済みのコンポーネントを構成することを強調している
        \item これにより, ロボット開発が大幅に簡素化および高速化されるが, 全体的なタイミング動作がわかりにくくなる
    \end{itemize}
\end{frame}

\begin{frame}{}
    \begin{itemize}
        \item この問題は, ROS のイベント駆動型の設計スタイルによってさらに悪化する
        \item これにより, データの依存関係が生じ, 処理チェーンが長くなる可能性がある
        \item その結果, 開発者が複数の疎結合コンポーネントにまたがる処理チェーンのタイミングを予測すること, または単に理解することは非常に困難である
        \item これらのコンポーネントのほとんどは, 世界中の独立したチームによって開発されている
        \item 現実的には, 自動化されたエンドツーエンドの応答時間分析は, 時間が重要な状況で ROS を安全に使用するために必要である
    \end{itemize}
\end{frame}


\begin{frame}{}
    \begin{itemize}
        \item 本論文では, ROS 2 "Crystal Clemmys" (2018 年 12 月にリリース) [5] の一時的な動作を調査することにより, このような自動分析ツールの理論的基礎を築こうとする
        \item Linux の SCHED\_DEADLINE などのリソース予約スケジューラの上で実行される ROS アプリケーションのモデルを提示し, 検証する
        \item このモデルに基づいて, ROS 環境の特殊性と工学的制約を考慮に入れた ROS 処理チェーンのエンドツーエンドの応答時間分析を開発する
        \item 最後に, 実用的な ROS コンポーネントへの分析の適用可能性を実証するために, ROS ナビゲーションスタックのコアである一般的な move\_base パッケージ [3] に対するアプローチを評価する
    \end{itemize}
\end{frame}
