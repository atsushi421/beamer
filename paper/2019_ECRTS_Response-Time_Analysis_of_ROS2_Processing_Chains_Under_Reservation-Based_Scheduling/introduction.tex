% !TeX root = main.tex

\section{INTRODUCTION}
\label{sec: introduction}

% \begin{frame}{背景}
%     \begin{itemize}
%         \item Linux のようなオペレーティングシステムのスケジューリングモデルは広範囲に研究されているが, これは ROS には当てはまらない
%         \item ROS は, カスタムスケジューリングポリシーを使用して, 独立したメッセージハンドラーを共有スレッドに多重化する
%         \item その結果, ROS 上で実行されるアプリケーションは, プラットフォームとなるオペレーティングシステムとミドルウェアレイヤのスケジューリング決定の影響を受け, 複雑で相互に依存するタイミングの影響を受ける
%     \end{itemize}
% \end{frame}

\begin{frame}{本論文の貢献}
    \begin{itemize}
        \item ROS 2 ``Crystal Clemmys" の動作を調査することにより, 自動エンドツーエンド分析ツールの理論的基礎を築く
        \item Linux の SCHED\_DEADLINE などのリソース予約スケジューラの上で実行される ROS アプリケーションのモデルを提示し, 検証
        \item モデルに基づいて, ROS 環境の特殊性と工学的制約を考慮に入れた ROS 処理チェーンのエンドツーエンドの応答時間分析を開発
    \end{itemize}
\end{frame}
