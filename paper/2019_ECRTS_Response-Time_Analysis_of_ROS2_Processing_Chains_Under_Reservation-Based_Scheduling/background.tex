% !TeX root = main.tex

\section{BACKGROUND}
\label{sec: background}


\subsection{Resource Reservations}
\label{ssec: resource reservations}

\begin{frame}{リソースの予約}
    \begin{itemize}
        \item ROS スレッドの予測可能なサービスを確保するためにリソースの予約を使用する
        \item リソースの予約は通常, 予約サーバによって実装される
        \item Linux システムでは, リソースの予約は SCHED\_DEADLINE~\cite{lelli2016deadline} を通じて利用できる
    \end{itemize}
\end{frame}

\begin{frame}{リソースの予約に関する仮定}
    \begin{itemize}
        \item 本論文では, 予約サーバーが周期的リソースモデル~\cite{shin2003periodic}に準拠している, すなわち, 以下を仮定する
              \begin{enumerate}
                  \item 各予約には暗黙のデッドラインがある
                  \item 予約サーバ $r_i$がワークロードを持つときはいつでも, 予約アルゴリズムが$P_i$時間単位ごとに少なくとも$Q_i$単位のサービスを保証する
                  \item バジェット枯渇と他の予約のレイテンシによる, ある予約で実行中のプロセスが受ける有限の最大リリースレイテンシが存在する
              \end{enumerate}
        \item ある長さの区間において予約が提供するサービスの最小量は, 既知の供給境界関数 $sbf()$ \cite{shin2003periodic, lipari2003resource} で得られる
    \end{itemize}
\end{frame}


\forme{
    \subsection{Compositional Performance Analysis}
    \label{ssec: compositional performance analysis}

    \begin{frame}{}
        \begin{itemize}
            \item 構成パフォーマンス分析 (CPA) は, ヘテロジニアスシステムと分散システムのパフォーマンスを分析的に評価するアプローチである [27]
            \item CPA は, システムをリソースのネットワークとしてモデル化し, ワークロードを依存関係のあるタスクとしてモデル化する
            \item リソースは, タスクによって消費される処理時間を提供する
        \end{itemize}
    \end{frame}

    \begin{frame}{}
        \begin{itemize}
            \item 依存関係のあるタスクは直接非巡回グラフとして編成され, グラフ内のパスは処理チェーンとして示される
            \item 同じリソースを共有するタスクは, リソース固有のスケジューリングポリシーに従ってスケジュールされる
            \item チェーンのソースタスクは, 外部から提供されたイベント到着曲線 $[27,30,62] \eta^{e}(\Delta)$ に従ってトリガされる
            \item これは, 任意の時間枠 $[t, t+\Delta)$ に到着できるイベント数の上界を定義する
        \end{itemize}
    \end{frame}

    \begin{frame}{}
        \begin{itemize}
            \item イベント到着曲線は, 周期的およびイベント駆動型 (割り込み駆動型など) の両方のアクティベーションをモデル化するのに十分一般的である [47]
            \item 例えば, タスク $T_{x}$ が定期的にトリガされる場合, その到着曲線は $\eta_{x}^{e}(\Delta)=\left[\frac{\Delta}{\operatorname{period}(x)}\right]$ として表すことができる
            \item 非ソースタスクは, 導出されたアクティベーション曲線に従ってトリガされる
            \item アクティベーション曲線は, 先行タスクによるアクティベーションレイテンシを反映し, 応答時間に依存するリリースジッターを考慮して, 到着曲線から取得される
        \end{itemize}
    \end{frame}

    \begin{frame}{}
        \begin{itemize}
            \item しかし, 応答時間はリリースジッターにも依存するため, 周期的な依存関係が生じる
            \item この問題を解決するために, 分析はゼロの初期ジッターで開始し, 収束が達成されるまで繰り返し応答時間分析を適用し, 全てのジッター境界を更新する [42,63]
            \item 収束は, 応答時間とジッタの間の単調な依存性によって (過負荷のないシステムの場合) 保証される (ジッタが大きいほど, 応答時間は長くなり, その逆も同様である)
            \item 基本的な CPA アプローチでは, チェーンのエンドツーエンドレイテンシを, 各タスクの個々の最悪応答時間の合計で制限する
            \item その後, [54, 56] のように, 解析精度を向上させる拡張機能が設計された
        \end{itemize}
    \end{frame}

    \begin{frame}{}
        \begin{itemize}
            \item ROS は, カスタムスケジューリングポリシーを使用して独特な方法でコールバックをディスパッチするエグゼキュータを提供するため, 既存の CPA の文献とツールは ROS に完全に一致するものではない
            \item しかし, CPA から自由にインスピレーションを得て, 次に紹介する ROS の特異性を反映した, 同様に柔軟なタイミングモデルを取得する
        \end{itemize}
    \end{frame}
}
