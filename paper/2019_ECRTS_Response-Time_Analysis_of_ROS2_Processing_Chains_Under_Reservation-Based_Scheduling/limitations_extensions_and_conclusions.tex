% !TeX root = main.tex

\section{Limitations, Extensions, and Conclusions}
\label{sec: limitations, extensions, and conclusions}

\begin{frame}{}
    \begin{itemize}
        \item ROS 2 処理チェーンのタイミング解析に関するこの最初の作業は, 既に実用的なコンポーネント (move\_base など) を処理することができ, 将来の開発のための豊富なプラットフォームを提供する.それにもかかわらず, 成熟し, 柔軟で, 広く使用されているフレームワークに伴う避けられない複雑さを考えると, あまり使用されない ROS の側面を除外する必要がありました.以下では, これらの制限について説明し, 将来の拡張の有望な方向性を強調する.
    \end{itemize}
\end{frame}

\begin{frame}{}
    \begin{itemize}
        \item 本論文では, 組み込みのシングル スレッド ROS エグゼキュータについて説明する. ROS は, そのエグゼキュータのマルチスレッド バリアントも提供し, さらに, 任意の特殊目的のエグゼキュータを定義できる.特定のロボットのニーズに合わせて調整された専用のスケジューラを簡単に統合できるようになれば, 将来的に興味深いドメイン固有の研究が可能になる.共有プロセスで複数のエグゼキュータ スレッドを使用すると, 同時実行性の問題が発生する. ROS は, この問題に対処するために相互に排他的なコールバック グループを導入し, 同じグループ内のコールバックが同時に実行されないことを保証する.コールバック間のブロッキング関係を処理するために分析を拡張することは, 今後の課題である.
    \end{itemize}
\end{frame}

\begin{frame}{}
    \begin{itemize}
        \item 本論文では, コールバックのグラフが固定されていると仮定した.ただし, ROS では, ノードが実行時に動的に参加および離脱したり, 動的にトピックにサブスクライブおよびサブスクライブ解除したりできる.これは, さまざまな動作モードを実装する場合に特に役立つ.この問題は, 文献 [38, 45] ではモード変更と呼ばれている.私たちの分析は, 安定動作中の各モードに適用できるが, モード変更中の過渡的な影響は考慮されていない.モードの変更を考慮した新しい分析手法の設計 ($[10,11,13]$ を ROS システムに拡張するなど) は, 別の関連する将来の方向性を表している.
    \end{itemize}
\end{frame}

\begin{frame}{}
    \begin{itemize}
        \item ネットワーク遅延のオーバーヘッドと基礎となる DDS 実装を 1 つの変数 $\delta_{i, j}$ としてモデル化した.これにより, ネットワークを通過するたびに通信遅延を合計することで, 全体的な応答時間におけるネットワーク関連の遅延を安全かつ簡単に計算できる.将来の改善の機会は, ネットワーク分析を統合して, ネットワークが複数回交差するときに, 1 回限りの支払いバーストの問題によって引き起こされる悲観論を排除することである.さらに, 利用可能な DDS 実装の詳細な調査により, メッセージ処理のオーバーヘッドをより正確にモデル化できるようになる.
    \end{itemize}
\end{frame}

\begin{frame}{}
    \begin{itemize}
        \item トピックとサービスに加えて, ROS は待機可能なコールバック タイプも提供する.この型は, ROS 1 [1] で知られている長期実行アクションや高レベル アクションのような, より複雑な通信プリミティブを実装することを目的としている.このメカニズムは最新のリリースでのみ導入されたため, 現時点でこのメカニズムの既知のユーザはいない.これらの追加メソッドが ROS 2 で採用されたときに, これらの追加メソッドに分析を拡張する必要がある.
    \end{itemize}
\end{frame}

\begin{frame}{}
    \begin{itemize}
        \item 各コールバックは, 実行ごとに最大 1 回, すべての後続のアクティベーションをトリガすると仮定した.将来の改善として, 提案された分析を拡張して, 事前定義された数のインスタンスを実行した後にのみコールバックが後続をトリガできるようにしたり, 1 回の実行で各後続の複数のインスタンスをトリガしたりしたいと考えている [23].
    \end{itemize}
\end{frame}

\begin{frame}{}
    \begin{itemize}
        \item CPAアプローチに基づく当社の分析により, 実際のシステムを簡単かつ効率的に分析し, 予約を個別に検討することで複雑さを制限できる.分析の精度は, チェーン内のアクティベーション イベント間の相関関係を考慮することでさらに改善できる.これにより, 複数の予約にまたがるチェーンの「1 回限りの支払い」の問題も軽減される.将来の研究の可能な研究の方向性は, フォンセカらによって提示されたアプローチを拡張することにある. [21] と Casini ら. [14] (それぞれ, 並列タスクのプリエンプティブおよび非プリエンプティブ固定優先度スケジューリングのコンテキストで), 複数の予約にまたがるチェーンが自己中断タスクによってモデル化されることに基づいて [16].このように, 到着バーストは予約ごとに 1 回だけ考慮することができるため, 同じ予約を複数回通過するチェーンの分析精度が向上する.
    \end{itemize}
\end{frame}

\begin{frame}{}
    \begin{itemize}
        \item 結論として, ソース コードとドキュメントのレビューに基づいて, ROS 2 システムの最初の包括的なスケジューリング モデルを提示した. ROS フレームワークの特定のプロパティを考慮に入れて, 現実的なケーススタディに適用する処理チェーンの応答時間分析を導き出した.将来の拡張の機会は十分に残されているが, 私たちの貢献は, リアルタイム システムの専門知識を持たない ROS ユーザがアプリケーションの一時的な安全性と遅延特性を迅速かつ便利に判断できるようにする自動分析ツールへの第一歩を表している.
    \end{itemize}
\end{frame}
