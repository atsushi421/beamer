% !TeX root = main.tex

\section{LIMITATIONS, EXTENSIONS, AND CONCLUSIONS}
\label{sec: limitations, extensions, and conclusions}

\begin{frame}{セクションサマリ}
    \begin{itembox}[l]{\textbf{目的}}
        本論文の制限について説明し, 将来の拡張の有望な方向性を強調する
    \end{itembox}
\end{frame}

\begin{frame}{マルチスレッドエグゼキュータ}
    \begin{itemize}
        \item ROS は組込みでマルチスレッドエグゼキュータを提供する
        \item 共有プロセスで複数のエグゼキュータスレッドを使用すると, 同時実行性の問題が発生する
        \item ROS はこの問題に対処するためにmutually exclusiveコールバックグループを導入し, 同じグループ内のコールバックが同時に実行されないことを保証する
        \item コールバック間のブロッキング関係を処理するために分析を拡張することは, 今後の課題である
    \end{itemize}
\end{frame}

\begin{frame}{モード変更}
    \begin{itemize}
        \item 本論文では, コールバックのグラフが固定されていると仮定した
        \item しかし, ROS ではモード変更が可能
              \begin{block}{モード変更}
                  ノードが実行時に動的に参加および離脱したり, 動的にトピックにsubscribeおよびsubscribe解除する
              \end{block}
              \vspace{5mm}
        \item モード変更を考慮した新しい分析手法の設計は, 今後の拡張可能性
    \end{itemize}
\end{frame}

\begin{frame}{通信レイテンシ}
    \begin{itemize}
        \item 本論文では, ネットワークレイテンシのオーバヘッドと基礎となる DDS 実装を 1 つの変数 $\delta_{i, j}$ としてモデル化した
        \item 通信レイテンシに関する将来の改善として以下が考えられる
              \begin{itemize}
                  \item ネットワーク分析を統合して, ネットワークが複数回交差するときに, 1 回限りの支払いバーストの問題によって引き起こされる悲観論を排除する
                  \item 利用可能な DDS 実装の詳細な調査により, メッセージ処理のオーバヘッドをより正確にモデル化する
              \end{itemize}
    \end{itemize}
\end{frame}

\begin{frame}{待機可能なコールバック}
    \begin{itemize}
        \item トピックとサービスに加えて, ROS は待機可能なコールバックタイプも提供する
        \item 待機可能なコールバックタイプは, 長期実行アクションや高レベルアクションのような, より複雑な通信を実装することを目的としている
        \item これらの追加メソッドが ROS 2 で採用されたときに, これらの追加メソッドに分析を拡張する必要がある
    \end{itemize}
\end{frame}

\begin{frame}{各コールバックのトリガ回数}
    \begin{itemize}
        \item 各コールバックは, 実行ごとに最大 1 回, 全ての後続のアクティベーションをトリガすると仮定した
        \item 将来の改善として, 以下が考えられる
              \begin{itemize}
                  \item  事前定義された数のインスタンスを実行した後にのみ, コールバックが後続をトリガできるようにする
                  \item 1 回の実行で各後続の複数のインスタンスをトリガできるようにする
              \end{itemize}
    \end{itemize}
\end{frame}

\begin{frame}{精度の改善}
    \begin{itemize}
        \item 分析の精度は, チェーン内のアクティベーションイベント間の相関関係を考慮することでさらに改善できる
        \item 今後の研究の方向性としては, Fonsecaら~\cite{fonseca2016response}とCasiniら~\cite{casini2018partitioned}のアプローチを拡張し, 複数の予約にまたがるチェーンを自己中断タスクによってモデル化することが考えられる
    \end{itemize}
\end{frame}

\begin{frame}{まとめ}
    \begin{itemize}
        \item ソースコードとドキュメントのレビューに基づいて, ROS 2 システムの最初の包括的なスケジューリングモデルを提示した
        \item ROS フレームワークのプロパティを考慮に入れて, 現実的なケーススタディに適用する処理チェーンの応答時間分析を導出した
        \item 将来の拡張の機会は十分に残されているが, 本論文の貢献は, リアルタイムシステムの専門知識を持たない ROS ユーザがアプリケーションの一時的な安全性とレイテンシ特性を迅速かつ便利に判断できるようにする自動分析ツールへの第一歩となる
    \end{itemize}
\end{frame}
