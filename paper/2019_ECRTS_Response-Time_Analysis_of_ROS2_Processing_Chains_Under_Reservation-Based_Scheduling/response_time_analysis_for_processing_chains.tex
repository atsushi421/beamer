% !TeX root = main.tex

\section{Response-Time Analysis for Processing Chains}
\label{sec: response-time analysis for processing chains}

\begin{frame}{セクションサマリ}
    \begin{itembox}[l]{\textbf{目的}}
        一般的な ROS 処理チェーンのエンドツーエンドレイテンシ (すなわち, 最大応答時間) の分析を示す
    \end{itembox}
    \begin{itembox}[l]{\textbf{流れ}}
        \begin{itemize}
            \item コールバックの ROS 固有の応答時間分析を提示
            \item サブチェーンと呼ばれるチェーンの部分の応答時間分析を提示
        \end{itemize}
    \end{itembox}
\end{frame}

\begin{frame}{}
    \begin{itemize}
        \item 本資料では, 分析方法のみを示す
        \item 証明は論文を参照
    \end{itemize}
\end{frame}

\forme{
    \begin{frame}{}
        \begin{itemize}
            \item エンドツーエンドレイテンシは, 各チェーンのコールバックの個々の応答時間を合計することで取得できる
        \end{itemize}
    \end{frame}

    \begin{frame}{}
        \begin{itemize}
            \item このアプローチはエンドツーエンドレイテンシの安全で単純な上界を提供するが, 干渉するコールバックの到着バーストが複数回 (分析中のチェーン内のコールバックごとに 1 回) 考慮される場合, 結果の境界は過度に悲観的になる可能性がある
            \item この効果は, 文献では「pay-burst-only-one」問題 [30, 56] として知られている
            \item 分析の精度を向上させるために, セクション $5.4$ では, サブチェーンと呼ばれるチェーンの部分が全体論的な方法で分析される境界を提示する
            \item $\gamma^{x}$ の $y$ 番目のサブチェーン $\gamma^{x, y}$ を, 単一の予約 $r_{k}$ に割り当てられた基のチェーンの連続したコールバック $c_{i} \in \gamma^{x}$ のシーケンス, すなわち $c_{i} \in \gamma^{x, y} \Rightarrow c_{i} \in \mathcal{C}_{k}$ として定義する
            \item このアプローチでは, 到着バーストはサブチェーンごとに 1 回だけ考慮される
            \item CPA アプローチは, サブチェーンが予約境界を越えるか, 複数の先行者を持つコールバックで別のチェーンと結合するたびに到着曲線を伝播し, 最悪応答時間を合計することにより, サブチェーンごとに適用できる
        \end{itemize}
    \end{frame}
}

\begin{frame}{サブチェーン}
    \begin{definition}[$\gamma^{x}$ の $y$ 番目のサブチェーン $\gamma^{x, y}$]
        予約 $r_{k}$ に割り当てられた, 元のチェーンの連続したコールバック $c_{i} \in \gamma^{x}$ のシーケンス, すなわち $c_{i} \in \gamma^{x, y} \Rightarrow c_{i} \in \mathcal{C}_{k}$
    \end{definition}
\end{frame}


\subsection{High-Level Overview}
\label{ssec: high-level overview}

\begin{frame}{複数の予約にまたがるコールバックチェーンの例}
    \fitimage{
        最初のサブチェーンのアクティベーションは外部イベント到着曲線によって境界が定まり, それ以降のサブチェーンは前のサブチェーンにおける応答時間と通信レイテンシに依存するアクティベーション曲線によって特徴づけられる
    }{sub_chain_example}
\end{frame}

\begin{frame}{応答時間の求め方の概要}
    \fitimage{
        \begin{itemize}
            \item $\eta_{x, 2}^{a}(\Delta)=\eta_{x}^{e}\left(\Delta+R_{x, 1}+\delta_{2,3}\right), \eta_{x, 3}^{a}(\Delta)=\eta_{x, 2}^{a}(\Delta+$  $\left.R_{x, 2}+\delta_{4,5}\right), \eta_{x, 4}^{a}(\Delta)=\eta_{x, 3}^{e}\left(\Delta+R_{x, 3}+\delta_{5,6}\right)$
            \item チェーンの応答時間は, サブチェーンの応答時間と通信レイテンシの合計, すなわち $R_{x}=R_{x, 1}+\delta_{2,3}+R_{x, 2}+\delta_{4,5}+R_{x, 3}+\delta_{5,6}+R_{x, 4}$ として計算できる
        \end{itemize}
    }{sub_chain_example}
\end{frame}

\begin{frame}{アクティベーション曲線の拡張}
    1 つ以上のコールバックを共有する処理チェーンの場合, コールバックのアクティベーション曲線は, 先行コールバックのアクティベーション曲線から導き出される
    \begin{equation*}
        \eta_{i}^{a}(\Delta)=\sum_{c_{j} \in \operatorname{pred}\left(c_{i}\right)} \eta_{j}^{a}\left(\Delta+R_{j}+\delta_{j, i}\right)
    \end{equation*}
\end{frame}


\subsection{Analysis for Individual Callbacks}
\label{ssec: analysis for individual callbacks}

\begin{frame}{セクションサマリ}
    \begin{itembox}[l]{\textbf{目的}}
        コールバックの応答時間分析を提示
    \end{itembox}
\end{frame}

\begin{frame}{クワイエットタイム, ビジーウィンドウ}
    \begin{definition}[クワイエットタイム]
        $t$ の前に到着した $c_{i}$ のpendingインスタンスがない場合, 時間 $t$ は, コールバック $c_{i}$ に関してクワイエットタイムである
    \end{definition}
    \begin{definition}[ビジーウィンドウ]
        $t_{1}$ と $t_{2}$ が $c_{i}$ のクワイエットタイムであり, $t_{1}$ と $t_{2}$ の間に$c_{i}$ に関するクワイエットタイムがない場合, 間隔 $\left[t_{1}, t_{2}\right)$ は $c_{i}$ のビジーウィンドウである
    \end{definition}
\end{frame}

\begin{frame}{コールバックの最悪応答時間}
    \begin{definition}[コールバック $c_{i}$ の最悪応答時間 $R_{i}$]
        特定のインスタンスの終了時間とリリース時刻の間の任意のインスタンスにわたる最大差
    \end{definition}
\end{frame}

\begin{frame}{要求境界関数}
    \begin{definition}[コールバック $c_{i}$ の要求境界関数 $rbf_{i}(\Delta)$]
        \setlength{\linewidth}{0.98\columnwidth}
        \begin{itemize}
            \item 各コールバック $c_{i}$ に対して, 要求境界関数 $rbf_{i}(\Delta)$ は, 長さ $\Delta$ の間隔でリリースされたコールバックインスタンスが要求する累積プロセッササービスの最大量
            \item $r b f_{i}(\Delta)=\eta_{i}^{a}(\Delta) \cdot e_{i}$
        \end{itemize}
    \end{definition}
    \begin{definition}[コールバックセット $\mathcal{C}^{*}$ の要求境界関数の合計]
        $R B F\left(\mathcal{C}^{*}, \Delta\right)=\sum_{c_{i} \in \mathcal{C}^{*}} r b f_{i}(\Delta)$
    \end{definition}
\end{frame}

\begin{frame}{pp ベースのコールバック}
    スケジューリングの観点から, コールバックは, イベントソース, タイマ, およびポーリングポイントベース (pp ベース) のコールバックの 3 つのカテゴリに分けられる
    \begin{definition}[$r_{k}$ に割り当てられた pp ベースのコールバックのセット $\mathcal{C}_{k}^{\mathrm{pp}}$]
        $\mathcal{C}_{k}^{\mathrm{pp}}=\mathcal{C}_{k} \backslash\left(\mathcal{C}_{k}^{\text {evt}} \cup \mathcal{C}_{k}^{\mathrm{tmr}}\right)$
    \end{definition}
\end{frame}

\begin{frame}[label=lemma1]{イベントソースの最悪応答時間}
    \begin{lemma}[イベントソースの最悪応答時間]
        \begin{itemize}
            \item $A \geq 0$ を分析対象のイベントソースコールバック $c_{i} \in \mathcal{C}_{k}^{\text {evt }}$ のインスタンスがリリースされた時刻とする
            \item $R_{i}^{*}(A)$ が以下の式を満たす最小の正の値である場合, $R_{i}=\max \left\{R_{i}^{*}(A) \mid A \geq 0\right\}$ は, $c_{i}$ の最悪応答時間である
                  \begin{equation*}
                      s b f_{k}\left(A+R_{i}^{*}(A)\right)=r b f_{i}(A+1)
                  \end{equation*}
        \end{itemize}
    \end{lemma}
\end{frame}

\forme{
    \begin{frame}{}
        \begin{itemize}
            \item Lemma 1 は, 可能性のあるリリースオフセット $A$ の無制限の数をチェックする必要があるため, 直接適用できない
            \item 応答時間分析を実際に実装するには, 分析間隔の長さの上界と, チェックする必要があるリリースオフセット数の削減の両方が必要である
            \item この問題については, セクション 5.3 で再検討する
        \end{itemize}
    \end{frame}
}

\begin{frame}[label=lemma2]{タイマコールバックが受ける最大ブロック時間}
    \begin{lemma}[タイマコールバックが受ける最大ブロック時間]
        タイマコールバック $c_{i} \in \mathcal{C}_{k}$ は, 優先度の低いコールバックによって最大 $B_{i}=\max \left\{e_{j} \mid c_{j} \in l p_{k}\left(c_{i}\right)\right\}$ 時間単位ブロックされる
    \end{lemma}
\end{frame}

\begin{frame}[label=lemma3]{タイマコールバックの最悪応答時間}
    \begin{lemma}[タイマコールバックの最悪応答時間]
        \begin{itemize}
            \item $A \geq 0$ を分析対象のタイマコールバック $c_{i} \in \mathcal{C}_{k}^{\mathrm{tmr}}$ のインスタンスがリリースされた時刻とする
            \item $R_{i}^{*}(A)$ が以下の式を満たす最小の正の値である場合, $R_{i}=\max \left\{R_{i}^{*}(A) \mid A \geq 0\right\}$ は, $c_{i}$ の最悪応答時間
                  \begin{equation*}
                      \begin{split}
                          s b f_{k}\left(A+R_{i}^{*}(A)\right)=&r b f_{i}(A+1)+\\
                          &R B F\left(h p_{k}\left(c_{i}\right), A+R_{i}^{*}(A)-e_{i}+1\right)+B_{i}
                      \end{split}
                  \end{equation*}
        \end{itemize}
    \end{lemma}
\end{frame}

\begin{frame}[label=lemma4]{pp ベースのコールバックの最悪応答時間}
    \begin{lemma}[pp ベースのコールバックの最悪応答時間]
        \begin{itemize}
            \item $A \geq 0$ を分析対象の pp ベースのコールバック $c_{i} \in \mathcal{C}_{k}^{\mathrm{pp}}$ のインスタンスがリリースされた時刻とする
            \item $X \geq 0$ を時刻 $A+R_{i}^{*}(A)-e_{i}$ と時刻 $A+R_{i}^{*}(A)-e_{i}$ の前の最後のポーリングポイントとの差とする
            \item $R_{i}^{*}(A)$ が以下の式を満たす最小の正の値である場合,  $R_{i}=\max \left\{R_{i}^{*}(A) \mid A \geq 0\right\}$ は $c_{i}$ の最悪応答時間である
                  \begin{equation*}
                      s b f_{k}\left(A+R_{i}^{*}(A)\right)=\operatorname{rbf}_{i}(A+1)+R B F\left(\left\{\mathcal{C}_{k} \backslash c_{i}\right\}, A+R_{i}^{*}(A)-e_{i}+1\right)
                  \end{equation*}
        \end{itemize}
    \end{lemma}
\end{frame}


\subsection{Bounding the Search Space}
\label{ssec: bounding the search space}

\begin{frame}{セクションサマリ}
    \begin{itembox}[l]{\textbf{目的}}
        実際の応答時間分析で前述のLemmaを使用するために,  分析間隔の制限と探索空間サイズを削減する
    \end{itembox}
\end{frame}

\begin{frame}[label=lemma6]{最長のビジーウィンドウ}
    \begin{lemma}[]
        $c_{i} \in \mathcal{C}_{k}$ が分析中のコールバックであり, $L^{*}$ が以下の条件を満たす最小の正の値である場合, $L^{*}$ は最長のビジーウィンドウである
        \begin{equation*}
            s b f_{k}\left(L^{*}\right)= \begin{cases}r b f_{i}\left(L^{*}\right) & \text { if } c_{i} \in \mathcal{C}_{k}^{\mathrm{evt}} \\ R B F\left(h p_{k}\left(c_{i}\right), L^{*}\right)+B_{i}+r b f_{i}\left(L^{*}\right) & \text { if } c_{i} \in \mathcal{C}_{k}^{\mathrm{tmr}} \\ R B F\left(\mathcal{C}_{k}, L^{*}\right) & \text { if } c_{i} \in \mathcal{C}_{k}^{\mathrm{pp}}\end{cases}
        \end{equation*}
    \end{lemma}
\end{frame}

\begin{frame}{探索空間の削減}
    コールバック $c_i$ に関する探索空間 $A_i$ は以下に制限される
    \begin{equation*}
        \mathcal{A}_{i}=\left\{0 \leq A \leq L^{*} \mid r b f_{i}(A+1) \neq r b f_{i}(A)\right\} \cup\{0\} .
    \end{equation*}
\end{frame}


\subsection{Analysis for Processing Chains}
\label{ssec: analysis for processing chains}

\begin{frame}{セクションサマリ}
    \begin{itembox}[l]{\textbf{目的}}
        複数のコールバックで構成されるサブチェーンのエンドツーエンドの分析を提供する
    \end{itembox}
\end{frame}

\begin{frame}[label=lemma8]{サブチェーンの最悪応答時間}
    \begin{lemma}[サブチェーンの最悪応答時間]
        \begin{itemize}
            \item $\gamma^{x, y}=\left(c_{s}, \ldots, c_{e}\right)$ を2つ以上のコールバックで構成されるサブチェーンとする
            \item $\Gamma_{k}$ を $r_{k}$ に割り当てられたサブチェーンのセットとする
            \item $R_{x, y}$ が以下の式を満たす最小の正の値である場合, $R_{x, y}$ は, $\gamma^{x, y}$ の最悪応答時間である
                  \begin{equation*}
                      s b f_{k}\left(R_{x, y}\right)=R B F^{\gamma}\left(\Gamma_{k}, R_{x, y}-e_{e}+1\right),
                  \end{equation*}
        \end{itemize}
    \end{lemma}
\end{frame}


\subsection{Analysis Summary}
\label{ssec: analysis summary}

\begin{frame}{チェーンのエンドツーエンドレイテンシを求める方法}
    \begin{itemize}
        \item セクション 5.2 では単一のコールバックの応答時間分析を提案し, セクション 5.4 はそれを単一の予約に割り当てられたサブチェーンに拡張した
        \item セクション 5.1 で説明したように, どちらのアプローチでも, 到着曲線を伝搬し, 個々の最悪応答時間を合計することにより, 処理チェーンの安全なエンドツーエンドレイテンシを計算できる
    \end{itemize}
\end{frame}
