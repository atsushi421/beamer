% !TeX root = main.tex

\section{EVALUATION}
\label{sec: evaluation}

\begin{frame}{セクションサマリ}
    \begin{itembox}[l]{\textbf{目的}}
        合成DAGセットを生成し, システムモデルのさまざまなパラメーターを変化させて, 提案アルゴリズムの性能を評価する
    \end{itembox}
\end{frame}

\begin{frame}{評価指標}
    各パラメータで200個のDAGセットのスケジューラビリティを判定し, 受入率を計算する
    \begin{block}{受入率}
        DAGセットのうち, 全てのDAGがデッドラインを満たしている割合
    \end{block}
\end{frame}

\begin{frame}{グローバルスケジューリングとの比較}
    \fitimage{
        フェデレートスケジューリングがグローバルスケジューリングよりも高い受入率を達成
    }{eva1}
\end{frame}

\begin{frame}{結果の考察}
    \begin{itemize}
        \item フェデレートスケジューリングの場合, 実行する専用クラスタを持つDAGは, 他のDAGに干渉しない
        \item 一方グローバルスケジューリングの場合, DAGは優先度の低い全てのDAGに干渉するため, 受入率が低くなる
    \end{itemize}
\end{frame}

\begin{frame}{既存フェデレートスケジューリングとの比較}
    \fitimage{
        提案アルゴリズムは, プロセッサバリューを考慮するため, 既存手法と比較してより高い受入率を達成
    }{eva2}
\end{frame}
