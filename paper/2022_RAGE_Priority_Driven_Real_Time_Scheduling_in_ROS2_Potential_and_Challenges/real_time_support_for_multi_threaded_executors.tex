% !TeX root = main.tex

\section{REAL-TIME SUPPORT FOR MULTI-THREADED EXECUTORS}
\label{sec: real-time support for multi-threaded executors}

\begin{frame}{セクションサマリ}
    \begin{itembox}[l]{\textbf{目的}}
        ROS 2 マルチスレッドエグゼキュータで発生する課題と, 進行中の作業を説明する
    \end{itembox}
\end{frame}

% \begin{frame}{}
%     \begin{itemize}
%         \item ROS 2 はマルチスレッドのエグゼキュータを提供するが, 以前の研究 $[5,6,8]$ ではシングルスレッドのエグゼキュータのみが考慮されていた
%         \item 一般に, マルチスレッド化により, 複数のプロセッサを効果的に利用できるようになり, システムの同時実行性とスループットが向上する
%         \item リアルタイムマルチスレッドの利点は, 自動運転車のコンテキストで実証されている [10]
%         \item また, 図 3 に示すように, ROS 2 のデフォルトのマルチスレッドエグゼキュータは, シングルスレッドのエグゼキュータよりもレイテンシパフォーマンスが優れていることも確認した
%         \item ROS 2 マルチスレッドエグゼキュータのこのような利点にも関わらず,  $\operatorname{ROS} 2$ のマルチスレッドエグゼキュータのタイミング動作を分析および改善するための先行研究はない
%         \item したがって, このセクションでは,
%     \end{itemize}
% \end{frame}

\begin{frame}{マルチスレッドエグゼキュータの分析の困難性}
    マルチスレッドのエグゼキュータでの処理チェーンの分析は以下の理由で困難
    \begin{itemize}
        \item 複数のスレッドにまたがるランタイムコールバックの分散
        \item スレッドの非同期ポーリングポイント
    \end{itemize}
\end{frame}

\begin{frame}{進行中の作業}
    \begin{itemize}
        \item マルチスレッドエグゼキュータを, グローバルスケジューラとしてモデル化し, 従来のノンプリエンプティブグローバルタスクスケジューリング手法を ROS~2 環境に拡張する
        \item PiCAS をマルチスレッドエグゼキュータに拡張して, 優先度に基づくスケジューリングを可能にし, エンドツーエンドレイテンシと予測可能性を向上させる
    \end{itemize}
\end{frame}

\begin{frame}{コールバックグループに関する可能性}
    コールバックグループに関する以下のような問題に関する研究は, ROS 2 でのより効率的なスケジューリングアプローチに繋がる
    \begin{itemize}
        \item コールバックグループが ROS 2 エグゼキュータのタイミング動作にどのように影響するか
        \item 各タイプのチェーンのエンドツーエンドレイテンシを分析的にモデル化する方法
        \item リアルタイムパフォーマンスを向上させるようなコールバックグループ構成
    \end{itemize}
\end{frame}
