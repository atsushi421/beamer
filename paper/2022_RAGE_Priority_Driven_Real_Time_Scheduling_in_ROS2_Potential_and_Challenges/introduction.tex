% !TeX root = main.tex

\section{INTRODUCTION}
\label{sec: introduction}

% \begin{frame}{}
%     \begin{itemize}
%         \item ROS (ロボットオペレーティングシステム) は, ロボットアプリケーションの開発におけるソフトウェアのモジュール性と構成可能性を促進することで, ロボットコミュニティの開発者の間で脚光を浴びている
%         \item しかし, 過去 10 年間, ROS は, 安全性が重要なアプリケーションに必要なリアルタイムサポートに大きな欠点があることを示してきた
%         \item ROS の新しいバージョンである ROS 2 は, 新しいソフトウェアアーキテクチャと Data Distribution Service (DDS) を採用することでリアルタイム機能のサポートを改善することを目指しているが, ROS ベースのシステムで厳しいタイミング制約を保証することは依然として困難である
%     \end{itemize}
% \end{frame}

% \begin{frame}{}
%     \begin{itemize}
%         \item 予測可能なエンドツーエンドレイテンシを確保することは, 安全性が重要なドメインのアプリケーションにとって重要である
%         \item しかし, ROS 2 のような実用的なフレームワークでこの要件を満たすことは, 次の理由により簡単な問題ではない
%               \begin{enumerate}
%                   \item ロボットアプリケーションは一般に一連の処理チェーンを形成し, そのデータと依存関係を分析するのは困難
%                   \item ROS 2 には, コールバック, ノード, エグゼキュータなどの複数のスケジュール可能なエンティティがさまざまな抽象化レイヤにまたがって発生する, 複雑で独自のスケジューリング動作があり, 既存のリアルタイム技術を適用することが難しい
%                   \item ROS のオープンソースの性質により, 多くのプログラマーが相互にやり取りするソフトウェアコンポーネントを個別に開発している
%               \end{enumerate}
%     \end{itemize}
% \end{frame}

% \begin{frame}{}
%     \begin{itemize}
%         \item 以前の研究 \cite{casini2019response, blass2021ros} では, デフォルトの ROS 2 スケジューリングスキームに焦点が当てられていたが, エンドツーエンドの長いレイテンシや分析における悲観性は, 既存のスケジューリングスキームでの優先度付けのサポートが不十分であることが原因である
%         \item そこで, ROS 2 の優先度駆動型チェーン考慮スケジューリングフレームワークである PiCAS \cite{choi2021picas} を開発し, チェーンのエンドツーエンドレイテンシを改善した
%         \item PiCAS は, クリティカルチェーンの応答性をさらに向上させるためにリソースを割り当てる方法についても回答する
%     \end{itemize}
% \end{frame}

\begin{frame}{未解決の問題1}
    \fitimage{
        \begin{itemize}
            \item ROS 2 処理チェーンに関するこれまでの全ての作業は, PiCASを含め, シングルスレッドのエグゼキュータのみを想定する
            \item マルチスレッドのエグゼキュータは, 複数の CPU コアを搭載したシステムでより良いレイテンシとより高いスループットを提供する
        \end{itemize}
    }{average_latency}
\end{frame}

\begin{frame}{未解決の問題2}
    ROS 2 は機械学習アルゴリズムを備えたインテリジェントな自律システムの開発に広く使用されているため, GPU や FPGA などの共有ハードウェアアクセラレータから予測できないタイミング動作が発生する可能性があり, これはリソースに制約のある組み込みロボットプラットフォームでは深刻な問題となる
\end{frame}

\begin{frame}{貢献}
    \begin{itemize}
        \item ROS 2 のリアルタイム性能と予測性を向上させるために, 優先度駆動型スケジューリングアプローチの可能性を探る
        \item 先行研究であるPiCASをレビューし, 実世界の自動運転シナリオの下でROS 2エグゼキュータの性能をベンチマークするために Apex.AI が開発したリファレンスシステム でRaspberry Pi 4を評価する
        \item ROS 2が実用的かつ信頼性の高いリアルタイムソフトウェア基盤として機能するために不可欠な, マルチスレッドエグゼキュータ設計とリアルタイムGPUアクセラレーションサポートについて説明する
    \end{itemize}
\end{frame}
