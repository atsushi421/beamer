% !TeX root = main.tex

\section{PRIOR KNOWLEDGE}
\label{sec: prior knowledge}

\begin{frame}{}
    \begin{itemize}
        \item リリースジッタを誘発することがある
        \item これは, アクティベーションイベントの実際の到着時刻, または一般的なアクティベーション条件が真となる時刻と, これがスケジューリングに認識される時刻との間の時間
        \item ジッターはタスクの平均周期を変えませんが, アクティベーションするイベントはその元の周期に対してずれることが許される
    \end{itemize}
\end{frame}

\begin{frame}{散発的タスク}
    \begin{itemize}
        \item 利用率を制限できる非周期的タスクの特別なクラスとして, 散発的タスクを区別した
        \item 彼らは, タスク $\mathcal{T}_{i}$ の最小到着間時間 $d_{i}^{-}$ を最小周期として定義し, 最大許容頻度 $[108,105]$ に対応させました
        \item 最悪応答時間の分析では, 散発的なタスクは周期的なタスクとして扱われるのが普通
    \end{itemize}
\end{frame}
