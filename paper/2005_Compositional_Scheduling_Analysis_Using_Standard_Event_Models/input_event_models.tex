% !TeX root = main.tex

\section{INPUT EVENT MODELS}
\label{sec: input event models}

\subsection{Task Activation and Event Streams}
\label{ssec: task activation and event streams}

\begin{frame}{イベントモデル}
    \begin{itemize}
        \item スケジューラビリティ分析技術では, 一般的にイベントモデルを用いて流入する負荷を把握する
        \item タスクの起動は, タイムトリガかイベントトリガか, 周期的か散発的かにかかわらず, 常にイベントによってモデル化できる
        \item タスクはイベントストリームによってアクティベーションされ, 各タスクはアクティベーションごとに1つのイベントを消費する
    \end{itemize}
\end{frame}

\begin{frame}{リアルタイムスケジューリング理論で重要な入力イベントモデル}
    \begin{itemize}
        \item 周期的 (strictly periodic)
        \item ジッターを含む周期的 (periodic with jitter)
        \item 散発的 (sporadic)
        \item 散発的バースト (sporadic bursts)
    \end{itemize}
\end{frame}


\subsection{Common Properties of Event Streams}
\label{ssec: common properties of event streams}

% \begin{frame}{}
%     全てのスケジューラビリティ分析手法は, 入力されるイベントストリームの2つの重要な特性を利用する
%     \begin{block}{与えられた時間内に到着するイベントの数}
%         この情報により, 分析アルゴリズムは, 例えば, タスクの起動回数や, より優先度の高いタスクによるプリエンプトの回数を予測する
%     \end{block}
%     \begin{block}{ストリーム中の連続するイベント間の最小時間間隔 (または距離) }
%         これは, 将来のイベント到着時刻を予測し, デッドラインを導き出したり, ビジーウィンドウを決定するために必要
%     \end{block}
% \end{frame}

\begin{frame}{イベントストリーム $\mathcal{S}$ の4つの特性関数}
    \begin{block}{$\eta^{+}(\Delta t)$}
        サイズ $\Delta t$ の時間間隔内で発生し得るイベントの最大数
    \end{block}
    \begin{block}{$\eta^{-}(\Delta t)$}
        間隔 $\Delta t$ で発生するイベントの最小数
    \end{block}
    \begin{block}{$\delta^{-}(n)$}
        ストリーム中の連続する $n$ 個のイベント間の最小距離
    \end{block}
    \begin{block}{$\delta^{+}(n)$}
        ストリーム中の連続する $n$ 個のイベント間の最大距離
    \end{block}
\end{frame}


\subsection{Strictly Periodic Events}
\label{ssec: strictly periodic events}

\begin{frame}{周期的イベントストリームのパラメータ}
    \begin{itemize}
        \item 周期的イベントストリームは, 周期 $T$ によって特徴付けられる
        \item すなわち, 完全に $T$ 周期でイベントが到着する
    \end{itemize}
\end{frame}

\begin{frame}{}
    時間間隔 $\Delta t$ における周期的イベントの最大数は以下
    \begin{equation*}
        \forall \Delta t>0: \eta_{\mathrm{P}}^{+}(\Delta t)=\left\lceil\frac{\Delta t}{T}\right\rceil
    \end{equation*}
\end{frame}

\begin{frame}{}
    \fitimage{
        連続関数を考えた場合の, 周期的イベントストリームの時間間隔とイベント数
    }{Time_Intervals_and_Number_of_Events_of_Periodic_Event_Streams}
\end{frame}

% \begin{frame}{}
%     \fitimage{
%         \begin{itemize}
%             \item 連続関数が既知の場合, 離散的な最大数は適切な丸め方によって得られる
%             \item 左閉じ右開きの組み合わせにより, $\eta$ の最大値が確保される p.38
%         \end{itemize}
%     }{Maximum_Number_of_Events_of_Periodic_Event_Streams}
% \end{frame}

\begin{frame}{}
    \fitimage{
        最大関数 $\eta^{+}(\Delta t)$ は, 最大イベント到着曲線を導き出す
    }{upper_bound_arrival_curve_of_strictly_periodic_events}
\end{frame}


\subsection{Periodic Events with Jitter}
\label{ssec: periodic events with jitter}

\begin{frame}{}
    \begin{itemize}
        \item 多くの周期的なシステムでは, 周期的なストリームが経験する可能性のある歪みをとらえた, いわゆるジッターを示す
    \end{itemize}
\end{frame}

\begin{frame}{}
    \fitimage{
        観測可能な最大数と最小数が最大3つ異なっている
    }{time_intervals_and_number_of_events_of_periodic_evants_with_jiter}
\end{frame}

\begin{frame}{}
    \begin{itemize}
        \item どのイベントに間隔を合わせればよいのだろうか
        \item ここでも, スケジューリング理論に関する既存の研究がその答えを提供してくれます
        \item ジッターモデルはLocke [75]によって最初に検討された
        \item 彼は, 最悪イベント到着シナリオを次のように構築した
        \item 彼は, イベントの流れの中で, あるイベントが最大許容時間偏差の範囲内で可能な限り遅く到着し, その後に到着するイベントは全てその最初のイベントの後で可能な限り早く到着すると仮定した
    \end{itemize}
\end{frame}

\begin{frame}{}
    \fitimage{
        このシナリオと, その結果得られる上界到着曲線
    }{upper_bound_arrival_curve_of_periodic_events_with_jitter}
\end{frame}

\begin{frame}{}
    \begin{itemize}
        \item 最悪の場合, ジッター付きストリームの全てのイベント (最初のイベントを除く) は, 厳密に周期的なストリームよりも $J$ 早く再到達する
        \item より正確には, 最初のイベントだけが遅れて到着するが, $\Delta t$ の時間オフセットも同じ量, すなわちジッター $J$ だけシフトする
    \end{itemize}
\end{frame}

\begin{frame}{}
    $\eta^{+}$ 関数は, 式3.18で与えられる
    \begin{equation*}
        \forall \Delta t>0: \eta_{\mathrm{P+J}}^{+}(\Delta t)=\left\lceil\tilde{\eta}^{+}(\Delta t)\right\rceil=\left\lceil\frac{\Delta t+J}{T}\right\rceil
    \end{equation*}
\end{frame}

\begin{frame}{}
    \begin{itemize}
        \item ジッターを含む周期的なイベントのモデルは非常に強力である
        \item 静的で厳密に周期的なイベントに加え, ジッターの概念により, 一般的に周期的なストリームの中で不確実で部分的にしか動作が分かっていない動的なイベントストリームをモデル化できる
    \end{itemize}
\end{frame}


\subsection{Sporadic Events}
\label{ssec: sporadic events}

\begin{frame}{}
    \begin{itemize}
        \item 周期的なイベントの他に, 不規則に到着する非周期的イベント [108] というクラスがある
        \item 一般に, イベント到着の最大数 $\eta(\Delta t)$ のような重要な特性は安全に境界を定めることができないので, このような非周期的イベントはいかなるリアルタイム分析の対象にもなりえません
        \item しかし, 散発的イベントと呼ばれる非周期的イベントのサブクラスがあり, 最小限の予測可能性と分析可能性を可能にする特別な特性を備えている
    \end{itemize}
\end{frame}

\begin{frame}{}
    \begin{itemize}
        \item 周期的なイベントとは対照的に, 散発的なイベントは不規則にやってくる
        \item 平均周期, 繰り返し時間, 頻度などは存在しない
        \item しかし, 散発的事象は, 連続する事象間の時間的な最小間隔 (到着間時間 $d^{-}[108,3]$ と呼ばれる) が既知であるため, 最大事象の到着を制限できる
    \end{itemize}
\end{frame}

\begin{frame}{}
    \begin{itemize}
        \item イベントの最大数は到着間時間から簡単に計算できる
        \item 直感的には, この数は到着間時間に等しい周期を持つ周期的ストリームのイベント数の最大値に等しい
        \item すなわち, 散発的なストリームの最悪動作 (イベントの最大数) は, 厳密には周期的なストリームである
    \end{itemize}
\end{frame}

\begin{frame}{}
    \fitimage{
        最大イベント到着曲線
    }{upper_bound_arrival_curve_of_sporadic_events}
\end{frame}

\begin{frame}{}
    $\eta^{+}$ 関数は式 $3.29$ で与えられる
    \begin{equation*}
        \forall \Delta t>0: \eta_{\mathrm{S}}^{+}(\Delta t)=\left\lceil\tilde{\eta}_{\mathrm{S}}^{+}(\Delta t)\right\rceil=\left\lceil\frac{\Delta t}{d^{-}}\right\rceil
    \end{equation*}
\end{frame}


\subsection{Sporadically Periodic Events}
\label{ssec: sporadically periodic events}

\begin{frame}{}
    \begin{itemize}
        \item 散発的な周期性イベント [4] は, 散発的なイベントの特殊なクラス を表している
        \item 他の研究 [121] は同じモデルを使用しているが, このようなストリームを散発的バースト性 (B) と呼び, いくつかのイベントの各出現をバーストとみなしている
    \end{itemize}
\end{frame}

\begin{frame}{}
    \begin{itemize}
        \item ストリームは3つのパラメータによって特徴付けられる
        \item 連続するバーストの最小時間間隔を定義する外側期間 $T^{O}$, 1バースト内のイベントの最大数を決定するバースト長 $b$, 1バースト内の2イベント間の最小距離を定義する内側期間 $T^{I}$ である
    \end{itemize}
\end{frame}

\begin{frame}{上界の到着関数}
    \begin{itemize}
        \item ここでも, いくつかの簡単な考察が, 最大数のイベントによるシナリオを構築するのに役立つ
        \item 測定間隔は, バースト内の最初のイベントと一致する
        \item そのバースト内の後続のイベントは, ジッターのある周期的なストリームの「早い」イベントと同様に, できるだけ早く, すなわち, その内部周期で到着する
        \item 後続の各バーストは同じ特性を持つそして, バーストはできるだけ早く/頻繁に, すなわち外周期で再到達する
    \end{itemize}
\end{frame}

\begin{frame}{}
    \fitimage{
        バースト長が $b=4$ のシナリオと, それに対応する上界到達曲線
    }{3_12}
\end{frame}

\begin{frame}{}
    \begin{itemize}
        \item この曲線はバーストとアイドル期間を示している
        \item これは,  (最悪の場合) バースト内の周期的な性質と, 「バースト」自体の出現が散発的であることを示している
        \item $\eta^{+}(\Delta t)$ 関数は, 式3.34で与えられる
        \item この式は, a) それぞれ $b$ 個の事象からなる多数の $\left\lfloor\frac{\Delta t}{T^{O}}\right\rfloor$ 個のフルバーストと, b) 最終的に最大バースト長 $b$ によって境界づけられる残りの時間間隔 $\Delta t-\left\lfloor\frac{\Delta t}{T^{O}}\right\rfloor T^{O}$ における残りの散発的事象を分離している

              \begin{equation*}
                  \forall \Delta t>0: \eta_{\mathrm{B}}^{+}(\Delta t)=\underbrace{\left\lfloor\frac{\Delta t}{T^{O}}\right\rfloor b}_{\text {full bursts }}+\underbrace{\min \left(\left[\frac{\Delta t-\left\lfloor\frac{\Delta t}{T^{O}}\right\rfloor T^{O}}{T^{I}}\right], b\right)}_{\text {remaining events }}
              \end{equation*}
    \end{itemize}
\end{frame}

\begin{frame}{}
    \begin{itemize}
        \item 他の3つのモデルとは対照的に, 同じように単純な連続的 $\tilde{\eta}^{+}$ 関数は存在しない
        \item しかし, 散発的な周期的事象という異なる問題に着目して, 2つの連続的なものを用意できる
        \item 式3.9の $\tilde{\eta}_{\mathrm{P}}^{+}$ に内周期を適用すると, 図3.12のバースト時の最大周波数に対応する $\tilde{\eta}_{\text {inner }}^{+}$ が得られる
        \item 同様に, 外周期とバースト長を用いると $\tilde{\eta}_{\text {outer }}^{+}$ が得られ, 長期平均イベント数が得られる
        \item これらの曲線はいずれも図3.12に示されている
    \end{itemize}
\end{frame}


\subsection{Summary}
\label{ssec: summary}

\begin{frame}{}
    \begin{itemize}
        \item RMS (Rate-monotonic scheduling) などの多くのスケジューリング分析技法では, 抽象的なイベントストリームの利用が不可欠である
        \item イベントストリームとイベントモデルを定義し, 文献上最も一般的な4つのイベントモデル:厳密な周期性 $(\mathrm{P})$, ジッターを含む周期性 $(\mathrm{P}+\mathrm{J})$, 散発的 (S) , 散発的な周期性または散発的バースト (B) について検討した
        \item これらのモデルは, ストリームの挙動を記述するために, いくつかの重要なパラメータを必要とするだけである
    \end{itemize}
\end{frame}

\begin{frame}{}
    \begin{itemize}
        \item $3.7$ 式や3.22式のような例を用いて, これらのパラメータがスケジューリング分析にどのように使用されるかを示した4つの特性関数は, 全てのイベントストリームに適用される
        \item 例えば, タスクのプリエンプト回数を決定するためには, イベントの最大数 $\eta^{+}(\Delta t)$ が必要である
        \item イベント間の最小距離 $\delta^{-}(n)$ は, タスクのデッドラインを制限したり (式3.26参照) , 式3.14のようにタスクの再帰性を正しく捉えるために使用される
    \end{itemize}
\end{frame}

\begin{frame}{}
    \begin{itemize}
        \item この2つの関数は, タスクと通信の最大 (最悪) 応答時間を目標とする古典的なスケジューリング分析の領域における重要なイベントストリームの特性を捕捉している
        \item 分散システムを対象とするアプローチでは, さらに, 最小応答時間を決定するために, 最小イベント数と最大距離を必要とし, その結果, システムレベルのスケジューリング異常の解決に必要となる
        \item これらの効果については, 2.2.2節で説明したとおりである
    \end{itemize}
\end{frame}

\begin{frame}{}
    \fitimage{
        4 つのイベントモデルの特徴的な機能
    }{table_3_1}
\end{frame}
