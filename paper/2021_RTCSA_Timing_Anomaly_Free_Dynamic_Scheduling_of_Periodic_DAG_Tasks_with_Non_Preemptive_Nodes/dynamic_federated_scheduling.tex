% !TeX root = main.tex

\section{DYNAMIC-FEDERATED SCHEDULING}
\label{sec: DYNAMIC-FEDERATED SCHEDULING}

\begin{frame}{ダイナミックフェデレートスケジューラ (DynFed) の概要}
    ダイナミックフェデレートスケジューラ (DynFed) は, 以下の2つのパートから構成されている
    \begin{block}{事前計算フェーズ}
        静的に各タスクがデッドラインを守るために必要なリソース需要を導き出す
    \end{block}
    \begin{block}{ランタイムフェーズ}
        事前計算データを用いてタイミング異常が無い方法でタスクをスケジューリングする
    \end{block}
\end{frame}


\subsection{Precomputation}
\label{ssec: Precomputation}

\begin{frame}{事前計算フェーズ全体像}
    \fitimage{
        事前計算フェーズでは, 各ノードがWCETで実行される仮定のもとで, 各タスクに最小限のコア数を割り当てる
    }{alg1}
    % \begin{itemize}
    %     \item
    %     \item これは, コア数の観点から, 単一のDAGを効率的にスケジューリングする任意のスケジューリングアルゴリズムによって行うことができる
    %     \item 事前計算の手順をAlg.1に示す1に示すように, 任意のタスク $\tau_{k}$ に対して, 3つの項目を計算する
    %     \item まず, $\tau_{k}$ をスケジューリングするために必要な最小のコア数($m_{k}^{m i n}$ と表記) を1~4行目を通して計算する
    % \end{itemize}
\end{frame}

\begin{frame}{事前計算フェーズ補足1}
    \fitimage{
        \begin{itemize}
            \item 1--4行で, $\tau_k$ をスケジュールするために必要な最小のコア数 $m_k^{min}$ を計算
            \item $R T\left(\tau_{k}, m_{k}^{\text {min }}\right)$は, 各ノードがWCETで実行される場合のスケジューリングシミュレーションで得られる
        \end{itemize}
    }{alg1_sup1}
\end{frame}

\begin{frame}{事前計算フェーズ補足2}
    \fitimage{
        実行順序を得る
    }{alg1_sup2}
\end{frame}

% \begin{frame}{}
%     \begin{itemize}
%         \item 1行目から4行目まで, アルゴリズムは, 対応するサブジョブをできるだけ少ないコアに割り当てることによって, タスクがそのデッドラインに基づいて必要とするコアの数を最小化していることに注意
%         \item これは, 「並列バースト」構造を持つDAGタスク (例えば, 大きなWCETを持つソースノードから始まり, 多くの並列後続ノードが続くDAG) のリソースの無駄を減らすことができる可能性がある
%         \item その理由は, 次のサブセクションで紹介するDynFedのメカニズムにある
%     \end{itemize}
% \end{frame}

% \begin{frame}{}
%     \begin{itemize}
%         \item 例2.例として, 図 1(a)のタスク $\tau_{1}$ に Alg.1 を適用する1 を図 1(a)のタスク $\tau_{1}$ に適用し, ノードを貪欲にディスパッチし, 同値をランダムに解消するスケジューラを想定する
%         \item Alg.1の1行目は $m_{1}^{\min }=\lceil 7 / 5\rceil=2$ を計算する1は $m_{1}^{\min }=\lceil 7 / 5\rceil=2$ を計算する2行目では $R T\left(\tau_{k}, m_{k}^{\min }\right)$ が起動され, ノードを想定した応答時間WCETが計 算される
%         \item その結果, 図2 (a) に示すように, $R T\left(\tau_{k}, m_{k}^{\text {min }}\right)=5 \ngtr D_{1}$ が得られるので, whileループはスキップされる
%         \item 次に, 得られたスケジュール, すなわち図2 (a) のスケジュールに従って, 6行目が $V_{k}^{e n d}=\left\{v_{1,2}, v_{1,3}\right\}$ を計算する
%     \end{itemize}
% \end{frame}

% \begin{frame}{}
%     \begin{itemize}
%         \item 事前計算で計算されたデータを保存し, 実行時に利用する必要がある
%         \item タスク $\tau_{k}$ の場合, これには実行順序と集合 $V_{k}^{e n d} \subseteq V_{k}$ に加え, 一つの整数値である $m_{k}^{\min }$ が含まれるが, いずれもタスクのノード数に応じて直線的に増大するメモリオーバヘッドが発生する
%         \item 次節では, このデータがオンラインスケジューリングでどのように利用されるかを示す
%     \end{itemize}
% \end{frame}


\subsection{Job Dispatching}
\label{ssec: Job Dispatching}

\begin{frame}{ランタイムフェーズ概要}
    \begin{itemize}
        \item DynFedは実行時に, 先入れ先出し (FIFO) ポリシーに従ってDAGジョブをスケジューリングする
        \item このアルゴリズムは, 新しいDAGジョブがリリースされたり, 実行中のサブジョブが完了するたびに呼び出される
    \end{itemize}
\end{frame}

\begin{frame}{疑似アルゴリズムの事前準備}
    \begin{itemize}
        \item 新しい DAG ジョブ $J_{k, i}$ がリリースされたとき, アルゴリズムが起動される前に, ジョブは FIFO キューに入れられ, $\operatorname{started}\left(J_{k, i}\right)= False$ と設定される
        \item \desc{$m_k$}{$\tau_k$ のサブジョブを実行しているコア数を保持する}
        \item \desc{$m_k^{ded}$}{$\tau_k$ 用に確保されているコア数を保持する}
    \end{itemize}
\end{frame}

\begin{frame}{ランタイムフェーズ全体像}
    \fitimage{
        アルゴリズムは3つの部分から構成されている
    }{alg2}
\end{frame}

\begin{frame}{ランタイムフェーズパート1}
    \fitimage{
        1--5行は, 実行完了したサブジョブに対する処理
    }{alg2_sup1}
\end{frame}

\begin{frame}{ランタイムフェーズパート2}
    \fitimage{
        6--15行は, 可能であればジョブの実行を開始する
    }{alg2_sup2}
\end{frame}

\begin{frame}{ランタイムフェーズパート3}
    \fitimage{
        17--31行は, タスクのリソース制限を超えない限り,各タスクのサブジョブを $O_{k}$ の順序に従って割り当てる
    }{alg2_sup3}
\end{frame}

\begin{frame}{DynFedはタイミング異常が無い}
    DynFedは以下の設計になっているため, タイミング異常が無い
    \begin{itemize}
        \item ジョブレベルのスケジューリング (7--15行)は, FIFOに基づいており, ジョブ間の順序が固定されている
        \item サブジョブレベルのスケジューリング (17--31行) は事前計算で得られた順序に基づいている
    \end{itemize}
\end{frame}
