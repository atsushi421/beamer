% !TeX root = main.tex

\section{PRELIMINARIES}
\label{sec: PRELIMINARIES}

\begin{frame}{セクションサマリ}
    \begin{itembox}[l]{\textbf{目的}}
        システムモデル, 表記法, および, タイミング異常の概念について概説する
    \end{itembox}
\end{frame}


\subsection{System Model}
\label{ssec: system model}

\begin{frame}{}
    \begin{itemize}
        \item 始めに, 本論文で登場する表記法・用語の表を示す
        \item 基本的な表記法・用語は資料中で説明無しで使用する
        \item 別ファイルで開く・印刷するなどして, 常に参照できる状態にしておくことを推奨する
    \end{itemize}
\end{frame}

% !TeX root = main.tex

\begin{frame}{表記法・用語 1}
    \full{
        \begin{table}[tb]
            \adjustbox{max width=\textwidth, max height=\slideheight}{
                \centering\begin{tabular}{|c|l|} \hline
                    $m$                                                                                             & スレッド数                                                                                        \\\hline
                    $\Gamma=\left\{\mathcal{C}_{1}, \mathcal{C}_{2}, \cdots, \mathcal{C}_{|\Gamma|}\right\}$        & チェインのセット                                                                                  \\\hline
                    $|\Gamma|$                                                                                      & $\Gamma$内のチェインの数                                                                          \\\hline
                    $\mathcal{C}_{i}=\left\{c_{i, 1}, c_{i, 2}, \cdots, c_{i,\left|\mathcal{C}_{i}\right|}\right\}$ & チェイン                                                                                          \\\hline
                    $c_{i,j}$                                                                                       & $\mathcal{C}_{i}$の$j$番目のコールバック                                                          \\\hline
                    $\left|\mathcal{C}_{i}\right|$                                                                  & $\mathcal{C}_{i}$内のコールバックの数                                                             \\\hline
                    先行要素                                                                                        & $J_{i}^{k}$ 内の連続する要素 $c_{i, j}^{k}$ と $c_{i, j+1}^{k}$ の各ペアにおける $c_{i, j}^{k}$   \\\hline
                    後続要素                                                                                        & $J_{i}^{k}$ 内の連続する要素 $c_{i, j}^{k}$ と $c_{i, j+1}^{k}$ の各ペアにおける $c_{i, j+1}^{k}$ \\\hline
                    ソースコールバック                                                                              & $\mathcal{C}_{i}$ の最初のコールバック                                                            \\\hline
                    シンクコールバック                                                                              & $\mathcal{C}_{i}$ の最後のコールバック                                                            \\\hline
                \end{tabular}
            }
        \end{table}
    }
\end{frame}

\begin{frame}{表記法・用語 2}
    \full{
        \begin{table}[tb]
            \adjustbox{max width=\textwidth, max height=\slideheight}{
                \centering\begin{tabular}{|c|l|} \hline
                    $T_i$                 & \tabml{$\mathcal{C}_{i}$の周期                \\\underline{周期}: 2 つの連続するチェーンインスタンスのリリース時刻の間の最小間隔}                       \\\hline
                    $D_i$                 & \tabml{$\mathcal{C}_{i}$の相対デッドライン    \\\underline{相対デッドライン}: 時間 $r$ でリリースされた $\mathcal{C}_{i}$ の各チェーンインスタンスは, \\その絶対デッドライン $r+D_{i}$ までに終了する必要がある}           \\\hline
                    $e_{i,j}$             & $c_{i,j}$の最悪実行時間 (WCET)                \\\hline
                    $E_i$                 & $\mathcal{C}_{i}$内のコールバックのWCETの合計 \\\hline
                    $U_{i}=E_{i} / T_{i}$ & $\mathcal{C}_{i}$の利用率                     \\\hline
                \end{tabular}
            }
        \end{table}
    }
\end{frame}

\begin{frame}{表記法・用語 3}
    \full{
        \begin{table}[tb]
            \adjustbox{max width=\textwidth}{
                \centering\begin{tabular}{|c|l|} \hline
                    $J_{i}^{k}$                & $\mathcal{C}_{i}$ の $k$番目のチェーンインスタンス             \\\hline
                    $c_{i, j}^{k}$             & $J_{i}^{k}$ に含まれる$c_{i, j}$ のコールバックインスタンス    \\\hline
                    $R\left(J_{i}^{k}\right)$  & $J_{i}^{k}$の応答時間                                          \\\hline
                    $\mathcal{R}_i^{wc}$       & $\mathcal{C}_{i}$の最悪応答時間                                \\\hline
                    $\Omega$                   & ready セット                                                   \\\hline
                    $h p\left(c_{i, j}\right)$ & コールバック $c_{i, j}$ よりも優先度の高いコールバックのセット \\\hline
                \end{tabular}
            }
        \end{table}
    }
\end{frame}

\begin{frame}{表記法・用語 4}
    \full{
        \begin{table}[tb]
            \adjustbox{max width=\textwidth, max height=\slideheight}{
                \centering\begin{tabular}{|c|l|} \hline
                    updated  & $\Omega$ に新しい要素が追加されること                                                      \\\hline
                    バッチ   & \tabml{複数のコールバックインスタンスが同じポーリングポイントで $\Omega$ に追加された場合, \\これらのインスタンスは同じバッチである} \\\hline
                    ビジー   & スレッドがコールバックインスタンスが実行している状態                                       \\\hline
                    アイドル & スレッドがコールバックインスタンスを実行していない状態                                     \\\hline
                \end{tabular}
            }
        \end{table}
    }
\end{frame}

\begin{frame}{表記法・用語 5}
    \full{
        \begin{table}[tb]
            \adjustbox{max width=\textwidth, max height=\slideheight}{
                \centering\begin{tabular}{|c|l|} \hline
                    $J$   & 分析対象のチェーン                     \\\hline
                    $r$   & $J$のリリース時刻                      \\\hline
                    $f$   & $J$の終了時刻                          \\\hline
                    $c_i$ & $J$の$i$番目のコールバックインスタンス \\\hline
                    $r_i$ & $c_i$がリリースされる時刻              \\\hline
                    $s_i$ & $c_i$が実行開始する時刻                \\\hline
                    $|J|$ & $J$内のコールバックの数                \\\hline
                \end{tabular}
            }
        \end{table}
    }
\end{frame}


\begin{frame}{表記法・用語 6}
    \full{
        \begin{table}[tb]
            \adjustbox{max width=\textwidth, max height=\slideheight}{
                \centering\begin{tabular}{|c|l|} \hline
                    $\mathcal{S}_{k, i}=\left\langle e_{k, a}^{\prime}, e_{k, b}^{\prime}, \cdots\right\rangle$                                                         & $c_i$に対する$\mathcal{C}_{k}$のサブ干渉シーケンス                                            \\\hline
                    $e_{k, a}^{\prime}$                                                                                                                                 & コールバックインスタンス $c_{k, a}$ が $\left[r_{i}, s_{i}\right)$ の間に実行された時間の長さ \\\hline
                    $\mathcal{S}_{k}=\left\{\mathcal{S}_{k, 1}, \mathcal{S}_{k, 2}, \cdots, \mathcal{S}_{k,|\mathcal{C}|}\right\}$                                      & $J$に対する$\mathcal{C}_{k}$の干渉シーケンス                                                  \\\hline
                    $\mathcal{I}_{k,i}$                                                                                                                                 & $c_i$に対する$\mathcal{C}_{k}$の干渉作業                                                      \\\hline
                    $\mathcal{I}_{k}$                                                                                                                                   & $J$に対する$\mathcal{C}_{k}$の干渉作業                                                        \\\hline
                    $\mathcal{I}_{k,i}^\mathcal{P} $                                                                                                                    & \tabml{$c_i$がブロックされている間に, $c_i$と同じコールバックグループに属す                   \\$\mathcal{C}_k$のコールバックインスタンスが実行した時間の総和} \\\hline
                    $\mathcal{I}_{k,i}^\mathcal{E} $                                                                                                                    & \tabml{$c_i$がブロックされている間に, $c_i$と異なるコールバックグループに属す                 \\$\mathcal{C}_k$のコールバックインスタンスが実行した時間の総和} \\\hline
                    $\mathcal{I}_{k,i}^\mathcal{B}  $                                                                                                                   & \tabml{$[r_i, s_i)$の間に, $c_i$と同じmutually exclusiveコールバックグループに属す            \\$\mathcal{C}_k$のコールバックインスタンスが実行した時間の総和} \\\hline
                    $\mathcal{Q}_{k}=\sum_{i=1}^{|\mathcal{C}|}\left(\mathcal{I}_{k, i}+(m-1) \mathcal{I}_{k, i}^{\mathcal{B}}-\mathcal{I}_{k, i}^{\mathcal{E}}\right)$ & $\mathcal{C}_{k}$の実行が$J$の終了時間に与える影響を特徴付けるために開発した項                \\\hline
                    $\Phi_{k,i}$                                                                                                                                        & $\mathcal{Q}_k$に対する $\mathcal{S}_{k,i}$の寄与                                             \\\hline
                \end{tabular}
            }
        \end{table}
    }
\end{frame}

\begin{frame}{表記法・用語 7}
    \full{
        \begin{table}[tb]
            \adjustbox{max width=\textwidth, max height=\slideheight}{
                \centering\begin{tabular}{|c|l|} \hline
                    $L$                                    & 問題ウィンドウの長さ                                                                                         \\\hline
                    $n_{k}(L)$                             & $J$ の問題ウィンドウ中に実行できる $\mathcal{C}_{k}$ のチェーンインスタンスの最大数                          \\\hline
                    $\overrightarrow{\mathcal{M}}_{k}$     & $J$に対する$\mathcal{C}_{k}$の超干渉シーケンス                                                               \\\hline
                    $ \hat{\Phi} $                         & $\Phi_{k,i}$の上界                                                                                           \\\hline
                    $\mathcal{G}(c_{i,j}) $                & \tabml{$c_{i,j}$が属すmutually exclusiveコールバックグループのインデックス}                                  \\\hline
                    $\theta_i$                             & \tabml{$\mathcal{C}_{i}$ の各コールバックが属すmutually exclusiveコールバックグループの集合                  \\ $\theta_{i}=\cup_{\forall c_{i, j} \in \mathcal{C}_{i}}\left\{\mathcal{G}\left(c_{i, j}\right)\right\}$} \\\hline
                    $\mathcal{I}_{k, i}^{\mathcal{E}^{*}}$ & $\mathcal{S}_{k, i}$ 内の $c_{i}$ とは異なるコールバックグループに属すコールバックインスタンスの合計実行時間 \\\hline
                    $\mathcal{Q}_{k}(\mathcal{Y})$         & 各サブ干渉シーケンス $\mathcal{S}_{k, i}$ に関する $\hat{\Phi}_{k, i}$ の合計                                \\\hline
                \end{tabular}
            }
        \end{table}
    }
\end{frame}

\begin{frame}{表記法・用語 8}
    \full{
        \begin{table}[tb]
            \adjustbox{max width=\textwidth, max height=\slideheight}{
                \centering\begin{tabular}{|c|l|} \hline
                    $\Phi_{k, i}^{p, q}$                            & \tabml{$\overrightarrow{\mathcal{M}}_{k}$ の$ p $番目のコールバックインスタンスの開始時刻から \\$ q $番目のコールバックインスタンスの開始時刻までの \\範囲内にある任意のウィンドウによって発生しうる最大の $\hat{\Phi}_{k, i}$} \\\hline
                    $\left|\overrightarrow{\mathcal{M}}_{k}\right|$ & $\overrightarrow{\mathcal{M}}_{k}$ のコールバックインスタンスの数                             \\\hline
                    $\Theta_{i, p}$ ($i \in[1,|\mathcal{C}|])$      & \tabml{$ p $番目のコールバックインスタンスの開始時刻から                                      \\$\overrightarrow{\mathcal{M}}_{k}$ の最後のコールバックインスタンスの終了時刻までの \\範囲に収まる任意のウィンドウによって発生しうる最大の $\sum_{j=i}^{|\mathcal{C}|} \hat{\Phi}_{k, j}$} \\\hline
                \end{tabular}
            }
        \end{table}
    }
\end{frame}


\begin{frame}{基本の定義1}
    \begin{itemize}
        \item $m$ 個のホモジニアスマルチコアプラットフォーム上で実行される $n$ 個の周期的タスクセット $\mathcal{T}=\left\{\tau_{1}, \cdots, \tau_{n}\right\}$ を考える
        \item 各タスク $\tau_{k}$ は, サブタスクを表すノードの集合 $G_{k}=\left(V_{k}, E_{k}\right)$でモデル化される
        \item ノード $v_{k, j}$ は, $\left(v_{k, j}, v_{k, j^{\prime}}\right) \in E_{k}$ の場合, $v_{k, j^{\prime}}$ の先行ノードである
        \item 各ノード $v_{k, j}$ は, $c_{k, j} \in \mathbb{N}^{+}$ で示されるWCETを持つ
        \item $\tau_{k}$ のWCETの合計は, $\operatorname{vol}\left(\tau_{k}\right)=\sum_{j=1}^{q_{k}} c_{k, j}$ で示される
    \end{itemize}
\end{frame}

\begin{frame}{基本の定義2}
    \begin{itemize}
        \item 各タスク $\tau_{k}$ は, 相対デッドライン $D_{k}$ を割り当てられ, $T_{k}$ の周期でジョブ $\left\{J_{k, 1}, J_{k, 2}, \cdots\right\}$ の無限列をリリースする
        \item ジョブ, 例えば $J_{k, 1}$ は, $\tau_{k}$ のサブタスクからリリースされたサブジョブ $\left\{j_{k, 1,1}, j_{k, 2,1}, \ldots, j_{k, q_{k}, 1}\right\}$ から構成される
        \item サブジョブは, 対応するジョブの全ての先行タスクが完了すると, 実行可能な状態 (適格と呼ぶ) になる
              % \item 簡単のために, タスクは暗黙のデッドライン, すなわち $T_{k}=D_{k}$ を持つと仮定するが, 我々のアルゴリズムはデッドラインが制約されたタスクに対しても有効である
        \item タスクセットのハイパーピリオドは, 周期の最小公倍数 (lcm) として定義される
    \end{itemize}
\end{frame}

\begin{frame}{限定プリエンプティブ}
    プリエンプション方式は以下のような限定プリエンプティブ
    \begin{itemize}
        \item タスクの実行はプリエンプティブ
        \item サブジョブの実行はノンプリエンプティブ, すなわち一旦実行が開始されると完了するまで継続される
    \end{itemize}
\end{frame}


\subsection{Timing Anomalies}
\label{ssec: Timing Anomalies}

\begin{frame}{タイミング異常}
    タイミング異常は動的スケジューリングにおいて, 重大な問題である
    \begin{definition}[タイミング異常]
        動的スケジューリングにおいて, ジョブの実行時間がWCETより短くなったにも関わらずタスクの応答時間が増加する減少
    \end{definition}
\end{frame}

\begin{frame}{タイミング異常の例}
    \begin{columns}
        \begin{column}{0.5\textwidth}
            \begin{itemize}
                \item $\tau_{1}$ と $\tau_{2}$ が, デッドライン $D_{1}=5$ と $D_{2}=10$ で, G-EDFアルゴリズムに従って2つのコアにスケジューリングされている
                \item $j_{2,1}$ の実行時間が2から1へと短くなっているにも関わらず, $\tau_1$の応答時間が増加している
            \end{itemize}
        \end{column}
        \begin{column}{0.5\textwidth}
            \fullimage{timing}
        \end{column}
    \end{columns}
\end{frame}

\begin{frame}{タイミング異常が無いスケジューリング}
    タイミング異常を伴うスケジューラの分析は複雑すぎるため, タイミング異常を解消できる効率的なスケジューラアルゴリズムを考案する
    \begin{definition}[タイミング異常が無い]
        あるサブジョブの実行時間がWCETより小さいとき, WCETの下での各タスクの応答時間が常にその応答時間より大きいか等しいとき, スケジューラはタイミング異常が無いと言える
    \end{definition}
\end{frame}

\begin{frame}{オフラインスケジューリング}
    \begin{itemize}
        \item タイミング異常を回避する簡単な方法は, オフラインスケジューリングである
        \item しかし, スケジューラがタイマを利用できない場合がある
              \begin{block}{タイマが利用できない例}
                  \setlength{\linewidth}{0.98\columnwidth}
                  \begin{itemize}
                      \item 研究衛星MISTのオンボードコンピュータは, タイマを提供していない
                      \item ソフトウェアレベルのタイマを持つには, そのような機能を提供するミドルウェア層が必要であるが, 組み込み型のリアルタイムシステムでは必ずしもそうではない
                  \end{itemize}
              \end{block}
    \end{itemize}
\end{frame}

\begin{frame}{実行順序}
    \begin{itemize}
        \item タイミング異常のないスケジューリングを実現するためのもう一つのアプローチは, 実行順序の概念に基づくもの
        \item スケジューラが $O_{\mathcal{I}}$ の順序で計算ユニットをディスパッチする場合, タイミング異常が起きない
    \end{itemize}

    \begin{definition}[実行順序 $O_{\mathcal{I}}$]
        計算ユニット群 (サブジョブ, DAGジョブなど) のスケジュール $\mathcal{I}$ があるとき, 実行順序 $O_{\mathcal{I}}$ は, 全ユニットの開始時刻を降順でなく並べた順序である
    \end{definition}
\end{frame}

\begin{frame}{実行順序の例}
    \fitimage{
        以下のスケジュールに対する実行順序は $O_{\mathcal{I}}=\left(j_{1,1} \succeq j_{2,1} \succeq j_{1,2} \succeq j_{1,3} \succeq j_{2,2}\right)$
    }{exam1}
\end{frame}

% \begin{frame}{実行順序によるタイミング異常の回避}
%     \begin{itemize}
%         \item スケジューラが $O_{\mathcal{I}}$ の順序で計算ユニットをディスパッチする場合, タイミング異常が起きない
%         \item この結果を, 本論文で検討しているマルチタスクシステムの場合に拡張し, 1つのハイパーピリオドで全てのサブジョブの実行順序に従うようにできる
              % \item これは健全な解決策であるが, ハイパーピリオドが大きい場合, 必然的に大きなメモリオーバヘッドにつながる
              % \item これを避けるために, 全てのサブジョブの間で厳密な実行順序に従うという要件を緩和する
              % \item この目的のために, 我々はまず, セクションIIIで規定するように, 並列DAGタスクのリソース要求を捕らえる抽象的なモデルを構築する
              % \item これに基づき, セクションIVでは, タイミング異常が無いスケジューラを提案する
%     \end{itemize}
% \end{frame}
