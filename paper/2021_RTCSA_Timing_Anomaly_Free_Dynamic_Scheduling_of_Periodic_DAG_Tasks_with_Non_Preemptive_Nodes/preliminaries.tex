% !TeX root = main.tex

\section{PRELIMINARIES}
\label{sec: PRELIMINARIES}

\begin{frame}{セクションサマリ}
    \begin{itembox}[l]{\textbf{目的}}
        システムモデル, 表記法, および, タイミング異常の概念について概説する
    \end{itembox}
\end{frame}


\subsection{System Model}
\label{ssec: system model}

\begin{frame}{}
    \begin{itemize}
        \item 始めに, 本論文で登場する表記法・用語の表を示す
        \item 基本的な表記法・用語は資料中で説明無しで使用する
        \item 別ファイルで開く・印刷するなどして, 常に参照できる状態にしておくことを推奨する
    \end{itemize}
\end{frame}

% !TeX root = main.tex

\begin{frame}{表記法・用語 1}
    \full{
        \begin{table}[tb]
            \adjustbox{max width=\textwidth, max height=\slideheight}{
                \centering\begin{tabular}{|c|l|} \hline
                    $N$      & タスクの数                 \\\hline
                    $M$      & プロセッサの個数           \\\hline
                    $\tau_i$ & タスク                     \\\hline
                    $T_i$    & $\tau_i$の周期             \\\hline
                    $D_i$    & $\tau_i$の相対デッドライン \\\hline
                    $J_{i}$  & $\tau_i$のジョブ           \\\hline
                \end{tabular}
            }
        \end{table}
    }
\end{frame}

\begin{frame}{表記法・用語 2}
    \full{
        \begin{table}[tb]
            \adjustbox{max width=\textwidth, max height=\slideheight}{
                \centering\begin{tabular}{|c|l|} \hline
                    $G\left(J_{i}\right)=\langle V, E\rangle$ & $J_{i}$のDAG                                 \\\hline
                    $V$                                       & 頂点のセット                                 \\\hline
                    $E$                                       & エッジのセット                               \\\hline
                    $v$                                       & ワークロードの一部                           \\\hline
                    $c(v)$                                    & $v$のWCET                                    \\\hline
                    $(u, v) \in E$                            & $u$ と $v$ の間の優先関係                    \\\hline
                    先行頂点                                  & エッジ$(u, v)$があるとき, $u$は$v$の先行頂点 \\\hline
                    後続頂点                                  & エッジ$(u, v)$があるとき, $v$は$u$の後続頂点 \\\hline
                    ソース頂点                                & 先行頂点を持たない頂点                       \\\hline
                    シンク頂点                                & 後続頂点を持たない頂点                       \\\hline
                \end{tabular}
            }
        \end{table}
    }
\end{frame}

\begin{frame}{表記法・用語 3}
    \full{
        \begin{table}[tb]
            \adjustbox{max width=\textwidth, max height=\slideheight}{
                \centering\begin{tabular}{|c|l|} \hline
                    適格     & 頂点は, その先行頂点が全て終了している場合に適格である \\\hline
                    完全パス & ソース頂点で始まり, シンク頂点で終わるパス             \\\hline
                    $C_i$    & $G(J_i)$内の全ての頂点の合計WCET                       \\\hline
                    $L_i$    & $G(J_i)$内のクリティカルパス上の全ての頂点の合計 WCET  \\\hline
                \end{tabular}
            }
        \end{table}
    }
\end{frame}

\begin{frame}{表記法・用語 4}
\full{
\begin{table}[tb]
\adjustbox{max width=\textwidth, max height=\slideheight}{
\centering\begin{tabular}{|c|l|} \hline
    $\Theta=\left\{\theta_{1}, \cdots, \theta_{|\Theta|}\right\}$ & アクティブVPのセット \\\hline
   $\theta_{z}$ & \tabml{$\Theta$ 内の $z$番目のアクティブVPの初期バジェット \\コンテキストから明白な場合は$\Theta$ 内の $z$番目のアクティブVPも示す} \\\hline

\end{tabular}
}
\end{table}
}
\end{frame}


\begin{frame}{基本の定義1}
    \begin{itemize}
        \item $m$ 個のホモジニアスマルチコアプラットフォーム上で実行される $n$ 個の周期的タスクセット $\mathcal{T}=\left\{\tau_{1}, \cdots, \tau_{n}\right\}$ を考える
        \item 各タスク $\tau_{k}$ は, サブタスクを表すノードの集合 $G_{k}=\left(V_{k}, E_{k}\right)$でモデル化される
        \item ノード $v_{k, j}$ は, $\left(v_{k, j}, v_{k, j^{\prime}}\right) \in E_{k}$ の場合, $v_{k, j^{\prime}}$ の先行ノードである
        \item 各ノード $v_{k, j}$ は, $c_{k, j} \in \mathbb{N}^{+}$ で示されるWCETを持つ
        \item $\tau_{k}$ のWCETの合計は, $\operatorname{vol}\left(\tau_{k}\right)=\sum_{j=1}^{q_{k}} c_{k, j}$ で示される
    \end{itemize}
\end{frame}

\begin{frame}{基本の定義2}
    \begin{itemize}
        \item 各タスク $\tau_{k}$ は, 相対デッドライン $D_{k}$ を割り当てられ, $T_{k}$ の周期でジョブ $\left\{J_{k, 1}, J_{k, 2}, \cdots\right\}$ の無限列をリリースする
        \item ジョブ, 例えば $J_{k, 1}$ は, $\tau_{k}$ のサブタスクからリリースされたサブジョブ $\left\{j_{k, 1,1}, j_{k, 2,1}, \ldots, j_{k, q_{k}, 1}\right\}$ から構成される
        \item サブジョブは, 対応するジョブの全ての先行タスクが完了すると, 実行可能な状態 (適格と呼ぶ) になる
              % \item 簡単のために, タスクは暗黙的デッドライン, すなわち $T_{k}=D_{k}$ を持つと仮定するが, 我々のアルゴリズムはデッドラインが制約されたタスクに対しても有効である
        \item タスクセットのハイパーピリオドは, 周期の最小公倍数 (lcm) として定義される
    \end{itemize}
\end{frame}

\begin{frame}{限定プリエンプティブ}
    プリエンプション方式は以下のような限定プリエンプティブ
    \begin{itemize}
        \item タスクの実行はプリエンプティブ
        \item サブジョブの実行はノンプリエンプティブ, すなわち一旦実行が開始されると完了するまで継続される
    \end{itemize}
\end{frame}


\subsection{Timing Anomalies}
\label{ssec: Timing Anomalies}

\begin{frame}{タイミング異常}
    タイミング異常は動的スケジューリングにおいて, 重大な問題である
    \begin{definition}[タイミング異常]
        動的スケジューリングにおいて, ジョブの実行時間がWCETより短くなったにも関わらずタスクの応答時間が増加する減少
    \end{definition}
\end{frame}

\begin{frame}{タイミング異常の例}
    \begin{columns}
        \begin{column}{0.5\textwidth}
            \begin{itemize}
                \item $\tau_{1}$ と $\tau_{2}$ が, デッドライン $D_{1}=5$ と $D_{2}=10$ で, G-EDFアルゴリズムに従って2つのコアにスケジューリングされている
                \item $j_{2,1}$ の実行時間が2から1へと短くなっているにも関わらず, $\tau_1$の応答時間が増加している
            \end{itemize}
        \end{column}
        \begin{column}{0.5\textwidth}
            \fullimage{timing}
        \end{column}
    \end{columns}
\end{frame}

\begin{frame}{タイミング異常が無いスケジューリング}
    タイミング異常を伴うスケジューラの分析は複雑すぎるため, タイミング異常を解消できる効率的なスケジューラアルゴリズムを考案する
    \begin{definition}[タイミング異常が無い]
        あるサブジョブの実行時間がWCETより小さいとき, WCETの下での各タスクの応答時間が常にその応答時間より大きいか等しいとき, スケジューラはタイミング異常が無いと言える
    \end{definition}
\end{frame}

\begin{frame}{オフラインスケジューリング}
    \begin{itemize}
        \item タイミング異常を回避する簡単な方法は, オフラインスケジューリングである
        \item しかし, スケジューラがタイマを利用できない場合がある
              \begin{block}{タイマが利用できない例}
                  \setlength{\linewidth}{0.98\columnwidth}
                  \begin{itemize}
                      \item 研究衛星MISTのオンボードコンピュータは, タイマを提供していない
                      \item ソフトウェアレベルのタイマを持つには, そのような機能を提供するミドルウェア層が必要であるが, 組込み型のリアルタイムシステムでは必ずしもそうではない
                  \end{itemize}
              \end{block}
    \end{itemize}
\end{frame}

\begin{frame}{実行順序}
    \begin{itemize}
        \item タイミング異常のないスケジューリングを実現するためのもう一つのアプローチは, 実行順序の概念に基づくもの
        \item スケジューラが $O_{\mathcal{I}}$ の順序で計算ユニットをディスパッチする場合, タイミング異常が起きない
    \end{itemize}

    \begin{definition}[実行順序 $O_{\mathcal{I}}$]
        計算ユニット群 (サブジョブ, DAGジョブなど) のスケジュール $\mathcal{I}$ があるとき, 実行順序 $O_{\mathcal{I}}$ は, 全ユニットの開始時刻を降順でなく並べた順序である
    \end{definition}
\end{frame}

\begin{frame}{実行順序の例}
    \fitimage{
        以下のスケジュールに対する実行順序は $O_{\mathcal{I}}=\left(j_{1,1} \succeq j_{2,1} \succeq j_{1,2} \succeq j_{1,3} \succeq j_{2,2}\right)$
    }{exam1}
\end{frame}

% \begin{frame}{実行順序によるタイミング異常の回避}
%     \begin{itemize}
%         \item スケジューラが $O_{\mathcal{I}}$ の順序で計算ユニットをディスパッチする場合, タイミング異常が起きない
%         \item この結果を, 本論文で検討しているマルチタスクシステムの場合に拡張し, 1つのハイパーピリオドで全てのサブジョブの実行順序に従うようにできる
              % \item これは健全な解決策であるが, ハイパーピリオドが大きい場合, 必然的に大きなメモリオーバヘッドにつながる
              % \item これを避けるために, 全てのサブジョブの間で厳密な実行順序に従うという要件を緩和する
              % \item この目的のために, 我々はまず, セクションIIIで規定するように, 並列DAGタスクのリソース要求を捕らえる抽象的なモデルを構築する
              % \item これに基づき, セクションIVでは, タイミング異常が無いスケジューラを提案する
%     \end{itemize}
% \end{frame}
